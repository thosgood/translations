\documentclass{article}

\usepackage[margin=1.6in]{geometry}

\title{Ordinary abelian varieties over a finite field}
\author{Pierre Deligne}
\date{}

\newcommand{\doctype}{French paper}
\newcommand{\origcit}{%
  \textsc{Deligne, P.}
  ``Vari\'{e}t\'{e}s ab\'{e}liennes sur un corps fini''.
  \emph{Inventiones Math.}, Volume~\textbf{8} (1969), 238--243.
  {\url{https://publications.ias.edu/node/352}}%
}


\usepackage{amssymb,amsmath}

\usepackage{hyperref}
\usepackage{xcolor}
\hypersetup{colorlinks,linkcolor={blue!50!black},citecolor={blue!50!black},urlcolor={blue!80!black}}
\usepackage{enumerate}

\usepackage{mathrsfs}
%% Fancy fonts --- feel free to remove! %%
\usepackage{fouriernc}


\usepackage{fancyhdr}
\usepackage{lastpage}
\usepackage{xstring}
\pagestyle{fancy}
\fancypagestyle{plain}{}
\fancyhf{}
\lhead{\footnotesize\nouppercase\leftmark}
\cfoot{\small\thepage\ of \pageref*{LastPage}}
% Git commit hash for server builds
\newif\ifserver
\serverfalse
\lfoot{\footnotesize\ifserver{Git commit: \href{https://github.com/thosgood/translations/commit/GitCommitHashVariable}{GitCommitHashVariable}}\fi}


%% Theorem environments %%

\usepackage{amsthm}


%% Shortcuts %%

\newcommand{\scr}[1]{{\mathscr{#1}}}
\newcommand{\FF}{\mathbb{F}}
\newcommand{\ZZ}{\mathbb{Z}}
\newcommand{\QQ}{\mathbb{Q}}
\newcommand{\RR}{\mathbb{R}}
\newcommand{\CC}{\mathbb{C}}
\newcommand{\Pc}{\mathrm{Pc}}
\newcommand{\Fr}{F_\mathrm{r}}

\renewcommand{\geq}{\geqslant}
\renewcommand{\leq}{\leqslant}

\DeclareMathOperator{\Hom}{Hom}
\DeclareMathOperator{\Ker}{Ker}

\newcommand{\oldpage}[1]{\marginpar{\footnotesize$\Big\vert$ \textit{p.~#1}}}


%% Document %%

\usepackage{embedall}
\begin{document}

\maketitle
\thispagestyle{fancy}

\renewcommand{\abstractname}{Translator's note.}

\begin{abstract}
  \renewcommand*{\thefootnote}{\fnsymbol{footnote}}
  \emph{This text is one of a series\footnote{\url{https://thosgood.com/translations}} of translations of various papers into English.}
  \emph{The translator takes full responsibility for any errors introduced in the passage from one language to another, and claims no rights to any of the mathematical content herein.}
  
  \emph{What follows is a translation of the \doctype:}

  \medskip\noindent
  \origcit
\end{abstract}

\setcounter{footnote}{0}


%% Content %%

\oldpage{238}
We give here a down-to-earth description of the category of ordinary abelian varieties over a finite field $\FF_q$.
The result that we obtain was inspired by Ihara~\cite[ch.~V]{2} (see also \cite{3}).


\section{}
\label{1}
Let $p$ be a prime number, $\FF_p$ the field $\ZZ/(p)$, and $\overline{\FF}_p$ an algebraic closure of $\FF_p$.
For every power $q$ of $p$, let $\FF_q$ be the subfield of $q$ elements of $\overline{\FF}_p$.
For every algebraic extension $k$ of $\FF_p$, we denote by $W_0(k)$ the discrete valuation Henselian ring essentially of finite type over $\ZZ$, absolutely unramified, with residue field $k$;
let $W(k)$ be the ring of Witt vectors over $k$, i.e. the completion of $W_0(k)$.
Let $W=W(\overline{F}_p)$, and let $\varphi$ be an embedding of $W$ into the field $\CC$ of complex numbers.
We denote by $\ZZ(1)$ the subgroup $2\pi i\ZZ$ of $\CC$.
The exponential map defines an isomorphism between $\ZZ(1)\otimes\ZZ_\ell$ and $\ZZ_\ell(1)(\CC)=\varprojlim\mu_{\ell^n}(\CC)$.

We denote by $A^*$ the dual abelian variety of an abelian variety $A$.
For every field $k$, we denote by $\overline{k}$ the algebraic closure of $k$.


\section{}
\label{2}
Let $A$ be an abelian variety of dimension~$g$, defined over a field $k$ of characteristic~$p$.
Recall that $A$ is said to be \emph{ordinary} if any of the following equivalent conditions are satisfied:
\begin{enumerate}[(I)]
  \item $A$ has $p^g$ points of order dividing $p$ with values in $\overline{k}$.
  \item The ``Hasse-Witte matrix'' $F^*\colon H^1(A^{(p)},\scr{O}_{A^{(p)}}) \to H^1(A,\scr{O}_A)$ is invertible.
  \item\label{2.III}
    The neutral component of the group scheme $A_p$ that is the kernel of multiplication by $p$ is of multiplicative type (and thus geometrically isomorphic to a power of $\mu_p$).
\end{enumerate}

If $k=\FF_q$, and if $F$ is the Frobenius endomorphism of $A$, and $\Pc_A(F;x)$ is its characteristic polynomial, then these conditions are then equivalent to:
\begin{enumerate}[(I)]
\setcounter{enumi}{3}
  \item At least half of the roots of $\Pc_A(F;X)$ in $\overline{\QQ}_p$ are $p$-adic units.
    In other words, if $n=\dim A$, then the reduction $\mod p$ of the polynomial $\Pc_A(F;x)$ is not divisible by $x^{n+1}$.
\end{enumerate}


\section{}
\label{3}
Let $A$ be an ordinary abelian variety over $\overline{\FF}_p$.
We denote by $\widetilde{A}$ the canonical Serre--Tate covering \cite{4} of $A$ over $W$.
Recall that $\widetilde{A}$ depends functorially on $A$, and is characterised by the fact that the $p$-divisible group $T_p(\widetilde{A})$ over $W$ attached to $\widetilde{A}$ \cite{5} is the product of the
\oldpage{239}
$p$-divisible groups (uniquely determined, by \hyperref[2.III]{2.(III)}) that cover, respectively, the neutral component and the largest \'{e}tale quotient of $T_p(A)$.
The canonical covering $\widetilde{A}$ is again the unique covering of $A$ such that every endomorphism of $A$ lifts to $\widetilde{A}$.
We denote by $T(A)$ the integer homology of the complex abelian variety $A_\CC$ induced by $\widetilde{A}$ and $\varphi$ by the extension of scalars of $W$ to $\CC$:
\[
  T(A) = H_1(\widetilde{A}\otimes_\varphi\CC).
\]
We know that $\widetilde{A}$ descends uniquely to $W_0(\overline{F}_p)$, and so $A_\CC$ depends only on $A$ and on the restriction of $\varphi$ to $W_0(\overline{F}_p)$.
The free $\ZZ$-module $T(A)$ is of rank~$2\dim(A)$;
it is functorial in $A$.
Furthermore, if $\ell\neq p$ is a prime number, then we have, functorially, that
\[
\label{3.1}
  T(A)\otimes\ZZ_\ell = T_\ell(A).
\tag{3.1}
\]

The canonical covering of the dual abelian variety $A^*$ of $A$ is the dual of $\widetilde{A}$, and so $(A_\CC)^*=A_\CC^*$, and $T(A)$ and $T(A^*)$ are in perfect duality with values in $\ZZ(1)$:
\[
\label{3.2}
  T(A)\otimes T(A^*) \to \ZZ(1)
\tag{3.2}
\]
(it is necessary to use $\ZZ(1)$ instead of $\ZZ$ in order to obtain a theory that is invariant under complex conjugation).
The pairings \hyperref[3.2]{(3.2)} are compatible, via \hyperref[3.1]{(3.1)}, with the pairings
\[
  T_\ell(A)\otimes T_\ell(A^*) \to \ZZ_\ell(1);
\]
a morphism $\xi\colon A\to A^*$ defines a polarisation of $A$ if and only if $\xi_\CC\colon A_\CC\to A_\CC^*$ defines a polarisation of $A_\CC$.
Set
\[
  \begin{aligned}
    T'_p(A) &= \Hom(\QQ_p/\ZZ_p,A(\overline{F}_p))
  \\T''_p(A) &= \Hom_{\ZZ_p}(T'_p(A^*),\ZZ(1)\otimes\ZZ_p)
  \end{aligned}
\]
These $\ZZ_p$-modules are covariant functors in $A$.

By definition of the canonical covering, the $p$-divisible group $T_p(\widetilde{A})$ is the sum of the constant pro\'{e}tale group $T'_p(A)$ and the Cartier dual of $T'_p(A^*)$.
For every morphism $u\colon A\to B$, the induced morphism $u\colon T_p(\widetilde{A})\to T_p(\widetilde{B})$ can be identified with the sum of $u|T'_p(A)\colon T'_p(A)\to T'_p(B)$ and the Cartier transpose of $u^t|T'_p(B^*)\colon T'_p(B^*)\to T'_p(A^*)$.
Over $\CC$, we canonically have that $\ZZ(1)/(p^n)\sim\mu_{p^n}$, whence an isomorphism of functors:
\[
\label{3.3}
  T_{(p)}(A) = T(A)\otimes\ZZ_p = T'_p(A)\oplus T''_p(A).
\tag{3.3}
\]


\section{}
\label{4}
Recall that, if $\varphi\colon X\to Y$ is an isogeny between complex abelian varieties, then the exact homotopy sequence reduces to a short exact sequence:
\[
  0 \to H_1(X) \to H_1(Y) \to \Ker(\varphi) \to 0.
\]
\oldpage{240}
The abelian varieties that are quotients of $X$ by a finite subgroup, and these finite subgroups of $X$, correspond bijectively with the sub-lattice of $H_1(X)\otimes\QQ$ containing $H_1(X)$.

Let $A$ be an ordinary abelian variety over $\overline{\FF}_p$.
If $n$ is an integer coprime to $p$, then the subschemes of finite groups of order~$n$ of $A$, of $\widetilde{A}$, and of $A_\CC$, correspond bijectively, and also correspond to lattices $R$ containing $T(A)$ such that $[R:T(A)]=n$.

Set $V'_p=T'_p(A)\otimes\QQ_p$ and $V''_p(A)=T''_p(A)\otimes\QQ_p$.
The subschemes of finite groups of order~$p^k$ of $A$ are products of a \'{e}tale subgroup and an infinitesimal subgroup.
The \'{e}tale subgroups of order~$p^k$ of $A$ correspond to those of subgroups of order~$p^k$ of $A_\CC$ such that the lattice $R$ corresponding to $T(A)$ is contained inside $T_{(p)}(A)+V'_p(A)$.
By duality, the infinitesimal subgroups of $A$ correspond to the lattices $R$ containing $T(A)$ that are $p$-isogenous to $T(A)$,, i.e. such that $[R:T(A)]$ is a power of $p$ and is contained in $T_{(p)}(A)+V''_p(A)$.

All told, the finite subgroups of $A^p$, or the abelian varieties that are quotients of $A$, correspond bijectively to the lattices $R$ containing $T(A)$ such that
\[
\label{4.1}
  R\otimes\ZZ_p = (R\otimes\ZZ_p \cap V'_p) + (R\otimes\ZZ_p \cap V''_p).
\tag{4.1}
\]


\section{}
\label{5}
In particular, $A^{(p)}$, the quotient of $A$ by the largest infinitesimal subgroup of $A$ that is annihilated by $p$ (for ordinary $A$), is defined by the lattice $T(A)^{(p)}$ containing $T(A)$ that is $p$-isogenous to $T(A)$, and such that
\[
  T(A)^{(p)}\otimes\ZZ_p = T'_p(A) + \frac1p T''_p(A).
\]


\section{}
\label{6}
Let $A$ be an abelian variety over $\FF_q$, and $F\colon x\mapsto x^q$ its Frobenius endomorphism.
Recall that $A$ is uniquely determined by the pair $(\overline{A},F)$ induced by $(A,F)$ by extension of scalars from $\FF_q$ to $\overline{\FF}_q$;
the endomorphism $F$ of $\overline{A}$ factors as the relative Frobenius morphism $\Fr^{(q)}\colon\overline{A}\to\overline{A}^{(q)}$ followed by an isomorphism $F'\colon\overline{A}^{(q)}\to\overline{A}$.
If $A$ is ordinary, then we denote by $T(A)$ the $\ZZ$-module $T(\overline{A})$ endowed with the endomorphism $F$ induced by the Frobenius endomorphism of $A$.
By \hyperref[5]{\S5}, the above, and \hyperref[3.3]{(3.3)}, the lattices $T(A)$ and $F(T(A))$ are $p$-isogenous, and we have that
\[
\label{6.1}
  F(T'_p(A)) = T'_p(A),
\tag{6.1}
\]
\[
\label{6.2}
  F(T''_p(A)) = qT''_p(A).
\tag{6.2}
\]


\section{}
\label{7}
\textbf{Theorem.}
{\itshape
  The functor $A\mapsto(T(A),F)$ is an equivalence of categories between the category of ordinary abelian varieties over $\FF_q$ and the category of free $\ZZ$-modules $T$ of finite type endowed with an endomorphism $F$ that satisfy the following conditions:
\oldpage{241}
  \begin{enumerate}[(a)]
    \item $F$ is semi-simple, and its eigenvalues have complex absolute value $q^{\frac12}$,
    \item at least half of the roots in $\overline{\QQ}_p$ of the characteristic polynomial of $F$ are $p$-adic units;
      in other words, if $T$ is of rank~$d$, then the reduction $\mod p$ of the polynomial $\Pc_T(F;x)$ is not divisible by $x^{[d/2]+1}$,
    \item there exists an endomorphism $V$ of $T$ such that $FV=q$.
  \end{enumerate}

  If condition~(a) is satisfied, then conditions~(b) and (c) are equivalent to the following:
  \begin{enumerate}[(a)]
  \setcounter{enumi}{3}
    \item the module $T\otimes\ZZ_p$ admits a decomposition, stable under $F$, into two sub-$\ZZ_p$-modules $T'_p$ and $T''_p$ of equal dimension, and such that $F|T'_p$ is invertible, and $F|T''_p$ is divisible by $q$.
  \end{enumerate}
}

\begin{proof}
  \begin{enumerate}[(A)]
    \item We first prove that (a)+(b)+(c)$\implies$(d).
      If $\alpha$ is a complex eigenvalue of $F$, then $\overline{\alpha}$ is another, of the same multiplicity, and $\alpha\overline{\alpha}=q$.
      If we exclude those that are equal to $\pm q^{\frac12}$, then the eigenvalues of $F$ in $\CC$, and thus in $\overline{\QQ}_p$, can be grouped into pairs of roots $\alpha$ and $q/\alpha$.
      The roots $\alpha$ and $q/\alpha$ can not simultaneously be $p$-adic units, and so it follows from (b) that $\pm q^{\frac12}$ is not an eigenvalue of $F$, that half of the eigenvalues of $F$ in $\overline{\QQ}_p$ are $p$-adic units, say $\alpha_1,\ldots,\alpha_{d/2}$, and that the other half are of the form $\beta_1=q/\alpha_1,\ldots,\beta_{d/2}=q/\alpha_{d/2}$.
      Let $T_{(p)}=T\otimes\ZZ_p$, $V_p=T\otimes\QQ_p$, $V'_p$ the subspace of $V_p$ given by the kernel of $\prod_i(F-\alpha_i)$, and $V''_p$ the kernel of the endomorphism $\varphi=\prod_i(F-\beta_i)$.
      We have that $V_p=V'_p\oplus V''_p$.
      Let $T'_p$ be the projection from $T_{(p)}$ to $V'_p$, and let $T''_p=T_{(p)}\cap V''_p$.
      Since $\varphi$ annihilates $V''_p$, and respects $T$, it sends $T'_p$ to $T_{(p)}\cap V'_p\subset T'_p$.
      Also, $\det(\varphi|V'_p)=\prod_{i,j}(\alpha_i-\beta_j)$ is a $p$-adic unit, and so $\varphi(T'_p)=T'_p$, and $T_{(p)}\cap V'_p=T'_p$, and so $T_{(p)}=T'_p\oplus T''_p$.
    \item \emph{Full faithfulness.}
      Let $A$ and $B$ be abelian varieties over $\FF_q$, and let $\psi$ be the arrow
      \[
        \psi\colon \Hom(A,B) \to \Hom_F(T(A),T(B)).
      \]
      By the theorem of Tate \cite{7} and by \hyperref[3.1]{(3.1)}, the arrow
      \[
        \psi_\ell\colon \Hom(A,B)\otimes\ZZ_\ell \to \Hom_F(T(A),T(B))\otimes\ZZ_\ell
      \]
      is an isomorphism for $(\ell,p)=1$, and so $\psi\otimes\QQ$ is an isomorphism.
      We know that $\Hom(A,B)$ is torsion free, and so $\psi$ is injective.
      Let $u\colon A\to B$ be a morphism such that $T(u)$ is divisible by $n$.
      The induced morphism $u_\CC\colon\overline{A}_\CC\to\overline{B}_\CC$ is thus divisible by~$n$, and thus so too is $\widetilde{u}\colon\widetilde{\overline{A}}\to\widetilde{\overline{B}}$ at the generic point of $W$.
      The kernel of multiplication by~$n$ is flat over $W$;
      $\widetilde{u}$ thus disappears on this kernel, $\widetilde{u}$ and $u$ are divisible by~$n$, and $\psi$ is bijective.
    \item \emph{Necessity.}
      The fact that $(T(A),F)$ satisfies (a) follows from Weil;
        condition~(d), which implies (b) and (c), follows from \hyperref[6]{\S6}.
\oldpage{242}
    \item \emph{Isogenies.}
      Let $(T_0,F)$ satisfy (a) and (d), and let $T$ be a lattice in $T_0\otimes\QQ$, stable under $F$, that also satisfies (d).
      Suppose that $(T_0,F)$ is the image of an abelian variety $A$ over $\FF_q$; we will prove that $(T,F)$ comes from an isogenous abelian variety.
      By $T$ with $\frac1k T$, which is isomorphic to $T$, we can suppose that $T\supset T_0$.
      Condition~(d) implies that $T$ satisfies \hyperref[4.1]{(4.1)}, and that $T$ defines a subgroup $H$ of $\overline{A}$, defined over $\FF_q$, and such that $(T,F)=T(A/H)$.
    \item \emph{Surjectivity.}
      The functor $T$ induces a functor $T_\QQ$ from the category of isogeny classes of ordinary abelian varieties over $\FF_q$ to the category of finite-dimensional $\QQ$-vector spaces endowed with an automorphism $F$ that satisfies (a) and (b).
      By (D), it suffices to prove that this functor $T_\QQ$ is essentially surjective.
      It even suffices to show that every simple object $(V,F)$ in the codomain is in the image.
      By Honda~\cite{1} (see also \cite{6}), there exists an abelian variety $A$ over $\FF_q$ such that the characteristic polynomial of the Frobenius $F_A$ of $A$ is a power of that of $F$.
      The third characterisation in \hyperref[2]{\S2} of ordinary abelian varieties shows that $A$ is ordinary.
      Furthermore, $(T(A)\otimes\QQ,F)$ is the sum of copies of $(V,F)$, and thus, by (B), the isogeny class of the abelian variety $A\otimes\QQ$ is the sum of copies of an abelian variety $B$ that satisfies $T(B)\otimes\QQ=(V,F)$.
  \end{enumerate}
\end{proof}


\section{}
\label{8}
Let $(T,F)$ be a pair satisfying the hypotheses of the theorem, $2g$ the rank of $T$, $A$ the corresponding abelian variety over $\FF_q$, and $A_\CC$ the induced complex abelian variety (\hyperref[3]{\S3}).
We have that
\[
  T= H_1(A_\CC),
\]
and so $T\otimes\RR$ can be identified with the Lie algebra of $A_\CC$, and is thus endowed with a complex structure.
Here, thanks to J.-P.~Serre, is how to reconstruct this complex structure in terms of $T$, $F$, and the restriction of $\varphi$ to $W_0(\FF_p)$:

\textbf{Proposition.}
{\itshape
  The complex structure on $T\otimes\RR$ defined above is characterised by the following properties:
  \begin{enumerate}[(I)]
    \item The endomorphism $F$ is $\CC$-linear.
    \item If $v$ is the valuation of the algebraic closure $\overline{\QQ}$ of $\QQ$ in $\CC$ that extends the valuation of $W_0(\FF_p)$, then the valuations of the $g$ eigenvalues of this endomorphism are strictly positive.
  \end{enumerate}
}

\begin{proof}
  Condition~(I) is evident, and condition~(II) follows from the fact that the action of $F$ on the Lie algebra of $A$ is congruent to zero $\mod p$.
  The uniqueness of a structure satisfying (I) and (II) follows easily from condition~(b), satisfied by $(T,F)$.
\end{proof}

\oldpage{243}

%% Bibliography %%

\nocite{*}

\begin{thebibliography}{7}

  \bibitem{1}
  {Honda,~T.}
  \newblock Isogeny classes of abelian varieties over finite fields.
  \newblock {\em J. Math. Soc. Jap.} \textbf{20} (1968), 83--95.

  \bibitem{2}
  {Ihara,~Y.}
  \newblock {\em On congruence monodromy problems, vol.~I}.
  \newblock University of Tokyo, 1968.

  \bibitem{3}
  {Ihara,~Y.}
  \newblock The congruence monodromy problems.
  \newblock {\em J. Math. Soc. Jap.} \textbf{20} (1968), 107--121.

  \bibitem{4}
  {Lubin,~J., Serre,~J.-P., and Tate,~J.}
  \newblock ``Elliptic curves and formal groups''.
  \newblock Woods Hole Summer Institute 1964 (mimeographed, printed in a limited number of copies).

  \bibitem{5}
  {Serre,~J.P.}
  \newblock ``Groups $p$-divisibles (d'apr\`{e}s J.~Tate)''.
  \newblock {\em S\'{e}minaire Bourbaki} \textbf{10} (1966--67), Talk no.~318.

  \bibitem{6}
  {Tate,~J.}
  \newblock ``Classes d'isog\'{e}nies de vari\'{e}t\'{e}s ab\'{e}liennes sur un corps fini (d'apr\`{e}s T.~Honda)''.
  \newblock {\em S\'{e}minaire Bourbaki} \textbf{11} (1968--69), Talk no.~352.

  \bibitem{7}
  {Tate,~J.}
  \newblock Endomorphisms of abelian varieties over finite fields.
  \newblock {\em Inventiones Math.} \textbf{2} (1966), 134--144.

\end{thebibliography}

\end{document}
