\documentclass{article}

\usepackage[margin=1.6in]{geometry}

\title{$T$-categories\\(Categories in a triple)}
\author{Albert Burroni}
\date{}

\newcommand{\origcit}{%
  \textsc{Burroni, Albert.}
  ``$T$-catégories (catégories dans un triple)''.
  \emph{Cahiers de Topologie et Géométrie différentielle catègoriques} \textbf{12} (1971), 215--321.
  {\url{http://www.numdam.org/item?id=CTGDC_1971__12_3_215_0}}
}


%% Standards %%

\usepackage{amssymb}
\usepackage{amsmath}
\usepackage{hyperref}
\usepackage{cleveref}
\usepackage{xcolor}
\hypersetup{colorlinks,linkcolor={blue!50!black},citecolor={blue!50!black},urlcolor={blue!80!black}}
\usepackage{enumerate}
\usepackage{graphicx}


%% Typography %%

\usepackage{fouriernc}
\usepackage[cal=esstix,scr=rsfs]{mathalpha}


%% Environments %%

\usepackage{amsthm}

\renewenvironment{abstract}{%
  \quotation
  \normalsize
  \begin{center}\textbf{{\abstractname}}\end{center}
  \par\smallskip\noindent
}{\endquotation}

\newenvironment{translator}[1]
  {\phantomsection\par\medskip\noindent\small\textbf{#1.}\itshape}
  {\par\medskip}

\newenvironment{itenv}[1]
  {\phantomsection\par\medskip\noindent\textbf{#1.}\itshape}
  {\par\medskip}

\newenvironment{rmenv}[1]
  {\phantomsection\par\medskip\noindent\textbf{#1.}\rmfamily}
  {\par\medskip}


%% Shortcuts %%

\newcommand{\oldpage}[1]{\marginpar{\footnotesize$\Big\vert$ \textit{p.~#1}}}

\newcommand{\todo}{{\color{purple}\textbf{TO-DO }}}
\newcommand{\unsure}[1]{{\color{purple}\textbf{#1}}}

\newcommand{\TT}{\mathbf{T}}
\newcommand{\cat}[1]{\mathcal{#1}}
\newcommand{\Cat}[1]{\mathsf{#1}}
\newcommand{\id}{\mathrm{id}}
\newcommand{\set}[1]{|#1|}

\DeclareMathOperator{\Hom}{Hom}


%% Bibliography %%

\usepackage[backend=bibtex]{biblatex}
\addbibresource{\jobname.bib}
\renewbibmacro{in:}{%
  \ifboolexpr{%
     test {\ifentrytype{article}}%
  }{}{\printtext{\bibstring{in}\intitlepunct}}%
}


%% Git version %%

\usepackage{fancyhdr}
\usepackage{lastpage}
\usepackage{xstring}
\pagestyle{fancy}
\fancypagestyle{plain}{}
\fancyhf{}
  \renewcommand{\headrulewidth}{0pt}%
\cfoot{\small\thepage\ of \pageref*{LastPage}}
\newif\ifserver
\serverfalse
\lfoot{\footnotesize\ifserver{Git commit: \href{https://github.com/thosgood/translations/commit/GitCommitHashVariable}{GitCommitHashVariable}}\fi}


%% Document %%

\usepackage{embedall}
\begin{document}

\maketitle

\begin{translator}{Note from the translator}
  This document is a translation from French of the article

  \medskip
  {\normalfont\origcit}

  \medskip
  {\noindent}produced with kind permission from \todo{permissions}
  
  \hfill--- Timothy Hosgood (translator)
\end{translator}


%% Introduction %%

\section*{Introduction}

\hfill\emph{In friendly homage to José Luis Viviente.}

\bigskip

\oldpage{215}
If $\TT$ is a triple on a category $\cat{E}$, then $\TT$-categories are more general structures than $\TT$-algebras; they correspond to the passage from ``everywhere-defined laws'' to ``partially defined laws'' (in a very broad sense, in fact) and thus also encompass structures as diverse as topologies and categories.

Manes showed, in \cite{Ma}, that the category of compact topological spaces is a category of $\TT$-algebras, where $\TT$ is the triple of ultrafilters on $\cat{E}=\Cat{Set}$.
This led Barr, in \cite{Ba}, to define ``relational $\TT$-algebras'' so that the category of topological spaces is a category of ``relational $\TT$-algebras''.
But Barr seemed disappointed in the fact that, apart from this particular case, and that of preorders relative to the identity triple, there were few examples.
Independently, the new system of axioms for topologies that we had given in \cite{Bu} led us to a similar idea (unfortunately with the triple of filters\ldots but our goal was to situate topologies amongst quasi-topologies) that consisted in seeing topologies as a notion analogous to that of preorders (it is indicative, indeed, that the structure of a finite topology is equivalent to that of a finite preorder).
But this idea necessitated us to try passing from preorders to categories, and we found, in the definitions of Bénabou \cite{Be} of categories in terms of spans, what we needed in order to define $\TT$-categories.
We show (\cref{proposition-i.2.4}) that the ``relational $\TT$-algebras'' are obtained as a particular case: that of $\TT$-preorders (which are to $\TT$-categories what preorders are to categories).

\oldpage{216}
The general outline of this article is as follows.
A Chapter~0 is dedicated to some details of terminology.
Chapter~1 first gives a definition of $\TT$-categories that demonstrates their relation to categories: if $\cat{E}=\Cat{Set}$, then the morphisms of a $\TT$-category appear, not as arrows going from one object to another object, but as arrows going from one structure on the objects to an object.
In this way, a ``convergence'' in a topology can be represented as an arrow $F\to x$ going from a filter to a point.
It is then easy to observe that these structures must be given by the endofunctor of a triple in order to be able to define the properties of reflexivity, transitivite, identity, and associativity of a $\TT$-category.
Then we will expand on some general properties: those that can we obtain without placing any hypotheses on the triple $\TT$, essentially the existence of projective limits and the fibration of the forgetful functor to $\cat{E}$ (which will allow us, in passing, to resolve the problem that consists of universally embedding the forgetful functor of $\TT$-algebras into a fibrant functor).
Other properties, such as the existence of inductive limits, the adjunction with the forgetful functor to $\TT$-graphs (the analogue of the Stone--Čech--Barr theorem), or the cofibration of the forgetful functor to $\cat{E}$ cannot be obtained unless we suppose the triple $\TT$ to be ``bounded''.

Chapter~II is essentially dedicated to two interpretations of $\TT$-categories, on one hand as monads (or monoids, if one prefers) in the ``Kleisli pseudo-category'', and on the other as pseudo-algebras in the ``bicatgeory of spans''.

Chapter~III gives various examples, of which the most detailed is that of multicategories (which generalise those defined by Lambek in \cite{La}).
For this, we have been led to study in Sections~III.1 and III.2 the particular case where $\TT$ is a ``strongly cartesian'' triple, which not only allows for the construction of free $\TT$-categories over the usual model of free monoids (see the Appendix), but also gives a manageable description of this structure.
\oldpage{217}
We will thus quantitatively improve the examples given by Barr, and we will also later study other useful examples, but this problem appeared less important to us long as our definition seems more natural and has a more general reach.

Chapter~IV is dedicated first of all to the description of various particular cases of $\TT$-functors, for example étale (or ``discrete fibration'') $\TT$-functors, and then a generalisation, of $\TT$-profunctors, that give an approach to $\TT$-natural transformations.

In a subsequence work, we will define tensor $\TT$-categories, that will give various ``coherence'' formulas, in particular those of the ``pseudo-categories'' introduced in Chapter~II.
We equally hope to undertake a homological study of these structure.

I would like, in closing, to express my thanks to Madame Bastiani, who encouraged me to work on this subject, and who reread the manuscript, which enabled me to correct numerous imperfections and sometimes errors.




%% Content %% 

\clearpage
\tableofcontents


\clearpage
\setcounter{section}{-1}
\section{Terminology}

\oldpage{219}

Categories generalise, at the same time, monoids and preorders;
we thus obtain two types of definitions of categories (and two choices of forgetful functors to sets).
The choice between these definitions depends most of all on practical necessities or the generalisations that we wish to obtain.
The relationship between these two definitions can be clarified by the notions of monad and polyad of Bénabou \cite{Be}.

We will find ourselves in such a situation due to the various structures that we introduce, and we will call these two types of definitions ``global'' and ``local'', respectively.
The only goal of the reminders below is the allow us to make precise the terminology and to give us a model of ``local'' definitions, but, of course, we suppose that the reader knows the elements of the theory of categories and of triples, and this chapter can be quickly skimmed over.


\subsection{Categories}

A \emph{category} is a quadruple $\cat{C}=(\set{\cat{C}}, \Hom_\cat{C}, \iota, \kappa)$ satisfying conditions~1 to 6 below:

\begin{enumerate}
  \item $\set{\cat{C}}$ is a set; its elements are called \emph{objects of $\cat{C}$}.

  \item $\Hom_\cat{C}$ is a family of sets (that we suppose to be disjoint from one another, and disjoint from $\set{\cat{C}}$, to simplify constructions; but this is rarely the case in practice and, in theory, it is not indispensable).
    This family is indexed by $\set{\cat{C}}^2$, and the relations
    \[
      e\in\set{\cat{C}},
      e'\in\set{\cat{C}},
      f\in\Hom_\cat{C}(e',e)
    \]
    are expressed by simply saying that $f\colon e\to e'$ is a \emph{morphism} (in $\set{\cat{C}}$);
    we call $e$ the \emph{source of $f$} and $e'$ the \emph{target of $f$}.

  \item $\iota$ is a family of maps of the form
    \[
      \iota(e)\colon \{\varnothing\}
      \to \Hom_\cat{C}(e,e),
    \]
    where the index $e$ runs over the set $\set{\cat{C}}$.
    We denote by $\id(e)$, $\id_e$, or even simply $e$, the morphism $\iota(e)(\varnothing)\colon e\to e$.
    \oldpage{220}
    (This is a sophisticated way of giving a family of morphisms $\id_e\colon e\to e$, where $e\in\set{\cat{C}}$).

  \item $\kappa$ is a family of maps of the form
    \[
      \kappa(e'',e',e)\colon
      \Hom_\cat{C}(e'',e')\times\Hom_\cat{C}(e',e)
      \to \Hom_\cat{C}(e'',e),
    \]
    where the index $(e'',e',e)$ runs over the set $\set{\cat{C}}^3$.
    We generally denote by $g\cdot f$ the composite, i.e. the image under such a map of a pair $(g,f)$.

  \item If $f\colon e\to e'$ is a morphism in $\cat{C}$, then
    \[
      f\cdot\id_e
      = f
      = \id_{e'}\cdot f.
    \]

  \item If $f\colon e\to e'$, $g\colon e'\to e''$, and $h\colon e''\to e'''$ are ``consecutive'' morphisms in $\cat{C}$, then
    \[
      h\cdot(g\cdot f)
      = (h\cdot g)\cdot f,
    \]
    which allows to us to denote this morphism by $h\cdot g\cdot f$.
\end{enumerate}

A pair $(\set{\cat{C}},\Hom_\cat{C})$ satisfying conditions~1 and 2 is called a \emph{graph}; a triple $(\set{\cat{C}},\Hom_\cat{C},\iota)$ satisfying conditions~1, 2, and 3 is called a \emph{pointed graph}; and in both cases we use the corresponding terminology of conditions~1, 2, and 3.

We say that $F\colon\cat{C}\to\cat{C}'$ is a \emph{functor} if $\cat{C}$ and $\cat{C}'$ are categories and $F$ is a pair $(\set{F},F_1)$ such that

\begin{enumerate}
  \item[1\textquotesingle.] $\set{F}\colon\set{\cat{C}}\to\set{\cat{C}'}$ is a map.
    We denote simply by $F(e)$ the object $\set{F}(e)$, for all $e\in\set{\cat{C}}$.

  \item[2\textquotesingle.] $F_1$ is a family of maps of the form
    \[
      F_1(e',e)\colon
      \Hom_\cat{C}(e',e)
      \to \Hom_{\cat{C}'}(F(e'),F(e)),
    \]
    where the index $(e',e)$ runs over the set $\set{\cat{C}}^2$.
    If $f\colon e\to e'$ is a morphism in $\cat{C}$, we denote simply by $F(f)$ the morphism $F_1(e',e)(f)$.

  \item[3\textquotesingle.] $F(\id_e)=\id_{F(e)}$ for all $e\in\set{\cat{C}}$.

  \item[4\textquotesingle.] $F(g\cdot f)=F(g)\cdot F(f)$, if $f\colon e\to e'$ and $g\colon e'\to e''$ are consecutive morphisms in $\cat{C}$.
\end{enumerate}

A \emph{natural transformation $I\colon F\to F'$}, where $F\colon\cat{C}\to\cat{C}'$ and $F'\colon\cat{C}\to\cat{C}'$ are functors with the same source and target, is given by a family of morphisms in $\cat{C}'$ of the form $I(e)\colon F(e)\to F'(e)$, where the index $e$ runs over the set $\set{\cat{C}}$, such that for every morphism $f\colon e\to e'$ in $\cat{C}$ we have the relation $I(e')\cdot F(f)=F'(f)\cdot I(e)$.
\oldpage{221}
We denote by $\cat{C}'^{\cat{C}}$ the usual category whose objects are the functors $F\colon\cat{C}\to\cat{C}'$ and whose morphisms are the natural transformations between these functors.
The composition law in $\cat{C}'^{\cat{C}}$ will generally take




%% Bibliography %%

\nocite{*}
\printbibliography[heading=bibintoc,title=Bibliography]

\end{document}
