\documentclass{article}

\usepackage[margin=1.6in]{geometry}

\title{$T$-categories\\(Categories in a triple)}
\author{Albert Burroni}
\date{}

\newcommand{\origcit}{%
  \textsc{Burroni, Albert.}
  ``$T$-catégories (catégories dans un triple)''.
  \emph{Cahiers de Topologie et Géométrie différentielle catègoriques} \textbf{12} (1971), 215--321.
  {\url{http://www.numdam.org/item?id=CTGDC_1971__12_3_215_0}}
}


%% Standards %%

\usepackage{amssymb}
\usepackage{amsmath}
\usepackage{hyperref}
\usepackage{xcolor}
\hypersetup{colorlinks,linkcolor={blue!50!black},citecolor={blue!50!black},urlcolor={blue!80!black}}
\usepackage{enumerate}
\usepackage{graphicx}


%% Typography %%

\usepackage{fouriernc}
\usepackage[cal=esstix,scr=rsfs]{mathalpha}


%% Environments %%

\usepackage{amsthm}

\renewenvironment{abstract}{%
  \quotation
  \normalsize
  \begin{center}\textbf{{\abstractname}}\end{center}
  \par\smallskip\noindent
}{\endquotation}

\newenvironment{translator}[1]
  {\phantomsection\par\medskip\noindent\small\textbf{#1.}\itshape}
  {\par\medskip}

\newenvironment{itenv}[1]
  {\phantomsection\par\medskip\noindent\textbf{#1.}\itshape}
  {\par\medskip}

\newenvironment{rmenv}[1]
  {\phantomsection\par\medskip\noindent\textbf{#1.}\rmfamily}
  {\par\medskip}


%% Shortcuts %%

\newcommand{\oldpage}[1]{\marginpar{\footnotesize$\Big\vert$ \textit{p.~#1}}}

\newcommand{\todo}{{\color{purple}\textbf{TO-DO }}}
\newcommand{\unsure}[1]{{\color{purple}\textbf{#1}}}

\newcommand{\TT}{\mathbf{T}}
\newcommand{\cat}[1]{\mathcal{#1}}
\newcommand{\Cat}[1]{\mathsf{#1}}


%% Bibliography %%

\usepackage[backend=bibtex]{biblatex}
\addbibresource{\jobname.bib}
\renewbibmacro{in:}{%
  \ifboolexpr{%
     test {\ifentrytype{article}}%
  }{}{\printtext{\bibstring{in}\intitlepunct}}%
}


%% Git version %%

\usepackage{fancyhdr}
\usepackage{lastpage}
\usepackage{xstring}
\pagestyle{fancy}
\fancypagestyle{plain}{}
\fancyhf{}
  \renewcommand{\headrulewidth}{0pt}%
\cfoot{\small\thepage\ of \pageref*{LastPage}}
\newif\ifserver
\serverfalse
\lfoot{\footnotesize\ifserver{Git commit: \href{https://github.com/thosgood/translations/commit/GitCommitHashVariable}{GitCommitHashVariable}}\fi}


%% Document %%

\usepackage{embedall}
\begin{document}

\maketitle

\begin{translator}{Note from the translator}
  This document is a translation from French of the article

  \medskip
  {\normalfont\origcit}

  \medskip
  {\noindent}produced with kind permission from \todo{permissions}
  
  \hfill--- Timothy Hosgood (translator)
\end{translator}


%% Introduction %%

\section*{Introduction}

\hfill\emph{In friendly homage to José Luis Viviente.}

\bigskip

\oldpage{215}
If $\TT$ is a triple on a category $\cat{E}$, then $\TT$-categories are more general structures than $\TT$-algebras; they correspond to the passage from ``everywhere-defined laws'' to ``partially defined laws'' (in a very broad sense, in fact) and thus also encompass structures as diverse as topologies and categories.

Manes showed, in \cite{Ma}, that the category of compact topological spaces is a category of $\TT$-algebras, where $\TT$ is the triple of ultrafilters on $\cat{E}=\Cat{Set}$.
This led Barr, in \cite{Ba}, to define ``relational $\TT$-algebras'' so that the category of topological spaces is a category of ``relational $\TT$-algebras''.
But Barr seemed disappointed in the fact that, apart from this particular case, and that of preorders relative to the identity triple, there were few examples.
Independently, the new system of axioms for topologies that we had given in \cite{Bu} led us to a similar idea (unfortunately with the triple of filters\ldots but our goal was to situate topologies amongst quasi-topologies) that consisted in seeing topologies as a notion analogous to that of preorders (it is indicative, indeed, that the structure of a finite topology is equivalent to that of a finite preorder).
But this idea necessitated us to try passing from preorders to categories, and we found, in the definitions of Bénabou \cite{Be} of categories in terms of spans, what we needed in order to define $\TT$-categories.
We show (\cref{proposition-i.2.4}) that the ``relational $\TT$-algebras'' are obtained as a particular case: that of $\TT$-preorders (which are to $\TT$-categories what preorders are to categories).

\oldpage{216}
The general outline of this article is as follows.
A Chapter~0 is dedicated to some details of terminology.
Chapter~1 first gives a definition of $\TT$-categories that demonstrates their relation to categories: if $\cat{E}=\Cat{Set}$, then the morphisms of a $\TT$-category appear, not as arrows going from one object to another object, but as arrows going from one structure on the objects to an object.
In this way, a ``convergence'' in a topology can be represented as an arrow $F\to x$ going from a filter to a point.
It is easy to observe that


%% Content %% 

\tableofcontents

\setcounter{section}{-1}

\section{Terminology}




%% Bibliography %%

\nocite{*}
\printbibliography[heading=bibintoc,title=Bibliography]

\end{document}
