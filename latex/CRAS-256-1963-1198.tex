\documentclass{article}

\usepackage[margin=1.6in]{geometry}

\title{Double categories and structured categories}
\author{Charles Ehresmann}
\date{}

\newcommand{\origcit}{%
  \textsc{Ehresmann, Charles.}
  ``Catégories doubles et catégories structurées''.
  \emph{Comptes rendus de l'Académie des Sciences} \textbf{256} (1963), 1198--1201.
  Presented by Jean Leray.
  4\textsuperscript{th} of February, 1963.
}


%% Standards %%

\usepackage{amssymb}
\usepackage{amsmath}
\usepackage{hyperref}
\usepackage{xcolor}
\hypersetup{colorlinks,linkcolor={blue!50!black},citecolor={blue!50!black},urlcolor={blue!80!black}}
\usepackage{enumerate}
\usepackage{graphicx}


%% Typography %%

\usepackage{fouriernc}
\usepackage[cal=esstix,scr=rsfs]{mathalpha}


%% Environments %%

\usepackage{amsthm}

\renewenvironment{abstract}{%
  \quotation
  \normalsize
  \begin{center}\textbf{{\abstractname}}\end{center}
  \par\smallskip\noindent
}{\endquotation}

\newenvironment{translator}[1]
  {\phantomsection\par\medskip\noindent\small\textbf{#1.}\itshape}
  {\par\medskip}

\newenvironment{itenv}[1]
  {\phantomsection\par\medskip\noindent\textbf{#1.}\itshape}
  {\par\medskip}

\newenvironment{rmenv}[1]
  {\phantomsection\par\medskip\noindent\textbf{#1.}\rmfamily}
  {\par\medskip}


%% Shortcuts %%

\newcommand{\oldpage}[1]{\marginpar{\footnotesize$\Big\vert$ \textit{p.~#1}}}

\newcommand{\todo}{{\color{purple}\textbf{TO-DO }}}
\newcommand{\unsure}[1]{{\color{purple}\textbf{#1}}}

\newcommand{\CC}{\mathcal{C}}
\DeclareMathOperator{\sq}{\square}
\DeclareMathOperator{\vsq}{\raisebox{-.1em}{\rotatebox{90}{$\boxminus$}\,}}
\DeclareMathOperator{\hsq}{\raisebox{-.05em}{$\boxminus$}}
\newcommand{\vmult}{\mathbin{\raisebox{-.1em}{\rotatebox{90}{$\boxminus$}}}}
\newcommand{\hmult}{\mathbin{\raisebox{-.05em}{$\boxminus$}}}
\DeclareMathOperator{\FF}{\mathcal{F}}
\DeclareMathOperator{\NN}{\mathfrak{N}}
\newcommand{\MM}{\mathfrak{M}}
\newcommand{\KK}{\mathcal{K}}
\renewcommand{\SS}{\mathcal{S}}
\newcommand{\TT}{\mathcal{T}}


%% Bibliography %%

\usepackage[backend=bibtex]{biblatex}
\addbibresource{\jobname.bib}
\renewbibmacro{in:}{%
  \ifboolexpr{%
     test {\ifentrytype{article}}%
  }{}{\printtext{\bibstring{in}\intitlepunct}}%
}


%% Git version %%

\usepackage{fancyhdr}
\usepackage{lastpage}
\usepackage{xstring}
\pagestyle{fancy}
\fancypagestyle{plain}{}
\fancyhf{}
  \renewcommand{\headrulewidth}{0pt}%
\cfoot{\small\thepage\ of \pageref*{LastPage}}
\newif\ifserver
\serverfalse
\lfoot{\footnotesize\ifserver{Git commit: \href{https://github.com/thosgood/translations/commit/GitCommitHashVariable}{GitCommitHashVariable}}\fi}


%% Document %%

\begin{document}

\maketitle

\begin{translator}{Note from the translator}
  This document is a translation from French of the article

  \medskip
  {\normalfont\origcit}

  \medskip
  {\noindent}produced with kind permission from Andrée Ehresmann and \emph{Comptes rendus de l'Académie des Sciences}.

  This version has also incorporated comments and errata from \cite[Comments on Part III-1, p.~339]{coll}.
  Some of these correct minor typos; some are supplementary comments by Andrée Ehresmann and have been included verbatim, preceded by \emph{[Comm.]} and their comment number (when relevant).
  Since all the footnotes in the original were citations, we have moved these to the bibliography, which means that the footnote numbering no longer agrees with that of the original.
  The works in the bibliography have been numbered according to their number in \cite{coll}.
  The page numbers of the original article are included at the relevant locations in the margins of this version.
  
  \hfill--- Timothy Hosgood (translator)
\end{translator}



%% Content %%

\begin{abstract}Definition of structured categories; the particular case of double categories, which admit a category of quadruplets as a quotient category.\end{abstract}

\section{Double categories}
\oldpage{1198}

\emph{[Comm.] This note is developed in \cite{63}.}

\begin{rmenv}{Definition}
  We define a \emph{double category} to be a class $\CC$ endowed with two composition laws, denoted $\bullet$ and $\perp$, satisfying the following conditions:

  \begin{enumerate}
    \item $(\CC,\bullet)$ is a category, denoted $\CC^\bullet$;
      the right and left units of $f\in\CC$ will be denoted by $\alpha^\bullet(f)$ and $\beta^\bullet(f)$ respectively, and the class of units by $\CC_0^\bullet$;
    \item $(\CC,\perp)$ is a category, denoted $\CC^\perp$;
      the units of $f\in\CC^\perp$ will be denoted by $\alpha^\perp(f)$ and $\beta^\perp(f)$ respectively, and the class of units by $\CC_0^\perp$;
    \item The maps $\alpha^\bullet$ and $\beta^\bullet$ (resp. $\alpha^\perp$ and $\beta^\perp$) are functors from $\CC^\perp$ to $\CC^\perp$ (resp. from $\CC^\bullet$ to $\CC^\bullet$);
    \item \emph{Axiom of permutability.}
      If the composites $k\bullet h$, $g\bullet f$, $k\perp g$, and $h\perp f$ are defined, then
      \[
        (k\bullet h)\perp(g\bullet f)
        = (k\perp g)\bullet(h\perp f).
      \]
  \end{enumerate}

  Let $\CC$ be a class endowed with two composition laws $\bullet$ and $\perp$ satisfying axioms 1 and 2; consider the following axioms:

  \begin{enumerate}
    \item[3\textquotesingle.]
      $\CC_0^\bullet$ (resp. $\CC_0^\perp$) is stable with respect to $\perp$ (resp. to $\bullet$);
    \item[4\textquotesingle.]
      If the composites $k\bullet h$, $g\bullet f$, $k\perp g$, and $h\perp f$ are defined, then both $(k\bullet h)\perp(g\bullet f)$ and $(k\perp g)\bullet(h\perp f)$ are defined and are equal to one another.
    \item[5.]
      For all $f\in\CC$, we have
      \[
        \begin{aligned}
          \alpha^\bullet(\alpha^\perp(f))
          = \alpha^\perp(\alpha^\bullet(f)),
          &\qquad
          \beta^\bullet(\beta^\perp(f))
          = \beta^\perp(\beta^\bullet(f));
        \\\alpha^\bullet(\beta^\perp(f))
          = \beta^\perp(\alpha^\bullet(f)),
          &\qquad
          \alpha^\perp(\beta^\bullet(f))
          = \beta^\bullet(\alpha^\perp(f)).
        \end{aligned}
      \]
  \end{enumerate}
\end{rmenv}

\begin{itenv}{Proposition}
  For $(\CC,\bullet,\perp)$ to be a double category, it is necessary and sufficient that conditions 1, 2, 3\textquotesingle, 4\textquotesingle, and 5 be satisfied.
  In this case, $\CC_0^\perp$ (resp. $\CC_0^\bullet$) is a subcategory of $\CC^\bullet$ (resp. $\CC^\perp$).
\end{itenv}

A \emph{double subcategory} of a double category $\CC$ is a subclass $\CC'$ of $\CC$ that is a subcategory of $\CC^\bullet$ and of $\CC^\perp$;
then $\CC'$ is a double category for the composition laws induced by $\bullet$ and $\perp$.

\begin{rmenv}{Definition}
  Let $\CC$ be a double category;
  we define a \emph{left ideal}\footnote{\emph{[Comm.~1.6] This definition does not agree with the usual one (\cite{73,122}) in which a left ideal (or \emph{sieve}) $J$ of $\CC$ is a subclass of $\CC$ such that $J\bullet C\subset J$.}} (resp. \emph{right ideal}) of $\CC^\perp$ to be a subcategory $I^\perp$ of $\CC^\perp$ such that $\CC\bullet I^\perp=I^\perp$ (resp. $I^\perp\bullet\CC=I^\perp$), where $\CC\bullet I^\perp$ (resp. $I^\perp\bullet\CC$) is the class of composites $f\bullet g$ (resp. $g\bullet f$) for $g\in I^\perp$ and $f\in\CC$.
  We similarly define an \emph{ideal} of $\CC^\bullet$.
\end{rmenv}

\oldpage{1199}

\begin{itenv}{Proposition}
  Let $\CC$ be a double category;
  a left ideal $I^\perp$ of $\CC^\perp$ is a species of structures\footnote{\emph{[Comm.~2.1] For the definition of species of structures and hypermorphism categories (introduced in \cite{47a}), cf. \cite[\S~I, 2--3]{63}; also ([Comm.~25.2]) the set-valued functor associated to $(\CC^\bullet,\beta,\alpha^{-1}(e))$ is the partial Hom functor $\operatorname{Hom}(e,-)\colon\CC\to\mathsf{Set}$}.} \cite{47b,55} over $\CC^\bullet$ for the composition law $(f,g)\mapsto f\bullet g$ if and only if $f\bullet g$ is defined, where $f\in\CC$ and $g\in I^\perp$.
  The corresponding category $\mathscr{E}(I^\perp)$ of hypermorphisms \cite{47b,55} is a double category for the composition laws
  \[
    (f',g')\bullet(f,g)
    = (f'\bullet f,g)
  \]
  if and only if $g'=f\bullet g$; further
  \[
    (f',g')\perp(f,g)
    = (f'\perp f,g'\perp g)
  \]
  if and only if $f'\perp f$ and $g'\perp g$ are defined.
\end{itenv}


\section{Double categories of squares}

Let $\CC_1$ and $\CC_2$ be two categories with the same class of units.
Let $\sq(\CC_2,\CC_1)$ be the set of quadruples $(g_2,g_1,f_1,f_2)$, with $f_i,g_i\in\CC_i$ for $i=1,2$, such that
\[
  \begin{aligned}
    \alpha(f_1)
    = \alpha(f_2),
    &\qquad
    \alpha(g_1)
    = \beta(f_2);
  \\\beta(f_1)
    = \alpha(g_2),
    &\qquad
    \beta(g_1)
    = \beta(g_2).
  \end{aligned}
\]
We define two composition laws on $\sq(\CC_2,\CC_1)$:
\begin{itemize}
  \item \emph{Longitudinal} multiplication
    \[
      (g'_2,g'_1,f'_1,f'_2)\vmult(g_2,g_1,f_1,f_2)
      = (g'_2,g'_1g_1,f'_1f_1,f_2)
    \]
    if and only if $f'_2=g_2$;
  \item \emph{Lateral} multiplication
    \[
      (g'_2,g'_1,f'_1,f'_2)\hmult(g_2,g_1,f_1,f_2)
      = (g'_2g_2,g'_1,f_1,f'_2f_2)
    \]
    if and only if $f'_1=g_1$.
\end{itemize}

\begin{itenv}{Proposition}
  $\sq(\CC_2,\CC_1)$ is a double category for longitudinal and lateral multiplication.
\end{itenv}

Suppose that $\CC=\CC_1=\CC_2$;
recall \cite{47b,55} that a \emph{square} in $\CC$ is an element $(g_2,g_1,f_1,f_2)\in\sq(\CC,\CC)$ such that $g_1f_2=g_2f_1$.

\begin{itenv}{Corollary}
  The class $\sq\CC$ of squares in $\CC$ is a double subcategory of $\sq(\CC,\CC)$.
\end{itenv}

\begin{itenv}{Theorem}
  Let $\CC$ be a double category;
  then $\CC^\bullet$ admits a subcategory\footnote{\emph{[Comm.~2.2] cf. \cite[Theorem~6]{63}.}} of the longitudinal category $\vsq(\CC_0^\bullet,\CC_0^\perp)$ as a quotient category \cite{47b,55}, where $\CC_0^\bullet$ (resp. $\CC_0^\perp$) is endowed with its structure as a subcategory of $\CC^\perp$ (resp. of $\CC^\bullet$).
\end{itenv}


\section{Functors into a double category}

Let $\Gamma$ be a category and $\CC$ a double category;
let $\FF(\CC^\bullet,\Gamma)$ be the class of functors from $\Gamma$ to $\CC^\bullet$.

\begin{itenv}{Proposition}
  $\FF(\CC^\bullet,\Gamma)$ is a category for the composition law $(\Phi',\Phi)\mapsto\Phi'\perp\Phi$, where $(\Phi'\perp\Phi)(f)=\Phi'(f)\perp\Phi(f)$, if and only if $\Phi'(f)\perp\Phi(f)$ is defined for all $f\in\CC$.
\end{itenv}

\oldpage{1200}

\begin{rmenv}{Definition}
  Let $\CC$ and $\CC_1$ be two double categories;
  we define a \emph{double functor} from $\CC$ to $\CC_1$ to be a map $\Phi$ from $\CC$ to $\CC_1$ such that $\Phi$ is a functor from $\CC^\bullet$ to $\CC_1^\bullet$ and a functor from $\CC^\perp$ to $\CC_1^\perp$.
  The class of double functors from $\CC$ to $\CC_1$ is denoted $\FF(\CC_1,\CC)$.
\end{rmenv}

\emph{[Comm.~3.1\P] The following proposition is not correct: the class of double functors is not closed under source and target maps.}

\begin{itenv}{Proposition}
  $\FF(\CC_1,\CC)$ is a subcategory of $\FF(\CC_1^\bullet,\CC^\bullet)$ and of $\FF(\CC_1^\perp,\CC^\perp)$;
  endowed with the two induced composition laws, $\FF(\CC_1,\CC)$ is a double category.
\end{itenv}

\begin{itenv}{Proposition}
  Let $\CC$ and $\CC'$ be two categories;
  the longitudinal category $\NN(\CC',\CC)$ of natural transformations \cite{52} between functors from $\CC$ to $\CC'$ can be identified with the category $\FF(\hsq\CC',\CC)$, by identifying the natural transformation $(\varphi',\tau,\phi)$ with the functor $\Phi$ such that
  \[
    \Phi(f)
    = \big(\varphi'(f),\,\tau(\beta(f)),\,\tau(\alpha(f)),\,\varphi(f)\big)
  \]
  for all $f\in\CC$.
\end{itenv}

Consequently, if $(\CC^\bullet,\CC^\perp)$ is a double category, then a functor $\Phi$ from a category $\Gamma$ into $\CC^\bullet$ can be considered as a generalised natural transformation from $\alpha^\perp\Phi$ to $\beta^\perp\Phi$.
We will see another generalisation of natural transformations (the double category of quintets) in a following publication.


\section{Structured categories}

Let $\MM_0$ be a class of classes such that if it contains $X$ then it also contains all the subsets of $X$, and if it contains $X$ and $X'$ then it also contains the product $X\times X'$;
let $\MM$ be the category of all functions from $X$ to $Y$, where $X,Y\in\MM_0$.
Let $(\MM,p,\KK,\SS)$ be a category of homomorphisms \cite{47b,55}, with $\SS$ containing the groupoid of invertible elements of $\KK$;
let $\KK_0$ be the class of units of $\KK$;
we identify $h\in\KK$ with $(\beta^\KK(h),p(h),\alpha^\KK(h))$.

\begin{rmenv}{Definition}
  We define a \emph{structured category in $\KK$} to be a pair $(\CC^\bullet,s)$, where $\CC^\bullet$ is the structure of a category on $\CC\in\MM_0$, and $s\in\KK_0$ with $p(s)=\CC$, satisfying the following conditions:\footnote{\emph{[Comm.~3.5+] Conditions 2 and 3 are not strict enough; they are modified in \cite{63} (and in subsequent papers), where $s'$ is required to be a substructure of the product $s\times s$ on $\KK$ (and this led to the formal definition of substructures in \cite{63}, refined in \cite{69,66}). Both notions coincide if there exists a substructure on $\KK$, i.e. if there exists a pullback of $(\alpha,\beta)$ in $\KK$. Cf. [Comm~55.2], where motivations are also given.}}
  \begin{enumerate}
    \item There exists $s_0\in\KK_0$ such that
      \[
        \begin{gathered}
          p(s_0)
          = \CC_0^\bullet
        \\(s,i_{\CC_0^\bullet},s_0),\,\,
          (s_0,\alpha,s),\,\,
          (s_0,\beta,s)
          \in\KK
        \end{gathered}
      \]
      where $i_{\CC_0^\bullet}$ is the canonical injection from $\CC_0^\bullet$ into $\CC$, and $\alpha$ and $\beta$ are the source and target maps (respectively) in $\CC^\bullet$.
    \item There exists a product $s\times s$ in $\KK$ such that $p(s\times s)=\CC\times\CC$;
      if $K$ is the subclass of $\CC\times\CC$ consisting of composible pairs, then there exists $s'\in\KK_0$ such that
      \[
        \begin{gathered}
          p(s')=K
        \\(s\times s,i_K,s')\in\KK.
        \end{gathered}
      \]
    \item writing $x$ to denote the map $(g,f)\mapsto g\bullet f$ from $K$ to $\CC$, the relation $(s\times s,i_K,s')\in\KK$ implies $(s,x,s')\in\KK$.
  \end{enumerate}
\end{rmenv}

\begin{rmenv}{Example}
  A structured category in $\widetilde{\TT}$, where $\widetilde{\TT}$ is the category of topologies, is a topological category \cite{50}.
\end{rmenv}

\oldpage{1201}

\begin{itenv}{Theorem}
  For $(\CC^\bullet,\CC^\perp)$ to be a double category, it is necessary and sufficient that $(\CC^\bullet,\CC^\perp)$ be a structured category in the category $\FF$ of functors from one category to another;
  in this case, $(\CC^\perp,\CC^\bullet)$ is also a structured category in $\FF$ (the structure on $\CC^\bullet$ is $\CC^\perp$).
\end{itenv}



%% Bibliography %%

\nocite{*}
\printbibliography[heading=bibintoc,title=Bibliography]

\end{document}
