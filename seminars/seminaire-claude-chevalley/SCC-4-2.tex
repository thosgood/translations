\documentclass{article}

\usepackage[margin=1.6in]{geometry}

\title{Serre's theorem}
\author{P. Gabriel}
\date{}

\newcommand{\doctype}{French seminar talk}
\newcommand{\origcit}{%
  \textsc{Gabriel, P.}
  ``Le th\'{e}or\`{e}me de Serre''.
  \emph{S\'{e}minaire Claude Chevalley}, Volume~\textbf{4} (1958--1959), Talk no.~2.
  {\url{http://www.numdam.org/item/SCC_1958-1959__4__A2_0}}%
}

\usepackage{amssymb,amsmath}

\usepackage{hyperref}
\usepackage{xcolor}
\hypersetup{colorlinks,linkcolor={blue!50!black},citecolor={blue!50!black},urlcolor={blue!80!black}}
\usepackage{enumerate}
\usepackage{tikz-cd}

\usepackage{mathrsfs}
%% Fancy fonts --- feel free to remove! %%
\usepackage{fouriernc}


\usepackage{fancyhdr}
\usepackage{lastpage}
\usepackage{xstring}
\pagestyle{fancy}
\fancypagestyle{plain}{}
\fancyhf{}
\lhead{\footnotesize\nouppercase\leftmark}
\cfoot{\small\thepage\ of \pageref*{LastPage}}
% Git commit hash for server builds
\newif\ifserver
\serverfalse
\lfoot{\footnotesize\ifserver{Git commit: \href{https://github.com/thosgood/translations/commit/GitCommitHashVariable}{GitCommitHashVariable}}\fi}


%% Theorem environments %%

\usepackage{amsthm}

\newenvironment{itenv}[1]
  {\phantomsection\par\medskip\noindent\textbf{#1.}\itshape}
  {\par\medskip}

\newenvironment{rmenv}[1]
  {\phantomsection\par\medskip\noindent\textbf{#1.}\rmfamily}
  {\par\medskip}


%% Shortcuts %%

\newcommand{\scr}[1]{{\mathscr{#1}}}
\renewcommand{\cal}[1]{{\mathcal{#1}}}

\renewcommand{\geq}{\geqslant}
\renewcommand{\leq}{\leqslant}

\newcommand{\oldpage}[1]{\marginpar{\footnotesize$\Big\vert$ \textit{p.~#1}}}

\renewcommand{\thepart}{\Alph{part}}

%% Document %%

\usepackage{embedall}
\begin{document}

\maketitle
\thispagestyle{fancy}

\renewcommand{\abstractname}{Translator's note.}

\begin{abstract}
  \renewcommand*{\thefootnote}{\fnsymbol{footnote}}
  \emph{This text is one of a series\footnote{\url{https://thosgood.com/translations}} of translations of various papers into English.}
  \emph{The translator takes full responsibility for any errors introduced in the passage from one language to another, and claims no rights to any of the mathematical content herein.}

  \medskip
  
  \emph{What follows is a translation of the \doctype:}

  \medskip\noindent
  \origcit
\end{abstract}

\setcounter{footnote}{0}

\tableofcontents
\bigskip


%% Content %%

\part{Affine algebraic sets and classical functors}
\label{chapterA}


\section{Ringed topological spaces}
\label{section1}

\oldpage{2-01}

From now on, and unless otherwise mentioned, the rings we consider will be assumed to be commutative, with a unit element, and \emph{Noetherian}.
We define a \emph{ringed topological space} $(V,\scr{A})$ to be a topological space $V$ endowed with the structure defined by the data of a sheaf of rings $\scr{A}$.
If $(V,\scr{A})$ and $(W,\scr{B})$ are two ringed topological spaces, then we define a morphism from the first to the second by the data of:
\begin{enumerate}[(a)]
  \item a continuous map $\psi\colon V\to W$;
  \item for every open subset $U$ of $W$, a ring homomorphism
    \[
      \varphi_U\colon\scr{B}(U)\to\scr{A}(\varphi^{-1}(U))
    \]
    that is compatible with the restriction maps.
\end{enumerate}

The composition of two morphisms is defined in the evident way, and we speak of the category of ringed topological spaces.
In what follows, $V$ will almost always be a Zariski space.
To such a $V$, we often associate a topological space $S(V)$, which is \emph{the scheme of $V$}, and which is defined in the following way:
\begin{itemize}
  \item the points of $S(V)$ are the closed irreducible subsets of $V$;
  \item the closed subsets of $S(V)$ are the sets $\scr{F}$, where $F$ is a closed subset of $V$, and $\scr{F}$ denotes the set of closed irreducible subsets of $V$ that are contained in $F$.
\end{itemize}

It is then clear that the correspondence $F\mapsto\scr{F}$ between closed subsets of $V$ and closed subsets of $S(V)$ is bijective, that $S(V)$ is a Zariski space, and that every closed irreducible subset of $S(V)$ is the closure of a unique point.
To every sheaf on $V$, we canonically associate a sheaf on $S(V)$.
In particular, if $(V,\scr{A})$ is a ringed topological space, and if $V$ is a Zariski space, then we denote by $(S(V),S(\scr{A}))$ the \emph{scheme of $(V,\scr{A})$}, which is defined to be the ringed topological space given by $S(V)$ and the sheaf associated to $\scr{A}$.
Of course,
\oldpage{2-02}
the scheme of $(S(V),S(\scr{A}))$ is isomorphic to $(S(V),S(\scr{A}))$.

If $A$ is a Noetherian Jacobson ring, and $(\Omega(A),\scr{U})$ is the ringed topological space defined by its maximal spectrum, then the associated scheme is exactly $(V(A),\scr{U})$ (see Talk no.~1).
If $(V,\scr{A})$ is an algebraic set endowed with the sheaf of germs of regular functions, then the associated scheme has been defined by Chevalley in \cite{3} and \cite{4}.
In this case, $V$ is the subspace of $S(V)$ given by the closed points, and $\scr{A}$ is the restriction of $S(\scr{A})$ to $V$.
From now on, we will almost always restrict to the case of algebraic sets over an algebraically closed field.


\section{Sheaves of ideals of an algebraic set}
\label{section2}

If $V$ is an algebraic set, and $\scr{A}$ its sheaf of germs of regular functions, then the notion of quasi-coherent (resp. coherent) sheaves generalises the notion of a module (resp. module of finite type) over the ring of coordinates of an affine algebraic set.
We similarly generalise the notions of support and dimension: if $\scr{F}$ is a coherent algebraic sheaf on $V$, then its \emph{support} is the set of points $x$ of $V$ where the fibre $\scr{F}_x$ of $\scr{F}$ is not zero (if the fibre of $\scr{F}$ is zero at $x$, then it is zero on a neighbourhood of $x$, since $\scr{F}$ is coherent, and thus the support is closed).
We then define the \emph{dimension of $\scr{F}$} to be the dimension of its support.
When $V$ is affine and the coherent sheaf $\scr{F}$ is associated to a module $M$ of finite type, then the support of $\scr{F}$ is given by the set of maximal ideals of the coordinate ring that contain the annihilator of $M$.

The notion of an ideal generalises in the following way: we define a \emph{sheaf of ideals} of $(V,\scr{A})$ to be any quasi-coherent (and thus coherent) subsheaf of $\scr{A}$.
We then find the inevitable correspondence between ideals and closed subsets:

If $\scr{I}$ is a sheaf of ideals, then we associate to it the following closed subset $W(\scr{I})$ of $V$:

$x\in W(\scr{I})$ if and only if $\scr{I}_x$, the fibre of $\scr{I}$ at $x$, is a proper ideal of $\scr{O}_x$ (i.e. $\scr{I}_x\neq\scr{O}_x$), or if and only if the germs of the functions defined by $\scr{I}$ at $x$ are zero.
$W(\scr{I})$ is thus the support of $\scr{A}|\scr{I}$.

Conversely, if $W$ is a closed subset of $V$, then we associate to it the following sheaf of ideals $\scr{I}(W)$:

The sections of $\scr{I}(W)$ on an open subset $U$ are the regular functions defined on $U$ that are zero on $U\cap W$.
If $\scr{I}(U,x)$ denotes the inverse image in $\scr{A}(U)$ of the maximal ideal of $\scr{O}_x$, then $\Gamma(U,\scr{I}(W))$ is the intersection
\oldpage{2-03}
of the $\scr{I}(U,x)$ where $x$ runs over $U\cap W$.
If $U$ is an affine open subset, then the sheaf of ideals $\scr{I}(W)|U$ is exactly the sheaf associated to the ideal $\Gamma(U,\scr{I}(W))$ of $\scr{A}(U)$, which shows that $\scr{I}(W)$ is coherent.

\begin{itenv}{Proposition 1}
\label{proposition1}
  \begin{enumerate}[(a)]
    \item The map $W\mapsto\scr{I}(W)$ gives a bijective correspondence between the closed subsets of $V$ and the sheaves of ideals $\scr{I}$ that satisfy the following condition:
      for every $x\in V$, $\scr{I}_x$ is either equal to $\scr{O}_x$ or equal to an intersection of prime ideals of $\scr{O}_x$.
    \item The map $W\mapsto\scr{I}(W)$ associates the closed subsets whose connected components are all irreducible to the sheaves of prime ideals (for every $x$, $\scr{I}_x$ is either equal to $\scr{O}_x$ or equal to a prime ideal).
  \end{enumerate}
\end{itenv}

\begin{proof}
  \begin{enumerate}[(a)]
    \item It suffices to give a proof in the case where $(V,\scr{A})$ is an affine algebraic set.
      So let $A=\scr{A}(V)$, and $\mathfrak{a}=\Gamma(V,\scr{I}(W))$.
      We have already seen that $\mathfrak{a}$ is then the intersection of the prime ideals $\scr{I}(V,x)$, where $x$ runs over $W$.
      So $\mathfrak{a}$ is an intersection of prime ideals.
      Since the correspondence between intersections of prime ideals of $A$ and closed subsets of $V$ is bijective, so too is the correspondence between ideals of $A$ and sheaves of ideals of $(V,\scr{A})$;
      it remains only to show, conversely, that the sheaf associated to an intersection $\mathfrak{a}$ of prime ideals satisfies the condition of the proposition: this follows from the conservation properties of the prime decomposition under localisation.
    \item The proof is analogous.
  \end{enumerate}
\end{proof}


\section{The ring of rational functions of an algebraic set}
\label{section3}

We recall that, if $V$ is an algebraic set, then we define a \emph{rational function on $V$} to be a regular map $f$ from an everywhere-dense open subset of $V$ to the field of constants $K$.
We further suppose that the domain of definition of $f$ cannot be extended.
The rational functions on $V$ form an algebra $K(V)$ over $K$.
If we denote by $V_i$ the irreducible components of $V$, then $K(V)$ is isomorphic to the product $\prod_i K(V_i)$, with the isomorphism being the obvious one.
Finally, if $V$ is irreducible, and if $U$ is an affine open subset of $V$, then $K(V)$ is exactly the field of fractions of the coordinate ring of $U$.

The sheaf $\scr{K}(V)$ of rational functions on $V$ is then defined in the following way:
to every open subset $U$ of $V$, we associate the ring $K(U)$ of
\oldpage{2-04}
rational functions on $U$, with the restrictions being obvious.
It is clear that this defines a quasi-coherent sheaf on $V$.
Furthermore, if we denote by $V_i$ the irreducible components of $V$, then $K(U)$ is exactly the product $\prod_{V_i\cap U\neq\varnothing}K(V_i)$.
It thus follows that the sheaf $\scr{K}(V)$ is obtained in the following way:
let $\scr{K}_i$ be the sheaf on $V$ that is zero outside of $V_i$, and that is constant with fibre $K(V_i)$ on $V_i$.
Then $\scr{K}(V)$ is the product of the sheaves $\scr{K}_i$.


\section{Characterisation of affine algebraic sets}
\label{section4}

Let $(V,\scr{A})$ be an algebraic set endowed with its sheaf of rings, and let $A=\Gamma(V,\scr{A})$ be the coordinate ring of $V$.
We know that we then have a canonical morphism $(V,\scr{A})\to(S(V),S(\scr{A}))$, and that $(V,\scr{A})$ is affine if and only if $(S(V),S(\scr{A}))$ is the prime spectrum of an algebra over $K$, of finite type and with no nilpotent elements.

We will now define a morphism $(S(V),S(\scr{A}))\to(V(A),\scr{U})$.
For this, let $x$ be an arbitrary element of $S(V)$, and let $S(\scr{A})_x$ be the fibre of $S(\scr{A})$ at $x$: this an a local ring, and we have a restriction homomorphism $A\to S(\scr{A})_x$.
We denote by $\mathfrak{p}_x$ the prime ideal of $A$ given by the inverse image under this homomorphism of the maximal ideal of $S(\scr{A})_x$.
It is an ideal of functions on $V$ that are zero on the closed point $x$.

The map $\varphi\colon x\mapsto\mathfrak{p}_x$ is a continuous map from $S(V)$ to $V(A)$.
Indeed, it suffices to show that the inverse image of a special open subset $U_f$ of $V(A)$ is an open subset of $S(V)$.
But $\varphi^{-1}(U_f)$ consists of points $x$ of $S(V)$ such that the image $f_x$ of $f$ in $\scr{A}_x$ is invertible, and if $f_x$ is invertible at $x$, then $f_y$ is invertible for $y$ in a neighbourhood of $x$.
QED.

It follows from the above that $1/f$ is a section of $S(\scr{A})$ over $\varphi^{-1}(U_f)$, and so the canonical map from $A$ to $\Gamma(\varphi^{-1}(U_f),S(\scr{A}))$ extends to a homomorphism:
\[
  \varphi_{U_f}\colon A_f \to \Gamma(\varphi^{-1}(U_f), S(\scr{A})).
\]

The $\varphi_{U_f}$ are compatible with the restriction maps, and, by ``passing to the inductive limit'', they define a morphism of ringed topological spaces:
\oldpage{2-05}
\[
  \varphi\colon (S(V),S(\scr{A})) \to (V(A),\scr{U}).
\]

It is clear that $\varphi$ is an isomorphism if and only if $(V,\scr{A})$ is an affine algebraic set.
The following theorem further examines this case:

\begin{itenv}{Theorem (Serre's Theorem)}
  The following are equivalent:
  \begin{enumerate}[(a)]
    \item $(V,\scr{A})$ is an affine algebraic set.
    \item If $0\to\scr{F}\to\scr{G}\to\scr{H}\to0$ is an exact sequence of quasi-coherent sheaves, then the sections over $V$ form an exact sequence.
    \item If $0\to\scr{F}\to\scr{G}\to\scr{H}\to0$ is an exact sequence of coherent sheaves, then the sections over $V$ form an exact sequence.
    \item There exist sections $f_i$ of $\scr{A}$ over $V$ such that:
      \begin{itemize}
        \item the ideal generated by the $f_i$ is equal to $A$; and
        \item the open subsets $V_{f_i}$ of $V$, where the $f_i$ are non-zero, are affine open subsets, and they cover $V$.
      \end{itemize}
  \end{enumerate}
\end{itenv}

\begin{proof}
  It remains only to show that (c) implies (d), and that (d) implies (a).

  \bigskip
  For (c)$\implies$(d), consider:
  \begin{itenv}{Lemma 1}
  \label{lemma1}
    If $\scr{F}$ is a non-zero coherent sheaf, then $\Gamma(V,\scr{F})$ is non-zero.
  \end{itenv}

  \begin{proof}
    Indeed, the support of $\scr{F}$ contains a closed point, say $x$.

    The fibre $\scr{F}_x$ of $\scr{F}$ is non-zero at $x$, and, if $\mathfrak{m}_x$ denotes the maximal ideal of $\scr{A}_x$, then $\scr{F}_x|\mathfrak{m}_x\scr{F}_x$ is non-zero (by the Nakayama lemma).
    The sheaf $\scr{G}$ that is zero away from $x$ and ``has the value'' $\scr{F}_x|\mathfrak{m}_x\scr{F}_x$ at $x$ is then coherent, and we clearly have a surjection:
    \[
      \scr{F}\to\scr{G}\to0.
    \]

    It thus follows that we have an epimorphism from $\Gamma(V,\scr{F})$ to $\Gamma(V,\scr{G}) = \scr{F}_x|\mathfrak{m}_x\scr{F}_x$, and $\Gamma(V,\scr{F})$ is non-zero.
  \end{proof}

  With this proven, let $f$ be a section of $\scr{A}$ over $V$ (a regular function on $V$).
  If $U$ is an arbitrary open subset of $V$, then we denote by $U_f$ the set of points of $U$ where $f$ is non-zero.
  The existence of the cover $(U_{f_i})$ will be a consequence of:
  \begin{itenv}{Lemma 2}
  \label{lemma2}
\oldpage{2-06}
    If $x$ is an arbitrary point of $V$, and $U$ is an affine open subset of $V$ containing $x$, then there exists a regular function $f$ on $V$ such that $V_f=U_f$, and such that $V_f$ contains $x$.
  \end{itenv}

  \begin{proof}
    Let $W$ be the complement of $U$ in $V$, and let $\scr{I}(W)$ (resp. $\scr{I}(W,x)$) be the sheaf of germs of functions that are zero on $W$ (resp. on $W$ and $x$).
    We then have an exact sequence:
    \[
      0 \to \scr{I}(W,x) \to \scr{I}(W) \to \scr{I}(W)|\scr{I}(W,x) \to 0
    \]
    and the sheaf $\scr{I}(W)|\scr{I}(W,x)$ is clearly non-zero.
    So there exists a section $f$ of $\scr{I}(W)$ that does not belong to $\Gamma(\scr{I}(W,x))$;
    this is the desired function.
  \end{proof}

  The quasi-compactness of $V$, along with \hyperref[lemma2]{Lemma~2}, imply the existence of an affine cover of $V$ by a finite number of $V_{f_i}$.
  It remains only to show that the ideal generated by the $f_i$ is $A$.
  But if $U$ is an affine open subset, then the $U_{f_i}$ cover $U$, and it follows that restrictions of the $f_i$ to $\scr{A}(U)$ generate the unit ideal.
  In other words, if there are $p$ sections $f_i$, and if $\scr{A}^p$ denotes the direct sum of $p$ sheaves, each isomorphic to $\scr{A}$, then the morphism from $\scr{A}^p$ to $\scr{A}$ defined by the $f_i$ is surjective.
  The same is true for the morphism $A^p\to A$ defined by the $f_i$.

  \bigskip

  For (d)$\implies$(a):
  we will show that the morphism
  \[
    \varphi\colon (S(V),S(\scr{A})) \to (V(A),\scr{U})
  \]
  is an isomorphism, and that $A$ is an algebra of finite type with no nilpotent elements.

  Note first of all that, if $f\in A$ is a regular function on $V$, then the coordinate ring of $V_f$ is exactly $A_f$.
  Indeed, if $A_i$ denotes the coordinate ring of $V_{f_i}$, and $A_{ij}=A_{if_j}=A_{jf_i}$ the coordinate ring of $V_{f_i}\cap V_{f_j}=V_{f_i\cdot f_j}$, then we have the famous exact sequence (see Talk no.~1):
  \[
    0 \to A \to \prod_i A_i \to \prod_{i\neq j}A_{ij}
  \]
  whence (by the exactness of the functor $M\mapsto M_f$) we have the exact sequence:
  \[
    0 \to A_f \to \prod_i(A_i)_f \to \prod(A_{ij})_f.
  \]

  But the $(A_i)_f$ and $(A_{ij})_f$ are clearly the coordinate rings of
\oldpage{2-07}
  $V_{f_i}\cap V_f$ and $V_{f_i}\cap V_{f_j}\cap V_f$;
  thus $A_f$ is coordinate ring of $V_f$.

  From this we see that the map $S(V)\to V(A)$ is injective, and that it induces an isomorphism
  \[
    (S(V_{f_i}), \scr{A}|S(V_{f_i})) \to (V(A)_{f_i}, \scr{U}|V(A)_{f_i}).
  \]

  Also, the $V(A)_{f_i}$ cover $V(A)$, since the $f_i$ generate the unite ideal.
  By gluing the pieces together, we thus obtain an isomorphism
  \[
    (S(V),S(\scr{A})) \to (V(A),\scr{U})
  \]
  and it remains only to show that $A$ is of finite type and has no nilpotent elements.

  But $A$ is an algebra of functions, and has no nilpotent elements.
  Now, for all $i$, let $a_{i_k}$ be a finite number of elements of $A$ such that the images of $f_i$ and $a_{i_k}$ in $A_{f_i}$ generate the ring $A_{f_i}$.
  Finally, let $a_i\in A$ be such that $\sum a_i f_i=1$ in $A$.
  We will show that the $f_i$, $a_{i_k}$, and $a_i$ all together generate the ring $A$:

  Indeed, by taking a suitable power of the formula $\sum a_i f_i=1$, we obtain formulas $\sum a_i(m)f_i^m=1$, where the $a_i(m)$ can be expressed in terms of the $a_i$ and $f_i$.
  If now $x$ is an element of $A$, then its image in $A_{f_i}$ can be written as $x_i|f_i^p$ for some $p$ large enough.
  It then follows (see Talk no.~1) that, for $n$ large enough, we have
  \[
    x = \sum_i a_i(n+p) x_i f_i^n
  \]
  which finishes the proof.
\end{proof}

\begin{itenv}{Corollary}
  If $V$ is an affine algebraic set, $U$ an open subset of $V$, $A$ the coordinate ring of $V$, and $\scr{U}(U)$ the coordinate ring of $U$, then the following are equivalent:
  \begin{enumerate}[(a)]
    \item $U$ is an affine open subset.
    \item If $M$ is an $A$-module, then the canonical homomorphism
      \[
        M\otimes_A\scr{U}(U) \to \scr{M}(U)
      \]
      is bijective.
  \end{enumerate}

\oldpage{2-08}
  Furthermore, if one of these equivalent properties is satisfied, then $\scr{U}(U)$ is $A$-flat.
\end{itenv}

\begin{proof}
  (a)$\implies$(b):
  Indeed, $M\mapsto\scr{M}(U)$ is then a right-exact functor in $M$, where $M$ runs over all $A$-modules.
  In particular, if
  \[
    L_1 \to L_0 \to M \to 0
  \]
  is a resolution of $M$ by free $A$-modules, then we have the diagram
  \[
    \begin{tikzcd}
      L_1\otimes_A\scr{U}(U) \rar \dar["w"]
      & L_0\otimes_A\scr{U}(U) \rar \dar["v"]
      & M\otimes_A\scr{U}(U) \rar \dar["u"]
      & 0
    \\\scr{L}_1(U) \rar
      & \scr{L}_0(U) \rar
      & \scr{M}(U) \rar
      & 0
    \end{tikzcd}
  \]

  Since $v$ and $w$ are isomorphisms, so too is $u$.

  \bigskip
  (b)$\implies$(a):
  We will show that, if
  \[
    0 \to \scr{F}' \to\scr{G}' \to \scr{H}' \to 0
  \]
  is an exact sequence of coherent sheaves on $U$, then the sections on $U$ give an exact sequence.

  But there exists an exact sequence
  \[
    0 \to \scr{F} \to \scr{G} \to \scr{H} \to 0
  \]
  of coherent sheaves on $V$ such that $\scr{F}|U=\scr{F}'$, $\scr{G}|U=\scr{G}'$, and $\scr{H}|U=\scr{H}'$ (see \cite{1}, with the proof in the appendix of \cite{2}).
  So, if $M$, $N$, and $P$ are the modules associated to $\scr{F}$, $\scr{G}$, and $\scr{H}$, then the sequence
  \[
    M\otimes_A\scr{U}(U) \to N\otimes_A\scr{U}(U) \to P\otimes_A\scr{U}(U) \to 0
  \]
  is exact, which proves that the homomorphism
  \[
    \scr{G}'(U) = N\otimes_A\scr{U}(U) \to \scr{H}'(U) = P\otimes_A\scr{U}(U)
  \]
  is surjective.
\end{proof}


%% Bibliography %%

\nocite{*}

\begin{thebibliography}{10}

  \bibitem{1}
  {Borel, A. and Serre, J.-P.}
  \newblock Le th\'{e}or\`{e}m de Riemann-Roch.
  \newblock {\em Bull. Soc. math. France} \textbf{86} (1958), 97--136.

  \bibitem{2}
  {Grothendieck, A.}
  \newblock La th\'{e}orie des classes de Chern.
  \newblock {\em Bull. Soc. math. France} \textbf{86} (1958), 137--154.

  \bibitem{3}
  {S\'{e}minaire Cartan-Chevalley:}
  \newblock G\'{e}om\'{e}trie alg\'{e}brique.
  \newblock {\em Volume}~\textbf{8} (1955/56).

  \bibitem{4}
  {S\'{e}minaire Chevalley:}
  \newblock Classification des groupes de Lie alg\'{e}brique.
  \newblock {\em Volume}~\textbf{1} (1958).

\end{thebibliography}

\end{document}
