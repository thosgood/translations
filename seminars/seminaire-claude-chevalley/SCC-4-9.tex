\documentclass{article}

\usepackage[margin=1.6in]{geometry}

\title{Abelian varieties}
\author{A. Douady}
\date{}

\newcommand{\doctype}{French seminar talk}
\newcommand{\origcit}{%
  \textsc{Douady, A.}
  ``Vari\'{e}t\'{e}s ab\'{e}liennes''.
  \emph{S\'{e}minaire Claude Chevalley}, Volume~\textbf{4} (1958--1959), Talk no.~9.
  {\url{http://www.numdam.org/item/SCC_1958-1959__4__A9_0}}%
}


\usepackage{amssymb,amsmath}

\usepackage{hyperref}
\usepackage{xcolor}
\hypersetup{colorlinks,linkcolor={blue!50!black},citecolor={blue!50!black},urlcolor={blue!80!black}}
\usepackage{enumerate}

\usepackage{mathrsfs}
%% Fancy fonts --- feel free to remove! %%
\usepackage{fouriernc}


\usepackage{fancyhdr}
\usepackage{lastpage}
\usepackage{xstring}
\pagestyle{fancy}
\fancypagestyle{plain}{}
\fancyhf{}
\lhead{\footnotesize\nouppercase\leftmark}
\cfoot{\small\thepage\ of \pageref*{LastPage}}
% Git commit hash for server builds
\newif\ifserver
\serverfalse
\lfoot{\footnotesize\ifserver{Git commit: \href{https://github.com/thosgood/translations/commit/GitCommitHashVariable}{GitCommitHashVariable}}\fi}


%% Theorem environments %%

\usepackage{amsthm}

\newenvironment{itenv}[1]
  {\phantomsection\par\medskip\noindent\textbf{#1.}\itshape}
  {\par\medskip}

\newenvironment{rmenv}[1]
  {\phantomsection\par\medskip\noindent\textbf{#1.}\rmfamily}
  {\par\medskip}


%% Shortcuts %%

\renewcommand{\geq}{\geqslant}
\renewcommand{\leq}{\leqslant}

\newcommand{\scr}[1]{{\mathscr{#1}}}
\newcommand{\fk}[1]{{\mathfrak{#1}}}

\DeclareMathOperator{\codim}{codim}

\newcommand{\oldpage}[1]{\marginpar{\footnotesize$\Big\vert$ \textit{p.~#1}}}


\renewcommand{\thesubsection}{\Alph{subsection}}
\usepackage{manfnt}


%% Document %%

\usepackage{embedall}
\begin{document}

\maketitle
\thispagestyle{fancy}

\renewcommand{\abstractname}{Translator's note.}

\begin{abstract}
  \renewcommand*{\thefootnote}{\fnsymbol{footnote}}
  \emph{This text is one of a series\footnote{\url{https://thosgood.com/translations}} of translations of various papers into English.}
  \emph{The translator takes full responsibility for any errors introduced in the passage from one language to another, and claims no rights to any of the mathematical content herein.}

  \medskip
  
  \emph{What follows is a translation of the \doctype:}

  \medskip\noindent
  \origcit
\end{abstract}

\setcounter{footnote}{0}

\tableofcontents


%% Content %%

\section{Algebraic groups}
\label{1}

\oldpage{9-01}
\begin{rmenv}{Definition}
  An \emph{algebraic group} is a pair $(G,\varphi)$, where $G$ is an algebraic variety and $\varphi$ is a morphism from $G\times G$ to $G$ that endows the set of points of $G$ with the structure of a group.
\end{rmenv}

\begin{rmenv}{Properties}
  For every point $a$ of $G$, the translations $l_a$ and $r_a$ defined by $l_a(x)=\varphi(a,x)$ and $r_a(x)=\varphi(x,a)$ are automorphisms of the algebraic variety structure of $G$.

  Every point $x$ of $G$ is simple (indeed, if $y$ is a simple point of $G$, then there exists an automorphism of $G$ that sends $y$ to $x$).

  If $G$ is connected, then $G$ is irreducible.
  The map $\pi\colon G\to G$ that, to any point, associates its inverse under the group law is a morphism (and thus an automorphism) of the algebraic variety structure.
  The proof of this property uses the fact that a bijective (and thus radicial) morphism from one variety to another is birational if it is unramified (\cite[p.~211, Corollary~2 to Proposition~3, Section~II, Chapter~VI]{1}).

  (There would be no problem with asking for $G$ to be connected as part of the definition of algebraic groups).
\end{rmenv}

\begin{rmenv}{Definition}
  An \emph{abelian variety} is an algebraic group whose variety is connected (and thus irreducible) and complete.
\end{rmenv}

We will show that this implies that the group is commutative.


\section{A property of complete varieties}
\label{2}

Recall that a variety $V$ is said to be complete if, for every variety $T$ and every closed subset $F\subset T\times V$, the projection from $F$ to $T$ is closed.
In the classical case, this property is equivalent to compactness.


\subsection{Proposition 0}
\label{2.A}

\begin{itenv}{Proposition 0}
\label{proposition0}
  Let $V$ be a complete connected variety, $T$ a connected variety, and $f$ a morphism from $T\times V$ to another variety $U$.
  Then, if, for some $t_0\in T$, $f(t_0,v)$ does not depend on $v$, then
  \[
    f(t,v)=\varphi(t)
    \quad\mbox{for all $t,v$}
  \]
  where $\varphi$ is a morphism from $T$ to $U$.
\end{itenv}

\oldpage{9-02}
\begin{proof}
  If $v_1,v_2\in V$, then the set $P_{v_1,v_2}$ of $t\in T$ such that $f(t,v_1)=f(t,v_2)$ is closed in $T$.
  The set $P=\bigcap P_{v_1,v_2}$ of $t\in T$ for which $f(t,v)$ does not depend on $v$ is thus also closed.
  We will now show that it is also open.
  If $t_1\in P$, then $f(t_1,v)=u_1$ for all $v$.
  Let $U\setminus F$ be an affine neighbourhood of $u_1$, with $F$ closed, so $f^{-1}(F)$ is closed, $G=\operatorname{pr}_T(f^{-1}(F))$ is closed, and $T\setminus G$ is a neighbourhood of $t_1$.
  For $t'\in T\setminus G$, $f$ defines a morphism $v\mapsto f(t',v)$ from $V$, which is complete and connected, to $U\setminus F$, which is affine.
  This map is necessarily constant.
  Thus $P\supset T\setminus G$ is a neighbourhood of each of its points, i.e. an open subset.

  Since $T$ is connected, if $t_0\in P$, then $P=T$, which finishes the proof.
  (To see that $\varphi$ is a morphism, it suffices to take a point $v_0\in V$, without worrying about the case where $V=\varnothing$).
\end{proof}

\begin{rmenv}{Remark}
  There is an analogous statement in analytic geometry:
  let $V$ be a compact connected complex-analytic space, $T$ a connected topological space, and $f$ a continuous map from $T\times V$ to another analytic space $U$, such that $f$ induces, for all $t\in T$, a holomorphic map $f_t$ from $\{t\}\times V$ to $U$.
  Then, if $f_t$ is constant for $t=t_0$, then it is constant for all $t\in T$.
  In other words, a holomorphic map from $V$ to $U$ that is homotopic to a constant map amongst holomorphic maps is constant.
  The hypothesis that $f_t$ be holomorphic for all $t$ is essential: there are counter-examples with non-K\"{a}hler varieties $V$.
\end{rmenv}


\subsection{Consequences of Proposition 0}
\label{1.B}

\begin{itenv}{Proposition 1}
\label{proposition1}
  If $V$ is a complete connected variety, $T$ a connected variety, and $G$ an algebraic group, then every morphism $f\colon T\times V\to G$ is of the form
  \[
    f(t,v) = \varphi_1(t)\times\varphi_2(v)
  \]
  where $\varphi_1$ and $\varphi_2$ are morphisms from $T$ and $V$ (respectively) to $G$.
\end{itenv}

\begin{proof}
  Let $t_0\in T$.
  Consider $f(t,v)\cdot f(t_0,v)^{-1}$;
  this is a morphism from $T\times V$ to $G$ that, for $t=t_0$ and arbitrary $v$, takes the value $e$ (the identity element in $G$).
  We thus have
  \[
    \begin{aligned}
      f(t,v)\cdot f(t_0,v)^{-1} &= \varphi_1(t)
    \\\implies f(t,v) &= \varphi_1(t)\cdot f(t_0,v).
    \end{aligned}\qedhere
  \]
\end{proof}

\begin{rmenv}{Remark 1}
  ``By an analogous argument we can show'', or ``by considering the dual group of $G$, we deduce'' that $f(t,v)$ can also be written in the form $\psi_1(v)\psi_2(t)$.
\end{rmenv}

\oldpage{9-03}
\begin{rmenv}{Remark 2}
  If $\varphi_1(t)\varphi_2(v)=\varphi'_1(t)\varphi'_2(v)$, then $\varphi'_1(t)=\varphi_1(t)\cdot a$ and $\varphi'_2(v)=\varphi_2(v)$, where $a$ is some fixed element of $G$.
\end{rmenv}

\begin{itenv}{Proposition 2}
\label{proposition2}
  Let $G$ be a connected group, and $V$ a complete connected variety;
  suppose that $e\in V\subset G$.
  Then $V$ is contained in the centre of $G$.
\end{itenv}

\begin{proof}
  Consider $f\colon G\times V\to G$ defined by $f(g,v)=v\cdot g\cdot v^{-1}$.
  If $g=e$, then $f$ does not depend on $v$.
  So $f(g,v)=\varphi(g)$.
  Setting $v=e$, we find that $\varphi(g)=g$, and so $vgv^{-1}=g$, which proves the proposition.
\end{proof}

In particular:

\begin{itenv}{Theorem 0}
\label{theorem0}
  The underlying group of an abelian variety is abelian.
\end{itenv}

(For another proof of this result, see the \hyperref[appendix]{Appendix}).


\section{Functions with values in an abelian variety}
\label{3}

\begin{itenv}{Theorem 1}
\label{theorem1}
  Every function $f$ on a non-singular variety $U$ with values in an abelian variety $A$ is a morphism.
\end{itenv}

This theorem results from the combination of two lemmas.

\begin{itenv}{Lemma 1}
\label{lemma1}
  If $f$ is a function defined on a non-singular variety $U$ with values in an algebraic group $G$, then the set $S$ of points of $U$ where $f$ is not defined is of pure codimension~$1$.
\end{itenv}

\begin{proof}
  Let $\varphi$ be the function from $U\times U$ to $G$ defined by $\varphi(u,u')=f(u)f(u')^{-1}$.
  Let $X$ be an affine neighbourhood of $e$ in $G$, and $\varphi_0$ the function from $U\times U$ to $X$ that only differs from $\varphi$ in the definition of its domain;
  there is no worry that $\varphi(U)$ might not be contained in $G\setminus X$, since, if $f$ is defined at $u$, then $\varphi$ is defined at $(u,u)$, and there it takes the value $e$;
  more precisely, we will show that the following three properties are equivalent:
  \begin{enumerate}[(a)]
    \item $f$ is defined at $u$;
    \item $\varphi$ is defined at $(u,u)$;
    \item $\varphi_0$ is defined at $(u,u)$.
  \end{enumerate}

  Firstly, (a)$\implies$(b)$\iff$(c) is evident.

  We now show that (b)$\implies$(a).
  If $\varphi$ is defined at $(u,u)$, let $v\in U$ be such that $f$ is defined at $v$, and such that $\varphi$ is defined at $(u,v)$;
  then, for all $u'$ where $f$ is defined,
  \[
    f(u') = f(u')\cdot f(v)^{-1}\cdot f(v) = \varphi(u',v)\cdot f(v).
  \]
  The function $f_0$ defined by $f_0(u')=\varphi(u',v)\cdot f(v)$ agrees with $f$, and is
\oldpage{9-04}
  defined at the point $u$, and so (b)$\implies$(a).

  This shows that the intersection of with the diagonal of the set $H$ of points of $U\times U$ where $\varphi_0$ is not defined is $S$, or rather the image of $S$ under the diagonal map.
  But the sets of points at which a numerical function on a normal variety is not defined is of pure codimension~$1$;
  this is thus also true if we replace ``numerical function'' with ``function with values in an affine variety''.
  So $H$ is of pure codimension~$1$.
  Since $U\times U$ is not singular, the codimension in $U\times U$ of $H\cap\Delta$ is $\leq\codim H+\codim\Delta = \dim U+1$, which shows that every component of $H\cap\Delta$ is of codimension $\leq1$ in $\Delta$.
\end{proof}

\begin{rmenv}{Remark}
  The hypothesis that $U$ be non-singular is essential, both for \hyperref[lemma1]{Lemma~1} and for \hyperref[theorem1]{Theorem~1}.
\end{rmenv}

\begin{rmenv}{Counter-example}
  Let $U$ be a cone in $K^3$ that has a cubic $G$ of genus~$1$ in the $2$-dimensional projective space as its base.
  Then $G$ can be endowed with a group structure.
  The projection $f$ from $U$ to $G$ is defined at every point except for the origin;
  $S$ is thus of codimension~$2$.
\end{rmenv}

\begin{itenv}{Lemma 2}
\label{lemma2}
  If $f$ is a function defined on a normal variety $U$ with values in a complete variety $V$, then the set $S$ of points of $U$ where $f$ is not defined is of codimension~$>1$.
\end{itenv}

\begin{proof}
  Since $V$ is complete, there exists a variety $W$ contained in $D^r$ (where $D$ denotes the projective line), a morphism $p$ from $W$ to $V$, and a function $s$ from $V$ to $W$ such that $p\circ s=I_V$.
  Then $f=p\circ(s\circ f)$ will be defined whenever $s\circ f$ is defined.
  But $s\circ f$ can be considered as taking values in $D^r$, since $W$ is closed in $D^r$, and will thus be defined whenever the $r$ coordinate functions of $f$ are defined.
  These functions take values in $D$, and so, since $U$ is normal, the set of points where they are not defined is of codimension~$>1$ \cite[p.~166, Corollary to Proposition~2, Section~1, Chapter~V]{1}.
\end{proof}


\section{Functions defined on a product with values in an abelian variety}
\label{4}

\begin{itenv}{Theorem 2}
\label{theorem2}
  Let $X$ and $Y$ be irreducible varieties, and $f$ a function defined on $X\times Y$ with values in an abelian variety $A$ (whose group law is written additively).
  Then $f$ is of the form $f(x,y)=f_1(x)+f_2(y)$, where $f_1$ and $f_2$ are functions from $X$ and $Y$ (respectively) to $A$.
\end{itenv}

\begin{rmenv}{Remark}
  This implies that $f$ is defined at the points where $f_1$ and $f_2$ are both defined, and at these points only.
\end{rmenv}

\oldpage{9-05}
\begin{proof}
  Let $(x_0,y_0)$ be a simple point of $X\times Y$.
  By considering the function $g$ on $X\times Y$, defined by
  \[
    g(x,y) = f(x,y) - f(x_0,y) - f(x,y_0) + f(x_0,y_0),
  \]
  we can reduce to showing that, if a function $g$ with values in $A$ is zero on
  \[
    X\vee Y = \{x_0\}\times Y \cup X\times\{y_0\},
  \]
  then it is zero on $X\times Y$.

  We will successively reduce to the following particular cases:
  \begin{enumerate}[(a)]
    \item $X$ is a curve;
    \item $X$ is a complete non-singular curve, and $Y$ is non-singular.
  \end{enumerate}

  \emph{Reduction to (a).}
  The set of points $x_1$ of $X$ such that there exists an irreducible curve containing $x_0$ and $x_1$, where $x_0$ is a simple point, is dense in $X$.
  But (a) implies that $g$ is zero at every point $(x_1,y)$ of the open set on which it is defined, and thus at every point of a dense set, and thus everywhere.

  \emph{Reduction to (b).}
  If $X$ is an irreducible curve, then there exists a curve $X_1$ that is both complete and normal (and thus non-singular), as well as a rational equivalence from $X$ to $X_1$, defined at $x_0$.
  Also, if $Y_1$ is the open subset of simple points of $Y$, then $Y_1$ is birationally equivalent to $Y$.
  The function $g$, defined on $X\times Y$, has a corresponding function $g_1$, defined on $X_1\times Y_1$, and it clearly suffices to prove the theorem for the function $g_1$.

  \emph{Proof in case (b).}
  The variety $X\times Y$ is a product of two non-singular varieties, and thus itself non-singular, and so $g$ is a morphism, by \hyperref[theorem1]{Theorem~1}.
  Its value does not depend on $x$ for $y=y_0$, and thus also for all $y$, by \hyperref[proposition0]{Proposition~0}, since $X$ is complete.
  It is zero for $x=x_0$, and thus for all $x$.

  This concludes the proof of the theorem.
\end{proof}

\begin{itenv}{Corollary 1}
\label{corollary1}
  Every function $f$ defined on an algebraic group $G$ with values in an abelian variety $A$ is of the form $h+a$, where $h$ is a homomorphism from $G$ to $A$, and $a$ is a constant.
\end{itenv}

\begin{proof}
  Set $h(x)=f(x)-f(e)$.
  Then the function $g\colon G\times G\to A$ defined by $g(x,y)=h(x\cdot y)$ is of the form $g_1(x)+g_2(y)$.
  We can impose that $g_1(e)=0$, and then $g_2(e)=0$.
  By successively taking $x=e$ and then $y=e$, we find that $g_1=g_2=h$.
  Whence $h(x,y)=h(x)+h(y)$.
\end{proof}

\begin{itenv}{Corollary 2}
\label{corollary2}
  Every function $f$ defined on a line $D$ with values in an abelian variety $A$ is constant.
\end{itenv}

\oldpage{9-06}
\begin{proof}
  By \hyperref[theorem1]{Theorem~1}, $f$ is everywhere defined, and by \hyperref[corollary1]{Corollary~1} applied to the multiplicative group, we have that $f(xy)=f(x)+f(y)+a$.
  Setting $x=0$, we have $f(0)=f(0)+f(y)+a$, whence $f(y)=-a$.
\end{proof}


\section{Appendix: adjoint representations}
\label{appendix}

We can also obtain \hyperref[theorem0]{Theorem~0} from the following proposition:

\begin{itenv}{Proposition 3}
\label{proposition3}
  Let $G$ be a connected algebraic group, and let $C$ be the centre of $G$.
  Then there exists a linear group $L=\mathrm{GL}(m)$ along with an algebraic homomorphism $f\colon G\to L$ such that $f^{-1}(e)=C$.
\end{itenv}

\begin{proof}
  Let $\fk{a}$ be the local ring of functions on $G$ defined at the identity element $e$, and let $\scr{J}$ be its maximal ideal;
  set $T_n=\scr{J}/\scr{J}^n$.
  For all $n$, $T_n$ is a finite-dimensional vector space.
  Every element $x\in G$ defines an inner automorphism $\alpha(x)\colon G\to G$ that induces an automorphism $\operatorname{Ad}_n(x)\colon T_n\to T_n$.
  Let $C_n$ be the kernel of $\operatorname{Ad}_n\colon G\to\mathrm{GL}(T_n)$.
  The $C_n$ form a decreasing sequence of subvarieties of $G$.
  Such a sequence is stationary, and so there exists some $n_0$ such that $C_n=C_{n_0}$ for all $n\geq n_0$.
  We now show that $C_{n_0}=C$:
  if $x\in C_{n_0}$, then $\operatorname{Ad}_n(x)$ is the identity for all $n$, and so the automorphism of $\fk{a}$ defined by the inner automorphism $\alpha(x)$ of $G$ is the identity, since the local ring $\fk{a}$ is separated.
  Consequently, since $G$ is connected (and thus irreducible), $\alpha(x)$ is the identity.
  Whence the proposition, taking $L=\mathrm{GL}(T_{n_0})$.
\end{proof}

\begin{rmenv}{Remark}
  \marginpar{\dbend}
  In characteristic $p\neq0$, the monomorphism $G/C\to L$ is not necessarily an isomorphism from $G/C$ to its image.
  For example, consider $G=k^*\times k$ endowed with the group law $\varphi((a,b),(a',b))=(aa',b+a^pb')$.
\end{rmenv}

We deduce \hyperref[proposition2]{Proposition~2} and \hyperref[theorem0]{Theorem~0} from \hyperref[proposition3]{Proposition~3} by noting that $L$ is an affine variety, and that, if $V$ is complete and connected, then every morphism from $V$ to $L$ is constant.



%% Bibliography %%

\nocite{*}

\begin{thebibliography}{1}

  \bibitem{1}
  {Chevalley, C.}
  \newblock Fondements de la G\'{e}om\'{e}trie alg\'{e}brique
  \newblock Paris, Secr\'{e}tariat math\'{e}matique, 1958, multigraphed.
  \newblock (Class taught at the Sorbonne in 1957--58).

\end{thebibliography}

\end{document}
