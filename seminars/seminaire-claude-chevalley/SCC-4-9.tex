\documentclass{article}

\title{Abelian varieties}
\author{A. Douady}
\date{}

\usepackage{amssymb,amsmath}

\usepackage{hyperref}
\usepackage{xcolor}
\hypersetup{colorlinks,linkcolor={red!50!black},citecolor={blue!50!black},urlcolor={blue!80!black}}
\usepackage[nameinlink]{cleveref}
\usepackage{enumerate}

\usepackage{mathrsfs}
%% Fancy fonts --- feel free to remove! %%
\usepackage{fouriernc}


\usepackage{fancyhdr}
\usepackage{lastpage}
\usepackage{xstring}
\makeatletter
\ifx\pdfmdfivesum\undefined
  \let\pdfmdfivesum\mdfivesum
\fi
\edef\filesum{\pdfmdfivesum file {\jobname}}
\pagestyle{fancy}
\makeatletter
\let\runauthor\@author
\let\runtitle\@title
\makeatother
\fancyhf{}
\lhead{\footnotesize\runtitle}
\lfoot{\footnotesize Version: \texttt{\StrMid{\filesum}{1}{8}}}
\cfoot{\small\thepage\ of \pageref*{LastPage}}


%% Theorem environments %%

\usepackage{amsthm}

\theoremstyle{plain}

  \newtheorem{innercustomtheorem}{Theorem}
  \crefname{innercustomtheorem}{Theorem}{Theorems}
  \newenvironment{theorem}[1]
    {\renewcommand\theinnercustomtheorem{#1}\innercustomtheorem}
    {\endinnercustomtheorem}

  \newtheorem{innercustomlemma}{Lemma}
  \crefname{innercustomlemma}{Lemma}{Lemmas}
  \newenvironment{lemma}[1]
    {\renewcommand\theinnercustomlemma{#1}\innercustomlemma}
    {\endinnercustomlemma}

  \newtheorem{innercustomproposition}{Proposition}
  \crefname{innercustomproposition}{Proposition}{Propositions}
  \newenvironment{proposition}[1]
    {\renewcommand\theinnercustomproposition{#1}\innercustomproposition}
    {\endinnercustomproposition}

  \newtheorem{innercustomcorollary}{Corollary}
  \crefname{innercustomcorollary}{Corollary}{Corollaries}
  \newenvironment{corollary}[1]
    {\renewcommand\theinnercustomcorollary{#1}\innercustomcorollary}
    {\endinnercustomcorollary}


\theoremstyle{definition}

  \newtheorem{innercustomremark}{Remark}
  \crefname{innercustomremark}{Remark}{Remarks}
  \newenvironment{remark}[1]
    {\renewcommand\theinnercustomremark{#1}\innercustomremark}
    {\endinnercustomremark}

  \newtheorem*{definition*}{Definition}
  \newtheorem*{remark*}{Remark}
  \newtheorem*{properties*}{Properties}


%% Shortcuts %%

\newcommand{\sh}[1]{{\mathscr{#1}}}
\newcommand{\cat}[1]{{\mathcal{#1}}}

\renewcommand{\geq}{\geqslant}
\renewcommand{\leq}{\leqslant}

\newcommand{\oldpage}[1]{\marginpar{\footnotesize$\Big\vert$ \textit{p.~#1}}}


\renewcommand{\thesubsection}{(\Alph{subsection})}


%% Document %%

\usepackage{embedall}
\begin{document}

\maketitle
\thispagestyle{fancy}

\renewcommand{\abstractname}{Translator's note.}

\begin{abstract}
  \renewcommand*{\thefootnote}{\fnsymbol{footnote}}
  \emph{This text is one of a series\footnote{\url{https://thosgood.com/translations}} of translations of various papers into English.}
  \emph{The translator takes full responsibility for any errors introduced in the passage from one language to another, and claims no rights to any of the mathematical content herein.}

  \medskip
  
  \emph{What follows is a translation of the French seminar talk:}

  \medskip\noindent
  \textsc{Douady, A.}
  ``Vari\'{e}t\'{e}s ab\'{e}liennes''.
  \emph{S\'{e}minaire Claude Chevalley}, Volume~\textbf{4} (1958--1959), Talk no.~9.
  {\url{http://www.numdam.org/item/SCC_1958-1959__4__A9_0}}
\end{abstract}

\setcounter{footnote}{0}

\tableofcontents
\bigskip


%% Content %%

\section{Algebraic groups}
\label{1}

\oldpage{9-01}
\begin{definition*}
  An \emph{algebraic group} is a pair $(G,\varphi)$, where $G$ is an algebraic variety and $\varphi$ is a morphism from $G\times G$ to $G$ that endows the set of points of $G$ with the structure of a group.
\end{definition*}

\begin{properties*}
  For
\end{properties*}


%% Bibliography %%

\nocite{*}

\begin{thebibliography}{1}

  \bibitem{1}
  {\sc Chevalley, C.}
  \newblock Fondements de la G\'{e}om\'{e}trie alg\'{e}brique
  \newblock Paris, Secr\'{e}tariat math\'{e}matique, 1958, multigraphed.
  \newblock (Class taught at the Sorbonne in 1957--58).

\end{thebibliography}

\end{document}
