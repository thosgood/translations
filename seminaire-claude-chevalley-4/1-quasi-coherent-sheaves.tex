\documentclass{article}

\usepackage{amssymb,amsmath}

\usepackage{hyperref}
\usepackage[nameinlink]{cleveref}
\usepackage{enumerate}

\usepackage{mathrsfs}
%% Fancy fonts --- feel free to remove! %%
\usepackage{Baskervaldx}
\usepackage{mathpazo}


\crefname{section}{Section}{Sections}
\crefname{equation}{}{}

%% Theorem environments %%

\usepackage{amsthm}

  \theoremstyle{plain}

  \newtheorem{innercustomtheorem}{Theorem}
  \crefname{innercustomtheorem}{Theorem}{Theorems}
  \newenvironment{theorem}[1]
    {\renewcommand\theinnercustomtheorem{#1}\innercustomtheorem}
    {\endinnercustomtheorem}

  \newtheorem{innercustomcorollary}{Corollary}
  \crefname{innercustomcorollary}{Corollary}{Corollaries}
  \newenvironment{corollary}[1]
    {\renewcommand\theinnercustomcorollary{#1}\innercustomcorollary}
    {\endinnercustomcorollary}

  \newtheorem*{proposition}{Proposition}


%% Shortcuts %%

\newcommand{\sh}{\mathscr}
\newcommand{\cat}{\mathcal}

\renewcommand{\geq}{\geqslant}
\renewcommand{\leq}{\leqslant}

\DeclareMathOperator{\HH}{H}

\newcommand{\todo}{\textbf{ !TODO! }}
\newcommand{\oldpage}[1]{\marginpar{\footnotesize$\Big\vert$ \textit{p.~#1}}}


%% Document %%

\usepackage{embedall}
\begin{document}

\renewcommand{\abstractname}{Translator's note.}

\title{Quasi-coherent sheaves}
\author{P. Gabriel}
\date{}
\maketitle

\begin{abstract}
  \renewcommand*{\thefootnote}{\fnsymbol{footnote}}
  \emph{This text is one of a series\footnote{\url{https://github.com/thosgood/translations}} of translations of various papers into English.}
  \emph{The translator takes full responsibility for any errors introduced in the passage from one language to another, and claims no rights to any of the mathematical content herein.}
  
  \emph{What follows is a translation (last updated \today) of the French paper:}

  \medskip\noindent
  \textsc{Gabriel, P.}
  ``Faisceaux quasi-coh\'{e}rents''.
  \emph{S\'{e}minaire Claude Chevalley}, Volume~\textbf{4} (1958-1959), Talk no.~1, 12~p.
  {\footnotesize\url{http://www.numdam.org/item/SCC_1958-1959__4__A1_0/}}
\end{abstract}

\setcounter{footnote}{0}

\tableofcontents


%% Content %%

\bigskip\bigskip
\oldpage{1-01}
We assume prior knowledge of the definitions and elementary properties of sheaves of modules on a topological space, i.e. \cite[chapitre~I, \S1; chapitre~II, \S\S1--2]{2}.
We define a presheaf $\sh{P}$ on a base $\mathfrak{B}$ of open subsets of a topological space $X$, with values in a category $\cat{C}$, to be the following data:
\begin{enumerate}[(a)]
  \item for every open subset $U$ in $\mathfrak{B}$, an object $\sh{P}(U)$ of $\cat{C}$, that we may also denote by $\Gamma(U,\sh{P})$;
  \item for every pair $(U,V)$ of open subsets $U$ in $\mathfrak{B}$ such that $U\subset V$, a morphism $\rho_{UV}\colon\sh{P}(V)\to\sh{P}(U)$.
    The morphism $\rho_{UV}$ will be called the restriction of $V$ to $U$.
    We further suppose that, if $U\subset V\subset W$ are open subsets in $\mathfrak{B}$, then $\rho_{UV}\circ\rho_{VW}=\rho_{UW}$.
\end{enumerate}

The construction of the sheaf $\widetilde{\sh{P}}$ associated to a presheaf $\sh{P}$ can be easily generalised to the case of presheaves on a base of open subsets.
If $\mathfrak{U}=(U_i)_{i\in I}$ is a cover of the open subset $U\in\mathfrak{B}$, where $U_i\in\mathfrak{B}$, and $U_i\cap U_j\in\mathfrak{B}$, and if $\sh{P}$ is a presheaf of rings (resp. of modules) on the base $\mathfrak{B}$, then we write $\HH^0(\mathfrak{U},\sh{P})$ to mean the subring (resp. submodule) of $\prod_{i\in I}\sh{P}(U_i)$ given by the $(X_i)_{i\in I}$ such that $\rho_{U_i\cap U_j,U_i}(X_i) = \rho_{U_i\cap U_j,U_j}(X_j)$ for every pair $i,j\in I$.
We then have canonical maps:
\[
  \sh{P}(U) \to \HH^0(\mathfrak{U},\sh{P}) \to \widetilde{\sh{P}}(U);
\]
if these maps are injective for every cover $\mathfrak{U}$ satisfying the conditions above, then $\widetilde{\sh{P}}(U)$ is the union of the $\HH^0(\mathfrak{U},\sh{P})$.


\section{Preliminaries on localisation}
\label{section1}

Let $A$ be a unital commutative ring.
A submonoid $S$ of the multiplicative monoid of $A$ (i.e. a non-empty subset of $A$ such that, if it contains $s$ and $t$, then it contains $s\cdot t$) is said to be \emph{complete} if it satisfies the following condition:
if $s\cdot t\in S$, then $s\in S$ and $t\in S$.
To every multiplicative submonoid
\oldpage{1-02}
$S$, we associate a complete monoid $\widetilde{S}$ in the following way:
$s\in\widetilde{S}$ if and only if there exists some $t$ in $A$ such that $s\cdot t\in S$.
The prime ideals that meet $S$ also meet $\widetilde{S}$, and vice versa.
Furthermore, the complement of $\widetilde{S}$ in $A$ is a union of prime ideals (the ideals that are maximal amongst those that do not meet $\widetilde{S}$ are prime).

If now $M$ denotes a unital $A$-module, and $S$ a multiplicative submonoid of $A$, then we denote by $M_S$ the following abelian group:

The set $M_S$ is the quotient of $M\times S$ by the equivalence relation
\[
  (m,s) = (n,t) \iff \exists r\in S\mbox{ such that }r(mt-ns)=0.
\]

The addition in $M\times S$ is defined by $(m,s)+(n,t)=(mt+ns,st)$;
our equivalence relation is compatible with the addition, whence we get the structure of an abelian group on $M_S$.
We denote by $m/s$ the class of $(m,s)$ in $M_S$.

We have a bilinear map $A_S\otimes_\mathbb{Z}M_S\to M_S$ defined by passing to quotients from the map $((a,s),(m,t))\mapsto(am,st)$.
Taking $M=A$, we see that $A_S$ is a ring and, more generally, $M_S$ is an $A_S$-module.
The map $\psi\colon m\mapsto m/1$ from $M$ to $M_S$ is compatible with the ring homomorphism $\varphi\colon a\mapsto a/1$ from $A$ to $A_S$ (i.e. it is a homomorphism of abelian groups such that $\psi(a\cdot m)=\varphi(a)\cdot\psi(m)$).

We can furthermore easily prove the following claims:
the correspondence $M\mapsto M_S$ is functorial (in the evident way);
the functor $M\mapsto M_S$ is exact (i.e. if $0\to M\to M'\to M''\to 0$ is an exact sequence of $A$-modules, then $0\to M_S\to M'_S\to M''_S\to 0$ is an exact sequence of $A$-modules) and commutes with inductive limits (i.e. if $(M_i)_{i\in I}$ is an inductive system of $A$-modules, and $L$ an inductive limit of this system, then the $(M_i)_S$ form an inductive system of $A_S$-modules, and the homomorphisms $(M_i)_S\to L_S$ induced by the $M_i\to L$ define $L_S$ as an inductive limit of the system $((M_i)_S)$); the canonical map $M\otimes_A A_S\to M_S$ is bijective.

If $S$ and $T$ are submonoids, then the abelian groups $(M_S)_T$, $(M_T)_S$, and $M_{S\cdot T}$ are all canonically isomorphic to one another, where $S\cdot T$ denotes the monoid given by products $s\cdot t$ with $s\in S$ and $t\in T$.
Identifying the rings $(A_S)_T$, $(A_T)_S$, and $A_{S\cdot T}$, which are themselves all canonically isomorphic to one another, the isomorphisms between $(M_S)_T$, $(M_T)_S$, and $M_{S\cdot T}$ are isomorphisms of modules.

\oldpage{1-03}
If $S$ is contained in a submonoid $S'$ of the multiplicative monoid of $A$, then there is a canonical map from $M_S$ to $M_{S'}$ that sends each element $m_s$ of $M_S$ (where $m\in M$ and $s\in S$) to the element of $M_{S'}$ denoted by the same symbol.
The map $A_S\to A_{S'}$ is a ring homomorphism;
the map $M_S\to M_{S'}$ is that which corresponds to the map $M\otimes_A A_S\to M\otimes_A A_{S'}$, induced by $A_S\to A_{S'}$.
If $S'=\widetilde{S}$ is the smallest complete monoid containing $S$, then the map $M_S\to M_{\widetilde{S}}$ is bijective.
Finally, if $S$ is the union of an increasingly-ordered filtered family of monoids $S_i$, then $M_S$ can be identified with $\varinjlim M_{S_i}$.


\section{The prime spectrum of a commutative ring}
\label{section2}

Let $A$ be a unital commutative ring, and $V(A)$ the set of prime ideals of $A$.
If $\mathfrak{a}$ is an ideal of $A$, then we denote by $W(\mathfrak{a})$ the set of prime ideals that contain $\mathfrak{a}$, and $U(\mathfrak{a})=V(A)\setminus W(\mathfrak{a})$.

Then
\begin{gather*}
  W(\mathfrak{a}) \subset V(A), \quad U(\mathfrak{a})\subset V(A),
\\W(\sum\mathfrak{a}_i) = \bigcap_i W(\mathfrak{a}_i),
\quad W(\mathfrak{a}\cap\mathfrak{b}) = W(\mathfrak{a}\cdot\mathfrak{b}) = W(\mathfrak{a})\cup W(\mathfrak{b}),
\\U(\sum\mathfrak{a}_i) = \bigcup_i U(\mathfrak{a}_i),
\quad U(\mathfrak{a}\cap\mathfrak{b}) = U(\mathfrak{a}\cdot\mathfrak{b}) = U(\mathfrak{a})\cap U(\mathfrak{b}).
\end{gather*}

The set $U(\mathfrak{a})$ increases with $\mathfrak{a}$ and $W(\mathfrak{a})$ decreases when $\mathfrak{a}$ increases.
Finally, $W(\mathfrak{a})=W(\mathfrak{b})$ if and only if every element of $\mathfrak{a}$ has a power in $\mathfrak{b}$, and vice versa:
indeed, to say that $W(\mathfrak{a})\subset W(\mathfrak{b})$ is to say that every prime ideal that contains $\mathfrak{a}$ also contains $\mathfrak{b}$, i.e. that $\mathfrak{b}$ is contained in the intersection of the prime ideals containing $\mathfrak{a}$, and it is classical that this latter intersection consists of the elements that have a power in $\mathfrak{a}$.

The sets $U(\mathfrak{a})$ are the open subsets of a topology on $V(A)$, and, endowed with this topology, $V(A)$ is called the \emph{prime spectrum of $A$}.

If the ideal $\mathfrak{a}$ is generated by the $(f_i)$, then $U(\mathfrak{a})$ is the union of the $U((f_i))=U_{f_i}$;
since $U_f\cap U_g=U_{fg}$, the $U_f$ thus form a base of open subsets when $f$ runs over the elements of $A$.
Every open subset of the form $U_f$ is said to be \emph{special}.
Every special open subset is \emph{quasi-compact}:
indeed, if $U_f$ is the union of some $U(\mathfrak{a}_i)$, then $U_f=\bigcup_i U(\mathfrak{a}_i)=U(\sum\mathfrak{a}_i)$, and a power $f^n$ of $f$ belongs to $\sum_i\mathfrak{a}_i$, and thus to the sum of a finite number of the $\mathfrak{a}_i$.

In particular, $V=U_i$ is quasi-compact.
Finally, if the ring $A$ is \emph{Noetherian}, then every increasing sequence of ideals stabilises, and the same is true for every
\oldpage{1-04}
increasing sequence of open subsets.
The prime spectrum is thus a \emph{Zariski topological space}.


\section{Quasi-coherent sheaves on $V(A)$}
\label{section3}


%% Bibliography %%

\nocite{*}
\bibliographystyle{acm}
\begin{thebibliography}{10}

  \bibitem{1}
  {\sc Cartan, H.}
  \newblock Vari\'{e}t\'{e}s alg\'{e}briques affines.
  \newblock {\em S\'{e}minaire Cartan-Chevalley, Volume~8} (1955/56), Talk no.~3.

  \bibitem{2}
  {\sc Godement, R.}
  \newblock ``Topologie alg\'{e}brique et th\'{e}orie des faisceaux''.
  \newblock {Paris, Hermann.}
  \newblock {\em Act. scient. et ind. 1252} (1958)

  \bibitem{3}
  {\sc Krull, W.}
  \newblock Jacobsonsche Rings, Hilbertscher Nullstellensatz, Dimensionstheorie.
  \newblock {\em Math. Z. 54} (1951), pp.~354--387

  \bibitem{4}
  {\sc Serre, J.-P.}
  \newblock Faisceaux alg\'{e}briques coh\'{e}rents.
  \newblock {\em Ann. Math. 61\/} (1955), pp.~197--279.

\end{thebibliography}

\end{document}
