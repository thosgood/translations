\documentclass{article}

\usepackage{amssymb,amsmath}

\usepackage{hyperref}
\usepackage[nameinlink]{cleveref}
\usepackage{enumerate}
\usepackage{tikz-cd}
\usepackage{graphicx}

\usepackage{mathrsfs}
%% Fancy fonts --- feel free to remove! %%
\usepackage{Baskervaldx}
\usepackage{mathpazo}


\crefname{section}{Section}{Sections}
\crefname{equation}{}{}

%% Theorem environments %%

\usepackage{amsthm}

  \theoremstyle{plain}

  \newtheorem{innercustomtheorem}{Theorem}
  \crefname{innercustomtheorem}{Theorem}{Theorems}
  \newenvironment{theorem}[1]
    {\renewcommand\theinnercustomtheorem{#1}\innercustomtheorem}
    {\endinnercustomtheorem}

  \newtheorem{innercustomproposition}{Proposition}
  \crefname{innercustomproposition}{Proposition}{Propositions}
  \newenvironment{proposition}[1]
    {\renewcommand\theinnercustomproposition{#1}\innercustomproposition}
    {\endinnercustomproposition}

  \newtheorem{innercustomlemma}{Lemma}
  \crefname{innercustomlemma}{Lemma}{Lemmas}
  \newenvironment{lemma}[1]
    {\renewcommand\theinnercustomlemma{#1}\innercustomlemma}
    {\endinnercustomlemma}

  \newtheorem{innercustomcorollary}{Corollary}
  \crefname{innercustomcorollary}{Corollary}{Corollaries}
  \newenvironment{corollary}[1]
    {\renewcommand\theinnercustomcorollary{#1}\innercustomcorollary}
    {\endinnercustomcorollary}


  \theoremstyle{definition}

  \newtheorem*{remark}{Remark}

  \newtheorem{innercustomdefinition}{Definition}
  \crefname{innercustomdefinition}{Definition}{Definitions}
  \newenvironment{definition}[1]
    {\renewcommand\theinnercustomdefinition{#1}\innercustomdefinition}
    {\endinnercustomdefinition}


%% Shortcuts %%

\newcommand{\sh}{\mathscr}
\newcommand{\cat}{\mathcal}

\renewcommand{\geq}{\geqslant}
\renewcommand{\leq}{\leqslant}

\newcommand{\todo}{\textbf{ !TODO! }}
\newcommand{\oldpage}[1]{\marginpar{\footnotesize$\Big\vert$ \textit{p.~#1}}}

\renewcommand{\thepart}{\Alph{part}}

%% Document %%

\usepackage{embedall}
\begin{document}

\renewcommand{\abstractname}{Translator's note.}

\title{Serre's theorem}
\author{P. Gabriel}
\date{}
\maketitle

\begin{abstract}
  \renewcommand*{\thefootnote}{\fnsymbol{footnote}}
  \emph{This text is one of a series\footnote{\url{https://github.com/thosgood/translations}} of translations of various papers into English.}
  \emph{The translator takes full responsibility for any errors introduced in the passage from one language to another, and claims no rights to any of the mathematical content herein.}
  
  \emph{What follows is a translation (last updated \today) of the French paper:}

  \medskip\noindent
  \textsc{Gabriel, P.}
  ``Le th\'{e}or\`{e}me de Serre''.
  \emph{S\'{e}minaire Claude Chevalley}, Volume~\textbf{4} (1958-1959), Talk no.~2, 8~p.
  {\footnotesize\url{http://www.numdam.org/item/SCC_1958-1959__4__A2_0/}}
\end{abstract}

\setcounter{footnote}{0}

\tableofcontents


%% Content %%

\bigskip\bigskip


\part{Affine algebraic sets and classical functors}
\label{chapterA}


\section{Ringed topological spaces}
\label{section1}

\oldpage{2-01}

From now on, and unless otherwise mentioned, the rings we consider will be assumed to be commutative, with a unit element, and \emph{Noetherian}.
We define a \emph{ringed topological space} $(V,\sh{A})$ to be a topological space $V$ endowed with the structure defined by the data of a sheaf of rings $\sh{A}$.
If $(V,\sh{A})$ and $(W,\sh{B})$ are two ringed topological spaces, then we define a morphism from the first to the second by the data of:
\begin{enumerate}[(a)]
  \item a continuous map $\psi\colon V\to W$;
  \item for every open subset $U$ of $W$, a ring homomorphism
    \[
      \varphi_U\colon\sh{B}(U)\to\sh{A}(\varphi^{-1}(U))
    \]
    that is compatible with the restriction maps.
\end{enumerate}

The composition of two morphisms is defined in the evident way, and we speak of the category of ringed topological spaces.
In what follows, $V$ will almost always be a Zariski space.
To such a $V$, we often associate a topological space $S(V)$, which is \emph{the scheme of $V$}, and which is defined in the following way:
\begin{itemize}
  \item the points of $S(V)$ are the closed irreducible subsets of $V$;
  \item the closed subsets of $S(V)$ are the sets $\sh{F}$, where $F$ is a closed subset of $V$, and $\sh{F}$ denotes the set of closed irreducible subsets of $V$ that are contained in $F$.
\end{itemize}

It is then clear that the correspondence $F\mapsto\sh{F}$ between closed subsets of $V$ and closed subsets of $S(V)$ is bijective, that $S(V)$ is a Zariski space, and that every closed irreducible subset of $S(V)$ is the closure of a unique point.
To every sheaf on $V$, we canonically associate a sheaf on $S(V)$.
In particular, if $(V,\sh{A})$ is a ringed topological space, and if $V$ is a Zariski space, then we denote by $(S(V),S(\sh{A}))$ the \emph{scheme of $(V,\sh{A})$}, which is defined to be the ringed topological space given by $S(V)$ and the sheaf associated to $\sh{A}$.
Of course,
\oldpage{2-02}
the scheme of $(S(V),S(\sh{A}))$ is isomorphic to $(S(V),S(\sh{A}))$.

If $A$ is a Noetherian Jacobson ring, and


%% Bibliography %%

\nocite{*}
\bibliographystyle{acm}

\end{document}
