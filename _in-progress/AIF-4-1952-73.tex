\documentclass{article}

\usepackage[margin=1.6in]{geometry}

\title{Summary of essential results in the theory of topological tensor products and nuclear spaces}
\author{A.~Grothendieck}
\date{}

\newcommand{\doctype}{French paper}
\newcommand{\origcit}{%
  \textsc{Grothendieck, Alexander}.
  ``R\'{e}sum\'{e} des r\'{e}sultats essentiels dans la th\'{e}orie des produits tensoriels topologiques et des espaces nucl\'{e}aires''.
  \emph{Annales de l'institut Fourier}, Volume~\textbf{4} (1952) , 73--112.
  {\url{http://www.numdam.org/item/?id=AIF_1952__4__73_0}}%
}


\usepackage{amssymb,amsmath}

\usepackage{hyperref}
\usepackage{xcolor}
\hypersetup{colorlinks,linkcolor={red!50!black},citecolor={blue!50!black},urlcolor={blue!80!black}}
\usepackage{enumerate}
\usepackage{tikz-cd}

\usepackage{mathrsfs}
%% Fancy fonts --- feel free to remove! %%
\usepackage{fouriernc}


\usepackage{fancyhdr}
\usepackage{lastpage}
\usepackage{xstring}
\makeatletter
\ifx\pdfmdfivesum\undefined
  \let\pdfmdfivesum\mdfivesum
\fi
\edef\filesum{\pdfmdfivesum file {\jobname}}
\pagestyle{fancy}
\makeatletter
\let\runauthor\@author
\let\runtitle\@title
\makeatother
\fancyhf{}
\lhead{\small\runtitle}
\lfoot{\footnotesize Version: \texttt{\StrMid{\filesum}{1}{8}}}
\cfoot{\small\thepage\ of \pageref*{LastPage}}


%% Theorem environments %%

\usepackage{amsthm}

\theoremstyle{plain}

\newenvironment{itenv}[1]
  {\phantomsection\par\medskip\noindent\textbf{#1.}\itshape}
  {\medskip}

\newenvironment{rmenv}[1]
  {\phantomsection\par\medskip\noindent\textbf{#1.}\rmfamily}
  {\medskip}


%% Shortcuts %%

\newcommand{\aster}[1]{$\star${#1}$\star$}
\newcommand{\BB}{\mathrm{B}}
\newcommand{\LL}{\mathrm{L}}
\renewcommand{\ll}{\ell}
\newcommand{\sBB}{\mathscr{B}}
\newcommand{\sLL}{\mathscr{L}}
\newcommand{\DF}{\mbox{\normalfont($\mathscr{DF}$)}}
\newcommand{\FF}{\mbox{\normalfont($\mathscr{F}$)}}
\newcommand{\hotimes}{\widehat{\otimes}}
\newcommand{\hhotimes}{\skew{0}{\widehat}{\skew{0}\widehat{\otimes}}}
\newcommand{\scr}{\mathscr}
\newcommand{\dd}{\mathrm{d}}

\newcommand{\todo}{\textbf{ !TODO! }}
\newcommand{\oldpage}[1]{\marginpar{\footnotesize$\Big\vert$ \textit{p.~#1}}}


%% Document %%

\usepackage{embedall}
\begin{document}

\maketitle
\thispagestyle{fancy}

\renewcommand{\abstractname}{Translator's note.}

\begin{abstract}
  \renewcommand*{\thefootnote}{\fnsymbol{footnote}}
  \emph{This text is one of a series\footnote{\url{https://thosgood.com/translations}} of translations of various papers into English.}
  \emph{The translator takes full responsibility for any errors introduced in the passage from one language to another, and claims no rights to any of the mathematical content herein.}

  \medskip
  
  \emph{What follows is a translation of the \doctype:}

  \medskip\noindent
  \origcit
\end{abstract}

\setcounter{footnote}{0}

\tableofcontents
\bigskip


%% Content %%

\section*{Introduction}
\addcontentsline{toc}{section}{\protect\numberline{}Introduction}%


\subsection*{Subject}
\addcontentsline{toc}{subsection}{\protect\numberline{}Subject}%

\oldpage{73}

\footnote{The numbers in brackets refer to the bibliography found at the end of this article.}
This article aims to give a summary, without proofs, of the principal results found in my work ``Produits tensoriels topologiques et espaces nucl\'{e}aires'', which will be published in the \emph{Memoirs of the Amer. Math. Society} (and which I will refer to as \cite{PTT}).
The main concern throughout \cite{PTT} was that of being exhaustive, both in terms of studying all the questions raised by the topics covered, as well as trying to state the more difficult results as theorems that were the most general possible.
This work was also very dense, and the important simple ideas risked being sometimes hidden by technical details.
This is why this bowdlerised summary is possibly useful in giving a more assimilable outline of the theory.
Some extra comments, interesting but not necessary for the general understanding of this summary, as well as some hints on certain proofs, have been placed between stars, like \aster{\ldots}.

The importance of topological tensor products shows itself in many different settings:
\begin{enumerate}[a)]
  \item The notion of the topological tensor product forms the foundations of a simple and general formulation of \emph{Fredholm theory}, including, alongside the classical case of an integral operator defined by a continuous kernel, many other operators that are defined in the most important functional spaces.%
    \footnote{Such a formulation of Fredholm theory seems to have appeared for the first time in \textsc{A.~Ruston}, ``Direct product of Banach spaces and linear functional equations'', \emph{Proc. of the London Math. Soc.} \textbf{3} (1951), 1. My work on this subject was conceived independently of his (in the autumn of 1951), and is rather different.}
  \item The many variants of the notion of topological tensor product give rise, by duality, to the definition of many remarkable classes of bilinear forms and linear operators, whose
\oldpage{74}
    study is only just barely covered in \cite[chap.~I, \S4]{PTT}.
    In particular, the techniques introduced there, conveniently systematised and exploited, allow us to obtain entirely unexpected results in the \emph{theory of linear transformations between the space $L^1$, $L^2$, and $L^\infty$}, and their topological-vectorial analogues (these results being, as of yet, not definitive, and thus unpublished).
    I might return to this subject, and restrict myself to explaining, in a rather different way, the systematic work of von Neumann--Schatten on the remarkable classes of compact operators in a Hilbert space \cite[chap.~4]{8}.
  \item From the point of view of this current work, the most important application of topological tensor products is the theory of \emph{nuclear spaces}.
    We explain this theory, generalise it, and make precise the famous ``theory of kernels'' of L.~Schwartz, and further discover new properties, even for the most classical of spaces.
    Here the topological tensor calculus is the most simple, since the majority of variants of the notion of topological tensor product coincide, and their properties thus sum.
    For now, there are not many applications to particular theories of the general theorems that we obtain.
    The most interesting seems to be a topological-vectorial variant of the ``K\"{u}nneth theorem'', giving the homology of a complex defined as the tensor product of two complexes, a variant which seems useful in topological algebra.
  \item Generally, it seems to me that the notions of topological tensor product are perfect for giving a suggestive and manageable \emph{language}, that would be good to use in many situations in functional analysis, especially since we have theorems (some of which are non-trivial) at our disposition that we can benefit from.
  I hope that this summary, or, better, \cite{PTT}, will succeed in giving the reader a similar impression, before the publication of the articles promised above.
\end{enumerate}


\subsection*{Terminology and notation}
\addcontentsline{toc}{subsection}{\protect\numberline{}Terminology and notation}%

Generally we follow the terminology and notation of \cite{3}, apart from the fact that we call the semi-reflexive spaces of \cite{3} \emph{reflexive}.
We only consider, unless otherwise stated, spaces that are \emph{locally convex} and \emph{separated};
by ``quotient space'' of a space $E$, we mean the quotient of $E$ by a \emph{closed} vector subspace.
The \emph{dual} of $E$, written $E'$, is assumed to be, unless otherwise stated, endowed with the strong topology (i.e. the topology
\oldpage{75}
of bounded convergence).
The dual of $E'$, or the \emph{bidual} of $E$, written $E''$, is assumed to be, unless otherwise stated, endowed with the topology given by uniform convergence on the equicontinuous subsets of $E'$, which induces the original topology on $E$.
We will eventually need to appeal to certain notions defined and studied in \cite{6}, most notably that of a \emph{$\DF$-space}.
For our purposes here, it will suffice to know that the dual of a $\FF$-space is a $\DF$-space; that every normed space is a $\DF$-space; and that the dual of a $\DF$-space is an $\FF$-space.

Let $E$, $F$, and $G$ be locally convex spaces.
Denote by $\BB(E,F;G)$ (resp. $\sBB(E,F;G)$) the space of continuous bilinear maps (resp. of separately continuous bilinear maps, i.e. linear with respect to each variable) from $E\times F$ to $G$.
Denote by $\LL(E;F)$ the space of continuous linear maps from $E$ to $F$.
Denote by $\sBB_e(E'_s,F'_s)$ the space of separately continuous bilinear forms on the product of the weak duals $E'_s$ and $F'_s$ of $E$ and $F$, endowed with the \emph{biequicontinuous topology} topology, i.e. the topology given by uniform convergence on the products of an equicontinuous subset of $E'$ with equicontinuous subset of $F'$.
This space is complete if and only if the spaces $E$ and $F$ are complete.

We define a \emph{bounded} (resp. \emph{compact}, resp. \emph{weakly compact}) \emph{linear map} from $E$ to $F$ to be a linear map from $E$ to $F$ that sends a suitable neighbourhood of $0$ to a bounded (resp. relatively compact, resp. relatively weakly compact) subset of $F$.

For short\footnote{This terminology was suggested to me by R.E.~Edwards.\\\emph{[Trans.] The seemingly more popular terminology these days is to say ``absolutely convex'' instead of ``disked'', and to speak of the ``absolutely convex hull'' instead of the ``disked hull''.}}, if $E$ is a vector space, we say \emph{disk} or \emph{disked set in $E$} to mean a convex and circled (a.k.a. balanced) subset of $E$.
If $E$ is a locally convex space, and $A$ a bounded disk in $E$, then we denote by $E_A$ the vector space generated by $A$, and endowed with the norm $\|x\|_A=\inf_{x\in\lambda A}|\lambda|$.
If $A$ is closed, then the unit ball of $E_A$ is $A$.
If $A$ is complete, then $E_A$ is complete.
If $V$ is a disked neighbourhood of $0$ in $E$, then $E_V$ denotes the normed space given by passing to the quotient under the semi-norm $\|x\|_V=\inf_{x\in\lambda V}|\lambda|$.

Recall that a locally convex space is said to be \emph{quasi-complete} if its closed bounded subsets are complete, \emph{barrelled} (resp. \emph{quasi-barrelled}) if the bounded subsets of its weak dual (resp. of its strong dual) are equicontinuous, and \emph{bornological} if every set of linear forms on $E$ that are uniformly bounded on every bounded subset is equicontinuous.
If $E$ is quasi-complete, then tunnelled is equivalent to quasi-tunnelled;
in any case, bornological implies quasi-tunnelled.



\section{Topological tensor products}
\label{1}
\oldpage{76}


\subsection{Generalities on \texorpdfstring{$E\hotimes F$}{EF}}
\label{1.1}

(\cite[chap.~1, \S1, n\textsuperscript{o}~1 and n\textsuperscript{o}~3]{PTT}).
\medskip

The axiomatic definition of the algebraic tensor product $E\otimes F$ of two vector spaces $E$ and $F$, and of the canonical bilinear map $(x,y)\mapsto x\otimes y$ from $E\times F$ to $E\otimes F$ (\cite{1}) asks only that, for every vector space $G$, the bilinear maps from $E\times F$ to $G$ correspond bijectively with linear maps $f$ from $E\otimes F$ to $G$, where the map corresponding to $f$ is given by $(x,y)\mapsto f(x\otimes y)$.

\begin{itenv}{Theorem 1}
\label{theorem1}
  If $E$ and $F$ are locally convex spaces, then we can endow $E\otimes F$ with a locally convex topology such that, for every locally convex space $G$, the \emph{continuous} bilinear maps from $E\times F$ to $G$ correspond exactly to \emph{continuous} linear maps from $E\otimes F$ to $G$.
  Further, such a topology is unique.
\end{itenv}

Then the \emph{equicontinuous} subsets of $\BB(E,F;G)$ correspond exactly to the \emph{equicontinuous} subsets of $\LL(E\otimes F;G)$ as well.
Unless otherwise mentioned, $E\otimes F$ is assumed to be endowed with the above topology, called the \emph{projective tensor product of the topologies of $E$ and $F$};
endowed with this topology, $E\otimes F$ is called the \emph{projective topological tensor product} of $E$ and $F$.

If $E$ and $F$ are normed, then $E\otimes F$ is normable, and we can even find a norm such that, for every \emph{normed} space $G$, the above isomorphism between $\BB(E,F;G)$ and $\LL(E\otimes F;G)$ preserves the natural norms.
Further, such a norm is unique.
This norm on $E\otimes F$, denoted by $u\mapsto\|u\|$, where the norms of $E$ and $F$ are implicit, is the lower bound of the quantities $\sum_i\|x_i\|\|y_i\|$ over all representations of $u$ in the form $u=\sum_i x_i\otimes y_i$ (and this norm has already been considered, in \cite{8}).
It is also the gauge of the set $\Gamma(U\otimes V)$, where $U$ (resp. $V$) is the unit ball in $E$ (resp. $F$), and where $U\otimes V$ denotes the set of $x\otimes y$ such that $x\in U$ and $y\in V$
\oldpage{77}
(with $\Gamma$ denoting, as per usual, the balanced hull).
In the case where $E$ and $F$ are general locally convex spaces, a fundamental system of neighbourhoods of $0$ in $E\otimes F$ is obtained by taking the sets $\Gamma(U\otimes V)$, where $U$ (resp. $V$) runs over a fundamental system of neighbourhoods of $0$ in $E$ (resp. $F$).

We can introduce the completion of $E\otimes F$, denoted by $E\hotimes F$, and called the \emph{completed projective tensor product} of $E$ and $F$.
If $E$ and $F$ are normed spaces, then $E\hotimes F$ is a Banach space (with a well-defined norm!).
If $E$ and $F$ are metrisable, then $E\hotimes F$ is of type~$\FF$.
We have, by definition, the \emph{scholium}: if $E$ and $F$ are locally convex spaces, and $G$ is a \emph{complete} locally convex space, then the continuous bilinear maps from $E\times F$ to $G$ correspond bijectively to the continuous linear maps from $E\hotimes F$ to $G$.

This claim still holds true for equicontinuous sets of maps.
In particular, the dual of $E\hotimes F$ is $\BB(E,F)$, with a correspondence between the equicontinuous subsets (which already suffices to characterise the induced topology on $E\otimes F$).

I do not know if, when $E$ and $F$ are of type~$\FF$, this algebraic isomorphism from the dual of $E\hotimes F$ to $\BB(E,F)$ is a \emph{topological} isomorphism, when we endow $\BB(E,F)$ with the topology given by bibounded convergence, i.e. uniform convergence on the products of two bounded sets (``the problem of topologies'').
An equivalent question is the following:
is every bounded subset of $E\hotimes F$ contained in the closed disked hull of a set $A\hotimes B$, where $A$ (resp. $B$) is a bounded subset of $E$ (resp. $F$)?

\aster{%
  We now give some general tips for calculations involving $E\hotimes F$ (\cite[chap.~I, \S1, n\textsuperscript{o}~3]{PTT}).
  If $E=\prod_i E_i$ and $F=\prod_j F_j$ (as topological-vectorial products), then $E\hotimes F$ can be identified with $\prod_{i,j}E_i\hotimes F_j$.
  If $E=\sum_i E_i$ (as a topological direct sum), and if $F$ is a normable space, then $E\hotimes F$ can be identified with the topological direct sum $\sum_i(E_i\hotimes F)$.
  This remains true if $F$ is an arbitrary $\DF$~space, provided that $I$ is countable, and these statements can also be generalised to the case where $E$ is the inductive limit (in the most general sense) of a family $E_i$ of spaces.
  If $E$ and $F$ are both of type~$\FF$ (resp. type~$\DF$), then so too is $E\hotimes F$.
  Similarly, if $E$ and $F$ are quasi-normable spaces, or Schwartz spaces (see definitions in \cite[\S3]{6}), then so too is $E\hotimes F$.
}

\oldpage{78}
\subsection{The space \texorpdfstring{$E\hotimes F$}{EF} when \texorpdfstring{$E$}{E} and \texorpdfstring{$F$}{F} are of type \texorpdfstring{$\FF$}{(F)}}
\label{1.2}

(\cite[chap.~1, \S2, n\textsuperscript{o}~1]{PTT}).
\medskip

\begin{itenv}{Theorem 2}
\label{theorem2}
  Let $E$ and $F$ be two $\FF$~spaces.
  Then every element of $E\hotimes F$ is the sum of an absolutely convergent series of the form
  \[
    u = \sum_i \lambda_i x_i \otimes y_i
  \]
  where $(x_i)$ (resp. $(y_i)$) is a bounded sequence in $E$ (resp. $F$), and where $(\lambda_i)$ is a summable sequence of scalars.
\end{itenv}

(It is also true that, if $(x_i)$, $(y_i)$, and $(\lambda_i)$ are given as above, then the series $\sum_i\lambda_i x_i\otimes y_i$ is always absolutely convergent in $E\hotimes F$, and so we have a \emph{characterisation} of the elements of $E\hotimes F$).
If $E$ and $F$ are normed, then we can suppose in the above that $\|x_i\|\leq1$, $\|y_i\|\leq1$, $\sum_i|\lambda_i|\leq\|u\|_1+\varepsilon$, where $\varepsilon>0$ is arbitrary and given in advance.
In these two statements, if $u$ runs over a compact subset of $E\hotimes F$, then we can suppose that the sequences $(x_i)$ and $(y_i)$ remain fixed (and we can even suppose that they are sequences that tend to $0$), and that $(\lambda_i)$ runs over a compact subset of $\ll^1$ (the space of summable sequences).
We have an analogous statement for the concrete representation of convergent sequences in $E\hotimes F$.
\hyperref[theorem2]{Theorem~2} and its previous variants serve mainly as a way to:
\begin{itenv}{Corollary}
  Let $E$ and $F$ be spaces of type~$\FF$.
  Then every compact subset $K$ of $E\hotimes F$ is contained inside the canonical image of the unit ball of a space $E_A\hotimes F_B$, where $A$ (resp. $B$) is a compact disked subset of $E$ (resp. $F$).
  A fortiori, $K$ is contained in the closed convex hull of $A\otimes B$.
\end{itenv}

This fact also implies that, on $\BB(E,F)$, the ``bicompact convergence'' topology is identical to the compact convergence topology on the dual of $E\hotimes F$.

\aster{%
  For the proof of \hyperref[theorem2]{Theorem~2}, we suppose, for simplicity, that $E$ and $F$ are Banach spaces, and let $I$ be the product of their unit balls, and $i\mapsto x_i$ and $i\mapsto y_i$ the projections from $I$ to its factors.
  It is easy t see that the linear map $(\lambda_i)\mapsto\sum_i\lambda_i x_i\otimes y_i$ from $\ll^1(I)$ to $E\hotimes F$ is a metric homomorphism from the former to a dense subspace of the latter, and thus \emph{onto} the latter, whence it indeed follows that $E\hotimes F$ can be identified with a quotient space of $\ll^1(I)$.

  From \hyperref[theorem2]{Theorem~2}, we can extract results of the following type:
  let $\scr{G}$ be a
\oldpage{79}
  locally compact group (resp. a Lie group), then every summable function $f$ (resp. every infinitely differentiable function of compact support) on $\scr{G}$ is of the form $\sum\lambda_i g_i*h_i$, where $(\lambda_i)\in\ll^1$, and where $(g_i)$ and $(h_i)$ are bounded sequences in $\LL^i(\scr{G})$ (resp. in $\scr{D}(\scr{G})$, the space of infinitely differentiable functions of compact support on $\scr{G}$);
  we thus immediately conclude that $f$ is a linear combination of functions of positive type that are in $\LL(\scr{G})$ (resp. $\scr{D}(\scr{G})$).
  In the former case, we can also restrict to functions that are all of compact support (with the supports of $g_i$ and $h_i$ being contained inside a compact subset that depends only on the compact support of $f$).
  There is a simple direct proof in the case of $\LL^1(\scr{G})$, but I do not think that there is one for $\scr{D}(\scr{G})$, where the question presents difficulties even for $\scr{G}=\mathbb{R}$ (by using the Fourier transformation).
}


\subsection{Calculation of \texorpdfstring{$\LL^1\hotimes E$}{L1E}}
\label{1.3}

(\cite[chap.~I, \S2, n\textsuperscript{o}~2]{PTT}).
\medskip

Let $M$ be a locally compact space endowed with a measure $\mu\geq0$, $E$ a Banach space, and $\LL_E^1(\mu)$ the space of $\mu$-integrable maps from $M$ to $E$ (\cite{2}) endowed with its usual norm: $\|f\|_1=\int\|f(t)\|\dd\mu(t)$.
We denote by $\LL^1(\mu)$ the space of $\mu$-summable scalar functions.
Then there exists an obvious bilinear map $(\varphi,a)\mapsto\varphi\cdot a$ from $\LL^1(\mu)\times E$ to $\LL_E^1(\mu)$ that is of norm~$\leq1$, and thus defines a linear map of norm~$\leq1$ from $\LL^1(\mu)\hotimes E$ to $\LL_E^1(\mu)$.

\begin{itenv}{Theorem 3}
\label{theorem3}
  The above map from $\LL^1(\mu)\hotimes E$ to $\LL_E^1(\mu)$ is a metric isomorphic from the former space to the latter.
\end{itenv}

To see this, we can immediately reduce to the case where $E$ is of finite dimension, and then proceed by transposition.
It then suffices to apply the classical theorem of Dunford--Pettis that characterises continuous linear maps from $\LL^1(\mu)$ to $E'$.

If $E$ is an arbitrary locally convex space, then we denote by $\LL_E^1(\mu)$ the completion of the separated space associated to the space of continuous maps with compact support from $M$ to $E$, endowed with its family of semi-norms $f\mapsto\int p(f(t))\dd\mu(t)$ (where $p$ runs over a fundamental family of continuous semi-norms on $E$).
Then \hyperref[theorem3]{Theorem~3} easily implies that $\LL_E^1(\mu)$ is again isomorphic to $\LL^1(\mu)\hotimes E$.

\begin{itenv}{Corollary}
  If $E$ is a closed vector subspace of the Banach space $F$, then the canonical linear map from $\LL^1(\mu)\hotimes E$ to $\LL^1(\mu)\hotimes F$ is a metric isomorphism.
\end{itenv}

\oldpage{80}
This recovers, for example, the well-known fact that every continuous linear map from $E$ to the dual $\LL^\infty(\mu)$ of $\LL^1(\mu)$ can be extended to a linear map of the same norm from $F$ to $\LL^\infty(\mu)$;
or, dually, that every continuous linear map from $\LL^1(\mu)$ to a quotient space $F'/E^0$ of a Banach dual by a weakly closed vector subspace comes from a linear map of the same norm from $\LL^1(\mu)$ to $F'$.

\aster{%
  The analogue of \hyperref[theorem3]{Theorem~3} for $\LL^p$ spaces is false for all $p>1$.
  However, \hyperref[theorem3]{Theorem~3} applies, in an essential way, in many important places in the theory outlined here.
  We now give some less important applications (see \cite[chap.~1, \S2, n\textsuperscript{o}~2]{PTT} for details).
  In \hyperref[theorem3]{Theorem~3}, if we take $E=c_0$ (the space of scalar sequences that converge to $0$), and note that $\LL_E^1(\mu)$ can then be identified with the space of latticially bounded sequences in $\LL^1(\mu)$ that converge to $0$ almost everywhere, then we see that such sequences in $\LL^1(\mu)$ form a category of invariant sequences from the topological-vectorial point of view.
  In particular, a continuous linear map from $\LL^1(\mu)$ to a space $\LL^1(\nu)$ sends latticially bounded sequences that converge to $0$ almost everywhere to sequences of the same type.
  We thus also see that the latticially bounded subsets of $\LL^1(\mu)$ form an invariant category from the topological-vectorial point of view.
  If we take $F=\ll^p$, with $1\leq p<+\infty$, then we similarly obtain a topological-vectorial interpretation of the sequences $(f_i)$ in $\LL^1(\mu)$ such that
  \[
    \int \left(
      \sum_i |f_i(t)|^p
    \right)^{1/p}
    \dd\mu(t) < +\infty.
  \]
  Such sequences are sent to sequences of the same type by any continuous linear map from $\LL^1(\mu)$ to a space $\LL^1(\nu)$.
  Another interesting application of \hyperref[theorem3]{Theorem~3} is the following: every bounded subset $M$ of $\LL^1(\mu)\hotimes E$ is contained in the canonical image of the unit ball of a space $\LL^1(\mu)\hotimes E_A$, where $A$ is a closed bounded disk in the space $E$ of type $\FF$;
  a fortiori, $M$ is contained in the closed disked hull of $B\otimes A$, where $B$ is the unit ball of $\LL^1(\mu)$, which solves the ``problem of topologies'' described in \hyperref[1.1]{\S1.1}.
}


\subsection{Other examples}
\label{1.4}

If $H$ is a Hilbert space, then the elements of $H'\hotimes H$, identified with endomorphisms of $H$ (the \emph{Fredholm maps} or \emph{nuclear maps} from $H$ to $H$ --- see \todo) are exactly the endomorphisms $u$ such that the positive Hermitian operator
\oldpage{81}
$\sqrt{u^*u}$ is compact and has a summable sequence of eigenvalues, and $\|u\|_1$ is then equal to the sum of the eigenvalues of $\sqrt{u^*u}$ (repeated with multiplicities, of course).
We obtain the operators that have already been studied in \cite{4} and \cite{8}.
In fact, $u$ is also a Fredholm operator if and only if its Hermitian components $\frac12(u+u^*)$ and $\frac{1}{2i}(u-u^*)$ are, i.e. if they are compact Hermitian operators whose sequences of eigenvalues are summable.
\emph{Relation to Hilbert--Schmidt operators: If $A$ and $B$ are Hilbert--Schmidt operators, then $AB$ is a Fredholm operator, and $\|AB\|_1\leq\|A\|_2\|B\|_2$};
and, conversely, every Fredholm operator $u$ is the product of two Hilbert--Schmidt operators $A$ and $B$ of norm $\|A\|_2=\|B\|_2=\sqrt{\|u\|_1}$.
All of these facts are elementary (one we know the spectral decomposition of compact Hermitian operators to a Hilbert space) and well known.

Numerous other examples of products $E\hotimes F$, relating to \emph{nuclear spaces}, will be seen in \todo.

\aster{%
  In the setting of infinite-dimensional Banach spaces, I do now know, even in particular cases, other concrete characterisations of the elements of $E\hotimes F$ apart from those that we have given.
  Also, the elements of $c_0\hotimes E$ (where $E$ can be any complete locally convex space) can be identified with \emph{certain} sequences in $E$ that converge to $0$, that we can call \emph{nuclearly convergent to $0$};
  but if $E$ is an infinite-dimensional Banach space, then we always obtain a strictly smaller class than the class of all sequences converging to $0$ (see \todo).
  We even show that (if $E$ is an infinite-dimensional Banach space), for any sequence $(\lambda_i)$ of positive scalars that is not square-summable, there exists a sequence $(x_i)$ in $E$ that does not converge nuclearly to $0$, and such that $\|x_i\|=\lambda_i$ for all $i$.
  However, if, for example, $E$ is the space $C(K)$ of continuous functions on a compact space, then we show that every square-summable sequence in $C(K)$ converges nuclearly to $0$.
  We also point out that, in any complete locally convex space, every summable sequence converges nuclearly to $0$.
}


\subsection{\texorpdfstring{$E\hhotimes F$}{EF} spaces}
\label{1.5}

(\cite[chap.~1, \S3, n\textsuperscript{o}~3]{PTT}).
\medskip

If $E$ and $F$ are Banach spaces, then $E\otimes F$ can be considered as a vector subspace of the Banach space $\BB(E',F')$ of continuous bilinear forms
\oldpage{82}
on $E'\times F'$.
The completion of $E\otimes F$ under the norm induced by $\BB(E',F')$ is denoted by $E\hhotimes F$, which is thus a complete normed vector subspace of $\BB(E',F')$.
Every reasonable normed topology on $E\otimes F$ is included between the topology induced by $E\hotimes F$ and the topology induced by $E\hhotimes F$.
If $E$ and $F$ are now both arbitrary locally convex spaces, then we can again consider $E\otimes F$ is a space of bilinear forms on $E'\times F'$, and endow it with the biequicontinuous-convergence topology (i.e. the topology given by uniform convergence on the products of an equicontinuous subset of $E'$ with an equicontinuous subset of $F'$):
the completion of $E\otimes F$ under this topology is again denoted by $E\hhotimes F$.
If $E$ and $F$ are complete, then the space $\sLL_e(E'_s,F'_s)$ of separately continuous bilinear forms on the product $E'_s\times F'_s$ of the weak duals $E'_s$ and $F'_s$ endowed with the biequicontinuous-convergence topology is complete, and so $E\hhotimes F$ can be identified with a topological vector subspace of $\sLL_e(E'_s,F'_s)$.
Then the elements of $E\hhotimes F$ can be identified with certain separately weakly continuous bilinear maps on $E'\times F'$, or even with certain weakly continuous linear maps from $E'$ to $F$;
these linear maps send equicontinuous subsets of $E'$ to relatively compact subsets of $F$, and the converse is true in all known cases (see \todo).

The topology on $E\otimes F$ induced by $E\hotimes F$ is finer than that induced by $E\hhotimes F$, whence we have a canonical continuous linear map
\[
  E\hotimes F \to E\hhotimes F.
\]

An important, unsolved, problem is the question of whether or not this map is always bijective (see \todo).
We note that it seems extremely plausible that, if $E$ and $F$ are Banach spaces such that the above map $E\hotimes F\to E\hhotimes F$ is a topological isomorphism (or even simply a topological homomorphism, i.e. here a map from the first space \emph{onto} the second), then $E$ or $F$ is of finite dimension.
This is true if, for example, $E$ contains a vector subspace isomorphic to an $\ll^p$ space or to $c_0$.

Let $E$ and $F$ be Banach spaces.
The norm induced by the dual of $E\hotimes F$ on $E'\otimes F'$ is clearly the norm induced by $E'\hhotimes F'$.
On the other hand, the norm induced on $E'\otimes F'$ by the dual of $E\hhotimes F$ is, in all known cases (see \todo), identical to the norm induced by $E'\hotimes F'$.
This duality, which barely appears in the present summary, is a precious tool for various questions
\oldpage{83}
(e.g. \cite[chap.~I, \S4, n\textsuperscript{o}~6]{PTT}).
The following theorem (which is, truth be told, trivial), can be seen as the dual response to \hyperref[theorem3]{Theorem~3}.

\begin{itenv}{Theorem 4}
\label{theorem4}
  Let $M$ be a locally compact space, let $C_0(M)$ the space of continuous scalar functions on $M$ that are ``zero at infinity'', endowed with the uniform convergence norm, and let $E$ be a locally convex complete space.
  Then $C_0(M)\hhotimes E$ is canonically isomorphic to the space $C_0(M,E)$ of continuous maps from $M$ to $E$ that are zero at infinity, endowed with the uniform convergence topology (and with the uniform norm if $E$ is a Banach space).
  In particular, $c_0\hhotimes E$ is isomorphic to the space of sequences in $E$ that tend to $0$.
\end{itenv}

As with the spaces $C(K)\hotimes E$ below, here the spaces $\LL'(\mu)\hhotimes E$ do not, in general, have a special interpretation as functional spaces.
We note, however, that \emph{the space $\ll^1\hhotimes E$ can be understood as the space of summable sequences in $E$} (i.e. of ``commutatively convergent'' sequences in $E$).


\subsection{Tensor product of linear maps}
\label{1.6}

(\cite[chap.~1, \S1, n\textsuperscript{o}~2]{PTT}).
\medskip

Let $E_i$ and $F_i$ ($i=1,2$) be locally convex spaces, and let $u_i$ be a continuous linear map from $E_i$ to $F_i$.
In algebra, we define a linear map $u_1\otimes u_2$ from $E_1\otimes E_2$ to $F_1\otimes F_2$ by the formula $(u_1\otimes u_2)(x_1\otimes x_2)=u_1x_1\otimes u_2x_2$.
This map is continuous if we endow $E_1\otimes E_2$ and $F_1\otimes F_2$ with the topologies induced by $E_1\hotimes E_2$ and $F_1\hotimes F_2$ (resp. by $E_1\hhotimes E_2$ and $F_1\hhotimes F_2$).
It then follows that $u_1\otimes u_2$ extends, by continuity, to a continuous linear map $u_1\hotimes u_2$ from $E_1\hotimes E_2$ to $F_1\hotimes F_2$ (resp. $u_1\hhotimes u_2$ from $E_1\hhotimes E_2$ to $F_1\hhotimes F_2$).
These maps will also simply be denoted by $u_1\otimes u_2$ whenever there is no fear of confusion.

\begin{itenv}{Theorem 5}
\label{theorem5}
  If each $u_i$ is a topological homomorphism from $E_i$ to a dense subspace of $F_i$, then $u_1\hotimes u_2$ is a topological homomorphism from $E_1\hotimes E_2$ to a dense subspace of $F_1\hotimes F_2$.
  If each $u_i$ is a topological isomorphism from $E_i$ to $F_i$, then $u_1\hhotimes u_2$ is a topological isomorphism from $E_1\hhotimes E_2$ to $F_1\hhotimes F_2$.
\end{itenv}

These claims remain true if the $E_i$ and $F_i$ are normed and we consider \emph{metric} homomorphisms and \emph{metric} isomorphisms.
Particularly interesting is the following corollary, which can also be obtained by an application of \hyperref[theorem2]{Theorem~2}.

\begin{itenv}{Corollary}
  If the $E_i$ and $F_i$ are $\FF$-spaces ($i=1,2$), and the $u_i$ are
\oldpage{84}
  homomorphisms from $E_i$ \emph{onto} $F_i$, then $u_1\hotimes u_2$ is a homomorphism from $E_1\hotimes E_2$ \emph{onto} $F_1\hotimes F_2$.
\end{itenv}





%% Bibliography %%

\nocite{*}

\begin{thebibliography}{9}

  \bibitem{1}
  {\sc Bourbaki, N.}
  \newblock {\em Alg\`{e}bre multilin\'{e}aire}.
  \newblock Hermann, {\em Act. Sc. Ind.} \textbf{1044}.

  \bibitem{2}
  {\sc Bourbaki, N.}
  \newblock {\em Int\'{e}gration}.
  \newblock Hermann, {\em Act. Sc. Ind.} \textbf{1175}.

  \bibitem{3}
  {\sc Dieudonn\'{e}, J. and Schwartz, L.}
  \newblock La dualit\'{e} dans les espaces $\FF$ et $\DF$.
  \newblock {\em Annales de Grenoble} \textbf{1} (1949), 61--101.

  \bibitem{4}
  {\sc Dixmier, J.}
  \newblock Les fonctionnelles lin\'{e}aires sur l'ensemble des op\'{e}rateurs born\'{e}s d'un espace de Hilbert.
  \newblock {\em Annals of Math.} \textbf{51} (1950), 387--408.

  \bibitem[PTT]{PTT}
  {\sc Grothendieck, A.}
  \newblock Produits tensoriels topologiques et espaces nucl\'{e}aires
  \newblock {\em Memoirs of the Amer. Math. Soc.}, to appear.

  \bibitem{6}
  {\sc Grothendieck, A.}
  \newblock Sur les espaces $\FF$ et $\DF$.
  \newblock {\em Summa Brasiliensis Mathematicae}, to appear.

  \bibitem{7}
  {\sc K\"{o}the, G.}
  \newblock Die Stufenr\"{a}ume, eine einfache klasse linearer vollkommener R\"{a}ume.
  \newblock {\em Math. Zeitschrift} \textbf{51} (1948), 317--345.

  \bibitem{8}
  {\sc Schatten, R.}
  \newblock {\em A theory of cross-spaces}.
  \newblock Princeton University Press (1950).

  \bibitem{9}
  {\sc Schwartz, L.}
  \newblock {\em Th\'{e}orie des distributions}, t.~1 and 2.
  \newblock Hermann, {\em Act Sc. et Ind.} \textbf{1091} and \textbf{1122}.

\end{thebibliography}


\end{document}
