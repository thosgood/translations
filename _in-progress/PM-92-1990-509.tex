\documentclass{article}

\usepackage[margin=1.6in]{geometry}

\title{Action of a torus in a projective variety}
\author{Michel BRION and Claudio PROCESI}
\date{}

\newcommand{\doctype}{French book chapter}
\newcommand{\origcit}{%
  \textsc{Brion, M. and Procesi, C.}
  ``Action d'un tore dans une vari\'{e}t\'{e} projective''
  in \emph{Operator algebras, unitary representations, enveloping algebras, and invariant theory (Paris, 1989)}, Birkh\"{a}user Boston, Progress in mathematics \textbf{92} (1990), 509--539.%
}


\usepackage{amssymb,amsmath}

\usepackage{hyperref}
\usepackage{xcolor}
\hypersetup{colorlinks,linkcolor={blue!50!black},citecolor={blue!50!black},urlcolor={blue!80!black}}
\usepackage{enumerate}
\usepackage{tikz-cd}

\usepackage[only,sslash]{stmaryrd} % For the GIT quotient double slash command

\usepackage{mathrsfs}
%% Fancy fonts --- feel free to remove! %%
\usepackage{fouriernc}


\usepackage{fancyhdr}
\usepackage{lastpage}
\usepackage{xstring}
\pagestyle{fancy}
\fancypagestyle{plain}{}
\fancyhf{}
\lhead{\footnotesize\nouppercase\leftmark}
\cfoot{\small\thepage\ of \pageref*{LastPage}}
% Git commit hash for server builds
\newif\ifserver
\serverfalse
\lfoot{\footnotesize\ifserver{Git commit: \href{https://github.com/thosgood/translations/commit/GitCommitHashVariable}{GitCommitHashVariable}}\fi}




%% Theorem environments %%

\usepackage{amsthm}

\newenvironment{itenv}[1]
  {\phantomsection\par\medskip\noindent\textbf{#1.}\itshape}
  {\par\medskip}

\newenvironment{rmenv}[1]
  {\phantomsection\par\medskip\noindent\textbf{#1.}\rmfamily}
  {\par\medskip}


%% Shortcuts %%

\newcommand{\scr}[1]{{\mathscr{#1}}}
\newcommand{\PP}{\mathbf{P}}
\newcommand{\QQ}{\mathbf{Q}}
\newcommand{\NN}{\mathbf{N}}
\newcommand{\ZZ}{\mathbf{Z}}
\newcommand{\RR}{\mathbf{R}}
\newcommand{\s}{\mathrm{s}}
\renewcommand{\ss}{\mathrm{ss}}
\newcommand{\II}{\mathbf{II}}
\DeclareMathOperator{\Pic}{Pic}
\DeclareMathOperator{\Spec}{Spec}
\DeclareMathOperator{\Proj}{Proj}

\renewcommand{\geq}{\geqslant}
\renewcommand{\leq}{\leqslant}

\newcommand{\todo}{\textbf{ !TODO! }}
\newcommand{\unsure}[1]{\underline{\textbf{TODO: #1}}}

\newcommand{\oldpage}[1]{\marginpar{\footnotesize$\Big\vert$ \textit{p.~#1}}}


%% Document %%

\usepackage{embedall}
\begin{document}

\maketitle
\thispagestyle{fancy}

\renewcommand{\abstractname}{Translator's note.}

\begin{abstract}
  \renewcommand*{\thefootnote}{\fnsymbol{footnote}}
  \emph{This text is one of a series\footnote{\url{https://thosgood.com/translations}} of translations of various papers into English.}
  \emph{The translator takes full responsibility for any errors introduced in the passage from one language to another, and claims no rights to any of the mathematical content herein.}

  \medskip
  
  \emph{What follows is a translation of the \doctype:}

  \medskip\noindent
  \origcit
\end{abstract}

\setcounter{footnote}{0}

\tableofcontents
\bigskip


%% Content %%

\emph{[Trans.] This translation would not have been possible without the help of Benjamin Brown, to whom the translator is very thankful.}

\section*{Introduction}
\label{introduction}

If a reductive algebraic group $G$ acts on a projective algebraic variety $X$, all over an algebraically closed field, the data of a $G$-linearised ample line bundle $L$ on $X$ allows us to define the open subset of stable points of $X$.
The quotient (in the usual sense of orbit spaces) of this open subset $X^\s=X^\s(L)$ by $G$ exists;
it is a quasi-projective variety, denoted by $X^\s/G$.
Furthermore, $X^\s$ is contained in the open subset $X^\ss=X^\ss(L)$ of semi-stable points, and we can again define a ``quotient'' $Y$ of $X^\ss$ by $G$ (it is the space of closed orbits of $G$ in $X^\ss$).
The variety $Y$ is projective, and contains $X^\s/G$ as an open subset.
To define it, we introduce the algebra $A:=\oplus_{n=0}^\infty \Gamma(X,L^n)$ and its sub-algebra $A^G$ consisting of invariants of $G$;
then $Y$ is the $\Proj$ of the graded algebra $A^G$.
Thus $Y$ is endowed with a sheaf $\scr{O}(n)$ for every integer $n$, and one of them is invertible.
So we can define the class $1/n[\scr{O}(n)]$ in the Picard group of $Y$, tensored with $\mathbb{Q}$.
This class is ample, and depends only on $X$ and $L$.

Our goal is to study these objects, introduced by Mumford in \cite{MF}, when $G$ is a torus (denoted by $T$).
The idea is to simultaneously consider all the quotients associated to the $T$-linearised bundles $L^n\otimes\scr{O}(\chi)$, where $n$ is a positive integer, and $\scr{O}(\chi)$ is the trivial line bundle on $X$, with the torus acting on each fibre of $\scr{O}(\chi)$ by multiplication by the character $\chi$.
The notion of a stable or semi-stable point for $L^n\otimes\scr{O}(\chi)$ depends only, in fact, on $\chi/n$.
\oldpage{510}
To every quotient $p$ of a character of $T$ by an integer, we can thus associate $X^\s(p)$, $X^\ss(p)$, $Y(p)$, and an ample class $L_p$ in $\Pic_\mathbb{Q}(Y(p))$;
when $p=0$, we recover the above notions.

For simplicity, we assume $L$ to be very ample, i.e. we consider the case where $X$ is a subvariety of a projective space $\PP(V)$, with the torus $T$ acting linearly on $V$, and where $L$ is the restriction of $\scr{O}(1)$ to $X$.
We show that $X^\s(p)$, $X^\ss(p)$, and $Y(p)$ depend only on the position of $p$ relative to a certain finite set $\II$ of characters of $T$ (the set of weights of $T$ in $V$).
More precisely, we define (\hyperref[1.1]{\S1.1} and \hyperref[1.2]{\S1.2}) a partition of the convex hull $\mathcal{C}$ of $\II$ into ``faces''.
Each face is the interior of a convex polyhedron, and $X^\s(p)$, $X^\ss(p)$, and $Y(p)$ depend only on the face of $p$;
for every face $F$, we can define $X^\s(F)$, $X^\ss(F)$, and $Y(F)$.
We further show \hyperref[1.3]{\S1.3} that the map $p\in F\mapsto L_p\in\Pic_\QQ(Y(F))$ is affine.

Let $F$ be an open face of $\mathcal{C}$.
We show that every point that is semi-stable for $F$ is stable;
thus, if $X$ is smooth, then $Y(F)$ only has singularities from quotients by finite abelian groups.
The quotients associated to the open faces are thus relatively simple.
For an arbitrary face $F$, let $F'$ be an open face whose closure contains $F$.
We define, in \hyperref[1.4]{\S1.4}, a morphism $\pi_{F,F'}\colon Y(F')\to Y(F)$, that is almost always birational.
In \hyperref[1.5]{\S1.5}, we study the case where $F$ is of codimension one, and contained in the closure of the two open faces $F_-$ and $F_+$.
The morphisms $\pi_{F,F_-}$ and $\pi_{F,F_+}$ are, in general, quite complicated;
however, if $\hat{Y}$ denotes the product of $Y(F_-)$ and $Y(F_+)$ over $Y(F)$, then the morphisms from $\hat{Y}$ to $Y(F_-)$, $Y(F)$, and $Y(F_+)$ are blow-ups of subvarieties, with the same exceptional divisor (see \hyperref[2.3]{\S2.3} for a precise statement).

We can extend the affine map $p\mapsto L_p$ from $F_+$ to $\Pic_\QQ(Y_+)$ to $\mathcal{C}$, and then lift each $L_p$ to $\hat{Y}$, giving a class $L_p^+$;
we define $L_p^-$ similarly.
We show in \hyperref[2.3]{\S2.3} that the difference $L_p^+ - L_p^-$ is a multiple of the exceptional divisor $\hat{Y}\to Y(F)$, and is zero for all $p\in F$.
Then, when we consider the product $\bar{Y}$ of the quotients associated to all the faces, over all the morphisms $\pi_{F,F'}$, the maps $p\mapsto L_p$ glue to give a continuous piecewise-affine map from $\mathcal{C}$ to $\Pic_\QQ(\overline{Y})$.

In the third section of this paper, we apply these results to the asymptotic study of multigraded modules;
we consider various extensions of the notions of multiplicity and of the Hilbert--Samuel function for a graded module to these modules.
More precisely, we consider a polynomial algebra $A$, graded by $\NN\times\ZZ^l$ (with the $\NN$-grading being defined by the degree of the polynomials).
We denote by $T$ the torus having $\ZZ^l$ as its group of characters.
Every $(\NN\times\ZZ^l)$-graded $A$-module defines
\oldpage{511}
a $T$-linearised sheaf $\scr{M}$ on the projective space $\PP(V)$ associated to $A$.
Let $\II$ be the set of weights of $T$ in $V$, i.e. the duals of the $\ZZ^l$-degrees of the generators of $A$.\footnote{\emph{[Trans.] See e.g. Example~2.2.2 in M.~Brion's ``Linearization of algebraic group actions'' \emph{Handbook of Group Actions} Volume~\textbf{4} (2018), or 1.1.5 in Dolgachev and Hu's ``Variation of geometric invariant theory quotients'' \emph{Pub. Math. de l'IH\'{E}S} \textbf{87} (1998).}}
For every face $F$ associated to $\II$, let $\scr{M}^{(F)}$ be the sheaf given by invariants of $T$ in $(\pi_F)_*(\scr{M})$, where $\pi_F\colon X^\ss(F)\to Y(F)$ is the ``quotient''.
We show that, if $M$ is an $A$-module of finite type, then $\scr{M}^{(F)}$ is a coherent sheaf on $Y(F)$.
For every $p\in F$, we denote by $\mu_M(p)$ the degree of $\scr{M}^{(F)}$ relative to the ample class $L_p$ (see \cite[I.3]{Kle}).
We show that the function $\mu_M$ is a polynomial on each face.
Furthermore, it appears as the density in various integral expressions of asymptotic invariants of $M$: the ``generalised Hilbert--Samuel polynomials'' \hyperref[3.5]{\S3.5}, and the ``Joseph polynomials'' (\cite{Jos}; see \hyperref[3.6]{\S3.6}).
If $M$ is the homogeneous coordinate ring of a smooth projective subvariety $X$ of $\PP(V)$, then the function $\mu_M$ is continuous, and its Fourier transform depends only on fixed points of $T$ in $X$, and on their normal bundle \hyperref[3.4]{\S3.4}.

Finally, in an appendix, we briefly explain the links between our results and some theorems due to Atiyah, Duistermaat--Heckman, Guillemin--Sternberg in symplectic geometry (see \cite{Ati}, \cite{DH12}, and \cite{GS12}).
Indeed, there are close links, studied in \cite{Kir}, between the algebraic action of a complex torus on a subvariety $X$ of a projective space and the Hamiltonian action of its compact maximal sub-torus $T_c$ on $X$ considered as a symplectic variety (the symplectic form being the imaginary part of a K\"{a}hler form on $\PP(V)$ that is invariant under $T_c$).
The objects of our study can be understood in terms of the moment map $J$:
its image can be identified with the convex polyhedron $\mathcal{C}$;
the open faces consist of regular values of $J$;
the quotient $Y(p)$ is ``the reduced phase space'' $J^{-1}(p)/T_c$;
the class of $L_p$ in $H^2(Y(p),\QQ)$ comes from the cohomology class of the symplectic form.
Furthermore, if $M$ is the homogeneous coordinate ring of $X$, then the function $\mu_M$ is the density of the measure given by the image of the moment map.

After having obtained the results of this paper, we became aware (in May 1989) of the article \cite{GS3} of V.~Guillemin and S.~Sternberg, which proves results that are related to ours found in \hyperref[2]{\S2}, concerning Hamiltonian actions of compact tori on symplectic varieties.


\section{Quotients by a torus}
\label{1}

\subsection{Notation}
\label{1.1}

Let $T$ be a torus acting linearly on a vector space $V$ (with the base field $k$ being algebraically closed).
Let $X$ be a closed irreducible subvariety of the projective space $\PP(V)$, stable under the action of $T$.
For simplicity, we suppose
\oldpage{512}
that $T$ acts faithfully on $V$, and that it does not contain homotheties, and further that $X$ is not contained in any linear subspace of $\PP(V)$.

We denote by $\mathfrak{X}(T)$ the group of characters of $T$, and by $E$ (resp. $E_\QQ$) the vector space $\mathfrak{X}(T)\otimes_\ZZ\RR$ (resp. $\mathfrak{X}(T)\otimes_\ZZ\QQ$).
Let $V=\oplus_{\chi\in\mathfrak{X}(T)}V_\chi$ be the decomposition of $V$ into proper subspaces of $T$.
We denote by $\II$ the set of $\chi$ such that $V_\chi\neq\{0\}$, i.e. the set of weights of $T$ in $V$.
For all $x\in\PP(V)$, we denote by $\bar{x}$ a representative of $x$ in $V$, and by $\bar{x}=\sum v_\chi$ its decomposition into eigenvectors of $T$.
We denote by $\II(x)$ the set of $\chi\in\II$ such that $v_\chi\neq0$;
we call it the set of weights of $x$.

For all $p\in E$, let $\overline{F}(p)$ be the intersection of all the simplices that both contain $p$ and have all their vertices in $\II$.
This is a convex polyhedron in $E$, with vertices in $E_\QQ$.
We denote by $F(p)$ its relative interior, i.e. the interior of $\overline{F}(p)$ in the affine space that it generates;
we call $F(p)$ the face of $p$.
By the Carath\'{e}odory theorem \cite[Theorem~1.21]{Val}, every point of the convex hull $\mathcal{C}$ of $\II$ belongs to a simplex that has all its vertices in $\II$.
Thus the $F(p)_{p\in E}$ form a partition of $\mathcal{C}$.
Furthermore, every face is contained in the closure of an open face.
From this it follows that every face of codimension one is contained in the closure of at most two open faces.
Every affine hyperplane generated by a face of codimension one is also generated by $l$ points of $\II$, where $l$ is the dimension of $E$.


\subsection{Faces and quotients}
\label{1.2}

For all $p\in E$, we denote by $X^\ss(p)$ (resp. $X^\s(p)$) the set of $x\in X$ such that the convex hull of $\II(x)$ contains $p$ (resp. contains $p$ in its interior).
It is clear that these two sets depend only on the face of $p$;
for every face $F$, we can thus define $X^\ss(F)$ and $X^\s(F)$.
The following claim is evident.

\begin{itenv}{Proposition}
  \begin{enumerate}[(i)]
    \item $X^\s(F)$ and $X^\ss(F)$ are open subsets of $X$, stable under $T$.
    \item $X^\ss(F)$ is non-empty $\iff$ $F$ is non-empty.
    \item $X^\s(F)$ is non-empty $\iff$ $F$ is in the interior of $\mathcal{C}$.
    \item $X^\ss(F)=X^\s(F)$ if $F$ is open in $E$.
      Conversely, if $X=\PP(V)$ and $X^\ss(F)=X^\s(F)$, then $F$ is open in $E$.
  \end{enumerate}
\end{itenv}

A point of $X^\s(F)$ (rep $X^\ss(F)$) is said to be \emph{stable} (resp. \emph{semi-stable}) for $F$.
The theorem below, and its proof, justify this terminology.

\begin{itenv}{Theorem}
\label{1.2-theorem}
  There exists a projective variety $Y(F)$, with the trivial action
\oldpage{513}
  of $T$, and a $T$-equivariant affine morphism $\pi_F\colon X^\ss(F)\to Y(F)$, such that $\scr{O}_{Y(F)}$ consists of the invariants of $T$ in $(\pi_F)_*(\scr{O}_{X^\ss(F)})$.
  The restriction of $\pi_F$ to $X^\s(F)$ has the orbits of $T$ as its fibres.
\end{itenv}

\begin{proof}
  Note first of all that, if $0\in\mathcal{C}$, then $X^\ss(0)$ (resp. $X^\s(0)$) is the set of semi-stable (resp. stable) points of $X\subset\PP(V)$;
  this follows from \cite[Theorem~2.1]{MF}.

  Let $p$ be a point in $F\cap E_\QQ$ (such a point exists since $F$ is a convex polyhedron with vertices in $E_\QQ$).
  Write $p=\chi/n$, where $n$ is the smallest positive integer such that $np\in\mathfrak{X}(T)$.
  Define a linear action of $T$ on the $n$th symmetric power $S^nV$ by $t\cdot m^n = \chi(t)^{-1}(t\cdot m)^n$.
  Embed $X$ into $\PP(S^nV)$ by the Veronese embedding.
  We immediately see that $X^\ss(F)=X^\ss(p)$ is the set of semi-stable points of $X\subset\PP(S^nV)$, and that $X^\s(F)$ is the set of stable points.
  The claim then follows from \cite[1.4]{MF}.
\end{proof}

\begin{rmenv}{Remark}
  In the language of \cite[Chapter~1]{MF}, which we adopt, $\pi_F$ is a universal categorical quotient, and its restriction to $X^\s(F)$ is a universal geometric quotient.
\end{rmenv}


\subsection{A class of sheaves on the quotients}
\label{1.3}

The line bundle $\scr{O}(1)$ on $\PP(V)$ restricted to $X$ gives a $T$-linearised bundle, again denoted by $\scr{O}(1)$.
For every character $\chi$ of $T$, we denote by $\scr{O}(\chi)$ the following $T$-linearised line bundle:
it is the trivial line bundle on $X$, and $T$ acts on each fibre via the character $-\chi$.
The map $\chi\mapsto\scr{O}(\chi)$ is a group homomorphism, from $\mathfrak{X}(T)$ to $\Pic^T(X)$ (the group of isomorphism classes of $T$-linearised line bundles on $X$).
We extend this homomorphism to give a homomorphism from $E_\QQ$ to $\Pic^T(X)\otimes_\ZZ\QQ$, where $p\mapsto\scr{O}(p)$.

We now reformulate the proof of \hyperref[1.2-theorem]{Theorem~1.2} in the language of line bundles.
Let $A=\oplus_{n=0}^\infty\Gamma(X,\scr{O}(n))$;
this is a graded algebra on which $T$ acts.
By \cite[1.11]{MF}, we have that $Y(0)=\Proj(A^T)$, where $A^T$ is the algebra of $T$-invariants in $A$.
Thus $Y(0)$ is canonically endowed with sheaves $\scr{O}(n)$ for each whole number $n$.
By \cite[8.14.4]{EGAII}, there exists an integer $n>0$ such that $\scr{O}(n)$ is invertible and very ample.
Let $[\scr{O}(n)]$ be its class in $\Pic(Y(0))$.
Set $[\scr{O}(1)] := 1/n[\scr{O}(n)]$;
this is an element of $\Pic_\QQ(Y(0)) := \Pic(Y(0))\otimes_\ZZ\QQ$.
By \cite[8.14.12]{EGAII}, this class does not depend on the choice of $n$ (see \cite{Dem} for more details on this construction).
By definition, $[\scr{O}(1)]$ is the class of $(\pi_0)_*^T\scr{O}(1)$ in $\Pic_\QQ(Y(0))$ (the invariant direct image of $\scr{O}(1)$).

More generally, let $p=\chi/n$ be a rational point of $\mathcal{C}$.
Note that $X^\ss(p)$ is the set of semi-stable points of $X$ associated to the
\oldpage{514}
$T$-linearised bundle $\scr{O}(n)\otimes\scr{O}(\chi)$.
Then $Y(p)$ is the $\Proj$ of the graded algebra
\[
  \bigoplus_{m=0}^\infty \Gamma(X,\scr{O}(mn)\otimes\scr{O}(m\chi))^T = \bigoplus_{m=0}^\infty \Gamma(X,\scr{O}(mn))_{m\chi}
\]
(we recall that, for every rational $T$-module $W$, and for every character $\psi$ of $T$, we denote by $W_\psi$ the subspace of eigenvectors of $T$ in $W$, of weight $\psi$).
We thus have a canonical sheaf $\scr{O}(1)$ on $Y(p)$, which depends on $\chi$ and $n$.
Set $[\scr{O}_p(1)] = 1/n[\scr{O}(1)]$ in $\Pic_\QQ(Y(F))$;
this class depends only on $p$.

\begin{itenv}{Proposition}
  Let $p$ be a rational point of a face $F$.
  Then $[\scr{O}_p(1)]$ is the class of $(\pi_F)_*^T(\scr{O}(1)\otimes\scr{O}(p))$ in $\Pic_\QQ(Y(F))$.
\end{itenv}

\begin{proof}
  By definition, $\scr{O}_{\chi,n}(1)$ is the invariant direct image of the canonical sheaf on the $\Proj$ of $\oplus_{m=0}^\infty\Gamma(X,\scr{O}(mn)\otimes\scr{O}(m\chi))$, i.e. $\scr{O}_{\chi,n}(1) = (\pi_p)_*^T(\scr{O}(n)\otimes\scr{O}(n\chi))$.
  The claim then follows immediately.
\end{proof}

\begin{itenv}{Corollary}
  For every face $F$, the map $p\mapsto[\scr{O}_p(1)]$ is the restriction of an affine map.
\end{itenv}

\begin{proof}
  Let $\chi/n$ be a rational point of the affine space generated by $F$.
  We immediately see that the restriction of $\scr{O}(n)\otimes\scr{O}(\chi)$ to any orbit of $T$ in $X^\ss(F)$ is trivial line $T$-bundle.
  By \cite[Proposition~3]{Kra2}, there exists a line bundle $\scr{L}_{n,\chi}$ on $Y(F)$ such that $\scr{O}(n)\otimes\scr{O}(\chi)=\pi_F^*\scr{L}_{n,\chi}$.
  By the projection formula, we have that $\scr{L}_{n,\chi}=(\pi_F)_*^T(\scr{O}(n)\otimes\scr{O}(\chi))$, and thus $\scr{L}_{n,\chi}$ is unique.
  Since the map $\chi/n\mapsto\scr{O}(1)\otimes\scr{O}(\chi/n)$ is affine, so too is $\chi/n\mapsto(1/n)\scr{L}_{n,\chi}$.
  We conclude thanks to the fact that $(1/n)\scr{L}_{n,\chi}=\scr{O}_{\chi/n}(1)$ for every $\chi/n\in F$.
\end{proof}


\subsection{Relations between the quotients}
\label{1.4}

Let $F$ and $F'$ be distinct faces such that $F\subset\overline{F'}$.

\begin{itenv}{Theorem}
  \begin{enumerate}[(i)]
    \item $X^\s(F) \subset X^\s(F') \subset X^\ss(F') \subset X^\ss(F)$.
    \item There exists a morphism $\pi_{F,F'}$ that makes the diagram
      \[
        \begin{tikzcd}[row sep=huge]
          X^\ss(F') \rar \dar[swap,"\pi_{F'}"]
          & X^\ss(F) \dar["\pi_F"]
        \\Y(F') \rar[swap,"\pi_{F,F'}"]
          & Y(F)
        \end{tikzcd}
      \]
      commute.
    \item $\pi_{F,F'}$ is birational if $F$ is not in the boundary of $\mathcal{C}$.
  \end{enumerate}
\end{itenv}

\oldpage{515}
\begin{proof}
  \begin{enumerate}[(i)]
    \item This is by direct verification.
    \item By \cite[1.11]{MF} and the proof of \hyperref[1.2-theorem]{Theorem~1.2}, $\pi_F\colon X^\ss(F)\to Y(F)$ is the quotient by the equivalence relation ``$x\sim y$ $\iff$ the closures of the $T$-orbits of $x$ and $y$ meet''.
      Then the inclusion of $X^\ss(F')$ into $X^\ss(F)$ passes to the quotient to give $\pi_{F,F'}\colon Y(F')\to Y(F)$.
    \item If $F$ is not contained in the boundary of $\mathcal{C}$, then $X^\s(F)$ is not empty, by \hyperref[1.2]{\S1.2};
      furthermore, the restrictions of $\pi_F$ and $\pi_{F'}$ to $X^\s(F)$ agree, since $X^\s(F)\subset X^\s(F')$.
      Thus $\pi_{F,F'}$ induces an isomorphism from $\pi_{F'}(X^\s(F))$ to its image.
  \end{enumerate}
\end{proof}

\begin{rmenv}{Remark}
  Suppose that $X$ is smooth.
  For every open face $F'$, the only singularities of the variety $Y(F')$ are quotients by finite cyclic groups.
  Consequently, if $F$ is a face contained inside $\overline{F'}$, and not in the boundary of $\mathcal{C}$, then the morphism $\pi_{F,F'}$ is a partial desingularisation of $Y(F)$.
  Using the \'{e}tale slice theorem \cite[III.1]{Lun}, we can show that the only singularities of $Y(F)$ are quotients by diagonalisable groups, of dimension at most the codimension of $F$ in $V$.
\end{rmenv}

We now study the quotient associated to a face $F$ included in the boundary of $\mathcal{C}$.
Let $\langle F\rangle$ be the affine space generated by $F$.
We can choose a one-parameter subgroup $\lambda$ of $\mathcal{C}$ (that is, an element of the dual lattice of $\mathfrak{X}(T)$) such that $\lambda$ is constant on $F$, equal to $a\in\ZZ$, and $\langle\lambda,p\rangle>a$ for all $p\in\mathcal{C}\setminus\langle F\rangle$.
Let $X^F$ be the set of fixed points of $\lambda$ in $X^\ss(F)$, that is, the set of $x\in X$ such that the convex hull of $\II(x)$ contains $F$, and is contained in $\langle F\rangle$.
This is an open subset of the set $X^\lambda$ of fixed points of $\lambda$ in $X$.

For all $x\in X$, the morphism $t\mapsto\lambda(t)x$ from $k^*$ to the complete variety $X$ can be extended to a morphism from $k$ to $X$.
We denote by $\lim_{t\to0}\lambda(t)x$ the image of $0$ under this morphism.

\begin{itenv}{Proposition}
  \begin{enumerate}[(i)]
    \item $X^\ss(F)$ is the set of $x\in X$ such that $\lim_{t\to0}\lambda(t)x\in X^F$.
    \item The restriction $r$ of $\pi_F$ to $X^F$ is a universal geometric quotient, and $\pi_F$ factors as $r\circ q$, where
      \[
        \begin{aligned}
          q\colon X^\ss(F) &\to X^F
        \\x &\mapsto \lim_{t\to0}\lambda(t)x.
        \end{aligned}
      \]
    \item If $X$ is smooth, then $X^F$ is also smooth, and $q$ is then a locally trivial fibration in affine spaces.
  \end{enumerate}
\end{itenv}

\begin{proof}
  \begin{enumerate}[(i)]
    \item Let $x\in X$ and $y=\lim_{t\to0}\lambda(t)x$.
      By the definition of $\lambda$, we have that
\oldpage{516}
      $F$ is contained in the convex hull of $\II(x)$ if and only if $F$ is contained in the convex hull of $\II(y)$.
    \item $X^\lambda$ is a closed $T$-stable sub-variety of $X$, and $\II(X^\lambda)\subset\langle F\rangle$, and so $F$ is open in the convex hull of $\II(X^\lambda)$.
      Furthermore, $X^F=X^\lambda(F)$.
      The first claim thus follows from \hyperref[1.2]{\S1.2};
      the second is evident.
    \item This follows from \cite[Theorem~4.4.]{BB}.
  \end{enumerate}
\end{proof}

\begin{itenv}{Corollary}
  Suppose that $X$ is smooth.
  Let $F'\neq F$ be a face whose closure contains $F$.
  Then $\pi_{F,F'}$ is not birational.
\end{itenv}

\begin{proof}
  By considering the intersection of $X^\ss(F)$ with a fibre of $\pi_F$, we can reduce to the case where $F$ and $X^F$ are points, and where $X^\ss(F)$ is a $T$-module.
  Then $Y(F)$ is a point, but we can easily show that $Y(F')$ has a positive dimension.
\end{proof}


\subsection{The case of faces of codimension \texorpdfstring{$1$}{1}}
\label{1.5}

Let $F$ be a face of codimension~$1$.
Suppose first of all that $F$ is contained in the boundary of $\mathcal{C}$.
Then $F$ is contained in the closure of a unique open face $F_+$.
Choose, as before, a one-parameter subgroup $\lambda$ such that the equation of $\langle F\rangle$ is $\langle\lambda,p\rangle=a$, and such that $\langle\lambda,p\rangle>a$ for all $p\in F_+$;
denote by $X^F$ the set of fixed points of $\lambda$ in $X^\ss(F)$.
The following claim is immediate.

\begin{itenv}{Proposition}
  \begin{enumerate}[(i)]
    \item $X^\ss(F_+)=X^\ss(F)\setminus X^F$.
    \item $\pi_{F,F_+}$ is a locally trivial fibration for the \'{e}tale topology, with fibres being the quotients by $\lambda$ of the sets
      \[
        \big\{ x\in X^\ss(F)\setminus X^F \mid \lim_{t\to0}\lambda(t)x=y \big\}
      \]
      for $y\in X^F$.
      If $X$ is smooth, then the fibres $\pi_{F,F_+}$ are weighted projective spaces.
  \end{enumerate}
\end{itenv}

(Recall that a \emph{weighted projective space} is a variety of the form $(V\setminus\{0\})/k^*$, where $k^*$ acts linearly on the vector space $V$ with all its weights positive).

Now consider a face of codimension~$1$ that meets the interior of $\mathcal{C}$.
Let $F_-$ and $F_+$ be the open faces whose closure contains $F$.
Let $\lambda$ be a one-parameter subgroup of $T$ such that $\langle\lambda,p\rangle=a$ on $F$, and $\langle\lambda,p\rangle>a$ on $F_+$.
Let $X^F$ be as before.
Set
\[
  \begin{aligned}
    X_- &= \big\{x\in X\mid\lim_{t\to\infty}\lambda(t)x=X^F\big\}
  \\X_+ &= \big\{x\in X\mid\lim_{t\to0}\lambda(t)x=X^F\big\}
  \\\dot{X}_+ &= X_+\setminus X^F
  \\\dot{X}_- &= X_-\setminus X^F.
  \end{aligned}
\]
Note that $\dot{X}_-$ is contained in $X^\ss(F_-)$;
we denote by $Y_-$ its image in $Y(F_-)$.
Finally, denote by $Y^F$ the image of $X^F$ in $Y(F)$.

\oldpage{517}
\begin{itenv}{Theorem}
  \begin{enumerate}[(i)]
    \item $X^\s(F_-)\setminus X^\s(F)=X_-$ and $X^\ss(F)\setminus X^\ss(F_-)=X_+$.
    \item $\pi_{F_-,F}\colon Y(F_-)\to Y(F)$ induces an isomorphism from $Y(F_-)\setminus Y_-$ to $Y(F)\setminus Y^F$.
    \item The restriction of $\pi_{F_-,F}$ to $Y_-$ makes the following diagram commute:
      \[
        \begin{tikzcd}[column sep=huge]
        \\\dot{X}_- \rar["x\mapsto\lim_{t\to\infty}\lambda(t)x"] \dar[swap,"\pi_{F_-}"]
          & X^F \dar["\pi_F"]
        \\Y_- \rar[swap,"\pi_{F_-,F}"]
          & Y^F
        \end{tikzcd}
      \]
      It is a locally trivial fibration for the \'{e}tale topology, with fibres being the quotients by $\lambda$ of the sets
      \[
        \big\{x\in X_-\mid\lim_{t\to\infty}\lambda(t)x=y\big\}
      \]
      for $y\in X^F$.
  \end{enumerate}
\end{itenv}

\begin{proof}
  \begin{enumerate}[(i)]
    \item This can be verified immediately.
    \item By (i), we have that
      \[
        X^\s(F_-)\setminus\dot{X}_- = X^\s(F) \subset X^\s(F_-)
      \]
      and so $\pi_{F_-}$ and $\pi_F$ agree on $X^\s(F_-)\setminus\dot{X}_-$.
    \item Note that $X_-$ is a closed $T$-stable subset of $X^\ss(F)$, and that $X_-\cap X^\ss(F_-)=\dot{X}_-$.
      Furthermore, the restriction of $\pi_{F_-}$ to $X_-$ factors as the map $x\mapsto\lim_{t\to\infty}\lambda(t)x$ followed by the quotient of $X^F$ by $T$.
      The claim then immediately follows.
  \end{enumerate}
\end{proof}

We conclude that $Y(F_+)$ can be obtained from $Y(F_-)$ by replacing $Y_-$ with $Y_+$.
The objective of the second section is to make these results more precise.


\section{The structure of morphisms between certain quotients}
\label{2}

\subsection{The linear action of a torus of dimension \texorpdfstring{$1$}{1}}
\label{2.1}

Consider a linear action of $T=k^*$ on a finite-dimensional $k$-vector space $N$.
Decompose $N$ into $N_+\oplus N_0\oplus N_-$, where $N_+$ (resp. $N_0$, $N_-$) is the sum of the proper subspaces associated to positive (resp. null, negative) characters of $T$.
We suppose, for simplicity, that $N_0=\{0\}$, i.e. that $T$ acts on $N$ with only the origin as a fixed point.
Let $Z=N \sslash T$ be the quotient (in the sense of Mumford) of $N$ by $T$: if $k[N]$ is the algebra of polynomial functions on $N$, and $k[N]^T$ is the sub-algebra given by $T$-invariant functions, then $Z=\Spec k[N]^T$.
This is an affine variety, generally singular at $0$ (the point associated to the maximal homogeneous ideal of $k[N]^T$).
Away from $0$, all its singularities are quotients by finite cyclic groups.

\oldpage{518}
We write $\dot{N}_+=N_+\setminus\{0\}$, and $\PP(N_+)$ to mean the quotient of $\dot{N}_+$ by $T$;
this is a weighted projective space.
Let $Z_+$ be the quotient of $\dot{N}_+\times N_-$ by the action of $T$.
The first projection induces a morphism $p_+\colon Z_+\to\PP(N_+)$ that makes $Z_+$ the quotient of a vector bundle on $\PP(N_+)$, by a finite cyclic group.

The inclusion of $\dot{N}_+\times N_-$ into $N$ induces a morphism $f_+\colon Z_+\to Z$.
We easily see that $f_+$ contracts the zero section $(\dot{N}_+\times\{0\})/T$ to $0$, and restricts to give an isomorphism on the complement of the zero section.
We similarly define $\dot{N}_-$, $\PP(N_-)$, $Z_-$, $p_-$, and $f_-$.
Let $\hat{Z}$ be the fibre product $Z_+\times_Z Z_-$.
We have the diagram
\[
  \begin{tikzcd}
    \hat{Z} \rar["\phi_-"] \dar[swap,"\phi_+"] \drar["\psi"]
    & Z_- \dar["f_-"] \rar["p_-"]
    & \PP(N_-)
  \\Z_+ \rar[swap,"f_+"] \dar[swap,"p_+"]
    & Z
  \\\PP(N_+)
  \end{tikzcd}
\]

The algebra $k[N_+]$ is graded (by weight with respect to $T$), and $\PP(N_+)$ is the $\Proj$ of $k[N_+]$.
Consequently, $\PP(N_+)$ is canonically endowed with a sheaf that we denote by $\scr{O}_+(n)$.

\begin{itenv}{Proposition}
  We have
  \[
    \hat{Z} = \Spec \left( \bigoplus_{n=0}^\infty \scr{O}_+(n)\otimes\scr{O}_-(n) \right)
  \]
  (a sheaf of graded algebras on $\PP(N_+)\times\PP(N_0)$), and
  \[
    Z_+ = \Spec \left( \bigoplus_{n=0}^\infty \scr{O}_+(n)\otimes k[N_-]_n \right)
  \]
  where $k[N_-]_n$ denotes the polynomial functions on $N_-$ of weight $-n$ with respect to $T$.
  The diagram
  \[
    \begin{tikzcd}
      \hat{Z} \rar["\phi_-"] \dar[swap,"\phi_+"] \drar["\psi"]
      & Z_- \dar["f_-"]
    \\Z_+ \rar[swap,"f_+"]
      & Z
    \end{tikzcd}
  \]
\oldpage{519}
  is dual to
  \[
    \begin{tikzcd}
      \bigoplus_{n=0}^\infty\scr{O}_+(n)\otimes\scr{O}_-(n)
      & \bigoplus_{n=0}^\infty k[N_+]_n\otimes\scr{O}_-(n) \lar
    \\\bigoplus_{n=0}^\infty\scr{O}_+(n)\otimes k[N_-]_n \uar
      & \bigoplus_{n=0}^\infty k[N_+]_n\otimes k[N_-]_n \uar \lar
    \end{tikzcd}
  \]
\end{itenv}

\begin{proof}
  It suffices to prove that $Z_+=\Spec\bigoplus_{n=0}^\infty\scr{O}_+(n)\otimes k[N_-]_n$;
  the claim on $\scr{O}_{\hat{Z}}$ then follows, since $\scr{O}_{\hat{Z}}=\scr{O}_{Z_+}\otimes_{\scr{O}_Z}\scr{O}_{Z_-}$.
  Let $f$ be a polynomial function of weight $p$ on $N_+$.
  Set $Z_{+f}=p_+^{-1}(\PP(N_+)_f)$, which is an affine open subset of $Z_+$.
  We have
  \[
    k[Z_{+f}]
    = k[N_{+f}\times N_-]^T
    = \bigoplus_{n=0}^\infty\big(k[N_+][1/f]\big)_n\otimes k[N_-]_n.
  \]
  On the other hand,
  \[
    \begin{aligned}
      \Gamma\left(\PP(N_+)_f, \bigoplus_{n=0}^\infty\scr{O}_+(n)\otimes k[N_-]_n\right)
      &= \bigoplus_{n=0}^\infty\Gamma\left(\PP(N_+)_f,\scr{O}_+(n)\right)\otimes k[N_-]_n
    \\&= \bigoplus_{n=0}^\infty\big(k[N_+][1/f]\big)_n\otimes k[N_-]_n.
    \end{aligned}
  \]
\end{proof}

\begin{rmenv}{Definition}
  Let $X$ be a variety with an action of $k^*$, such that the associated grading of $\scr{O}_X$ is concentrated in non-negative degrees: $\scr{O}_X=\bigoplus_{n=0}^\infty\scr{O}_{X,n}$.
  Let $Y$ be the subvariety of $X$ such that $\scr{O}_Y=\scr{O}_{X,0}$.
  The \emph{blow-up of $Y$ in $X$} is the $\Proj$ of the graded algebra $\bigoplus_{n=0}^\infty\mathcal{A}_n$, where $\mathcal{A}_n=\bigoplus_{m\geq n}\scr{O}_{X,m}$.
  Its \emph{exceptional divisor} (considered here as a reduced variety) is the $\Proj$ of the sheaf of graded algebras $\scr{O}_X$.
\end{rmenv}

\begin{itenv}{Corollary}
  \begin{enumerate}[(i)]
    \item $f_+$ and $f_-$ are proper.
    \item $\phi_+$, $\phi_-$, and $\psi$ are blow-ups (resp. zero sections of $p_+$, $p_-$, and $0$), with exceptional divisor $\PP(N_+)\times\PP(N_-)$.
  \end{enumerate}
\end{itenv}

\begin{proof}
  We will show that $\psi$ is the blow-up of $0$ in $X$, where the grading of $k[X]$ is defined by $k[X]_n=k[N_+]_n\otimes k[N_-]_n$ (the proofs of the claims concerning $\phi_+$ and $\phi_-$ are analogous).
  This follows from the following lemma \cite[2.5]{Fle}:
\end{proof}

\begin{itenv}{Lemma}
  Let $B=\bigoplus_{n=0}^\infty B_n$ be a graded algebra of finite type, with $B_0=k$.
  The blow-up of $0$ in $B$ is the spectrum of the sheaf of algebras $\bigoplus_{n=0}^\infty\scr{O}_{\PP(B)}(n)$, where $\PP(B)$ is the $\Proj$ of $B$.
\end{itenv}


\oldpage{520}
\subsection{Comparison of sheaves on the quotients}
\label{2.2}

We keep the above notation.
Let $\pi_+\colon\dot{N}_+\times N_-\to Z_+$ be the quotient by $T$.
As in \hyperref[1.2]{\S1.2}, denote by $\scr{O}(\chi)$ the $T$-linearised line bundle on $N$ associated to the character $-\chi$ of $T$.
Set $\scr{O}_\chi^+(1)=(\pi_+)_*^T(\scr{O}(\chi))$;
define $\scr{O}_\chi^-(1)$ similarly.
We also have a sheaf $\scr{O}(1)$ on $\hat{Z}$, associated to the $\scr{O}_{\hat{Z}}$-module $\bigoplus_{n=0}^\infty\scr{O}_+(n)\otimes\scr{O}_-(n)$.
We can also see $\scr{O}(1)$ as the sheaf associated to the (Weil) divisor $\psi^{-1}(0)$ on $\hat{Z}$, where $\psi^{-1}(0)$ is the set-theoretic fibre of $\psi$ at $0$.
For all $\chi\in\ZZ$, we thus have the class $\chi\cdot\scr{O}(1)$ in $\Pic_\QQ(\hat{Z})$.

\begin{itenv}{Theorem}
  For all $\chi\in\ZZ$, we have
  \[
    \phi_+^*\scr{O}_\chi^+(1) - \phi_-^*\scr{O}_\chi^-(1)
    = \chi\cdot\scr{O}(1)
    = \chi\cdot\psi^{-1}(0)
  \]
  in $\Pic_\QQ(\hat{Z})$.
\end{itenv}

\begin{proof}
  Let $f$ be a polynomial function on $N_-$ that is an eigenvector of $T$ of weight~$p$.
  With the notation of the proof in \hyperref[2.1]{\S2.1}, we have
  \[
    \Gamma(Z_{+f},\scr{O}_\chi^+(1))
    = k[N_{+f}\times N_-]_\chi
    = \bigoplus_{n=0}^\infty\big(k[N_+]/[1/f]\big)_{n+\chi}\otimes k[N_-]_n.
  \]
  In other words, $\scr{O}_\chi^+(1)$ is the module $\bigoplus_{n=0}^\infty\scr{O}_+(n+\chi)\otimes k[N_-]_n$ on $\scr{O}_{Z_+}$.
  Similarly, $\scr{O}_\chi^-(1)$ is the module $\bigoplus_{n=0}^\infty k[N_+]_n\otimes\scr{O}_-(n-\chi)$ on $\scr{O}_{Z_-}$.
  Consequently, $\phi_+^*\scr{O}_\chi^+(1)-\phi_-^*\scr{O}_\chi^-(1)$ is the module $\bigoplus_{n=0}^\infty\scr{O}_+(n+\chi)\otimes\scr{O}_-(n+\chi)$ on $\bigoplus_{n=0}^\infty\scr{O}_+(n)\otimes\scr{O}_-(n)$.
\end{proof}

\begin{rmenv}{Remark}
  These results remain true in a more general setting.
  Indeed, let $A^+$ and $A^-$ be positively graded $k$-algebras of finite type such that $A_0^+=A_0^-=k$.
  The \emph{Segre product} \cite[Chapter~4]{GW} of $A^+$ and $A^-$ is the graded $k$-algebras $A=\oplus_{n=0}^\infty A_n^+\otimes A_n^-$.
  Let $X_+$ (resp. $\dot{X}_+$) be the spectrum of $A^+$ (resp. of $A^_+$ without its homogenous maximal ideal);
  the multiplicative group $T$ acts on $X_+$ and $\dot{X}_+$.
  Denote by $\pi_+\colon X_+\times X_-\to Z_+$ the quotient by $T$, and $\scr{O}_\chi^+(1)=(\pi_+)_*^T\scr{O}(\chi)$, as above.
  We have a morphism $f_+\colon Z_+\to Z:=\Spec(A)$.
  Set $\hat{Z}=Z_+\times_Z Z_-$.
  The above statements easily generalise to the following:
\end{rmenv}

\oldpage{521}

\begin{itenv}{Proposition}
  \begin{enumerate}[(i)]
    \item In the diagram
      \[
        \begin{tikzcd}
          \hat{Z} \ar[r,"\phi_-"] \ar[d,swap,"\phi_+"] \ar[dr,"\psi"]
          & Z_- \ar[d,"f_-"]
        \\Z_+ \ar[r,swap,"f_+"]
          & Z
        \end{tikzcd}
      \]
      both $f_+$ and $f_-$ are proper and birational;
      $\psi$ (resp. $\phi_+$; $\phi_-$) is the blow-up of $0$ in $Z$ (resp. of $\Proj(A_+)=\dot{X}_+/k^*$ in $Z_+$; of $\Proj(A_-)=\dot{X}_-/k^*$ in $Z_-$).
    \item For every $\chi\in\ZZ$, we have
      \[
        \phi_+^*\scr{O}_\chi^+(1) - \phi_-^*\scr{O}_\chi^-(1)
        = \chi\psi^{-1}(0)
      \]
      in $\Pic_{\QQ}(\hat{Z})$.
  \end{enumerate}
\end{itenv}

In the case studied before, we take $A_\pm=k[N_\pm]$, with the grading induced by the action of $T$ on $N$, and not the grading as an algebra of polynomials.


\subsection{Quotients associated to two contiguous open faces}
\label{2.3}



%% Bibliography %%

\nocite{*}

\begin{thebibliography}{25}

  \bibitem[Ati]{Ati}
  {M.F. Atiyah.}
  \newblock Convexity and commuting Hamiltonians.
  \newblock {\em Bull. London Math. Soc.\/} \textbf{14} (1982), 1--15.

  \bibitem[BB]{BB}
  {A. Bialynicki-Birula.}
  \newblock Some theorems on actions of algebraic groups.
  \newblock {\em Ann. Math.} \textbf{98} (1973), 480--497.

  \bibitem[BBM]{BBM}
  {W. Borho, J.L. Brylinski, R. MacPherson.}
  \newblock Equivariant K-theory approach to nilpotent orbits.
  \newblock {\em Preprint I.H.E.S.} (March 1986).

  \bibitem[Bri]{Bri}
  {M. Brion.}
  \newblock Points entiers dans les poly\`{e}dres convexes.
  \newblock {\em Ann. sci. \'{E}cole Norm. Sup.} \textbf{21} (1988), 653--663.

  \bibitem[Dem]{Dem}
  {M. Demazure.}
  \newblock Anneaux gradu\'{e}s normaux.
  \newblock {\em S\'{e}minaire sur les singularit\'{e}s des surfaces, \'{E}cole polytechnique 1978--79}.

  \bibitem[DH1\&2]{DH12}
  {J.J. Duistermaat and G.J. Heckman.}
  \newblock On the variation in the cohomology of the symplectic form of the reduced phase space 1.
  \newblock {\em Invent. math.} \textbf{69} (1982), 259--268;
  \newblock 2.
  \newblock {\em Invent. math.} \textbf{72} (1983), 153--158.

  \bibitem[EGA~II]{EGAII}
  {A. Grothendieck.}
  \newblock Elements de g\'{e}om\'{e}trie alg\'{e}brique II.
  \newblock {\em Publications math\'{e}matiques de l'I.H.E.S.} \textbf{8} (1961).

  \bibitem[Fle]{Fle}
  {H. Flenner.}
  \newblock Rationale quasihomogene Singularit\"{a}ten.
  \newblock {\em Archiv der Math.} \textbf{36} (1981), 35--44.

  \bibitem[Ful]{Ful}
  {W. Fulton.}
  \newblock {\em Intersection theory}.
  \newblock Springer--Verlag, {Ergebnisse der Math.} \textbf{2} (1984).

  \bibitem[GS1\&2]{GS12}
  {V. Guillemin and S. Sternberg.}
  \newblock Convexity properties of the moment mapping 1.
  \newblock {\em Invent. math.} \textbf{67} (1982), 491--513;
  \newblock 2.
  \newblock {\em Invent. math.} \textbf{77} (1984), 533--546.

  \bibitem[GS3]{GS3}
  {V. Guillemin and S. Sternberg.}
  \newblock Birational equivalence in the symplectic category.
  \newblock {\em Invent. math.} \textbf{97} (1989), 485--522.

  \bibitem[GW]{GW}
  {S. Goto and K.i. Watanabe.}
  \newblock On graded rings I.
  \newblock {\em J. Math. Soc. Japan} \textbf{30} (1978), 179--213.

  \bibitem[Jos]{Jos}
  {A. Joseph.}
  \newblock On the variety of a highest weight module.
  \newblock {\em J. of Alg.} \textbf{88} (1984), 238--278.

  \bibitem[KN]{KN}
  {G. Kempf and L. Ness.}
  \newblock The length of vectors in representation spaces.
  \newblock In {\em Algebraic Geometry}, Springer Lecture Notes in Math. \textbf{732} (1979).

  \bibitem[Kir]{Kir}
  {F. Kirwan.}
  \newblock {\em Cohomology of quotients in symplectic and algebraic geometry}.
  \newblock Princeton University Press, Mathematical Notes \textbf{31} (1984).

  \bibitem[Kle]{Kle}
  {S.L. Kleiman.}
  \newblock Towards a numerical theory of ampleness.
  \newblock {\em Ann. Math.} \textbf{84} (1966), 293--344.

  \bibitem[Kol]{Kol}
  {J. Kollar.}
  \newblock The structure of algebraic threefolds: An introduction to Mori's program.
  \newblock {\em Bull. Amer. Math. Soc.} \textbf{17} (1987), 211-273.

  \bibitem[Kra1]{Kra1}
  {H. Kraft.}
  \newblock {\em Geometrische Methoden in der Invariantentheorie}.
  \newblock Vieweg, Aspekte der Mathematik \textbf{1} (1985).

  \bibitem[Kra2]{Kra2}
  {H. Kraft.}
  \newblock $G$-vector bundles and the linearization problem.
  \newblock To appear in a volume of {\em Contemporary Mathematics}.

  \bibitem[Lun]{Lun}
  {D. Luna.}
  \newblock Slices \'{e}tales.
  \newblock {\em M\'{e}moire de la Soci\'{e}t\'{e} math\'{e}matique de France} \textbf{33} (1973), 81--105.

  \bibitem[MF]{MF}
  {D. Mumford and J. Fogarty.}
  \newblock {\em Geometric invariant theory}.
  \newblock Springer--Verlag, second enlarged edition (1982).

  \bibitem[Nes]{Nes}
  {L. Ness.}
  \newblock A stratification of the nilcone via the moment map.
  \newblock {\em Amer. J. Math.} \textbf{106} (1984), 1281--1330.

  \bibitem[Nie]{Nie}
  {H. Nielsen.}
  \newblock Diagonalizably linearized coherent sheaves.
  \newblock {\em Bull. Soc. Math. France} \textbf{102} (1974), 85--97.

  \bibitem[Smo]{Smo}
  {W. Smoke.}
  \newblock Dimension and multiplicity for graded algebras.
  \newblock {\em J. of Alg.} \textbf{21} (1972), 149--173.

  \bibitem[Val]{Val}
  {F.A. Valentine.}
  \newblock {\em Convex sets}.
  \newblock McGraw Hill (1964).

\end{thebibliography}

\end{document}
