\documentclass{article}

\title{A comment about the integral representation of the Riemann $\xi$-function}
\author{P\'{o}lya, G.}
\date{}

\usepackage{amssymb,amsmath}

\usepackage{hyperref}
\usepackage[nameinlink]{cleveref}
\usepackage{enumerate}

\usepackage{mathrsfs}
%% Fancy fonts --- feel free to remove! %%
\usepackage{Baskervaldx}
\usepackage{mathpazo}


\usepackage{fancyhdr}
\usepackage{lastpage}
\usepackage{xstring}
\makeatletter
\ifx\pdfmdfivesum\undefined
  \let\pdfmdfivesum\mdfivesum
\fi
\edef\filesum{\pdfmdfivesum file {\jobname}}
\pagestyle{fancy}
\makeatletter
\let\runauthor\@author
\let\runtitle\@title
\makeatother
\fancyhf{}
\lhead{\footnotesize\runtitle}
\rhead{\footnotesize Version: \texttt{\StrMid{\filesum}{1}{8}}}
\cfoot{\small\thepage\ of \pageref*{LastPage}}


\crefname{section}{\S}{\SS}
\crefname{equation}{}{}


%% Theorem environments %%

\usepackage{amsthm}


%% Shortcuts %%

\newcommand{\sh}{\mathscr}
\newcommand{\cat}{\mathcal}
\newcommand{\dd}{\operatorname{d}\!}

\renewcommand{\geq}{\geqslant}
\renewcommand{\leq}{\leqslant}

\newcommand{\todo}{\textbf{ !TODO! }}
\newcommand{\oldpage}[1]{\marginpar{\footnotesize$\Big\vert$ \textit{p.~#1}}}


%% Document %%

\usepackage{embedall}
\begin{document}

\maketitle
\thispagestyle{fancy}

\renewcommand{\abstractname}{Translator's note.}

\begin{abstract}
  \renewcommand*{\thefootnote}{\fnsymbol{footnote}}
  \emph{This text is one of a series\footnote{\url{https://thosgood.com/translations/}} of translations of various papers into English.}
  \emph{The translator takes full responsibility for any errors introduced in the passage from one language to another, and claims no rights to any of the mathematical content herein.}
  
  \emph{What follows is a translation of the German paper:}

  \medskip\noindent
  \textsc{P\'{o}lya, G.}
  ``Bemerkung \"{U}ber die Integraldarstellung der Riemannschen $\xi$-Funktion.''
  \emph{Acta Math.}, Volume~\textbf{48} (1926), pp.~305--317.
  \textsc{DOI:} \href{https://doi.org/10.1007/BF02565336}{10.1007/BF02565336}.
  % \emph{Journal}, Volume~\textbf{X} (Date), pp.~Y--Z.
  % {\footnotesize\url{URL}}
\end{abstract}

\setcounter{footnote}{0}

\bigskip


%% Content %%

\emph{[Translator.] The numbering of the footnotes in the original has not been replicated in this translation, since this would have resulted in multiple footnotes with the same number on one page.}

\bigskip

The Riemann $\xi$-function, defined by the formula
\oldpage{305}
\[
\label{1}
  \xi(iz)
  =
  \frac12 \left(
    z^2 - \frac14
  \right) \pi^{-\frac{z}{2}-\frac14} \Gamma \left(
    \frac{z}{2} + \frac14
  \right) \zeta \left(
    \frac12 + z
  \right),
\tag{1}
\]
was represented by Riemann himself by an infinite trigonometric integral, namely\footnote{\textsc{B. Riemann}, Werke (1876), S.~138.}
\[
\label{2}
  \xi(z) = 2\int_0^\infty \Phi(u)\cos(zu)\dd u
\tag{2}
\]
\[
\label{3}
  \Phi(u) = 2\pi e^{\frac{5u}{2}} \sum_{n=1}^\infty (2\pi e^{2u}n^2 - 3) n^2 e^{-n^2\pi e^{2u}}.
\tag{3}
\]
It is evident that
\[
\label{4}
  \Phi(u) \sim 4\pi^2 e^{\frac{9u}{2}-\pi e^{2u}}
  \quad\mbox{as $u\to+\infty$.}
\tag{4}
\]
Furthermore (see \cref{section4}), $\Phi(u)$ is an even function.
Therefore
\[
\label{5}
  \Phi(u) \sim 4\pi^2 \left(
    e^{\frac{9u}{2}} + e^{-\frac{9u}{2}}
  \right) e^{-\pi(e^{2u}+e^{-2u})}
  \quad\mbox{as $u\to\pm\infty$.}
\tag{5}
\]

\oldpage{306}
With regards to the Riemann hypothesis, one could ask the following question\footnote{This was casually mentioned by Prof.~Landau in a conversation in 1913.}: does the function given by replacing $\Phi(u)$ in the right-hand side of \cref{2} with the right-hand side of \cref{4} have only real zeros?

The answer is no (see \cref{section4}): the resulting function has infinitely many imaginary zeros.
If, however, the right-hand side of \cref{5} is used, instead of the right-hand side of \cref{4}, then we obtain the function
\[
\label{6}
  \xi^*(z) = 8\pi^2 \int_0^\infty \left(
    e^{\frac{9u}{2}} + e^{-\frac{9u}{2}}
  \right) e^{-\pi(e^{2u}+e^{-2u})} \cos(zu) \dd u,
\tag{6}
\]
which one could call the ``modified $\xi$-function'', and $\xi^*(z)$ in fact \emph{only has real zeros}.
Incidentally,
\[
  \xi(z) \sim \xi^*(z)
\]
if $z$ tends towards $\infty$ in some closed ray based at the point $0$ and not containing the real axis.
If we denote by $N(r)$ the number of zeros of $\xi(z)$ in the circular region $|z|\leq r$, and by $N^*(r)$ the number of zeros of $\xi^*(z)$ in the same region, then
\[
  N(r) \sim N^*(r)
\]
and even
\[
  N(r) - N^*(r) = O(\log r).
\]



%% Bibliography %%

\nocite{*}
\bibliographystyle{acm}

\end{document}
