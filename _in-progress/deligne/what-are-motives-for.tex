\documentclass{article}

\title{What use are motives?}
\author{Pierre Deligne}
\date{}

\usepackage{amssymb,amsmath}

\usepackage{hyperref}
\usepackage[nameinlink]{cleveref}
\usepackage{enumerate}

\usepackage{mathrsfs}
%% Fancy fonts --- feel free to remove! %%
\usepackage{Baskervaldx}
\usepackage{mathpazo}


\usepackage{fancyhdr}
\usepackage{lastpage}
\usepackage{xstring}
\makeatletter
\ifx\pdfmdfivesum\undefined
  \let\pdfmdfivesum\mdfivesum
\fi
\edef\filesum{\pdfmdfivesum file {\jobname}}
\pagestyle{fancy}
\makeatletter
\let\runauthor\@author
\let\runtitle\@title
\makeatother
\fancyhf{}
\lhead{\footnotesize\runtitle}
\rhead{\footnotesize Version: \texttt{\StrMid{\filesum}{1}{8}}}
\cfoot{\small\thepage\ of \pageref*{LastPage}}


\crefname{section}{\S\!}{\S\S\!}
\crefname{equation}{}{}


%% Theorem environments %%

\usepackage{amsthm}


%% Shortcuts %%

\newcommand{\sh}{\mathscr}
\newcommand{\cat}{\mathcal}
\newcommand{\pr}{\mathrm{pr}}
\newcommand{\ZZ}{\mathbb{Z}}
\newcommand{\QQ}{\mathbb{Q}}
\newcommand{\CC}{\mathbb{C}}
\newcommand{\PP}{\mathbb{P}}
\newcommand{\mot}{\mathrm{mot}}

\renewcommand{\geq}{\geqslant}
\renewcommand{\leq}{\leqslant}

\DeclareMathOperator{\Pic}{Pic}
\DeclareMathOperator{\Hom}{Hom}
\DeclareMathOperator{\Ext}{Ext}
\DeclareMathOperator{\real}{real}
\DeclareMathOperator{\Gr}{Gr}

\newcommand{\todo}{\textbf{ !TODO! }}
\newcommand{\oldpage}[1]{\marginpar{\footnotesize$\Big\vert$ \textit{p.~#1}}}


%% Document %%

\usepackage{embedall}
\begin{document}

\maketitle
\thispagestyle{fancy}

\renewcommand{\abstractname}{Translator's note.}

\begin{abstract}
  \renewcommand*{\thefootnote}{\fnsymbol{footnote}}
  \emph{This text is one of a series\footnote{\url{https://thosgood.com/translations/}} of translations of various papers into English.}
  \emph{The translator takes full responsibility for any errors introduced in the passage from one language to another, and claims no rights to any of the mathematical content herein.}
  
  \emph{What follows is a translation of the French paper:}

  \medskip\noindent
  \textsc{Deligne, P.}
  ``A quoi servent les motifs?''.
  \emph{Proc. Symp. in Pure Math.} \textbf{55} (1994), 143--161.
  {\footnotesize\url{https://publications.ias.edu/deligne/paper/413}}.
\end{abstract}

\setcounter{footnote}{0}

\tableofcontents
\bigskip


%% Content %%

\section*{}

\oldpage{143}
The first of the ``standard conjectures'' (Grothendieck~\cite{19}, Kleiman~\cite{20}), the Lefschetz-style one, says that certain cohomology classes are algebraic.
Anyway, if motives are the ``direct factors'' of algebraic varieties $X$ defined by projectors (algebraic cycles on $X\times X$), then their definition is only reasonable if we have enough algebraic cycles.
On this problem --- the construction of interesting algebraic cycles --- progress has been sparse.

Grothendieck tried to establish a catalogue of projective constructions of cycles (\todo), cf. \cite[p.~197]{19}.
For those for which we have calculated their cohomology class, the class can be expressed in terms of Chern classes of obvious vector bundles.
Even though I do not know of any counterexamples, it seems unlikely to me that the ring of cycle classes on $X\times X$ generated by the diagonal, the divisors, and the inverse images of the Chern classes of $X$ by the $\pr_i$ (for $i=1,2$) will always contain the cycles required by the first standard conjecture (for example, the K\"{u}nneth components of the diagonal).

Instead of trying to construct cycles, we can try to construct vector bundles, and then take their Chern classes.
It is, in fact, like this that K.~Kodaira and D.C.~Spencer (1953) prove the Hodge conjecture for divisors (a theorem of Lefschetz), with the group $\Pic(X)=H^1(X,\sh{O}^*)$ being accessible, over $\CC$, by the exponential exact sequence
\[
  0 \to 2\pi i\ZZ \to \sh{O} \to \sh{O}^* \to 0.
\]
Similarly, on an abelian variety, it is easier to write the functional equation of the $\Theta$ functions (a cocycle for a line bundle) than to define a $\Theta$ function, itself giving the divisor $\Theta=0$.

\oldpage{144}
Unfortunately, in higher rank, we do not know how to construct vector bundles whose Chern classes are interesting any more than we know how to construct interesting cycles.

On $\CC$, the Hodge conjecture would give the desired cycles.
Two difficulties in proving this conjecture have been found.
The first: Atiyah and Hirzebruch have shown that it could only be true rationally (cf. Atiyah-Hirzebruch~\cite{1}).

The second: from the point of view of Hodge theory, the class of a cycle $Z$ of codimension $d$ on a smooth projective $X$ lives naturally, not in $H_\ZZ^{d,d}:=H^{d,d}\cap H_\ZZ$, but in an extension $E_d$ of this group by the intermediate Jacobian:
\[
  J^d(X) := H^{2d-1}(X,\CC)/F^d + H^{2d-1}(X,\ZZ),
\]
where $F^d$ denotes the $d$-th term of the Hodge filtration of $H^{2d-1}(X,\CC)$.
In the category of mixed Hodge structures, this is, respectively, $\Hom(\ZZ(d),H^{2d}(X))$ and $\Ext^1(\ZZ(d),H^{2d-1}(X))$.
The Hodge conjecture says that the group of cycle classes is sent to a subgroup of finite index of the quotient $H_\ZZ^{d,d}$.
However, we have no idea what should be the image in $E_d$.
Furthermore, the aesthetic awkwardness that causes this ignorance renders the methods of P.A.~Griffiths, to prove the Hodge conjecture by induction on the dimension of the variety $X$, generally inapplicable.
This method is inspired by that which Lefschetz~\cite{24} uses for surfaces.
The idea is that, if $(H_t)_{t\in\PP^1}$ is a \todo of hyperplane sections, then constructing the cycle $Z$ on $X$ reduces to constructing the cycles $Z_i=Z\cap H_t$.
In good cases:
\begin{enumerate}[(a)]
  \item the cohomology class $c\in H_\ZZ^{d,d}(X)$ that we wish to be that of a cycle $Z$ restricts to zero on the $H_t$;
  \item if $Z$ exists, then its cohomology class $c$ determines the class of the $Z_t$ in the intermediate Jacobians $J^d(H_t)$; and
  \item the construction of $Z$ reduces to constructing the cycles $Z_t$ of classes defining a given section of the family of the $J^d(H_t)$.
\end{enumerate}
We can thus conclude that every element of $J^d(H_t)$ is the class of a cycle of codimension $d$ that is cohomologous to zero.
Using this method, Zucker~\cite{32} proves the Hodge conjecture for cubic hypersurfaces in $\PP^5$.
Note, however, that, if every element of $J^d(H_t)$ is the class of a cycle that is cohomologous to zero, then $H^{2d-1}(H_t)$ is of Hodge type $\{(d,d-1)(d-1,d)\}$ (see \cref{1.6}), and the converse is a consequence of the Hodge conjecture applied to the product of $H_t$ with a suitable abelian variety.
Furthermore, this surjectivity implies the existence of a curve $C_t$ and an algebraic cycle $W_t$ on $H_t\times C_t$ that sends $H^1(C_t)$ to $H^{2d-1}(H_t)$.
We can then expect that $H^{2d}(X)$ is controlled by the $H^2$ of the surface fibred over $\PP^1$ with fibres given by the $C_t$, and, for an $H^2$, we can use the Hodge conjecture anyway.

\oldpage{145}
The aim of these notes is to show that, despite this lack of progress on the problem of constructing cycles, the philosophy of motives is a powerful tool.

I thank S.~Bloch for his comments on a first version of these notes.


\section{Motives}
\label{1}

According to what we can and want to do, we have various definitions of motives available to us --- or even none.
We need to distinguish pure motives, typically given by the cohomology of non-singular projective varieties, and mixed motives, where open and singular varieties are allowed.
The notion of a motive on $S$ (a family of motives parametrised by $S$) poses yet more problems.

\subsection{}
\label{1.1}

For a field $k$, we always want the category of motives over $k$ to be a $\QQ$-linear abelian category $\sh{M}(k)$ with finite-dimensional $\Hom$ groups.
In the pure case, we want it to be semi-simple and graded (by weights).
In the mixed case, every motive $M$ should admit a finite increasing filtration $W$, with $\Gr_n^W(M)$ pure of weight~$n$.
Some other essential structures are the following:
\begin{enumerate}[(a)]
  \item Every algebraic variety $X$ on $k$ should have motivic cohomology groups $H_\mot^i(X)$, which are objects of $\sh{M}(k)$.
    In the pure case, we restrict to non-singular projective varieties $X$, and $H_\mot^i(X)$ should then be a pure motive of weight~$i$.
  \item For each of the usual cohomology theories $H$, we should have a ``realisation'' functor $\real$, and isomorphisms
    \[
      H^i(X) = \real H_\mot^i(X).
    \]
  \item There should be a tensor product $\otimes$, with which the realisation functors are compatible.
\end{enumerate}

\subsection{}
\label{1.2}

The tensor product structure described above allows us to apply the theory of Tannakian categories, invented by Grothendieck to study the formalism of motives.
References: Saavedra~\cite{28}, Deligne--Milne~\cite{12}, Deligne~\cite{15}.

For $k$ of characteristic $0$, this theory says that the category of motives over $k$, with its tensor product, should be equivalent to the category of linear representations of a scheme $G$ of affine groups over $\QQ$.



%% Bibliography %%

\nocite{*}

\end{document}
