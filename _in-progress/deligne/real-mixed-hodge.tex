\documentclass{article}

\title{Real mixed Hodge structures}
\author{Pierre Deligne}
\date{}

\usepackage{amssymb,amsmath}

\usepackage{hyperref}
\usepackage[nameinlink]{cleveref}
\usepackage{enumerate}

\usepackage{mathrsfs}
%% Fancy fonts --- feel free to remove! %%
\usepackage{Baskervaldx}
\usepackage{mathpazo}


\usepackage{fancyhdr}
\usepackage{lastpage}
\usepackage{xstring}
\makeatletter
\ifx\pdfmdfivesum\undefined
  \let\pdfmdfivesum\mdfivesum
\fi
\edef\filesum{\pdfmdfivesum file {\jobname}}
\pagestyle{fancy}
\makeatletter
\let\runauthor\@author
\let\runtitle\@title
\makeatother
\fancyhf{}
\lhead{\footnotesize\runtitle}
\rhead{\footnotesize Version: \texttt{\StrMid{\filesum}{1}{8}}}
\cfoot{\small\thepage\ of \pageref*{LastPage}}


\crefname{section}{\S\!}{\S\S\!}
\crefname{equation}{}{}


%% Theorem environments %%

\usepackage{amsthm}

\theoremstyle{plain}

\newtheorem*{lemma*}{Lemma}


%% Shortcuts %%

\newcommand{\sh}{\mathscr}
\newcommand{\cat}{\mathcal}

\renewcommand{\geq}{\geqslant}
\renewcommand{\leq}{\leqslant}

\DeclareMathOperator{\Gr}{Gr}

\newcommand{\todo}{\textbf{ !TODO! }}
\newcommand{\oldpage}[1]{\marginpar{\footnotesize$\Big\vert$ \textit{p.~#1}}}


%% Document %%

\usepackage{embedall}
\begin{document}

\maketitle
\thispagestyle{fancy}

\renewcommand{\abstractname}{Translator's note.}

\begin{abstract}
  \renewcommand*{\thefootnote}{\fnsymbol{footnote}}
  \emph{This text is one of a series\footnote{\url{https://thosgood.com/translations/}} of translations of various papers into English.}
  \emph{The translator takes full responsibility for any errors introduced in the passage from one language to another, and claims no rights to any of the mathematical content herein.}
  
  \emph{What follows is a translation of the French paper:}

  \medskip\noindent
  \textsc{Deligne, P.}
  ``Structures de Hodge mixtes r\'{e}elles''.
  \emph{Proc. Symp. in Pure Math.} \textbf{55} (1994), 509--514.
  {\footnotesize\url{https://publications.ias.edu/deligne/paper/393}}
\end{abstract}

\setcounter{footnote}{0}

\tableofcontents
\bigskip


%% Content %%

\oldpage{509}
Real mixed Hodge structures (\cref{2}) form a strict abelian $\otimes$-category, and the ``underlying real vector space'' functor is a \todo functor.
The functor $\Gr^W$ is another such functor.
In this paper, we set out to calculate what the general theory of Tannakian categories gives in this particular case.


\section{Three opposite filtrations}
\label{1}

\subsection{}
\label{1.1}

This section is a variation on section~1.2 of \textsc{P.~Deligne}, ``Th\'{e}orie de Hodge II'', \emph{Publ. Math. IHES} \textbf{40}, pp.~5--58.
Let $\mathfrak{A}$ be an abelian category, and let $A$ be an object of $\mathfrak{A}$ endowed with three opposite filtrations $W$, $F$, and $\overline{F}$.
Here, ``filtration'' means ``finite filtration'' (loc. cit. 1.1.4), the filtration $W$ is assumed to be increasing, and the filtrations $F$ and $\overline{F}$ decreasing.
The opposite condition takes the form $\Gr_F^p\Gr_{\overline{F}}^q\Gr_n^WA=0$ for $n\neq p+q$ (loc. cit. 1.2.7, 1.2.13).

For every bigraded object $B=\bigoplus B^{p,q}$, ew denote by $W$, $F_W$, and $\overline{F}_W$ the three opposite filtrations
\[
  \begin{aligned}
    W_n &= \bigoplus_{p+q\leq n} B^{p,q},
  \\F_W^i &= \bigoplus{p\geq i} B^{p,q},
  \\\overline{F}_W^i &= \bigoplus_{q\geq i} B^{p,q}.
  \end{aligned}
\]
With this notation, saying that the three filtrations $W$, $F$, and $\overline{F}$ of $A$ are opposite again implies (loc. cit. 1.2.6) that each $\Gr_n^W$ admits a (unique)
\oldpage{510}
decomposition
\[
  \Gr_n^W A = \bigoplus_{p+q=n} A_W^{p,q}
\]
such that the filtrations of $\Gr_n^W A$ induced by $F$ and $\overline{F}$ are given by
\[
  \begin{aligned}
    F^i\Gr_n^W A &= \bigoplus_{\substack{p+q=n\\p\geq i}} A_W^{p,q},
  \\\overline{F}^i\Gr_n^W A &= \bigoplus_{\substack{p+q=n\\q\geq i}} A_W^{p,q}.
  \end{aligned}
\]
In other words, $F$ and $\overline{F}$ induce the filtrations $F_W$ and $\overline{F}_W$ on $\Gr^W A$.

Set
\[
  A_F^{p,q} := (W_{p+q}\cap F^p) \cap \left((W_{p+q}\cap\overline{F}^q) + \sum_{i\geq0}(W_{p+q-i}\cap\overline{F}^{q-i+1})\right)
\]
and, swapping the roles of $F$ and $\overline{F}$, and of $p$ and $q$,
\[
  A_{\overline{F}}^{p,q} := (W_{p+q}\cap \overline{F}^q) \cap \left((W_{p+q}\cap F^p) + \sum_{i\geq0}(W_{p+q-i}\cap F^{p-i+1})\right).
\]

For $p+q=n$, the projection from $W_n$ to $\Gr_n^W A$ induces an isomorphism from $A_F^{pq}$ to $A_W^{pq}$: loc. cit. 1.2.8, restricted to the particular case 1.2.11 (where $A_F^{p,q}$ is denoted $A_0^{pq}$; note that, in the formula defining $A_F^{pq}$, we can replace the sum over $i\geq0$ with the same sum over $i\geq2$).
We thus deduce (loc. cit.) that the $A_F^{pq}$ (resp. $A_{\overline{F}}^{pq}$) give a bigrading of $A$, and that the sum $a_F$ (resp. $a_{\overline{F}}$) of the isomorphisms $A_F^{pq}\xrightarrow{\sim}A_W^{pq}$ (resp. $A_{\overline{F}}^{pq}\xrightarrow{\sim}A_W^{pq}$) is a bifiltered isomorphism from $(A;W,F)$ to $(\Gr^W A;W,F_W)$ (resp. from $(A;W,\overline{F})$ to $(\Gr^W A;W,\overline{F}_W)$).

\begin{lemma*}
  The automorphism $d=a_{\overline{F}}a_F^{-1}$ of $\Gr^W A$ satisfies
  \[
  \label{1.1.1}
    (d-1)(A_W^{pq}) \subset \bigoplus_{\substack{r<p\\s<q}} A_W^{rs}.
  \tag{1.1.1}
  \]
\end{lemma*}


\end{document}
