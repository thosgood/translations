\documentclass{report}

\usepackage[margin=1.6in]{geometry}

\title{!!!!!!!! singular differential equations}
\author{Pierre Deligne}
\date{}

\newcommand{\doctype}{French book}
\newcommand{\origcit}{%
  \textsc{Deligne, P.}
  \emph{Equations Diff\'{e}rentielles \`{a} Points Singuliers R\'{e}guliers.}
  Springer--Verlag, Lecture Notes in Mathematics \textbf{163} (1970).
  {\url{https://publications.ias.edu/node/355}}%
}


\usepackage{amssymb,amsmath}

\usepackage{hyperref}
\usepackage{xcolor}
\hypersetup{colorlinks,linkcolor={blue!50!black},citecolor={blue!50!black},urlcolor={blue!80!black}}
\usepackage{enumerate}
\usepackage{graphicx}
\usepackage{tikz-cd}

\usepackage{mathrsfs}
%% Fancy fonts --- feel free to remove! %%
\usepackage{fouriernc}


\usepackage{fancyhdr}
\usepackage{lastpage}
\usepackage{xstring}
\makeatletter
\ifx\pdfmdfivesum\undefined
  \let\pdfmdfivesum\mdfivesum
\fi
\edef\filesum{\pdfmdfivesum file {\jobname}}
\pagestyle{fancy}
\fancypagestyle{plain}{}
\fancyhf{}
\lhead{\footnotesize\nouppercase\leftmark}
\lfoot{\footnotesize Version: \texttt{\StrMid{\filesum}{1}{8}}}
\cfoot{\small\thepage\ of \pageref*{LastPage}}


%% Theorem environments %%

\usepackage{amsthm}

\newenvironment{itenv}[1]
  {\phantomsection\par\medskip\noindent\textbf{#1.}\itshape}
  {\medskip}

\newenvironment{itenv*}
  {\phantomsection\par\medskip\noindent\itshape}
  {\medskip}

\newenvironment{rmenv}[1]
  {\phantomsection\par\medskip\noindent\textbf{#1.}\rmfamily}
  {\medskip}


%% Shortcuts %%

\newcommand{\scr}[1]{{\mathscr{#1}}}
\renewcommand{\cal}[1]{{\mathcal{#1}}}
\newcommand{\fk}[1]{{\mathfrak{#1}}}
\newcommand{\sbullet}{{\mathbin{\vcenter{\hbox{\scalebox{.5}{$\bullet$}}}}}}
\newcommand{\id}{\mathrm{Id}}
\newcommand{\PP}{\mathbb{P}}
\renewcommand{\AA}{\mathbb{A}}
\newcommand{\CC}{\mathbb{C}}
\newcommand{\QQ}{\mathbb{Q}}
\newcommand{\ZZ}{\mathbb{Z}}
\newcommand{\NN}{\mathbb{N}}
\newcommand{\dd}{\mathrm{d}}
\newcommand{\DD}{\mathrm{D}}
\newcommand{\pr}{\mathrm{pr}}
\newcommand{\II}{\mathrm{II}}
\newcommand{\RR}{\mathbf{R}}
\newcommand{\MM}{\mathrm{M}}
\newcommand{\GL}{\mathrm{GL}}
\newcommand{\SL}{\mathrm{SL}}
\newcommand{\SO}{\mathrm{SO}}
\newcommand{\tg}{\mathrm{tg}}
\newcommand{\an}{\mathrm{an}}
\newcommand{\detrow}[2]{\operatorname{det}(#1,#2)}
\newcommand{\simto}{\xrightarrow{\raisebox{-0.7ex}[0ex][0ex]{$\sim$}}}

\renewcommand{\geq}{\geqslant}
\renewcommand{\leq}{\leqslant}

\DeclareMathOperator{\Spec}{Spec}
\DeclareMathOperator{\Ker}{Ker}
\DeclareMathOperator{\Coker}{Coker}
\DeclareMathOperator{\Hom}{Hom}
\DeclareMathOperator{\End}{End}
\DeclareMathOperator{\shHom}{\underline{Hom}}
\DeclareMathOperator{\shEnd}{\underline{End}}
\DeclareMathOperator{\HH}{H}
\DeclareMathOperator{\Gr}{Gr}
\DeclareMathOperator{\tr}{tr}
\DeclareMathOperator{\Res}{Res}
\DeclareMathOperator{\codim}{codim}

\newcommand{\todo}{\textbf{ !TODO! }}
\newcommand{\oldpage}[1]{\marginpar{\footnotesize$\Big\vert$ \textit{p.~#1}}}


%% Document %%

\usepackage{embedall}
\begin{document}

\hypersetup{pageanchor=false}
\begin{titlepage}
\maketitle
\end{titlepage}

\renewcommand{\abstractname}{Translator's note.}

\begin{abstract}
  \renewcommand*{\thefootnote}{\fnsymbol{footnote}}
  \emph{This text is one of a series\footnote{\url{https://thosgood.com/translations}} of translations of various papers into English.}
  \emph{The translator takes full responsibility for any errors introduced in the passage from one language to another, and claims no rights to any of the mathematical content herein.}

  \medskip
  
  \emph{What follows is a translation of the \doctype:}

  \medskip\noindent
  \origcit

  \medskip
  \noindent\emph{We have also made changes following the errata, which was written in April 1971, by P. Deligne, at Warwick University.}
\end{abstract}

\hypersetup{pageanchor=true}

\setcounter{footnote}{0}

\tableofcontents


%% Content %%

\setcounter{chapter}{-1}

\chapter{Introduction}
\label{0}

\oldpage{1}
If $X$ is a (non-singular) complex-analytic manifold, then there is an equivalence between the notions of
\begin{enumerate}[a)]
  \item local systems of complex vectors on $X$ ; and
  \item vector bundles on $X$ endowed with an integrable connection.
\end{enumerate}

The latter of these two notions can be adapted in an evident way to the case where $X$ is a non-singular algebraic variety over a field $k$ (which we will take here to be of characteristic $0$).
However, general algebraic vector bundles with integrable connections are pathological (see \hyperref[II.6.19]{(II.6.19)});
we only obtain a reasonable theory if we impose a ``regularity'' condition at infinity.
By a theorem of Griffiths \cite{8}, this condition is automatically satisfied for ``Gauss-Manin connections'' (see \hyperref[II.7]{(II.7)}).
In dimension one, this is closely linked to the idea of regular singular points of a differential equation (see \hyperref[I.4]{(I.4)} and \hyperref[II.1]{(II.1)}).

In Chapter~I, we explain the different forms that the notion of an integrable connection can take.
In Chapter~II, we prove the fundamental facts concerning regular connections.
In Chapter~III, we translate certain results that we have obtained into the language of Nilsson class functions, and, as an application of the regularity theorem (\hyperref[II.7]{(II.7)}), we explain the proof by Brieskorn \cite{5} of the monodromy theorem.

These notes came from the non-crystalline part of a seminar given at Harvard during the autumn of 1969, under the title: ``Regular singular differential equations and crystalline cohomology''.

I thank the assistants of this seminar, who had to be subjected to often unclear talks, and who allowed me to find numerous simplifications.

I also thank N.~Katz, with whom I had numerous and useful conversations, and to whom are due the principal results of section~\hyperref[II.1]{(II.1)}.


\section*{Notation and terminology}

\oldpage{2}
Within a single chapter, the references follow the decimal system.
A reference to a different chapter (resp. to the current introduction) is preceded by the Roman numeral of the chapter (resp. by 0).

We will use the following definitions:
\begin{enumerate}[({0.}1)]
  \item\label{0.1}
    \emph{analytic space}:
    the analytic spaces are complex and of locally-finite dimension.
    They are assumed to be $\sigma$-compact, but not necessarily separated.
  \item\label{0.2}
    \emph{multiform function}:
    a synonym for multivalued function --- for a precise definition, see \hyperref[I.6.2]{(I.6.2)}.
  \item\label{0.3}
    \emph{immersion}:
    following the tradition of algebraic geometers, immersion is a synonym for ``embedding''.
  \item\label{0.4}
    \emph{smooth}:
    a morphism $f\colon X\to S$ of analytic spaces is smooth if, locally on $X$, it is isomorphic to the projection from $D^n\times S$ to $S$, where $D^n$ is an open polydisc.
  \item\label{0.5}
    \emph{locally paracompact}:
    a topological space is locally paracompact if every point has a paracompact neighbourhood (and thus a fundamental system of paracompact neighbourhoods).
  \item\label{0.6}
    non-singular (or smooth) \emph{complex algebraic variety}:
    a smooth scheme of finite type over $\Spec(\mathbb{C})$.
  \item\label{0.7}
    (complex) \emph{analytic manifold}:
    a non-singular (or smooth) analytic space.
  \item\label{0.8}
    \emph{covering}:
    following the tradition of topologists, a covering is a continuous map $f\colon X\to Y$ such that every point $y\in Y$ has a neighbourhood $V$ such that $f|V$ is isomorphic to the projection from $F\times V$ to $V$, where $F$ is discrete.
\end{enumerate}


\renewcommand{\thechapter}{\Roman{chapter}}

\chapter{Dictionary}
\label{I}

\oldpage{3}
In this chapter, we explain the relations between various aspects and various uses of the notion of ``local systems of complex vectors''.
The equivalence between the points of view considered has been well known for a long time.

We do not consider the ``crystalline'' point of view;
see \cite{4,10}.


\section{Local systems and the fundamental group}
\label{I.1}

\begin{rmenv}{Definition 1.1}
\label{I.1.1}
  Let $X$ be a topological space.
  A \emph{complex local system} on $X$ is a sheaf of complex vectors on $X$ that, locally on $X$, is isomorphic to a constant sheaf $\mathbb{C}^n$ (n$\in\mathbb{N}$).
\end{rmenv}

\begin{rmenv}{1.2}
\label{I.1.2}
  Let $X$ be a locally path-connected and locally simply path-connected topological space, along with a basepoint $x_0\in X$.
  To avoid any ambiguity, we point out that:
  \begin{enumerate}[a)]
    \item The fundamental group $\pi_1(X,x_0)$ of $X$ at $x_0$ has elements given by homotopy classes of loops based at $x_0$;
    \item If $\alpha,\beta\in\pi_1(X,x_0)$ are represented by loops $a$ and $b$, then $\alpha\beta$ is represented by the loop $ab$ obtained by juxtaposing $b$ and $a$, in that order.
  \end{enumerate}

  Let $\scr{F}$ be a locally constant sheaf on $X$.
  For every path $a\colon[0,1]\to X$, the inverse image $a^*\scr{F}$ of $\scr{F}$ on $[0,1]$ is a locally constant, and thus constant, sheaf, and there exists exactly one isomorphism between $a^*\scr{F}$ and the constant sheaf defined by the set $(a^*\scr{F})_0 = \scr{F}_{a(0)}$.
  This isomorphism defines an isomorphism $a(\scr{F})$ between $(a^*\scr{F})_0$ and $(a^*\scr{F})_1$, i.e. an isomorphism
  \[
    a(\scr{F})\colon \scr{F}_{a(0)} \to \scr{F}_{a(1)}.
  \]
  This isomorphism depends only on the homotopy class of $a$, and satisfies $ab(\scr{F}) = a(\scr{F})\cdot b(\scr{F})$.
  In particular, $\pi_1(X,x_0)$ acts (on the left) on the fibre $\scr{F}_{x_0}$ of $\scr{F}$ at $x_0$.
  It is well known that:
\end{rmenv}

\begin{itenv}{Proposition 1.3}
\label{I.1.3}
  Under the hypotheses of \hyperref[I.1.2]{(1.2)}, with $X$ connected, the functor $\scr{F}\mapsto\scr{F}_{x_0}$ is an equivalence between the category of locally constant sheaves on $X$ and the category of sets endowed with an action by the group $\pi_1(X,x_0)$.
\end{itenv}

\oldpage{4}
\begin{itenv}{Corollary 1.4}
\label{I.1.4}
  Under the hypotheses of \hyperref[I.1.2]{(1.2)}, with $X$ connected, the functor $\scr{F}\mapsto\scr{F}_{x_0}$ is an equivalence between the category of complex local systems on $X$ and the category of complex finite-dimensional representations of $\pi_1(X,x_0)$.
\end{itenv}

\begin{rmenv}{1.5}
\label{I.1.5}
  Under the hypotheses of \hyperref[I.1.2]{(1.2)}, if $a\colon[0,1]\to X$ is a path, and $b$ a loop based at $a(0)$, then $aba^{-1}=a(b)$ is a path based at $a(1)$.
  Its homotopy class depends only on the homotopy classes of $a$ and $b$.
  This construction defines an isomorphism between $\pi_1(X,a(0))$ and $\pi_1(X,a(1))$.
\end{rmenv}

\begin{itenv}{Proposition 1.6}
\label{I.1.6}
  Under the hypotheses of \hyperref[I.1.5]{(1.5)}, there exists, up to unique isomorphism, exactly one locally constant sheaf of groups $\Pi_1(X)$ on $X$ (\emph{the fundamental groupoid}), endowed, for all $x_0\in X$, with an isomorphism
  \[
  \label{I.1.6.1}
    \Pi_1(X)_{x_0} \simeq \pi_1(X,x_0)
  \tag{1.6.1}
  \]
  and such that, for every path $a\colon[0,1]\to X$, the isomorphism in \hyperref[I.1.5]{(1.5)} between $\pi_1(X,a(0))$ and $\pi_1(X,a(1))$ can be identified, via \hyperref[I.1.6.1]{(1.6.1)}, with the isomorphism in \hyperref[I.1.2]{(1.2)} between $\Pi_1(X)_{a(0)}$ and $\Pi_1(X)_{a(1)}$.
\end{itenv}

If $X$ is connected, with base point $x_0$, then the sheaf $\Pi_1(X)$ corresponds, via the equivalence in \hyperref[I.1.3]{(1.3)}, to the group $\pi_1(X,x_0)$ endowed with its action over itself by inner automorphisms.

\begin{itenv}{Proposition 1.7}
\label{I.1.7}
  If $\scr{F}$ is a locally constant sheaf on $X$, then there exists exactly one action (said to be \emph{canonical}) of $\Pi_1(X)$ on $\scr{F}$ that, at each $x_0\in X$, induces the action from \hyperref[I.1.2]{(1.2)} of $\pi_1(X,x_0)$ on $\scr{F}$.
\end{itenv}


\section{Integrable connections and local systems}
\label{I.2}

\oldpage{5}

\begin{rmenv}{2.1}
  Let $X$ be an analytic space \hyperref[0.1]{(0.1)}.
  We define a (holomorphic) \emph{vector bundle} on $X$ to be a locally free sheaf of modules that is of finite type over the structure sheaf $\cal{O}$ of $X$.
  If $\cal{V}$ is a vector bundle on $X$, and $x$ a point of $X$, then we denote by $\cal{V}_{(x)}$ the free $\cal{O}_{(x)}$-module of finite type of germs of sections of $\cal{V}$.
  If $\mathfrak{m}_x$ is the maximal ideal of $\cal{O}_{(x)}$, then we define the \emph{fibre at $x$ of the vector bundle $\cal{V}$} to be the \todo of finite rank
  \[
  \label{I.2.1.1}
    \cal{V}_x = \cal{V}_{(x)} \otimes_{\cal{O}_{(x)}} \cal{O}_{(x)}/\mathfrak{m}_x.
  \tag{2.1.1}
  \]

  If $f\colon X\to Y$ is a morphism of analytic spaces, then the \emph{inverse image} of a vector bundle $\cal{V}$ on $Y$ is the vector bundle $f^*\cal{V}$ on $X$ given by the inverse image of $\cal{V}$ as a coherent module:
  if $f^\sbullet\cal{V}$ is the sheaf-theoretic inverse image of $\cal{V}$, then
  \[
  \label{I.2.1.2}
    f^*\cal{V} \simeq \cal{O}_X \otimes_{f^*\cal{O}_Y} f^\sbullet\cal{V}
  \tag{2.1.2}
  \]

  In particular, if $x\colon P\to X$ is the morphism from the point space $P$ to $X$ defined by a point $x$ of $X$, then
  \[
    \label{I.2.1.3}
      \cal{V}_x \simeq x^*\cal{V}.
    \tag{2.1.3}
  \]
\end{rmenv}

\begin{rmenv}{2.2}
\label{I.2.2}
  Let $X$ be a complex-analytic manifold \hyperref[0.7]{(0.7)} and $\cal{V}$ a vector bundle on $X$.
  The old school would have defined a (holomorphic) connection on $\cal{V}$ as the data, for every pair of points $(x,y)$ that are first order infinitesimal neighbours in $X$, of an isomorphism $\gamma_{y,x}\colon\cal{V}_x\to\cal{V}_y$ that depends holomorphically on $(x,y)$ and is such that $\gamma_{x,x}=\id$.

  Suitably interpreted, this ``definition'' coincides with the currently fashionable definition \hyperref[I.2.2.4]{(2.2.4)} given below (which we not be use in the rest of the section).

  It suffices to understand ``point'' to mean ``point with values in any analytic space'':

\oldpage{6}
  \begin{rmenv}{2.2.1}
  \label{I.2.2.1}
    \emph{A point in an analytic space $X$ with values in an analytic space $S$} is a morphism from $S$ to $X$.
  \end{rmenv}

  \begin{rmenv}{2.2.2}
  \label{I.2.2.2}
    If $Y$ is a subspace of $X$, then the \emph{$n$\textsuperscript{th} infinitesimal neighbourhood} of $Y$ in $X$ is the subspace of $X$ defined locally by the $(n+1)$th power of the ideal of $\cal{O}_X$ that defines $Y$.
  \end{rmenv}

  \begin{rmenv}{2.2.3}
  \label{I.2.2.3}
    Two points $x,y\in X$ with values in $S$ are said to be \emph{first order infinitesimal neighbours} if the map $(x,y)\colon S\to X\times X$ that they define factors through the first order infinitesimal neighbourhood of the diagonal of $X\times X$.
  \end{rmenv}

  \begin{rmenv}{2.2.4}
  \label{I.2.2.4}
    If $X$ is a complex-analytic manifold and $\cal{V}$ is a vector bundle on $X$, then a (\emph{holomorphic}) \emph{connection} $\gamma$ on $\cal{V}$ consists of the following data:

    for every pair $(x,y)$ of points of $X$ with values in an arbitrary analytic space $S$, with $x$ and $y$ first order infinitesimal neighbours, an isomorphism $\gamma_{x,y}\colon x^*\cal{V}\to y^*\cal{V}$;
    this data is subject to the conditions:
    \begin{enumerate}[(i)]
      \item (functoriality) For any $f\colon T\to S$ and any first order infinitesimal neighbours $x,y\colon S\rightrightarrows X$, we have $f^*(\gamma_{y,x})=\gamma_{yf,xf}$.
      \item We have $\gamma_{x,x}=\id$.
    \end{enumerate}
  \end{rmenv}
\end{rmenv}

\begin{rmenv}{2.3}
\label{I.2.3}
  Let $X_1$ be the first-order infinitesimal neighbourhood of the diagonal $X_0$ of $X\times X$, and let $p_1$ and $p_2$ be the two projections of $X_1$ to $X$.
  By definition, the vector bundle $P^1(\cal{V})$ of first-order jets of sections of $\cal{V}$ is the bundle $(p_1)_*p_2^*\cal{V}$.
  We denote by $j^1$ the first-order differential operator that sends each section of $\cal{V}$ to its first-order jet:
  \[
    j^1\colon \cal{V} \to P^1(\cal{V}) \simeq \cal{O}_{X_1}\otimes_{\cal{O}_X}\cal{V}.
  \]

  A connection (\hyperref[I.2.2.4]{(2.2.4)}) can be understood as a homomorphism (which is automatically an isomorphism)
  \[
  \label{I.2.3.1}
    \gamma = p_1^*\cal{V} \to p_2^*\cal{V}
  \tag{2.3.1}
  \]
  which induces the identity over $X_0$.
  Since
  \[
    \Hom_{X_1}(p_1^*\cal{V},p_2^*\cal{V}) \simeq \Hom(\cal{V},(p_1)_*p_2^*\cal{V}),
  \]
\oldpage{7}
  a connection can also be understood as a ($\cal{O}$-linear) homomorphism
  \[
  \label{I.2.3.2}
    \DD\colon \cal{V} \to P^1(\cal{V})
  \tag{2.3.2}
  \]
  such that the obvious composite arrow
  \[
    \cal{V}\xrightarrow{\DD} P^1(\cal{V}) \to \cal{V}
  \]
  is the identity.
  The sections $\DD s$ and $j^1(s)$ of $P^1(v)$ thus have the same image in $\cal{V}$, and $j^1(s)-\DD(s)$ can be identified with a section $\nabla s$ of $\Omega_X^1\otimes\cal{V} \simeq \Ker(P^1(\cal{V})\to\cal{V})$:
  \[
  \label{I.2.3.3}
    \nabla\colon \cal{V} \to \Omega^1(X)
  \tag{2.3.3}
  \]
  \[
  \label{I.2.3.4}
    j^1(s) = \DD(s)+\nabla s.
  \tag{2.3.4}
  \]

  In other words, a connection \hyperref[I.2.2.4]{(2.2.4)}, allowing us to compare two neighbouring fibres of $\cal{V}$, also allows us to define the differential $\nabla s$ of a section of $\cal{V}$.

  Conversely, equation~\hyperref[I.2.3.4]{(2.3.4)} allows us to define $\DD$, and thus $\gamma$, from the covariant derivative $\nabla$.
  For $\DD$ to be linear, it is necessary and sufficient for $\nabla$ to satisfy the identity
  \[
  \label{I.2.3.5}
    \nabla(fs) = \dd f\cdot s + f\cdot\nabla s
  \tag{2.3.5}
  \]

  Definition~\hyperref[I.2.2.4]{(2.2.4)} is thus equivalent to the following definition, due to J.L.~Koszul.
\end{rmenv}

\begin{rmenv}{Definition 2.4}
\label{I.2.4}
  Let $\cal{V}$ be a (holomorphic) vector bundle on a complex-analytic manifold $X$.
  A \emph{holomorphic connection} (or simply, \emph{connection}) on $\cal{V}$ is a $\mathbb{C}$-linear homomorphism
  \[
    \nabla\colon \cal{V} \to \Omega_X^1(\cal{V}) = \Omega_X^1\otimes_{\cal{O}}\cal{V}
  \]
  that satisfies the Leibniz identity (\hyperref[I.2.3.5]{(2.3.5)}) for local sections $f$ of $\cal{O}$ and $s$ of $\cal{V}$.
  We call $\nabla$ the \emph{covariant derivative} defined by the connection.
\end{rmenv}

\begin{rmenv}{2.5}
\label{I.2.5}
  If the vector bundle $\cal{V}$ is endowed with a connection $\Gamma$ with covariant derivative $\nabla$, and if $w$ is a holomorphic vector field on $X$, then we set, for every local section $v$ of $\cal{V}$ over an open subset $U$ of $X$,
  \[
    \nabla_w(v) = \langle \nabla v,w \rangle \in \cal{V}(U).
  \]
  We call $\nabla_w\colon \cal{V} \to \cal{V}$ the \emph{covariant derivative along the vector field $w$}.
\end{rmenv}

\oldpage{8}
\begin{rmenv}{2.6}
\label{I.2.6}
  If ${}_1\!\Gamma$ and ${}_2\!\Gamma$ are connections on $X$, with covariant derivatives ${}_1\!\nabla$ and ${}_2\!\nabla$ (respectively), then ${}_2\!\nabla-{}_1\!\nabla$ is a $\cal{O}$-linear homomorphism from $\cal{V}$ to $\Omega_X^1(\cal{V})$.
  Conversely, the sum of ${}_1\!\nabla$ and such a homomorphism defines a connection on $\cal{V}$.
  Thus connections on $\cal{V}$ form a principal homogeneous space (or torsor) on $\shHom(\cal{V},\Omega_X^1(\cal{V})) \simeq \Omega_X^1(\shEnd(\cal{V}))$.
\end{rmenv}

\begin{rmenv}{2.7}
\label{I.2.7}
  If vector bundles are endowed with connections, then every vector bundle obtained by a ``tensor operation'' is again endowed with a connection.
  This is evident with \hyperref[I.2.2.4]{(2.2.4)}.
  More precisely, let $\cal{V}_1$ and $\cal{V}_2$ be vector bundles endowed with connections with covariant derivatives $\nabla_1$ and $\nabla_2$.

  \begin{rmenv}{2.7.1}
  \label{I.2.7.1}
    We define a connection on $\cal{V}_1\oplus\cal{V}_2$ by the formula
    \[
      \nabla_w(v_1+v_2) = {}_1\!\nabla_w(v_1) + {}_2\!\nabla_w(v_2)
    \]
  \end{rmenv}

  \begin{rmenv}{2.7.2}
  \label{I.2.7.2}
    We define a connection on $\cal{V}_1\otimes\cal{V}_2$ by the Leibniz formula
    \[
      \nabla_w(v_1\otimes v_2) = \nabla_w v_1\cdot v_2 + v_1\cdot\nabla_w v_2.
    \]
  \end{rmenv}

  \begin{rmenv}{2.7.3}
  \label{I.2.7.3}
    We define a connection on $\shHom(\cal{V}_1,\cal{V}_2)$ by the formula
    \[
      (\nabla_w f)(v_1) = {}_2\!\nabla_2(f(v_1)) - f({}_1\!\nabla v_1).
    \]
  \end{rmenv}

  The canonical connection on $\cal{O}$ is the connection for which $\nabla f=\dd f$.
  
  Let $\cal{V}$ be a vector bundle endowed with a connection.
  \begin{rmenv}{2.7.4}
  \label{I.2.7.4}
    We define a connection on the dual $\cal{V}^\vee$ of $\cal{V}$ via \hyperref[I.2.7.3]{(2.7.3)} and the defining isomorphism $\cal{V}^\vee = \shHom(\cal{V},\cal{O})$.
    We have
    \[
      \langle \nabla_w v',v \rangle = \partial_w\langle v',v \rangle - \langle v',\nabla_w v \rangle.
    \]
  \end{rmenv}

  We leave it to the reader to verify that these formulas do indeed define connections.
  For \hyperref[I.2.7.2]{(2.7.2)}, for example, one must verify that, firstly, the given formula defines a $\mathbb{C}$-bilinear map from $(\cal{V}_1\otimes\cal{V}_2)$, which means that the right-hand side $\II(v_1,v_2)$ is $\mathbb{C}$-bilinear and such that $\II(fv_1,v_2)=\II(v_1,fv_2)$;
  secondly, one must also verify identity~\hyperref[I.2.3.5]{(2.3.5)}.
\end{rmenv}

\begin{rmenv}{2.8}
\label{I.2.8}
  An $\cal{O}$-homomorphism $f$ between vector bundles $\cal{V}_1$ and $\cal{V}_2$ endowed with connections
\oldpage{9}
  is said to be \emph{compatible with the connections} if
  \[
    {}_2\!\nabla\cdot f = f\cdot{}_1\!\nabla.
  \]
  By \hyperref[I.2.7.3]{(2.7.3)}, this reduces to saying that $\nabla f=0$, if $f$ is thought of as a section of $\shHom(\cal{V}_1,\cal{V}_2)$.
  For example, by \hyperref[I.2.7.3]{(2.7.3)}, the canonical map
  \[
    \Hom(\cal{V}_1,\cal{V}_2)\otimes\cal{V}_1 \to \cal{V}_2
  \]
  is compatible with the connections.
\end{rmenv}

\begin{rmenv}{2.9}
\label{I.2.9}
  A local section $v$ of $\cal{V}$ is said to be \emph{horizontal} if $\nabla v=0$.
  If $f$ is a homomorphism between bundles $\cal{V}_1$ and $\cal{V}_2$ endowed with connections, then it is equivalent to say either that $f$ is horizontal, or that $f$ is compatible with the connections \hyperref[I.2.8]{(2.8)}.
\end{rmenv}

\begin{rmenv}{2.10}
\label{I.2.10}
  Let $\cal{V}$ be a holomorphic vector bundle on $X$.
  Define $\Omega_X^p=\bigwedge^p\Omega_X^1$ and $\Omega_X^p(\cal{V})=\Omega_X^p\otimes_\cal{O}\cal{V}$ (the sheaf of \emph{exterior differential $p$-forms with values in $\cal{V}$}).
  Suppose that $\cal{V}$ is endowed with a holomorphic connection.
  We then define $\CC$-linear morphisms
  \[
  \label{I.2.10.1}
    \nabla\colon \Omega_X^p(\cal{V}) \to \Omega_X^{p+1}(\cal{V})
  \tag{2.10.1}
  \]
  characterised by the following formula:
  \[
  \label{I.2.10.2}
    \nabla(\alpha,v) = \dd\alpha\cdot v + (-1)^p\alpha\wedge\nabla v,
  \tag{2.10.2}
  \]
  where $\alpha$ is any local section of $\Omega^p$, $v$ is any local section of $\cal{V}$, and $\dd$ is the exterior differential.
  To prove that the right-hand side $\II(\alpha,v)$ of \hyperref[I.2.10.2]{(2.10.2)} defines a homomorphism \hyperref[I.2.10.1]{(2.10.1)}, it suffices to show that $\II(\alpha,v)$ is $\CC$-bilinear and satisfies
  \[
    \II(f\alpha,v) = \II(\alpha,fv).
  \]
  But we have that
  \[
    \begin{aligned}
      \II(f\alpha,v)
      &= \dd(f\alpha)v + (-1)^pf\alpha\wedge\nabla v
    \\&= \dd\alpha\cdot fv + \dd f\wedge\alpha v + (-1)^pf\alpha\wedge\nabla v
    \\&= \dd\alpha\cdot fv + (-1)^p\alpha\wedge(f\nabla v+\dd f\cdot v)
    \\&= \II(\alpha,fv).
    \end{aligned}
  \]

  Let $\cal{V}_1$ and $\cal{V}_2$ be vector bundles endowed with connections, and let $\cal{V}$ be their tensor product \hyperref[I.2.7.2]{(2.7.2)}.
  We denote by $\wedge$ the evident maps
  \[
    \wedge\colon \Omega^p(\cal{V}_1)\otimes\Omega^1(\cal{V}_2) \to \Omega^{p+q}(\cal{V})
  \]
\oldpage{10}
  such that, for any local section $\alpha$ (resp. $\beta$, resp. $v_1$, resp. $v_2$) of $\Omega^p$ (resp. $\Omega^q$, resp. $\cal{V}_1$, resp. $\cal{V}_2$), we have that $(\alpha\otimes v_1)\wedge(\beta\otimes v_2) = (\alpha\wedge\beta)\otimes(v_1\otimes v_2)$.
  If $\nu_1$ (resp. $\nu_2$) is any local section of $\Omega^p(\cal{V}_1)$ (resp. $\Omega^q(\cal{V}_2)$), then
  \[
  \label{I.2.10.3}
    \nabla(\nu_1\wedge\nu_2) = \nu_1\wedge\nu_2 + (-1)^p\nu_1\wedge\nu_2.
  \tag{2.10.3}
  \]
  Indeed, if $\nu_1=\alpha v_1$ and $\nu_2=\beta v_2$, then
  \[
    \begin{aligned}
      \nabla(\nu_1\wedge\nu_2)
      &= \nabla(\alpha\wedge\beta\otimes v_1\otimes v_2)
    \\&= \dd(\alpha\wedge\beta)v_1\otimes v_2 + (-1)^{p+q}\alpha\wedge\beta\wedge\nabla(v_1\otimes v_2)
    \\&= \dd\alpha\wedge\beta v_1\otimes v_2 + (-1)^p\alpha\wedge\dd\beta v_1\otimes v_2
    \\&\quad+ (-1)^{p+q}\alpha\wedge\beta\wedge\nabla v_1\otimes v_2 + (-1)^{p+q}\alpha\wedge\beta v_1\wedge\nabla v_2
    \\&= \dd\alpha v_1\wedge\nu_2 + (-1)^p\nu_1\wedge\dd\beta v_2 + (-1)^p\alpha\wedge\nabla v_1\wedge\nu_2
    \\&\quad+ (-1)^{p+q}\nu_2\wedge\beta\wedge\nabla v_2
    \\&= (\dd\alpha v_1 + (-1)^p\alpha\wedge\nabla v_1)\wedge\nu_2 + (-1)^p\nu_1\wedge(\dd\beta v_2 + (-1)^q\beta\wedge\nabla v_2)
    \\&= \nabla\nu_1\wedge\nu_2 +(-1)^p\nu_1\wedge\nabla\nu_2.
    \end{aligned}
  \]

  Let $\cal{V}$ be a vector bundle endowed with a connection.
  If we apply the above formula to $\cal{O}$ and $\cal{V}$, then, for any local section $\alpha$ (resp. $\nu$) of $\Omega^p$ (resp. $\Omega^q(\cal{V})$), we have that
  \[
  \label{I.2.10.4}
    \nabla(\alpha\wedge\nu) = \dd\alpha\wedge\nu + (-1)^p\alpha\wedge\nabla\nu.
  \tag{2.10.4}
  \]

  Iterating this formula gives
  \[
  \label{I.2.10.5}
    \begin{aligned}
      \nabla\nabla(\alpha\wedge\nu)
      &= \nabla(\dd\alpha\wedge\nu + (-1)^p\alpha\wedge\nabla\nu)
      \\&= \dd\alpha\wedge\nu + (-1)^{p+1}\dd\alpha\wedge\nabla\nu + (-1)^p\dd\alpha\wedge\nabla\nu + \alpha\wedge\nabla\nabla\nu
      \\&= \alpha\wedge\nabla\nabla\nu.
    \end{aligned}
  \tag{2.10.5}
  \]
\end{rmenv}

\begin{rmenv}{Definition 2.11}
\label{I.2.11}
  Under the hypotheses of \hyperref[I.2.10]{(2.10)}, the \emph{curvature} $R$ of the given connection on $\cal{V}$ is the composite homomorphism
  \[
    R\colon \cal{V} \to \Omega_X^2(\cal{V})
  \]
  considered as a section of $\Hom(\cal{V},\Omega_X^2(\cal{V})) \simeq \Omega_X^2(\End(\cal{V}))$.
\end{rmenv}

\begin{rmenv}{2.12}
\label{I.2.12}
  Taking $q=0$ in \hyperref[I.2.10.4]{(2.10.4)} gives
  \[
  \label{I.2.12.1}
    \nabla\nabla(\alpha v) = \alpha\wedge R(v),
  \tag{2.12.1}
  \]
  which we write as
\oldpage{11}
  \[
  \label{I.2.12.2}
    \nabla\nabla(\nu) = R\wedge\nu
    \qquad\mbox{(the \emph{Ricci identity}).}
  \tag{2.12.2}
  \]

  We endow $\shEnd(\cal{V})$ with the connection given in \hyperref[I.2.7.3]{(2.7.3)}.
  The equation $\nabla(\nabla\nabla)=(\nabla\nabla)\nabla$ can be written as $\nabla(R\wedge\nu) = R\wedge\nabla\nu$.
  By \hyperref[I.2.7.3]{(2.7.3)}, we have that $\nabla R\wedge\nu = \nabla(R\wedge\nu) - R\wedge\nabla\nu$, so that
  \[
  \label{I.2.12.3}
    \nabla R=0
    \qquad\mbox{(the \emph{Bianchi identity}).}
  \tag{2.12.3}
  \]
\end{rmenv}

\begin{rmenv}{2.13}
\label{I.2.13}
  If $\alpha$ is an exterior differential $p$-form, then we know that
  \[
    \begin{aligned}
      \langle \dd\alpha, X_0\wedge\ldots\wedge X_p \rangle
      &= \sum_i(-1)^i j_{X_i}\langle \alpha, X_0\wedge\ldots\wedge\widehat{X_i}\wedge\ldots\wedge X_p \rangle
    \\&\quad+ \sum_{i<j})(-1)^{i+j} \langle \alpha, [X_i,X_j]\wedge X_0\wedge\ldots\wedge\widehat{X_i}\wedge\ldots\wedge\widehat{X_j}\wedge\ldots\wedge X_p \rangle.
    \end{aligned}
  \]
  From this formula, and from \hyperref[I.2.10.2]{(2.10.2)}, we see that, for any local section $\nu$ of $\Omega_X^p(\cal{V})$, and holomorphic vector fields $X_0,\ldots,X_p$,
  \[
  \label{I.2.13.1}
    \begin{aligned}
      \langle \nabla\nu, X_0\wedge\ldots\wedge X_p \rangle
      &= \sum_i(-1)^i \nabla_{X_i}\langle \nu, X_0\wedge\ldots\wedge\widehat{X_i}\wedge\ldots\wedge X_p \rangle
    \\&\quad+ \sum_{i<j})(-1)^{i+j} \langle \nu, [X_i,X_j]\wedge X_0\wedge\ldots\wedge\widehat{X_i}\wedge\ldots\wedge\widehat{X_j}\wedge\ldots\wedge X_p \rangle.
    \end{aligned}
  \tag{2.13.1}
  \]

  In particular, for any local section $v$ of $\cal{V}$, we have that
  \[
    \langle \nabla\nabla v, X_1\wedge X_2 \rangle
    = \nabla_{X_1}\langle \nabla v, X_2 \rangle - \nabla_{X_2}\langle v, X_1 \rangle - \langle \nabla v, [X_1,X_2] \rangle.
  \]
  That is,
  \[
  \label{I.2.13.2}
    R(X_1,X_2)(v) = \nabla_{X_1}\nabla_{X_2}v - \nabla_{X_2}\nabla_{X_1}v - \nabla_{[X_1,X_2]}v.
  \tag{2.13.2}
  \]
\end{rmenv}

\begin{rmenv}{Definition 2.14}
\label{I.2.14}
  A connection is said to be \emph{integrable} if its curvature is zero, i.e. \hyperref[I.2.13.2]{(2.13.2)} if the following holds identically:
  \[
    \nabla_{[X,Y]} = [\nabla_X,\nabla_Y].
  \]
\end{rmenv}
If $\dim(X)\leq1$, then every connection is integrable.

If $\Gamma$ is an integrable connection on $\cal{V}$, then the morphism $\nabla$ of \hyperref[I.2.10.1]{(2.10.1)} satisfy $\nabla\nabla=0$, and so the $\Omega^p(\cal{V})$ give a differential complex $\Omega^\bullet(\cal{V})$.

\begin{rmenv}{Definition 2.15}
\label{I.2.15}
  Under the above hypotheses, the complex $\Omega^\bullet(\cal{V})$ is called the \emph{holomorphic de Rham complex} with values in $\cal{V}$.
\end{rmenv}

\oldpage{12}
The results \hyperref[I.2.16]{(2.16)} to \hyperref[I.2.19]{(2.19)} that follow will be proven in a more general setting in \hyperref[I.2.23]{(2.23)}.

\begin{itenv}{Proposition 2.16}
\label{I.2.16}
  Let $V$ be a local complex system on a complex-analytic variety $X$ \hyperref[0.6]{(0.6)}, and let $\cal{V}=\cal{O}\otimes_\CC V$.
  \begin{enumerate}[(i)]
    \item There exists, on the vector bundle $\cal{V}$, exactly one connection (said to be \emph{canonical}) whose horizontal sections are the local sections of the subsheaf $V$ of $\cal{V}$.
    \item The canonical connection on $\cal{V}$ is integrable.
    \item For any local section $f$ (resp. $v$) of $\cal{O}$ (resp. $V$),
      \[
      \label{I.2.16.1}
        \nabla(fv) = \dd f\cdot v.
      \tag{2.16.1}
      \]
  \end{enumerate}
\end{itenv}

\begin{proof}
  If $\nabla$ satisfies (i), then \hyperref[I.2.16.1]{(2.16.1)} is a particular case of \hyperref[I.2.3.5]{(2.3.5)}.
  Conversely, the right-hand side $\II(f,v)$ of \hyperref[I.2.16.1]{(2.16.1)} is $\CC$-bilinear, and thus extends uniquely to a $\CC$-linear map $\nabla\colon\cal{V}\to\Omega^1(\cal{V})$, which we can show defines a connection.
  Claim~(ii) is local on $X$, which allows us to reduce to the case where $V=\underline{\CC}$.
  Then $\cal{V}=\cal{O}$, $\nabla=\dd$, and $\nabla_{[X,Y]}=[\nabla_X,\nabla_Y]$ by the definition of $[X,Y]$.
\end{proof}

It is well known that:
\begin{itenv}{Theorem 2.17}
\label{I.2.17}
  Let $X$ be a complex-analytic variety.
  Then the following functors are quasi-inverse to one another, and thus give an equivalence between the category of complex local systems on $X$ and the category of holomorphic vector bundles with on $X$ with integrable connections (with the morphisms being the horizontal morphisms of vector bundles):
  \begin{enumerate}[a)]
    \item the complex local system $V$ is sent to $\cal{V}=\cal{O}\otimes V$ endowed with its canonical connection;
    \item the holomorphic vector bundle $\cal{V}$ endowed with its integrable connection is sent to the subsheaf $V$ of $\cal{V}$ consisting of horizontal sections (i.e. those $v$ such that $\nabla v=0$).
  \end{enumerate}
\end{itenv}

These equivalences are compatible with taking the tensor product, the internal $\Hom$, and the dual;
to the unit complex local system $\underline{\CC}$ corresponds the bundle $\cal{O}$ endowed with the connection $\nabla$ such that $\nabla f=\dd f$.

Definition~\hyperref[I.2.10.2]{(2.10.2)} implies the following:
\begin{itenv}{Proposition 2.18}
\label{I.2.18}
  If $V$ is a complex local system on $X$, and if $\cal{V}=\cal{O}\otimes_\CC V$,
\oldpage{13}
  then the system of isomorphisms
  \[
    \Omega_X^p\otimes_\CC V
    \simeq
    \Omega_X^p\otimes_\cal{O}\cal{O}\otimes_\CC V
    \simeq
    \Omega_X^p\otimes_\cal{O}\cal{V}
  \]
  is an isomorphism of complexes
  \[
    \Omega_X^\bullet\otimes_\CC V \to \Omega_X^\bullet(\cal{V}).
  \]
\end{itenv}
From this, the holomorphic Poincar\'{e} lemma gives the following:
\begin{itenv}{Proposition 2.19}
\label{I.2.19}
  Under the hypotheses of \hyperref[I.2.16]{(2.16)}, the complex $\Omega_X^\bullet(\cal{V})$ is a resolution of the sheaf $\cal{V}$.
\end{itenv}

\begin{rmenv}{2.20}
\label{I.2.20}
  \textbf{Variants.}
  
  \begin{rmenv}{2.20.1}
  \label{I.2.20.1}
    If $X$ is a differentiable manifold, and we consider $C^\infty$ connections on $C^\infty$ vector bundles, then all of the above results still hold true, mutatis mutandis.
    We will not use this fact.
  \end{rmenv}

  \begin{rmenv}{2.20.2}
  \label{I.2.20.2}
    Theorem~\hyperref[I.2.17]{(2.17)} makes essential use of the non-singularity of $X$;
    it is thus unimportant to note that this hypothesis has not been used in an essential way before \hyperref[I.2.17]{(2.17)}
  \end{rmenv}

  \begin{rmenv}{2.20.3}
  \label{I.2.20.3}
    The definition \hyperref[I.2.4]{(2.4)} of a connection and the definition \hyperref[I.2.11]{(2.11)} of an integrable connection are formal enough that we can transport them to the category of schemes, or in relative settings:
  \end{rmenv}
\end{rmenv}

\begin{rmenv}{Definition 2.21}
\label{I.2.21}
  \begin{enumerate}[(i)]
    \item Let $f\colon X\to S$ be a smooth morphism of schemes, and $\cal{V}$ a quasi-coherent sheaf on $X$.
      A \emph{relative connection} on $\cal{V}$ is an $f^*\cal{O}_S$-linear sheaf morphism
      \[
        \nabla\colon \cal{V} \to \Omega_{X/S}^1(\cal{V})
      \]
      (called the \emph{covariant derivative} defined by the connection) that identically satisfies, for any local section $f$ (resp. $v$) of $\cal{O}_X$ (resp. $\cal{V}$),
      \[
        \nabla(fv) = f\cdot\nabla v + \dd f\cdot v.
      \]
    \item Given $\cal{V}$ endowed with a relative connection, there exists exactly one system of $f^*\cal{O}_S$-homomorphisms of sheaves
      \[
        \nabla^{(p)}\colon \Omega_{X/S}^p(\cal{V}) \to \Omega_{X/S}^{p+1}(\cal{V})
      \]
      that satisfies \hyperref[I.2.10.4]{(2.10.4)} and is such that $\nabla^{(0)}=\nabla$.
\oldpage{14}
    \item The \emph{curvature} of a connection is defined by
      \[
        R = \nabla^{(1)}\nabla^{(0)} \in \shHom(\cal{V},\Omega_{X/S}^2(\cal{V})) \cong \Omega_{X/S}^2(\shEnd(\cal{V})).
      \]
      The curvature satisfies the Ricci identity \hyperref[I.2.12.2]{(2.12.2)} and the Bianchi identity \hyperref[I.2.12.3]{(2.12.3)}.
    \item An \emph{integrable connection} is a connection with zero curvature.
    \item The \emph{de Rham complex} defined by an integrable connection is the complex $(\Omega_{X/S}^p(\cal{V}),\nabla)$.
  \end{enumerate}
\end{rmenv}

\begin{rmenv}{2.22}
\label{I.2.22}
  Let $f\colon X\to S$ be a \emph{smooth} morphism of complex-analytic spaces;
  by hypothesis, $f$ is thus locally (in the domain) isomorphic to a projection $\pr_2\colon\CC^n\times S\to S$ (for some $n\in\NN$).
  A \emph{local relative system} on $X$ is a sheaf of $f^*\cal{O}_S$-modules that is locally isomorphic to the sheaf-theoretic inverse image of a coherent analytic sheaf on $S$.
  If $\cal{V}$ is a coherent analytic sheaf on $X$, then a \emph{relative connection} on $\cal{V}$ is an $f^*\cal{O}_S$-linear homomorphism
  \[
    \nabla\colon \cal{V} \to \Omega_{X/S}^1(\cal{V})
  \]
  that identically satisfies, for any local section $f$ (resp. $v$) of $\cal{O}$ (resp. $\cal{V}$),
  \[
    \nabla(fv) = f\cdot\nabla v + \dd f\cdot v.
  \]
  A \emph{morphism} between vector bundles endowed with relative connections is a morphism of vector bundles that commutes with $\nabla$.
  We define, as in \hyperref[I.2.11]{(2.11)} and \hyperref[I.2.21]{(2.21)}, the \emph{curvature} $R\in\Omega_{X/S}^2(\shEnd(\cal{V}))$ of a relative connection.
  A relative connection is said to be \emph{integrable} if $R=0$, in which case we have the \emph{relative de Rham complex with values in $\cal{V}$}, denoted by $\Omega_{X/S}^\bullet(\cal{V})$, and defined as in \hyperref[I.2.15]{(2.15)} and \hyperref[I.2.21]{(2.21)}.
\end{rmenv}

The ``absolute'' statements \hyperref[I.2.17]{(2.17)}, \hyperref[I.2.18]{(2.18)}, and \hyperref[I.2.19]{(2.19)} have ``relative'' (i.e. ``with parameters'') analogues:

\begin{itenv}{Theorem 2.23}
\label{I.2.23}
  Under the hypotheses of \hyperref[I.2.22]{(2.22)}, we have the following.
  \begin{enumerate}[(i)]
    \item For every relative local system $V$ on $X$, there exists a coherent analytic sheaf \mbox{$\cal{V}=\cal{O}_X\otimes_{f^*\cal{O}_S}V$}, and exactly one relative connection, said to be canonical, such that a local section $v$ of $\cal{V}$ is horizontal (i.e. such that $\nabla v=0$) if and only if $v$ is a section of $V$;
      this connection is integrable.
    \item Given a relative local system $V$ on $X$, the de Rham complex defined
\oldpage{15}
      by $\cal{V}=\cal{O}_X\otimes_{f^*\cal{O}_S}V$, endowed with its canonical connection, is a resolution of the sheaf $V$.
    \item The following functors are quasi-inverse to one another, and thus give an equivalence between the category of relative local systems on $X$ and the category of coherent analytic sheaves on $X$ endowed with a relative integrable connection:
      \begin{enumerate}[a)]
        \item the relative local system $V$ is sent to $\cal{V}=\cal{O}_X\otimes_{f^*\cal{O}_S}V$ endowed with its canonical connection;
        \item the coherent analytic sheaf $\cal{V}$ on $X$ endowed with a relative integrable connection is sent to the subsheaf consisting of its horizontal sections (i.e. the sections $v$ such that $\nabla v=0$).
      \end{enumerate}
  \end{enumerate}
\end{itenv}

\begin{proof}
  We first prove (i).
  To show that $\cal{V}$ is coherent, it suffices to do so locally, for $V=f^\sbullet V_0$, in which case $\cal{V}$ is the inverse image, in the sense of coherent analytic sheaves, of $\cal{V}_0$.
  The canonical relative connection necessarily satisfies, for any local section $f$ (resp $v_0$) of $\cal{O}_X$ (resp. $V$),
  \[
  \label{I.2.23.1}
    \nabla(fv_0) = \dd f\cdot v_0.
  \tag{2.23.1}
  \]
  The right-hand side $\II(f,v_0)$ of this equation is biadditive in $f$ and $v_0$, and satisfies, for any local section $g$ of $f^*\cal{O}_S$, the identity
  \[
    \II(fg,v_0) = \II(f,gv_0)
  \]
  (using the fact that $\dd g=0$ in $\Omega_{X/S}^1$).
  We thus deduce the existence and uniqueness of a relative connection $\nabla$ that satisfies \hyperref[I.2.23.1]{(2.23.1)}.
  Finally, we have that
  \[
    \nabla\nabla(fv_0) = \nabla(\dd f\cdot v_0) = \dd\dd f\cdot v_0 = 0,
  \]
  and so the canonical connection $\nabla$ is integrable.
  The fact that only the sections of $V$ are horizontal is a particular case of (ii), which is proven below.
\end{proof}

\begin{rmenv}{2.23.2}
\label{I.2.23.2}
  We first of all consider the particular case of (ii) where $S=D^n$, $X=D^n\times D^m$, $f=\pr_2$, and the relative local system $V$ is the inverse image of $\cal{O}_S$.
  The complex of global sections
  \[
    0 \to \Gamma(f^\sbullet\cal{O}_S) \to \Gamma(\cal{O}_X) \to \Gamma(\Omega_{X/S}^1) \to \ldots
  \]
  is acyclic, since it admits the homotopy operator $H$ defined below.
\oldpage{16}
  \begin{enumerate}[a)]
    \item $H\colon \Gamma(\cal{O}_X) \to \Gamma(f^\sbullet\cal{O}_S) = \Gamma(S,\cal{O}_S)$ is the inverse image under the zero section of $f$;
    \item an element $\omega\in\Gamma(\Omega_{X/S}^p)$ (where $p>0$) can be represented in a unique way as a sum of convergent series:
      \[
        \omega = \sum_{\substack{I\subset[1,m]\\|I|=p}} \; \sum_{\underline{n}\in \NN^{m+n}} a_{\underline{n}}^I
        \left(
          \prod_{i\in I} x_i^{n_i}\dd x_i
        \right)
        \left(
          \prod_{i\in[1,m+n]\setminus I} x_i^{n_i}
        \right)
      \]
      and we set
      \[
        H(\omega) = \sum_{I\subset[1,m]} \; \sum_{j\in I} \; \sum_{\underline{n}\in\NN^{m+n}} a_{\underline{n}}^I
        \left(
          \prod_{\substack{j\in I\\i\neq j}} x_j^{n_j}\dd x_j \frac{x_j^{n_j+1}}{n_j+1}
        \right)
        \left(
          \prod_{i\in[1,m+n]\setminus I} x_i^{n_i}.
        \right)
      \]
  \end{enumerate}

  This remains true if we replace $D^{m+n}$ by a smaller polycylinder, and so $\Omega_{X/S}^\bullet$ is a resolution of $f^\sbullet\cal{O}_S$.
\end{rmenv}

\begin{proof}[\normalfont\textbf{2.23.3}]
\label{I.2.23.3}
  We now prove (ii), which is of a local nature on $X$ and $S$.
  Denoting by $D$ the open unit disc, we can thus restrict to the case where $S$ is a closed analytic subset of the polycylinder $D^n$, where $X=D^m\times S$, with $f=\pr_2$, and where $V$ is the inverse image of a coherent analytic sheaf $V_0$ on $S$.
  Applying the syzygy theorem, and possibly shrinking $X$ and $S$, we can further suppose that the direct image of $V_0$ on $D^n$, which we again denote by $V_0$, admits a finite resolution $\scr{L}$ by free coherent $\cal{O}_{D^n}$-modules.
  To prove (ii), we are allowed to replace $V_0$ by its direct image on $D^n$, and to suppose that $D^n=S$, which we now do.

  If $\Sigma_0$ is a short exact sequence of coherent $\cal{O}_S$-modules
  \[
    \Sigma_0\colon 0 \to V'_0 \to V_0 \to V''_0 \to 0,
  \]
  then let $\Sigma=f^\sbullet\Sigma_0$ be the exact sequence of relative local systems given by the inverse image of $\Sigma_0$ (which is exact since $f^\sbullet$ is an exact functor), and let $\Omega_{X/S}^\bullet(\Sigma)$ be the corresponding exact sequence of relative de Rham complexes:
  \[
    \Omega_{X/S}^\bullet(\Sigma)\colon 0 \to \Omega_{X/S}^p\otimes_{f^\sbullet\cal{O}_S}f^\sbullet V'_0 \to \Omega_{X/S}^p\otimes_{f^\sbullet\cal{O}_S}f^\sbullet V_0 \to \Omega_{X/S}^p\otimes_{f^\sbullet\cal{O}_S}f^\sbullet V''_0 \to 0.
  \]
  This sequence is exact since $\Omega_{X/S}^p$ is flat over $f^\sbullet\cal{O}_S$, since it is locally free over $\cal{O}_X$ which is itself flat over $f^\sbullet\cal{O}_S$.

  The snake lemma applied to $\Omega_{X/S}^\bullet(\Sigma)$ shows that, if claim~(ii)
\oldpage{17}
  is satisfied for any two of relative local systems $f^\sbullet V_0$, $f^\sbullet V'_0$, and $f^\sbullet V''_0$, then it is again satisfied for the third.
  We thus deduce, by induction, that, if $V_0$ admits a finite resolution $M_\bullet$ by modules that satisfy (ii), then $V_0$ satisfies (ii).
  This, applied to $V_0$ and $\scr{L}^\bullet$, finishes the proof of (i) and (ii).

  It follows from (ii) that the composite $\mbox{(iii)b}\circ\mbox{(iii)a}$ of the functors from (iii) is canonically isomorphic to the identity;
  furthermore, if $V_1$ and $V_2$ are relative local systems, and if $u\colon\cal{V}_1\to\cal{V}_2$ is a homomorphism that induces $0$ on $V_1$, then $u=0$, since $V_1$ generates $\cal{V}_1$;
  it thus follows that the functor $\mbox{(iii)a}$ is fully faithful.
  It remains to show that every vector bundle $\cal{V}$ endowed with a relative connection $\nabla$ is given locally by a relative local system.

  \begin{itemize}
    \item[] \textbf{Case 1:} \emph{$S=D^n$, $X=D^{n+1}=D^n\times D$, $f=\pr_1$, and $\cal{V}$ is free.}

      Under these hypotheses, if $v$ is an arbitrary section of the inverse image of $\cal{V}$ under the zero section $s_0$ of $f$, then there exists exactly one horizontal section $\widetilde{v}$ of $\cal{V}$ that agrees with $v$ on $s_0(S)$ (as follows from the existence and uniqueness of solutions for Cauchy problems with parameters).
      If $(e_i)_{1\leq i\leq k}$ is a basis of $s_0^*\cal{V}$, then the $\widetilde{e_i}$ form a horizontal basis of $\cal{V}$, and $(\cal{V},{\nabla})$ is defined by the relative local system $f^\sbullet s_0^*\cal{V}\simeq f^\sbullet\cal{O}_S^k$.
    \item[] \textbf{Case 2:} \emph{$S=D^n$, $X=D^{n+1}=D^n\times D$, and $f=\pr_1$.}

      By possibly shrinking $X$ and $S$, we can suppose that $\cal{V}$ admits a free presentation:
      \[
        \cal{V}_1 \xrightarrow{d} \cal{V}_0 \xrightarrow{\varepsilon} \cal{V} \to 0.
      \]
      By then possibly shrinking again, we can further suppose that $\cal{V}_0$ and $\cal{V}_1$ admit connections $\nabla_1$ and $\nabla_0$ (respectively) such that $\varepsilon$ and $d$ are compatible with the connections (if $(e_i)$ is a basis of $\cal{V}_0$, then $\nabla_0$ is determined by the $\nabla_0 e_i$, and it suffices to choose $\nabla_0 e_i$ such that $\varepsilon(\nabla_0 e_i)=\nabla(\varepsilon(e_i))$; we proceed similarly for $\nabla_1$).
      The connections $\nabla_0$ and $\nabla_1$ are automatically integrable, since $f$ is of relative dimension~$1$.
      There thus exist (by Case~1) relative local systems $V_0$ and $V_1$ such that $(\cal{V}_i,\nabla_i)\simeq\cal{O}_X\otimes_{f^\sbullet\cal{O}_S}V_i$.
      We then have that
      \[
        (\cal{V},\nabla) \simeq \cal{O}_X\otimes_{f^\sbullet\cal{O}_S}(V_0/dV_1).
      \]
\oldpage{18}
    \item[] \textbf{Case 3:} \emph{$f$ is of relative dimension~$1$.}

      We can suppose that $S$ is a closed analytic subset of $D^n$, and that $X=S\times D$ and $f=\pr_1$.
      The relative local systems (resp. the modules with relative connections) on $X$ can then be identified with the local relative systems (resp. the modules with relative connections) on $D^n\times D$ that are annihilated by the inverse image of the ideal that defines $S$, and we conclude by using Case~2.
    \item[] \textbf{General case.}
      We proceed by induction on the relative dimension $n$ of $f$.
      The case $n=0$ is trivial.
      If $n\neq0$, then we are led to the case where $X=S\times D^{n-1}\times D$ and $f=\pr_1$.
      The bundle with connection $(\cal{V},\nabla)$ induces a bundle $\cal{V}_0$ with connection on $X_0=S\times D^{n-1}\times\{0\}$ which is, by induction, of the form $(\cal{V}_0,\nabla_0) = \cal{O}_{X_0}\otimes_{\pr_1^\sbullet\cal{O}_S}V$.
      The projection $f$ from $X$ to $S\times D^{n-1}$ is of relative dimension~$1$, and the relative connection $\nabla$ induces a relative connection for $\cal{V}$ on $X/S\times D^{n-1}$.
      By Case~3, there exists a vector bundle $V_1$ on $S\times D^{n-1}$, as well as an isomorphism
      \[
        \cal{V} \simeq \cal{O}_X\otimes_{p^\sbullet\cal{O}_{S\times D^{n-1}}}p^\sbullet V_1.
      \]
      of bundles with relative connections (with respect to $p$).

      The vector bundle $V_1$ can be identified with the restriction of $\cal{V}$ to $X_0$, whence we obtain an isomorphism
      \[
        \alpha\colon \cal{V}\simeq\cal{O}_X\otimes_{f^\sbullet\cal{O}_S}V
      \]
      of vector bundles, such that
      \begin{enumerate}[(i)]
        \item the restriction of $\alpha$ to $X_0$ is horizontal ; and
        \item $\alpha$ is ``relatively horizontal'' with respect to $p$.
      \end{enumerate}

      If $v$ is a section of $V$, then condition~(ii) implies that
      \[
        \nabla_{x_n}v = 0.
      \]

      If $1\leq i<n$, since $R=0$, then, by an analogous statement to \hyperref[I.2.13.2]{(2.13.2)}, we have that
      \[
        \nabla_{x_n}\nabla_{x_i}v = \nabla_{x_i}\nabla_{x_n}v = 0.
      \]

      In other words, $\nabla_{x_i}v$ is a relative horizontal section, with respect to $p$, of $\cal{V}$;
      by (i), it is zero on $X_0$, and is thus zero, and we conclude that $\nabla v=0$.
      The isomorphism $\alpha$ is thus horizontal, and this finishes the proof
\oldpage{19}
      of \hyperref[I.2.23]{(2.23)}.
  \end{itemize}
\end{proof}

Some results in general topology (\hyperref[I.2.24]{(2.24)} to \hyperref[I.2.27]{(2.27)}) will be necessary to deduce \hyperref[I.2.28]{(2.28)} from \hyperref[I.2.23]{(2.23)}.

\begin{rmenv}{Reminder 2.24}
\label{I.2.24}
  Let $Y$ be a closed subset of a topological space $X$, and suppose that $Y$ has a paracompact neighbourhood.
  For every sheaf $\scr{F}$ on $X$, we have that
  \[
    \varinjlim_{U\supset Y} \HH^\bullet(U,\scr{F}) \simto \HH^\bullet(Y,\scr{F}).
  \]
\end{rmenv}

\begin{proof}
  See Godement~\cite[II, 4.11.1, p.~193]{7}.
\end{proof}

\begin{itenv}{Corollary 2.25}
\label{I.2.25}
  Let $f\colon X\to S$ be a proper separated morphism between topological spaces.
  Suppose that $S$ is locally paracompact \hyperref[0.5]{(0.5)}.
  Then, for every $s\in S$, and for every sheaf $\scr{F}$ on $X$, we have that
  \[
    (\RR^i f_*\scr{F})_s \simeq \HH^i(f^{-1}(s), \scr{F}|f^{-1}(s)).
  \]
\end{itenv}

\begin{proof}
  Since $f$ is closed, the $f^{-1}(U)$ form a fundamental system of neighbourhoods of $f^{-1}(s)$, where the $U$ are neighbourhoods of $s$.
  Furthermore, if $U$ is paracompact, then $f^{-1}(U)$ is paracompact, since $f$ is proper and separated.
  We conclude by \hyperref[I.2.24]{(2.24)}.
\end{proof}

\begin{rmenv}{Reminder 2.26}
\label{I.2.26}
  Let $X$ be a contractible locally paracompact topological space, $i$ an integer, and $V$ a complex local system on $X$, such that $\dim_\CC\HH^i(X,V)<\infty$.
  Then, for every vector space $A$ over $\CC$, possibly of infinite dimension, we have that
  \[
  \label{I.2.26.1}
    A\otimes_\CC\HH^i(X,V) \simto \HH^i(X,A\otimes_\CC V).
  \tag{2.26.1}
  \]
\end{rmenv}

\begin{proof}
  We denote by $\HH_\bullet(X,V^*)$ the singular homology of $X$ with coefficients in $V^*$.
  The universal coefficient formula, which holds here, gives
  \[
  \label{I.2.26.2}
    \HH^i(X,A\otimes V) \simeq \Hom_\CC(\HH_i(X,V^*),A).
  \tag{2.26.2}
  \]
  For $A=\CC$, we thus conclude that $\dim\HH_i(X,V^*)<\infty$.
  Equation~\hyperref[I.2.26.1]{(2.26.1)} then follows from \hyperref[I.2.26.2]{(2.26.2)}.
\end{proof}

\begin{rmenv}{2.27}
\label{I.2.27}
  Let $f\colon X\to S$ be a smooth morphism of complex-analytic spaces, and let $V$ be a local system on $X$.
  Then the sheaf
\oldpage{20}
  \[
  \label{I.2.27.1}
    V_\mathrm{rel} = f^\sbullet\cal{O}_S \otimes_\CC V
  \tag{2.27.1}
  \]
  is a relative local system.
  We denote by $\Omega_{X/S}^\bullet(V)$ the corresponding de Rham complex.
  By \hyperref[I.2.23]{(2.23)}, $\Omega_{X/S}^\bullet$ is a resolution of $V_\mathrm{rel}$.
  We thus have that
  \[
  \label{I.2.27.2}
    \RR^i f_* V_\mathrm{rel} \simto \RR^i f_*(\Omega_{X/S}^\bullet(V))
  \tag{2.27.2}
  \]
  where the right-hand side is the relative hypercohomology.
  From \hyperref[I.2.27.1]{(2.27.1)}, we thus obtain an arrow
  \[
  \label{I.2.27.3}
    \cal{O}_S \otimes_\CC \RR^i f_*V \to \RR^i f_*(V_\mathrm{rel}),
  \tag{2.27.3}
  \]
  whence, by composition, an arrow
  \[
  \label{I.2.27.4}
    \cal{O}_S \otimes \RR^i f_*V \to \RR^i f_*(\Omega_{X/S}(V)).
  \tag{2.27.4}
  \]
\end{rmenv}

\begin{itenv}{Proposition 2.28}
\label{I.2.28}
  Let $f\colon X\to S$ be a smooth separated morphism of analytic spaces, $i$ an integer, and $V$ a complex local system on $X$.
  We suppose that
  \begin{enumerate}[a)]
    \item $f$ is topologically trivial locally on $S$ ; and
    \item the fibres of $f$ satisfy
      \[
        \dim\HH^i(f^{-1}(s),V) < \infty.
      \]
      Then the arrow \hyperref[I.2.27.4]{(2.27.4)} is an isomorphism:
      \[
        \cal{O}_S \otimes_\CC \RR^i f_*V \simto \RR^i f_*(\Omega_{X/S}^\bullet(V)).
      \]
  \end{enumerate}
\end{itenv}

\begin{proof}
  Let $s\in S$, $Y=f^{-1}(s)$, and $V_0=V|Y$.
  To show that \hyperref[I.2.27.4]{(2.27.4)} is an isomorphism, it suffices to construct a fundamental system $T$ of neighbourhoods of $s$ such that the arrows
  \[
  \label{I.2.28.1}
    \HH^0(T,\cal{O}_S) \otimes \HH^i(T\times Y,\pr_2^\sbullet V_0) \simto \HH^i(T\times Y,\pr_1^\sbullet\cal{O}_S \otimes \pr_2^\sbullet V_0)
  \tag{2.28.1}
  \]
  are isomorphisms.
  In fact, the fibre at $s$ of \hyperref[I.2.27.3]{(2.27.3)}, which is the inductive limit of the arrows \hyperref[I.2.28.1]{(2.28.1)}, will then be an isomorphism.

  We will prove \hyperref[I.2.28.1]{(2.28.1)} for a compact Stein neighbourhood $T$ of $s$, assumed to be contractible.
  The arrow in \hyperref[I.2.28.1]{(2.28.1)} can then be written as
\oldpage{21}
  \[
  \label{I.2.28.2}
    \HH^0(T,\cal{O}_S) \otimes \HH^i(Y,V_0) \simto \HH^i(T\times Y,\pr_1^\sbullet\cal{O}_S \otimes \pr_2^\sbullet V_0).
  \tag{2.28.2}
  \]

  We can calculate the right-hand side of \hyperref[I.2.28.2]{(2.28.2)} by using the Leray spectral sequence for $\pr_2\colon T\times Y\to Y$.
  By \hyperref[I.2.25]{(2.25)}, since $\HH^i(T,\cal{O}_S)=0$, we have that
  \[
    \HH^i(T\times Y,\pr_1^\sbullet\cal{O}_S \otimes \pr_2^\sbullet V_0) \simeq \HH^i(Y,\HH^0(T,\cal{O}_S) \otimes V_0),
  \]
  and we conclude by \hyperref[I.2.26]{(2.26)}.
\end{proof}

\begin{rmenv}{2.29}
\label{I.2.29}
  Under the hypotheses of \hyperref[I.2.28]{(2.28)}, with $S$ smooth, we define the \emph{Gauss--Manin connection} on $\RR^i f_*\Omega_{X/S}^\bullet(V)$ as being the unique integrable connection that admits the local sections of $\RR^i f_*V$ as its horizontal sections \hyperref[I.2.17]{(2.17)}.
\end{rmenv}



\section{Translation in terms of first-order partial differential equations}
\label{I.3}

\begin{rmenv}{3.1}
\label{I.3.1}
  Let $X$ be a complex-analytic variety.
  If $\cal{V}$ is the holomorphic vector bundle defined by a $\CC$-vector space $V_0$, then we have seen that $\cal{V}$ admits a canonical connection with covariant derivative ${}_0\!\nabla$.
  If $\nabla$ is the covariant derivative defined by another connection on $\cal{V}$, then we have seen \hyperref[I.2.6]{(2.6)} that $\nabla$ can be written in the form
  \[
    \nabla = {}_0\!\nabla + \Gamma,
    \quad\mbox{where $\Gamma\in\Omega(\shEnd(\cal{V}))$.}
  \]

  If we identify sections of $\cal{V}$ with holomorphic maps from $X$ to $V_0$, then we have that
  \[
  \label{I.3.1.1}
    \nabla v = \dd v + \Gamma\cdot v
  \tag{3.1.1}
  \]

  If we suppose that we have chosen a basis of $V$, i.e. an isomorphism $e\colon\CC^n\to V_0$ with coordinates (identified with basis vectors) $e_\alpha\colon\CC\to V_0$, then $\Gamma$ can be written as a matrix $\omega_\beta^\alpha$ of differential forms (the \emph{matrix of forms of the connection}), and \hyperref[I.3.1.1]{(3.1.1)} can then be written as
  \[
  \label{I.3.1.2}
    (\nabla v)^\alpha = \dd v^\alpha + \sum_\beta \omega_\beta^\alpha v^\beta.
  \tag{3.1.2}
  \]

  Let $\cal{V}$ be an arbitrary holomorphic vector bundle on $X$.
  The choice of a basis $e\colon\CC^n\simto\cal{V}$ of $\cal{V}$ allows us to think of $\cal{V}$ as being defined by a
\oldpage{22}
  constant vector \todo ($\CC^n$), and the above remarks apply:
  connections on $\cal{V}$ correspond, via \hyperref[I.3.1.2]{(3.1.2)}, with $(n\times n)$-matrices of differential forms on $X$.
  If $\omega_e$ is the matrix of the connection $\nabla$ in the basis $e$, and if $f\colon\CC^n\to\simto\cal{V}$ is another basis of $\cal{V}$, with matrix $A\in\GL_n(\cal{O})$ (where $A=e^{-1}f$), then \hyperref[I.3.1.2]{(3.1.2)}
  \[
    \begin{aligned}
      \nabla v
      &= e\dd(e^{-1}v) + e\omega_e e^{-1}v
    \\&= fA^{-1}\dd(Af^{-1}v) + fA^{-1}\omega_e Af^{-1}v
    \\&= f\dd f^{-1}v + f(A^{-1}\dd A + A^{-1}\omega_e A)f^{-1}v.
    \end{aligned}
  \]
  Comparing this with \hyperref[I.3.1.2]{(3.1.2)} in the basis $f$, we find that
  \[
  \label{I.3.1.3}
    \omega_f = A^{-1}\dd A + A^{-1}\omega_e A.
  \tag{3.1.3}
  \]

  If, further, $(x^i)$ is a system of local coordinates on $X$, which defines a basis $(\dd x^i)$ of $\Omega_X^1$, we set
  \[
    \omega_\beta^\alpha = \sum_i \Gamma_{\beta i}^\alpha \dd x^i
  \]
  and we call the holomorphic functions $\Gamma_{\beta i}^\alpha$ the \emph{coefficients of the connection}.
  Equation~\hyperref[I.3.1.2]{(3.1.2)} can the be written as
  \[
  \label{I.3.1.4}
    (\nabla_i v)^\alpha = \partial_i v^\alpha + \sum_\beta \Gamma_{\beta i}^\alpha v^\beta.
  \tag{3.1.4}
  \]

  The differential equation $\nabla v=0$ of horizontal sections of $\cal{V}$ can be written as the linear homogeneous system of first-order partial differential equations
  \[
  \label{I.3.1.5}
    \partial_i v^\alpha = -\sum_\beta \Gamma_{\beta i}^\alpha v^\beta.
  \tag{3.1.5}
  \]
\end{rmenv}

\begin{rmenv}{3.2}
\label{I.3.2}
  With the notation of \hyperref[I.3.1.2]{(3.1.2)}, and using Einstein summation notation, we have that
  \[
    \begin{aligned}
      \nabla\nabla v
      &= \nabla((\dd v^\alpha + \omega_\beta^\alpha v^\beta)e_\alpha)
    \\&= \dd(\dd v^\alpha + \omega_\beta^\alpha v^\beta)e_\alpha - (\dd v^\alpha + \omega_\beta^\alpha v^\beta)\wedge\omega_\alpha^\gamma\cdot e_\gamma
    \\&= \dd\omega_\beta^\alpha\cdot v^\beta\cdot e_\alpha - \omega_\beta^\alpha\wedge\dd v^\beta\cdot e_\alpha - \dd v^\alpha\wedge\omega_\alpha^\gamma\cdot e_\gamma - \omega_\beta^\alpha\wedge\omega_\alpha^\gamma\cdot v^\beta e_\gamma
    \\&= (\dd\omega_\beta^\gamma - \omega_\beta^\alpha\wedge\omega_\alpha^\gamma)v^\beta e_\gamma.
    \end{aligned}
  \]

  The curvature tensor matrix is thus
\oldpage{23}
  \[
  \label{I.3.2.1}
    R_\beta^\alpha = \dd\omega_\beta^\alpha + \sum_\gamma \omega_\gamma^\alpha\wedge\omega_\beta^\gamma,
  \tag{3.2.1}
  \]
  which we can also write as
  \[
  \label{I.3.2.2}
    R = \dd\omega + \omega\wedge\omega.
  \tag{3.2.2}
  \]
  Equation~\hyperref[I.3.2.1]{(3.2.1)} gives, in a system $(x^i)$ of local coordinates,
  \[
  \label{I.3.2.3}
    \begin{cases}
      R_{\beta i j}^\alpha
      &= (\partial_i\Gamma_{\beta j}^\alpha - \partial_j\Gamma_{\beta i}^\alpha) + (\Gamma_{\gamma i}^\alpha\Gamma_{\beta j}^\gamma - \Gamma_{\gamma j}^\gamma\Gamma_{\beta i}^\gamma)
    \\R_\beta^\alpha
      &= \sum_{i<j} R_{\beta i j}^\alpha \dd x^i\wedge\dd x^j.
    \end{cases}
  \tag{3.2.3}
  \]

  The condition $R_{\beta i j}^\alpha = 0$ is the integrability condition of the system \hyperref[I.3.1.5]{(3.1.5)}, in the classical sense of the word;
  it can be obtained by eliminating $v^\alpha$ from the equations given by substituting \hyperref[I.3.1.5]{(3.1.5)} into the identity $\partial_i\partial_j v^\alpha = \partial_j\partial_i v^\alpha$.
\end{rmenv}



\section{\texorpdfstring{$n^\mathrm{th}$}{nth}-order differential equations}
\label{I.4}

\begin{rmenv}{4.1}
\label{I.4.1}
  The solution of a linear homogeneous $n^\mathrm{th}$-order differential equation
  \[
  \label{I.4.1.1}
    \frac{\dd^n}{\dd x^n}y = \sum_{i=1}^n a_i(x) \frac{\dd^{n-i}}{\dd x^i}y
  \tag{4.1.1}
  \]
  is equivalent to that of the system
  \[
  \label{I.4.1.2}
    \begin{cases}
      \frac{\dd}{\dd x}y_i
      = y_{i+1} &\mbox{(for $1\leq i<n$);}
    \\\frac{\dd}{\dd x}y_n
      = \sum_{i=1}^n a_i(x) y_{n+1-i}
    \end{cases}
  \tag{4.1.2}
  \]
  of $n$ first-order equations.

  By \hyperref[I.3]{(3)}, this system can be described as the differential equation of horizontal sections of a rank-$n$ vector bundle endowed with a suitable connection, and this is what we aim to further explore.
\end{rmenv}

\begin{rmenv}{4.2}
\label{I.4.2}
  Let $X$ be a non-singular complex-analytic variety of pure dimension~$1$.
  Let $X_n$ be the $n^\mathrm{th}$ infinitesimal neighbourhood of the diagonal of $X\times X$, and $p_1$ and $p_2$ the two projections from $X_n$ to $X$.
  We denote by $\pi_{k,l}$ the injection from $X_l$ to
\oldpage{24}
  $X_k$, for $l\leq k$.

  Let $\Omega^{\otimes n}$ be the $n^\mathrm{th}$ tensor power of the invertible sheaf $\Omega_X^1$ (for $n\in\ZZ$).
  If $I$ is the ideal that defines the diagonal of $X\times X$, then $I/I^2\simeq\Omega_X^1$ canonically, and
  \[
  \label{I.4.2.1}
    I^n/I^{n+1} \simeq \Omega^{\otimes n}.
  \tag{4.2.1}
  \]

  If $\scr{L}$ is an invertible sheaf on $X$, then we denote by $P^n(\scr{L})$ the vector bundle
  \[
  \label{I.4.2.2}
    P^n(\scr{L}) = (p_1)_*p_2^*\scr{L}
  \tag{4.2.2}
  \]
  of $n^\mathrm{th}$-order jets of sections of $\scr{L}$.
  The $I$-adic filtration of $p_2^*\scr{L}$ defines a filtration of $P^n(\scr{L})$ for which
  \[
  \label{I.4.2.3}
    \begin{aligned}
      \Gr P^n(\scr{L}) &\simeq \Gr P^n(\cal{O})\otimes\scr{L}
    \\\Gr^i P^n(\scr{L}) &\simeq \Omega^{\otimes i}\otimes\scr{L} \qquad\mbox{(for $0\leq i\leq n$).}
    \end{aligned}
  \tag{4.2.3}
  \]

  Recall that we define, by induction on $n$, a \emph{differential operator of order $\leq n$} as being a morphism $A\colon\scr{M}\to\scr{N}$ of abelian sheaves such that
  \[
    \begin{cases}
      \mbox{$A$ is $\cal{O}$-linear} & \mbox{for $n=0$ ;}
    \\\mbox{$[A,f]$ is of order $\leq m$ for every local section $f$ of $\cal{O}$} & \mbox{for $n=m+1$.}
    \end{cases}
  \]

  For every local section $s$ of $\scr{L}$, $p_2^*s$ defines a local section $\DD^n(s)$ of $P^n(\scr{L})$ \hyperref[I.4.2.2]{(4.2.2)}.
  The $\CC$-linear sheaf morphism $\DD^n\colon\scr{L}\to P^n(\scr{L})$ is the universal differential operator of order $\leq n$ with domain $\scr{L}$.
\end{rmenv}

\begin{rmenv}{Definition 4.3}
\label{I.4.3}
  \begin{enumerate}[(i)]
    \item A \emph{linear homogeneous $n^\mathrm{th}$-order differential equation} on $\scr{L}$ is an $\cal{O}_X$-homomorphism $E\colon P^n(\scr{L})\to\Omega^{\otimes n}\otimes\scr{L}$ that induces the identity on the submodule $\Omega^{\otimes n}\otimes\scr{L}$ of $P^n(\scr{L})$.
    \item A local section $s$ of $\scr{L}$ is a \emph{solution} of the differential equation $E$ if $E(\DD^n(s))=0$.
  \end{enumerate}
\end{rmenv}

In fact, I have cheated with this definition, in that I have only considered equations that can be put in the ``resolved'' form \hyperref[I.4.1.1]{(4.1.1)}.

\oldpage{25}
\begin{rmenv}{4.4}
\label{I.4.4}
  Suppose that $\scr{L}=\cal{O}$, and let $x$ be a local coordinate on $X$.
  The choice of $x$ allows us to identify $p^k(\cal{O})$ with $\cal{O}^{[0,k]}$, with the arrow $\DD^k$ becoming
  \[
    \begin{aligned}
      \DD^k\colon \cal{O} &\to P^k(\cal{O}) \simeq \cal{O}^{[0,k]}
    \\f &\mapsto (\partial_x^i f)_{0\leq i\leq k}.
    \end{aligned}
  \]
  The choice of $x$ also allows us to identify $\Omega^1$ with $\cal{O}$, so that the $n^\mathrm{th}$-order differential equation can be identified with a morphism $E\in\Hom(\cal{O}^{[0,n]},\cal{O})$, and, as such, has coordinates $(b_i)_{0\leq i\leq n}$ with $b_n=1$.
  The solutions of $E$ are then exactly the (holomorphic) functions $f$ that satisfy
  \[
  \label{I.4.4.1}
    \sum_{i=0}^n b_i(x) \partial_x^i f = 0
    \qquad\mbox{(with $b_n=1$).}
  \tag{4.4.1}
  \]

  The existence and uniqueness theorem for solutions of the Cauchy problem in \hyperref[I.4.4.1]{(4.4.1)} implies the following.
\end{rmenv}

\begin{itenv}{Theorem 4.5}
\label{I.4.5}
  \emph{(Cauchy).}
  Let $X$ and $\scr{L}$ be as in \hyperref[I.4.2]{(4.2)}, and let $E$ be an $n^\mathrm{th}$-order differential equation on $\scr{L}$.
  Then
  \begin{enumerate}[(i)]
    \item the subsheaf of $\scr{L}$ given by solutions of $E$ is a local system $\scr{L}^E$ of rank~$n$ on $X$ ; and
    \item the canonical arrow $\DD^{n-1}\colon\scr{L}^E \to P^{n-1}(\scr{L})$ induces an isomorphism
      \[
        \cal{O}\otimes_\CC\scr{L}^E \simto P^{n-1}(\scr{L}).
      \]
  \end{enumerate}
\end{itenv}

In particular, it follows from \hyperref[I.4.5]{(4.5.ii)} and from \hyperref[I.2.17]{(2.17)} that $E$ defines a canonical connection on $P^{n-1}(\scr{L})$, whose horizontal sections are the images under $\DD$ of solutions of $E$.

\begin{rmenv}{4.6}
\label{I.4.6}
  To a differential equation $E$ on $\scr{L}$, we have thus associated
  \begin{enumerate}[a)]
    \item a holomorphic vector bundle $\cal{V}$ endowed with a connection (which is automatically integrable) \todo ; and
    \item a surjective homomorphism $\lambda\colon\cal{V}\to\scr{L}$ (by $i=0$ in \hyperref[I.4.2.3]{(4.2.3)}).
  \end{enumerate}
  Furthermore, the solutions of $E$ are exactly the images under $\lambda$ of the horizontal sections of $\cal{V}$.
  This is just another way of expressing how to obtain \hyperref[I.4.1.2]{(4.1.2)} from \hyperref[I.4.1.1]{(4.1.1)}.
\end{rmenv}

\oldpage{26}
\begin{rmenv}{4.7}
\label{I.4.7}
  Let $\cal{V}$ be a rank-$n$ vector bundle on $X$, endowed with a connection with covariant derivative $\nabla$.
  Let $v$ be a local section of $\cal{V}$, and $w$ a vector field on $X$ that doesn't vanish at any point.
  We say that $v$ is \emph{cyclic} if the local sections $(\nabla_w)^i(v)$ of $\cal{V}$ (for $0\leq i<n$) form a basis of $\cal{V}$.
  This condition does not depend on the choice of $w$, and if $f$ is an invertible holomorphic function, then $v$ is cyclic if and only if $fv$ is cyclic.
  In fact, we can show, by induction on $i$, that $(\nabla_{gw})^i(fv)$ lies in the submodule of $\cal{V}$ generated by the $(\nabla_w)^j(v)$ (for $0\leq j\leq i$).

  If $\scr{L}$ is an invertible module, then we say that a section $v$ of $\cal{V}\otimes\scr{L}$ is cyclic if, for every every local isomorphism between $\scr{L}$ and $\cal{O}$, the corresponding section of $\cal{V}$ is cyclic.
  This applies, in particular, to sections $v$ of $\shHom(\cal{V},\scr{L})=\cal{V}^\vee\otimes\scr{L}$.
\end{rmenv}

\begin{itenv}{Lemma 4.8}
\label{I.4.8}
  With the hypotheses and notation of \hyperref[I.4.6]{(4.6)}, $\lambda$ is a cyclic section of $\shHom(\cal{V},\scr{L})$.
\end{itenv}

\begin{proof}
  The problem is local on $X$;
  we can reduce to the case where $\scr{L}=\cal{O}$ and where there exists a local coordinate $x$.

  We use the notation of \hyperref[I.4.4]{(4.4)}.
  A section $(f^i)$ of $P^{n-1}(\cal{O})\simeq\cal{O}^{[0,n-1]}$ is horizontal if and only if it satisfies
  \[
    \begin{cases}
      \partial_x f^i = f^{i+1}
      & \mbox{(for $0\leq i\leq n-2$)}
    \\\partial_x f^{n-1} = -\sum_{i=0}^{n-1} b_i f^i.
    \end{cases}
  \]
  This gives us the coefficients of the connection: the matrix of the connection is
  \[
  \label{I.4.8.1}
    \left(
      \def\arraystretch{1.8}
      \begin{array}{rrrrrr}
        0 & -1 & 0 & 0 & \cdots & 0
      \\0 & 0 & -1 & 0 & \cdots & 0
      \\0 & 0 & 0 & -1 & & 0
      \\\vdots & \vdots & & \ddots & \ddots & \vdots
      \\\vdots & \vdots & & & 0 & -1
      \\b^0 & b^1 & \cdots & \cdots & b^{n-2} & b^{n-1}
      \end{array}
    \right)
  \tag{4.8.1}
  \]

\oldpage{27}
  In the chosen system of coordinates, $\lambda=e^0$, and we calculate that
  \[
    \nabla_x^i\lambda = e^i
    \qquad\mbox{(for $0\leq i\leq n-1$)}
  \]
  which proves \hyperref[I.4.8]{(4.8)}.
\end{proof}

\begin{itenv}{Proposition 4.9}
\label{I.4.9}
  The construction in \hyperref[I.4.6]{(4.6)} establishes an equivalence between the following categories, where we take morphisms to be isomorphisms:
  \begin{enumerate}[a)]
    \item the category of invertible sheaves on $X$ endowed with an $n^\mathrm{th}$-order differential equation \hyperref[I.4.3]{(4.3)} ; and
    \item the category of triples consisting of a rank-$n$ vector bundle $\cal{V}$ endowed with a connection, an invertible sheaf $\scr{L}$, and a cyclic homomorphism $\lambda\colon\cal{V}\to\scr{L}$.
  \end{enumerate}
\end{itenv}

\begin{proof}
  We will construct a functor that is quasi-inverse to that in \hyperref[I.4.6]{(4.6)}.
  Let $\cal{V}$ be a vector bundle with connection, and $\lambda$ a homomorphism from $\cal{V}$ to an invertible sheaf $\scr{L}$.
  We denote by $V$ the local system of horizontal sections of $\cal{V}$.
  For every $\cal{O}$-module $\scr{M}$, we have \hyperref[I.2.17]{(2.17)}
  \[
    \Hom_\cal{O}(\cal{V},\scr{M}) \simto \Hom_\CC(V,\scr{M}).
  \]
  In particular, we define a map $\gamma^k$ from $\cal{V}$ to $P^k(\scr{L})$ by setting, for any \emph{horizontal section} $v$ of $\cal{V}$,
  \[
  \label{I.4.9.1}
    \gamma^k(v) = \DD^k(\lambda(v)).
  \tag{4.9.1}
  \]

  \begin{itenv}{Lemma 4.9.2}
  \label{I.4.9.2}
    The homomorphism $\lambda$ is cyclic if and only if
    \[
      \gamma^{n-1}\colon \cal{V} \to P^{n-1}(\scr{L})
    \]
    is an isomorphism.
  \end{itenv}

  The problem is local on $X$.
  We can restrict to the case where $\scr{L}=\cal{O}$ and where we have a local coordinate $x$.
  With the notation of \hyperref[I.4.4]{(4.4)}, the morphism $\gamma^k$ then admits the morphisms $\partial_x^i\lambda = \nabla_x^i$ (for $0\leq i\leq k$) as coordinates.
  For $k=n-1$, these form a basis of $\Hom(\cal{V},\cal{O})$ if and only if $\gamma^{n-1}$ is an isomorphism.

  For $k\geq l$, the diagram
\oldpage{28}
  \[
  \label{I.4.9.3}
    \begin{tikzcd}[row sep=huge]
      & \cal{V} \dlar[swap,"\gamma^k"] \drar["v^l"] &
    \\P^k(\scr{L}) \ar[rr,"\pi_{l,k}"] && P^k(\scr{L})
    \end{tikzcd}
  \tag{4.9.3}
  \]
  commutes;
  if $\lambda$ is cyclic, then this, along with \hyperref[I.4.9.2]{(4.9.2)}, implies that $\gamma^n(v)$ is locally a direct factor, of codimension~$1$ in $P^n(\scr{L})$, and admits $\omega^{\otimes n}\otimes\scr{L} \simeq \Ker(\pi_{n-1,n})$ as a complement.
  There thus exists exactly one $n^\mathrm{th}$-order differential equation
  \[
    E\colon P^n(\scr{L}) \to \Omega^{\otimes n}\otimes\scr{L}
  \]
  on $\scr{L}$ such that $E\circ\gamma^n=0$.

  By \hyperref[I.4.9.1]{(4.9.1)}, if $v$ is a horizontal section of $\cal{V}$, then $E\DD^n\lambda v = E\gamma^n v = 0$, and so $\lambda v$ is a solution of $E$.
  We endow $P^{n-1}(\scr{L})$ with the connection \hyperref[I.4.6]{(4.6)} defined by $E$.
  If $v$ is a horizontal section of $\cal{V}$, then $\gamma^{n-1}(v)=\DD^{n-1}\lambda v$, where $\lambda v$ is a solution of $E$, and $\gamma^{n-1}(v)$ is thus horizontal.
  We thus deduce that $\gamma^{n-1}$ is compatible with the connections.
  A particular case of \hyperref[I.4.9.3]{(4.9.3)} shows that the diagram
  \[
    \begin{tikzcd}[row sep=huge]
      \cal{V} \ar[rr,"\gamma^{n-1}"] \drar[swap,"\lambda"]
      && P^{n-1}(\scr{L}) \dlar["\todo"]
    \\&\scr{L}&
    \end{tikzcd}
  \]
  commutes, whence we have an isomorphism between $(\cal{V},\scr{L},\lambda)$ and the triple given by \hyperref[I.4.6]{(4.6)} applied to $(\scr{L},E)$.
  The functor
  \[
    (\cal{V},\scr{L},\lambda) \mapsto (\scr{L},E)
  \]
  is thus quasi-inverse to the functor in \hyperref[I.4.6]{(4.6)}.
\end{proof}

\begin{rmenv}{4.10}
\label{I.4.10}
  We now summarise the relations between two systems $(\cal{V},\scr{L},\lambda)$ and $(\scr{L},E)$ that correspond under \hyperref[I.4.6]{(4.6)} and \hyperref[I.4.9]{(4.9)}.

  We have homomorphisms $\gamma^k\colon\cal{V}\to P^k(\scr{L})$, such that
  \begin{enumerate}
    \item[(4.10.1)]\label{I.4.10.1}
      $\gamma^k(v) = \DD^k\lambda v$ for $v$ horizontal ;
\oldpage{29}
    \item[(4.10.2)]\label{I.4.10.2}
      $\gamma^0=\lambda$ and $\pi_{l,k}\gamma^k=\gamma^l$ ;
    \item[(4.10.3)]\label{I.4.10.3}
      $\gamma^{n-1}$ is an isomorphism ($\lambda$ is cyclic) ;
    \item[(4.10.4)]\label{I.4.10.4}
      $E\gamma^n=0$ ; and
    \item[(4.10.5)]\label{I.4.10.5}
      $\lambda$ induces an isomorphic between the local system $V$ of sections of $\cal{V}$ and the local system $\scr{L}^E$ of solutions of $E$.
  \end{enumerate}
\end{rmenv}


\section{Second-order differential equations}
\label{I.5}

In this section, we specialise the results of \hyperref[I.4]{(4)} to the case where $n=2$, and we express certain results given in R.C.~Gunning~\cite{11} in a more geometric form.

\begin{rmenv}{5.1}
\label{I.5.1}
  Let $S$ be an analytic space, and let $q\colon X_2\to S$ be an analytic space over $S$ that is locally isomorphic to the finite analytic space over $S$ defined by the $\cal{O}_S$-algebra $\cal{O}_S[T]/(T^3)$.

  The fact that the group $\mathrm{PGL}_2$ acts \todo on $\PP^1$ has the following infinitesimal analogue.
\end{rmenv}

\begin{itenv}{Lemma 5.2}
\label{I.5.2}
  Under the hypotheses of \hyperref[I.5.1]{(5.1)}, let $u$ and $v$ be $S$-immersions of $X_2$ into $\PP_S^1$, i.e.
  \[
    \begin{tikzcd}
      X_2 \rar[shift left=1,"u"] \rar[shift right=1,swap,"v"]
      & \PP_S^1.
    \end{tikzcd}
  \]
  Then there exists exactly one \todo ($S$-automorphism) of $\PP_S^1$ that sends $u$ to $v$.
\end{itenv}

\begin{proof}
  The problem is local on $S$, which allows us to suppose that $X_2$ is defined by the $\cal{O}_S$-algebra $\cal{O}_S[T]/(T^3)$, and that $u(X_2)$ and $v(X_2)$ are contained inside the same affine line, say, $\AA_S^1$.
  By translation, we can assume that $u(0)=v(0)=0$.
  We must then prove the existence and uniqueness of a \todo $p(x)=(ax+b)(cx+d)$ that satisfies $p(0)=0$, with first derivative $p'(0)\neq0$ and given second derivative $p''(0)$.
  We have that $b=0$, and $p$ can be written uniquely in the form
\oldpage{30}
  \[
    \begin{aligned}
      p(x)
      &= e\frac{x}{1-fx} \qquad\mbox{(for $e\neq0$)}
    \\&= ex + efx^2 \mod x^3.
    \end{aligned}
  \]
  The claim then follows immediately.
\end{proof}

\begin{rmenv}{5.3}
\label{I.5.3}
  By \hyperref[I.5.2]{(5.2)}, there exists exactly (up to isomorphism) one pair $(u,P)$ consisting of a projective line $P$ on $S$ (with structure group $\mathrm{PGL}_2(\cal{O}_S)$) and an $S$-immersion $u$ of $X_2$ into $P$.
  We call $P$ the \emph{osculating projective line of $X_2$}.

  Let $X$ be a smooth curve.
  Let $X_2$ the second infinitesimal neighbourhood of the diagonal of $X\times X$, and let $q_1$ and $q_2$ be the two projections from $X_2$ to $X$.

  The morphism $q_1\colon X_2\to X$ is of the type considered in \hyperref[I.5.1]{(5.1)}.
\end{rmenv}

\begin{rmenv}{Definition 5.4}
\label{I.5.4}
  We define the \emph{osculating projective line bundle of $X$}, denoted by $P_\tg$, to be the osculating projective line bundle of $q_1\colon X_2\to X$.
\end{rmenv}

By definition, we thus have a canonical commutative diagram
\[
\label{I.5.4.1}
  \begin{tikzcd}
    X_2 \rar[hook] \drar[swap,"q_1"]
    & P_\tg \dar
  \\& X
  \end{tikzcd}
\tag{5.4.1}
\]
and, in particular, $P_\tg$ is endowed with a canonical section $e$, which is the image of the diagonal section of $X_2$, and we have that
\[
\label{I.5.4.2}
  e^*\Omega_{P_\tg/X}^1 \simeq \Omega_X^1.
\tag{5.4.2}
\]

\begin{rmenv}{5.5}
\label{I.5.5}
  If $X$ is a projective line, then $\pr_1\colon X\times X\to X$ is a projective bundle on $X$ such that $P_\tg$ can be identified with the constant projective bundle of fibre $X$ on $X$ endowed with the inclusion homomorphism of $X_2$ into $X\times X$, i.e.
  \[
    \begin{tikzcd}
      X_2 \rar[hook] \drar[swap,"q_1"]
      & X\times X \dar["\pr_1"]
    \\& X
    \end{tikzcd}
  \]
\oldpage{31}
  In this particular case, we have a canonical commutative diagram
  \[
    \begin{tikzcd}[column sep=huge]
      X_3 \drar[hook] \ar[ddr]
    \\X_2 \uar[hook] \rar[hook] \drar
      & P_\tg \dar
    \\& X
    \end{tikzcd}
  \]
\end{rmenv}

Now let $X$ be an arbitrary smooth curve.

\begin{rmenv}{Definition 5.6}
\label{I.5.6}
  \emph{(Local version).}
  A \emph{projective connection} on $X$ is a sheaf on $X$ of germs of local isomorphisms from $X$ to $\PP^1$, which is a principal homogeneous sheaf (i.e. a torsor) for the constant sheaf of groups with value $\mathrm{PLG}_2(\CC)$.
\end{rmenv}

If $X$ is endowed with a projective connection, then every local construction on $\PP^1$ that is invariant under the projective group can be transported to $X$;
in particular, the construction in \hyperref[I.5.5]{(5.5)} gives us a morphism $\gamma$ that fits into a commutative diagram:
\[
\label{I.5.6.1}
  \begin{tikzcd}[column sep=huge]
    X_3 \drar[hook,"\gamma"] \ar[ddr]
  \\X_2 \uar \rar[hook] \drar
    & P_\tg \dar
  \\& X
  \end{tikzcd}
\tag{5.6.1}
\]

It is not difficult to show that such a morphism $\gamma$ is defined by a unique projective connection (a proof of this will be given in \hyperref[I.5.10]{(5.10)}), and so Definition~\hyperref[I.5.6]{(5.6)} is equivalent to the following.

\begin{rmenv}{Definition 5.6~bis}
\label{I.5.6bis}
  \emph{(Infinitesimal version).}
  A \emph{projective connection} on $X$ is a morphism $\gamma\colon X_3\hookrightarrow P_\tg$ that makes the diagram in \hyperref[I.5.6.1]{(5.6.1)} commute.
\end{rmenv}

Intuitively, giving a projective connection (the infinitesimal version) allows us to define \todo

\begin{rmenv}{5.7}
\label{I.5.7}
  Set $\Omega^{\otimes n}=(\Omega_X^1)^{\otimes n}$ \hyperref[I.4.2]{(4.2)}.
  The sheaf of ideals on $X_3$ that defines $X_2$ is
\oldpage{32}
  canonically isomorphic to $\Omega^{\otimes3}$, and is annihilated by the sheaf of ideals that defines the diagonal.
  Also, if $\Delta$ is the diagonal map, then, by \hyperref[I.5.4.2]{(5.4.2)}, we have that
  \[
    \Delta^*\gamma^*\Omega_{P_\tg/X}^1 \simeq \Omega^1.
  \]
  We thus deduce that the set of $X$-homomorphisms from $X_3$ to $P_\tg$ that induce the canonical homomorphism from $X_2$ to $P_\tg$ is either empty, or a principal homogeneous space for
  \[
    \Hom_X(\Delta^*\gamma^*\Omega_{P_\tg/X}^1, \Omega^{\otimes3})
    = \Hom_X(\Omega^1,\Omega^{\otimes3})
    = \HH^0(X,\Omega^{\otimes2}).
  \]
  If we replace $X$ by a small enough open subset, then this set is non-empty:
\end{rmenv}

\begin{itenv}{Proposition 5.8}
\label{I.5.8}
  Projective connections of open subsets of $X$ form a principal homogeneous sheaf (i.e. a torsor) for the sheaf $\Omega^{\otimes2}$.
\end{itenv}

\begin{proof}
  If $\eta$ is a section of $\Omega^{\otimes2}$, and $\gamma_1\colon X_3\to P_\tg$ is a projective connection, then the connection $\gamma_2=\gamma_1+\eta$ is defined, for any function $f$ on $P_\tg$, by
  \[
  \label{I.5.8.1}
    \gamma_2^*f = \gamma_1^*f + \eta\cdot e^*\dd f
  \tag{5.8.1}
  \]
  (modulo the identification of $\Omega^{\otimes3}$ with an ideal of $\cal{O}_{X_3}$).
\end{proof}

\begin{rmenv}{5.9}
\label{I.5.9}
  Let $f\colon X\to Y$ be a homomorphism between smooth curves endowed with projective connections $\gamma_X$ and $\gamma_Y$, and suppose that $f$ is a local isomorphism (i.e. $\dd f\neq0$ at all points).
  Set
  \[
    \theta f = f^*\gamma_Y - \gamma_X \in \Gamma(X,\Omega^{\otimes2}).
  \]
  For a composite map $g\circ f\colon X\xrightarrow{f}Y\xrightarrow{g}Z$, we trivially have that
  \[
  \label{I.5.9.1}
    \begin{aligned}
      \theta(g\circ f) &= \theta(f) + f^*\theta(g),
    \\\mbox{whence }\theta(f^{-1}) &= -f^*\theta(f).
    \end{aligned}
  \tag{5.9.1}
  \]

  Suppose that $X$ and $Y$ are open subsets of $\CC$, and endowed with the projective connection induced by that of $\PP^1(\CC)$.
  Denoting by $x$ the injection from $X$ into $\CC$, we then have that
\oldpage{33}
  \[
  \label{I.5.9.2}
    \theta f = \frac{f'(f'''/6) - (f''/2)^2}{(f')^2} \dd x^{\otimes2}.
  \tag{5.9.2}
  \]

  To prove this, we identify, using \hyperref[I.5.5]{(5.5)}, the projective double tangent bundle to $X$ or $Y$ with the constant projective bundle.
  The morphism $\delta f\colon P_{\tg,X}\to P_{\tg,Y}$ induced by $f$ can be written as
  \[
    \delta f\colon (x,p) \mapsto
    \left(
      f(x),
      f(x) + \frac{f'(x)(p-x)}{1-\frac12(f''(x)/f'(x))(p-x)}
    \right).
  \]

  Consider the diagram
  \[
    \begin{tikzcd}
      X_3 \rar["f"] \dar[hook]
      & Y_3 \dar[hook]
    \\P_{\tg,X} \rar["\delta f"]
      & P_{\tg,Y}
    \end{tikzcd}
  \]
  Then $\theta(f)$ describes the lack of commutativity of the diagram, i.e. the difference between the jets
  \[
    (x,x+\varepsilon)
    \mapsto
    \left(
      f(x),
      f(x) + f'(x)\varepsilon + f''(x)\frac{\varepsilon^2}{2} + f'''(x)\frac{\varepsilon^3}{6}
    \right)
    \qquad\mbox{(mod. $\varepsilon^4=0$)}
  \]
  and
  \[
    \begin{aligned}
      (x,x+\varepsilon)
      &\mapsto
      \left(
        f(x),
        f(x) + \frac{f'(x)\varepsilon}{1-\frac12(f''(x)/f'(x))\varepsilon}
      \right)
    \\&= \left(
        f(x),
        f(x) + f'(x)\varepsilon + f''(x)\frac{\varepsilon^2}{2} + \frac14(f''(x)^2/f'(x))\varepsilon^3
      \right).
    \end{aligned}
  \]
  We thus have that
  \[
    \theta(f) =
    \left(
      \frac16f'''(x) - \frac14f''(x)^2/f'(x)
    \right) \dd x^{\otimes3} \dd f^{\otimes-1},
  \]
  and \hyperref[I.5.9.2]{(5.9.2)} then follows.

  Equation~\hyperref[I.5.9.2]{(5.9.2)} shows that $6\theta f$ is the classical \emph{Schwarz derivative} of $f$.

  \bigskip

  If a map $f$ from $X\subset\CC$ to $\PP^1(\CC)$ is described by projective coordinates $f=(g,h)$, then
  \[
  \label{I.5.9.3}
    \theta(f) =
    \frac{
      \left\vert
        \begin{array}{cc}
          g&g'\\h&h'
        \end{array}
      \right\vert
      \left(
        \left\vert
          \begin{array}{cc}
            g&g'''/6\\h&h'''/6
          \end{array}
        \right\vert
        +
        \left\vert
          \begin{array}{cc}
            g'&g'/2\\h'&h'/2
          \end{array}
        \right\vert
      \right)
      -
      \left\vert
        \begin{array}{cc}
          g&g''/2\\h&h''/2
        \end{array}
      \right\vert^2
    }{
      \left\vert
        \begin{array}{cc}
          g&g'\\h&h'
        \end{array}
      \right\vert^2
    }
  \tag{5.9.3}
  \]

  To prove \hyperref[I.5.9.3]{(5.9.3)}, the simplest method is to note the following.
\oldpage{34}
  \begin{enumerate}[(i)]
    \item The right-hand side of \hyperref[I.5.9.3]{(5.9.3)} is invariant under a linear substitution of constant coefficients $L$ acting on $g$ and $h$: the numerator and denominator $\det(L)^2$.
    \item The right-hand side of \hyperref[I.5.9.3]{(5.9.3)} is invariant under the substitution
      \[
        (g,h) \mapsto (\lambda g,\lambda h).
      \]

      Denoting the determinant of $\left(\begin{array}{cc}a&b\\c&d\end{array}\right)$ by $\detrow{a}{b}$, we have that
      \[
        \begin{aligned}
          \detrow{g}{(\lambda g)'}
          \quad&=\quad \detrow{\lambda g}{\lambda g'+\lambda' g}
        \\[0.5em]
          &=\quad \lambda^2\detrow{g}{g'}
        \\[1em]
          \detrow{\lambda g}{(\lambda g)''/2}
          \quad&=\quad \lambda^2\detrow{g}{g''/2}
          + \lambda\lambda'\detrow{g}{g'}
        \\[1em]
          \detrow{\lambda g}{(\lambda g)'''/6}
          \quad&=\quad \lambda^2\detrow{g}{g'''/6}
          + \lambda\lambda'\detrow{g}{g''/2}
        \\[0.5em]
          &\qquad+ (\lambda\lambda''/2)\detrow{g}{g'}
        \\[1em]
          \detrow{(\lambda g)'}{(\lambda g)''/2}
          \quad&=\quad \lambda^2\detrow{g'}{g''/2}
          - (\lambda\lambda''/2)\detrow{g}{g'}
        \\[0.5em]
          &\qquad+ (\lambda')^2\detrow{g}{g'}
          + \lambda\lambda'\detrow{g}{g''/2}.
        \end{aligned}
      \]

      The new denominator $D_\lambda$ (resp. numerator $N_\lambda$) is thus given in terms of the old denominator $D$ (resp. numerator $N$) by
      \[
        \begin{aligned}
          D_\lambda
          &= \lambda^4 D
        \\N_\lambda
          &= \lambda^4 N + \lambda^4\detrow{g}{g'}\cdot
          \left[
            \begin{aligned}
              &\Big(
                \lambda'/\lambda\detrow{g}{g''/2}
                + (\lambda''/2\lambda)\detrow{g}{g'}
              \Big)
            \\+&\Big(
                -(\lambda''/2\lambda)\detrow{g}{g'}
                +(\lambda'/\lambda)^2\detrow{g}{g'}
                +(\lambda'/\lambda)\detrow{g}{g''}
              \Big)
            \\-&\Big(
                2(\lambda'/\lambda)\detrow{g}{g''/2}
                + (\lambda'/\lambda)^2\detrow{g}{g'}
              \Big)
            \end{aligned}
          \right]
        \end{aligned}
      \]
      and $N_\lambda/D_\lambda=N/D$.

      With these variance properties agreeing with those of the left-hand side of \hyperref[I.5.9.3]{(5.9.3)}, it suffices to prove \hyperref[I.5.9.3]{(5.9.3)} in the particular case where $h=1$.
      The equation then reduces to \hyperref[I.5.9.2]{(5.9.2)}.
  \end{enumerate}

  We will only need to use the fact that $\theta(f)$ can be expressed in terms of \todo:
  we have, for $Z_i\sim Z$,
  \[
  \label{I.5.9.4}
    \frac{(f(Z_1),f(Z_2),f(Z_3),f(Z_4))}{(Z_1,Z_2,Z_3,Z_4)} - 1
    = \theta(f)(Z_1-Z_2)(Z_3-Z_4) + \mathcal{O}((Z_i-Z)^3).
  \tag{5.9.4}
  \]
\end{rmenv}

\oldpage{35}
\begin{rmenv}{5.10}
\label{I.5.10}
  The differential equation $\theta(f)=0$ (for $f\colon X\to\PP^1(\CC)$ with non-zero first derivative) is a third-order differential equation.
  It thus admits $\infty^3$ solutions, locally, and these solutions are permuted amongst themselves by the projective group (since the group transitively permutes the Cauchy data: \hyperref[I.5.2]{(5.2)}).
  The set of solutions is thus a projective connection (the local version, \hyperref[I.5.6]{(5.6)}).
  This construction is inverse to that which associates, to any projective connection in the sense of \hyperref[I.5.6]{(5.6)}, a projective connection in the sense of \hyperref[I.5.6bis]{(5.6bis)}.
\end{rmenv}

\begin{rmenv}{5.11}
\label{I.5.11}
  Let $X$ be a smooth curve, $\scr{L}$ an invertible sheaf on $X$, and $E$ a second-order ordinary differential equation on $\scr{L}$.
  We have seen, in \hyperref[I.4.5]{(4.5)}, that $E$ defines a connection on the bundle $P^1(\scr{L})$ of first-order jets of sections of $\scr{L}$, and we obtain from it a connection on
  \[
    \bigwedge^2 P^1(\scr{L}) \simeq \scr{L}\otimes\Omega^1(\scr{L}) = \Omega^1\otimes\scr{L}^{\otimes2}.
  \]

  If $X$ is a compact connected curve of genus~$g$, then the bundle $\Omega^1\otimes\scr{L}^{\otimes2}$ is thus necessarily of degree~$0$, and we have that
  \[
    \deg(\scr{L}) = 1-g.
  \]

  Let $V$ be the rank-$2$ local system of solutions of $E$;
  we have \hyperref[I.4.5]{(4.5)} that $\cal{O}\otimes V\simto P^1(\scr{L})$, and the linear form $\lambda\colon P^1(\scr{L})\to\scr{L}$ defines a section $\lambda_0$ of the projective bundle associated to the vector bundle $P^1(\scr{L})$.

  Locally on $X$, $V$ is isomorphic to the constant local system $\underline{\CC^2}$;
  the choice of an isomorphism $\sigma\colon V\to\underline{\CC^2}$ identifies $\lambda_0$ with a map $\lambda_{0,\sigma}$ from $X$ to $\PP^1(\CC)$;
  by \hyperref[I.4.8]{(4.8)}, the differential of this map is everywhere non-zero, and so $\lambda_{0,\sigma}$ allows us to transport the canonical projective connection of $\PP^1(\CC)$ to $X$.
  This connection does not depend on the choice of $\sigma$, and so the differential equation $E$ defines a projective connection on $X$.
\end{rmenv}

\begin{itenv}{Proposition 5.12}
\label{I.5.12}
  Let $\scr{L}$ be an invertible sheaf on a smooth curve $X$.
  The construction in \hyperref[I.5.10]{(5.10)} gives a bijection between
  \begin{enumerate}[a)]
    \item the set of second-order ordinary differential equations on $\scr{L}$ ; and
\oldpage{36}
    \item the set of pairs consisting of a projective connection on $X$ and a connection on $\Omega^1(\scr{L}^{\otimes2})$.
  \end{enumerate}
\end{itenv}

\begin{proof}
  The problem is local on $X$;
  we can thus suppose that $X$ is an open subset of $\CC$, and that $\scr{L}=\cal{O}$.
  An equation $E$ can then be written as
  \[
    E\colon y'' + a(x)y' + b(x)y = 0.
  \]
  If we identify $P^1(\scr{L})$ with $\cal{O}^2$, then the matrix of the connection \hyperref[I.5.10]{(5.10)} defined by $E$ on $\wedge^2 P^1(\scr{L})\sim\cal{O}$ is then $-a(x)\tr M$, where $M$ is the matrix in \hyperref[I.4.8.1]{(4.8.1)}.

  Let $\varphi$ be the identity map from $X$, an open subset of $\PP^1(\CC)$, to itself endowed with the projective connection defined by $E$.
  Identifying $\Omega$ with $\cal{O}$ by means of the given local coordinate, we then have that
  \[
  \label{I.5.12.1}
    \theta(\varphi) = \frac13b - \frac{1}{12}(a^2+2a').
  \tag{5.12.1}
  \]
  Indeed, if $f$ and $g$ are two linearly independent solutions of $E$, then the map with projective coordinates
  \[
    (f,g)\colon X\to\PP^1(\CC)
  \]
  \todo the projective connection.
  We have that
  \[
    \begin{aligned}
      f''
      &= -(af'+bf)
    \\f'''
      &= -a(-af'-bf) - bf' - a'f' - b'f
    \\&= (a^2-a'-b)f' + (ab-b')f.
    \end{aligned}
  \]

  Equation~\hyperref[I.5.9.3]{(5.9.3)} gives (using the same notation for determinants as before)
  \[
    \begin{aligned}
      \theta(\varphi)
      &= \frac{
        \detrow{f}{f'}
        \left(
          \frac16(a^2-a'-b)\detrow{f}{f'}
          +\frac12 b\detrow{f}{f'}
        \right)
        - (\frac12a)^2\detrow{f}{f'}^2
      }{
        \detrow{f}{f'}^2
      }
    \\&= \frac16(a^2-a'-b) + \frac12 b - \frac14 a^2
    \\&= \frac13 b - \frac{1}{12}(a^2+2a').
    \end{aligned}
  \]

  We conclude by noting that $(a,b)$ is uniquely determined by $(-a,\frac13b-\frac{1}{12}(a^2+2a'))$, and that, for any holomorphic function $g$ on an open subset $U$ of $\CC$, there exists a unique projective connection on $U$ satisfying $\theta(\varphi)=g$, for $\varphi$ \todo the connection (same proof as for \hyperref[I.5.10]{(5.10)}, or \hyperref[I.5.8]{(5.8)}).
\end{proof}


\oldpage{37}
\section{Multiform functions of finite determination}
\label{I.6}

\begin{rmenv}{6.1}
\label{I.6.1}
  Let $X$ be a non-empty connected topological space that is both locally path connected and locally simply path connected, and let $x_0$ be a point of $X$.
  We denote by $\pi\colon \widetilde{X}_{x_0}\to X$ the universal cover of $(X,x_0)$, and by $\widetilde{x}_0$ the base point of $\widetilde{X}_{x_0}$.
\end{rmenv}

If $\scr{F}$ is a sheaf on $X$, then we pose:
\begin{rmenv}{Definition 6.2}
\label{I.6.2}
  A \emph{multiform section} of $\scr{F}$ on $X$ is a global section of the inverse image $\pi^*\scr{F}$ of $\scr{F}$ on $\widetilde{X}_{x_0}$.

  If $s$ is a multiform section of $\scr{F}$ on $X$, then a \emph{determination of $s$ at a point $x$ of $X$} is an element of the fibre $\scr{F}_{(x)}$ of $\scr{F}$ at $x$ that is an inverse image of $s$ under a local section of $\pi$ at $x$.
  \textbf{\todo (check)}
  Each point in $\pi^{-1}(x)$ thus defines a determination of $s$ at $x$.
  We define the \emph{base determination} of $s$ at $x_0$ to be the determination defined by $\widetilde{x}_0$.
  We define a \emph{determination of $s$ on an open subset $U$ of $X$} to be a section of $\scr{F}$ over $U$ whose \todo at every point of $U$ is a determination of $s$ at that point.
\end{rmenv}

\begin{rmenv}{Definition 6.3}
\label{I.6.3}
  We say that $\scr{F}$ satisfies the \emph{principle of analytic continuation} if the set where any two local sections of $\scr{F}$ agree is always (open and) closed.
\end{rmenv}

\begin{rmenv}{Example 6.4}
\label{I.6.4}
  If $\scr{F}$ is a coherent analytic sheaf on a complex-analytic space, then $\scr{F}$ satisfies the principle of analytic continuation if and only if $\scr{F}$ has no \todo components.
\end{rmenv}

\begin{itenv}{Proposition 6.5}
\label{I.6.5}
  Let $X$ and $x_0$ be as in \hyperref[I.5.1]{(5.1)}, and let $\scr{F}$ be a sheaf of $\CC$-vector spaces on $X$ that satisfies the principle of analytic continuation.
  For every multiform section $s$ of $\scr{F}$, the following conditions are equivalent:
  \begin{enumerate}[(i)]
    \item the determinations of $s$ at $x_0$ generate a finite-dimensional sub-vector space of $\scr{F}_{x_0}$ ; and
    \item the subsheaf of $\scr{F}$ of $\CC$-vector spaces generated by the determinations of $s$ is a complex local system \hyperref[I.1.1]{(1.1)}.
  \end{enumerate}
\end{itenv}

\oldpage{38}
\begin{proof}
  It is trivial that (ii) implies (i).
  We now prove that (i) implies (ii).
  Let $x$ be a point of $X$ at which the determinations of $s$ generate a finite-dimensional sub-vector space of $\scr{F}_x$, and let $U$ be a connected open neighbourhood of $x$ over which $\widetilde{X}_{x_0}$ is trivial: $(\pi^{-1}(U),\pi) \simeq (U\times I,\pr_1)$ for some suitable set $I$.
  We will prove that, over $U$, the determinations of $s$ generate a complex local system.
  Each $i\in I$ defines a determination $s_i$ of $s$, and, over $U$, the vector subsheaf of $\scr{F}$ generated by the determinations of $s$ is generated by the $(s_i)_{i\in I}$;
  if this sheaf is constant, then the hypotheses on $x$ implies that it is a local system.
  We have:

  \begin{itenv}{Lemma 6.6}
  \label{I.6.6}
    If a sheaf $\scr{F}$ of $\CC$-vector spaces on a connected space satisfies the principle of analytic continuation, then the vector subsheaf of $\scr{F}$ generated by a family of global sections $s_i$ is a constant sheaf.
  \end{itenv}

  The sections $s_i$ define
  \[
    a\colon\underline{\CC}^{(I)}\to\scr{F}
  \]
  with the image being the vector subsheaf $\scr{G}$ of $\scr{F}$ generated by the $s_i$.
  If an equation $\sum_i\lambda_i s_i=0$ between the $s_i$ holds at a point, then it holds everywhere, by the principle of analytic continuation.

  The sheaf $\Ker(a)$ is thus constant subsheaf of $\underline{\CC}^{(I)}$, and the claim then follows.

  We conclude the proof of \hyperref[I.6.5]{(6.5)} by noting that, by the above, the largest open subset of $X$ over which the determinations of $s$ generate a local system is closed and contains $x_0$.
\end{proof}

\begin{rmenv}{Definition 6.7}
\label{I.6.7}
  Under the hypotheses of \hyperref[I.6.5]{(6.5)}, a multiform section $s$ of $\scr{F}$ is said to be a \emph{finite determination} if it satisfies either of the equivalent conditions of \hyperref[I.6.5]{(6.5)}.
\end{rmenv}

\begin{rmenv}{6.8}
\label{I.6.8}
  Under the hypotheses of \hyperref[I.6.5]{(6.5)}, let $s$ be a multiform section of finite determination of $\scr{F}$.
  This section defines
\oldpage{39}
  \begin{enumerate}[a)]
    \item the local system $V$ generated by its determinations ;
    \item a \todo of $V$ at $x_0$, say, $v_0$, corresponding to the base determination of $s$ ; and
    \item an inclusion morphism $\lambda\colon V\to\scr{F}$.
  \end{enumerate}

  The triple consisting of $V_{x_0}$, $v_0$, and the representation of $\pi_1(X,x_0)$ on $V_{x_0}$ defined by $V$ \hyperref[I.1.4]{(1.4)} is called the \emph{monodromy} of $s$.
  The triple $(V,V_0,\lambda)$ satisfies the following two conditions.
  \begin{enumerate}
    \item[(6.8.1)]\label{I.6.8.1}
      $v_0$ is a cyclic vector of the $\pi_1(X,x_0)$-module $V_{x_0}$, i.e. it generates the $\pi_1(X,x_0)$-module $V_{x_0}$.

      This simply means that $V$ is generated by the set of determinations of the unique multiform section of $V$ with base determination $v_0$.
    \item[(6.8.2)]\label{I.6.8.2}
      $\lambda\colon V_{x_0}\to\scr{F}_{x_0}$ is injective.
  \end{enumerate}
\end{rmenv}

\begin{rmenv}{6.9}
\label{I.6.9}
  Let $W_0$ be a finite-dimensional complex representation of $\pi_1(X,x_0)$, endowed with a cyclic vector $w_0$.
  The multiform section $s$ of $\scr{F}$ is said to be of \emph{monodromy subordinate to $(w_0,v_0)$} if it is the finite determination, and if, with the notation of \hyperref[I.6.8]{(6.8)}, there exists a homomorphism of $\pi_1(X,x_0)$-representations of $W_0$ in $V_{x_0}$ that sends $w_0$ to $v_0$.
  Let $W$ be the local system defined by $W_0$, and let $w$ be the unique multiform section of $w$ of base determination $w_0$.
  It is clear that, under the hypotheses of \hyperref[I.6.5]{(6.5)}, we have
\end{rmenv}

\begin{itenv}{Proposition 6.10}
\label{I.6.10}
  The function $\lambda\mapsto\lambda(w)$ is a bijection between the set $\Hom_\CC(W,\scr{F})$ and the set of multiform sections of $\scr{F}$ with monodromy subordinate to $(W_0,w_0)$.
\end{itenv}

\begin{itenv}{Corollary 6.11}
\label{I.6.11}
  Let $X$ be a reduced connected complex analytic space endowed with a base point $x_0$.
  Let $W_0$ be a finite-dimensional complex representation of $\pi_1(X,x_0)$ endowed with a cyclic vector $w_0$, and $W$ the corresponding local system on $X$, with $\scr{W}=\cal{O}\otimes_\CC W$ being
\oldpage{40}
  the associated vector bundle, and $w$ the unique multiform section of $\scr{W}$ of base determination $w_0$.
  Write $\scr{W}^\vee$ be the dual vector bundle of $\scr{W}$.
  Then the function
  \[
    \lambda \mapsto \langle\lambda,w\rangle,
  \]
  from $\Gamma(X,\scr{W}^\vee)$ to the set of multiform holomorphic functions on $X$ of monodromy subordinate to $(W_0,w_0)$, is a bijection.
\end{itenv}

\begin{itenv}{Corollary 6.12}
\label{I.6.12}
  If $X$ is Stein, then there exist multiform holomorphic functions on $X$ of any given monodromy $(W_0,w_0)$.
\end{itenv}



\chapter{Regular connections}
\label{II}


\section{Regularity in dimension~\texorpdfstring{$1$}{1}}
\label{II.1}

\oldpage{41}
\begin{rmenv}{1.1}
\label{II.1.1}
  Let $U$ be an open neighbourhood of $0$ in $\CC$, and consider an $n^\mathrm{th}$-order differential equation
  \[
  \label{II.1.1.1}
    y^{(n)} + \sum_{i=0}^{n-1} a_i(x) y^{(i)} = 0
  \tag{1.1.1}
  \]
  where the $a_i$ are holomorphic functions on $U\setminus\{0\}$.
  We classically say that $0$ is a \emph{regular singular point} of \hyperref[II.1.1.1]{(1.1.1)} if the functions $x^{n-i}a_i(x)$ are holomorphic at $0$.
  If this is true, then, after multiplying by $x^n$, we can write \hyperref[II.1.1.1]{(1.1.1)} in the form
  \[
  \label{II.1.1.2}
    \left(x\frac{\dd}{\dd x}\right)^n y
    + \sum b_i(x)\left(x\frac{\dd}{\dd x}\right)^i y
    = 0
  \tag{1.1.2}
  \]
  where the $b_i(x)$ are holomorphic at $0$.
\end{rmenv}

In this section, we will translate this idea into the language of connections (cf. \hyperref[I.4]{(I.4)}), and we will establish some of its properties.

The results in this section were taught to me by N.~Katz.
They are either due to N.~Katz (see, most notably, \cite{14,15}), or classical (see, for example, Ince~\cite{13}, and Turrittin~\cite{25,26}).

\begin{rmenv}{1.2}
\label{II.1.2}
  Let $K$ be a (commutative) field, $\Omega$ a rank-$1$ vector space over $K$, and $\dd\colon K\to\Omega$ a non-trivial derivation, i.e. an non-zero additive map that satisfies the identity
  \[
  \label{II.1.2.1}
    \dd(xy) = x\dd y + y\dd x.
  \tag{1.2.1}
  \]

  Let $V$ be an $n$-dimensional vector space over $K$.
  Then a \emph{connection} on $V$ is an additive map $\nabla\colon V\to\Omega\otimes V$ that satisfies the identity
  \[
  \label{II.1.2.2}
    \nabla(xv) = \dd x\cdot v + x\nabla v.
  \tag{1.2.2}
  \]

  If $\tau$ is an element of the dual $\Omega^\vee$ of $\Omega$, then we set
  \[
  \label{II.1.2.3}
    \partial_\tau(x) = \langle\dd x,\tau\rangle \in K,
  \tag{1.2.3}
  \]
  \[
  \label{II.1.2.4}
    \nabla_\tau(v) = \langle\dd v,\tau\rangle \in V.
  \tag{1.2.4}
  \]

\oldpage{42}
  We thus have that
  \begin{enumerate}
    \item[(1.2.5)] \label{II.1.2.5}
      $\partial_\tau$ is a derivation;
    \item[(1.2.6)] \label{II.1.2.6}
      $\nabla_\tau(xv) = \partial_\tau(x)\cdot v + x\nabla_\tau v$; and
    \item[(1.2.7)] \label{II.1.2.7}
      $\nabla_{\lambda\tau}(v) = \lambda\nabla_\tau v$.
  \end{enumerate}

  Let $v\in V$.
  We can easily show that the vector subspace of $V$ generated by the vectors
  \[
    v, \nabla_{\tau_1}v, \nabla_{\tau_2}\nabla_{\tau_1}v, \ldots, \nabla_{\tau_k}\cdots\nabla_{\tau_1} v
  \]
  (where $\tau_i\neq0$ in $\Omega$) does not depend on the choice of the $\tau_i\neq0$, and does not change is we replace $v$ by $\lambda v$ (for some $\lambda\in K^*$).
  Furthermore, if the last of these vectors is a linear combination of the preceding vectors, then this vector subspace is stable under derivations.
  We say that $v$ is a \emph{cyclic vector} if, for $\tau\in\Omega$, the vectors
  \[
    \nabla_\tau^i v
    \qquad\mbox{(for $0\leq i\leq n$)}
  \]
  form a basis of $V$.
\end{rmenv}

\begin{itenv}{Lemma 1.3}
\label{II.1.3}
  Under the above hypotheses, and if $K$ is of characteristic~$0$, then there exists a cyclic vector.
\end{itenv}

\begin{proof}
  Let $t\in K$ be such that $\dd t\neq0$, and let $\tau=t/\dd t\in\Omega^\vee$.
  Then $\partial_\tau(t^k)=kt^k$.

  Let $m\leq n$ be the largest integer such that there exists a vector $e$ such that the vectors $\partial_\tau^i e$ (for $0\leq i\leq m$) are linearly independent.
  If $m\neq n$, then there exists a vector $f$ that is linearly independent of the $\partial_\tau^i e$.
  For any rational number $\lambda$ and integer $k$, the vectors
  \[
    \partial_\tau^i(e+\lambda t^k f)
    \qquad\mbox{(for $0\leq i\leq m$)}
  \]
  are linearly dependent, and their exterior product $\omega(\lambda,k)$ is thus zero.
  We have that
  \[
    \partial_\tau^i(e+\lambda t^k f)
    = \partial_\tau^i e + \sum_{0\leq k\leq i} k^j t^k \partial_\tau^{i-j} f.
  \]

  From this equation, we obtain a finite decomposition
  \[
    \omega(\lambda,k)
    = \sum_{\substack{0\leq a\leq m\\0\leq b}} \lambda^a t^{ka} k^b \omega_{a,b}
  \]
\oldpage{43}
  where $\omega_{a,b}$ is independent of $\lambda$ and $k$.
  Since $\omega(\lambda,k)=0$ for all $\lambda\in\QQ$, and since
  \[
    \omega(\lambda,k) = \sum \lambda^a \omega_a(k)
  \]
  where $\omega_a(k) = t^{ka}(\sum k^b \omega_{a,b}) = t^{ka} \omega'_a(k)$, we have that $\omega_a(k) = \omega'_a(k) = 0$.
  Since
  \[
    \omega'_a(k) = \sum k^b \omega_{a,b} = 0
  \]
  for all $k\in\ZZ$, we have that $\omega_{a,b}=0$.
  In particular,
  \[
    \omega_{1,m}
    = e\wedge\partial_\tau^1 e\wedge\ldots\wedge\partial_\tau^{m-1} e\wedge f
    = 0,
  \]
  and $f$ is then linearly dependent of the $\partial_\tau^i e$ (for $0\leq i\leq m$), which contradicts the hypothesis.
  Thus $m=n$, and so $e$ is a cyclic vector
\end{proof}

\begin{rmenv}{1.4}
\label{II.1.4}
  Let $\cal{O}$ be a discrete valuation ring of \emph{equal characteristic~$0$}, with maximal ideal $\fk{m}$, residue field $k=\cal{O}/\fk{m}$, and field of fractions $K$.
  Suppose that $\cal{O}$ is endowed with a free rank-$1$ $\cal{O}$-module $\Omega$ along with a derivation $\dd\colon\cal{O}\to\Omega$ that satisfies
  \begin{rmenv}{1.4.1}
  \label{II.1.4.1}
    \itshape
    There exists a uniformiser $t$ such that $\dd t$ generates $\Omega$.
  \end{rmenv}
  (For less hyper-generality, see \hyperref[II.1.7]{(1.7)}).

  If $t_1$ is another uniformiser, then $t_1=at$ for some $a\in\cal{O}^*$, and, by hypothesis, $\dd a$ is a multiple of $\dd t$, i.e. $\dd a=\lambda\dd t$.
  We thus have that
  \[
    \dd t_1
    = a\dd t + \dd a\cdot t
    = (a+\lambda t)\dd t
  \]
  and so
  \begin{rmenv}{1.4.2}
    \itshape
    For every uniformiser $t$, $\dd t$ generates $\Omega$.
  \end{rmenv}

  We denote by
  \[
    v\colon K^* \to \ZZ
  \]
  the valuation of $K$ defined by $\cal{O}$;
  we also denote by $v$ the valuation of $\Omega\otimes K$ defined by the lattice $\Omega$.
  If $t$ is a uniformiser, then
  \[
    v(\omega) = v(\omega/\dd t).
  \]

  If $f\in K^*$ with $f=at^n$ (for $a\in\cal{O}$), then
  \[
    \dd f = \dd a\cdot t^n + nat^{n-1}\dd t
  \]
  and thus
\oldpage{44}
  \begin{enumerate}
    \item[(1.4.3)] \label{II.1.4.3}
      $v(\dd f) \leq v(f) - 1$; and
    \item[(1.4.4)] \label{II.1.4.4}
      $v(f)\neq0 \implies v(\dd f) = v(f) - 1$.
  \end{enumerate}

  In particular, $\dd$ is continuous and extends to $\dd\colon\cal{O}^\wedge\to\Omega^\wedge$, and the triple $(\cal{O}^\wedge,\dd,\Omega^\wedge)$ again satisfies \hyperref[II.1.4.1]{(1.4.1)}.
\end{rmenv}

\begin{itenv}{Lemma 1.5}
\label{II.1.5}
  If $\cal{O}$ is complete, then the triple $(\cal{O},\dd,\Omega)$ is isomorphic to the triple $(k[[t]],\partial_t,k[[t]])$.
\end{itenv}

\begin{proof}
  The homomorphisms
  \[
    \Gr(\dd)\colon \fk{m}^i/\fk{m}^{i+1} \to \fk{m}^{i-1}\Omega/\fk{m}^i\Omega
  \]
  induced by $\dd$ are linear and bijective \hyperref[II.1.4.4]{(1.4.4)}.
  Since $\cal{O}$ is complete, $\dd\colon\fk{m}\to\Omega$ is surjective, and $\Ker(\dd)\simto k$.
  This gives us a field of representatives that is annihilated by $\dd$, and the choice of a uniformiser $t$ gives the desired isomorphism $k[[t]]\simto\cal{O}$.
\end{proof}

\begin{rmenv}{1.6}
\label{II.1.6}
  If an $\cal{O}$-algebra $\cal{O}'$ is a discrete valuation ring with a field of fractions $K'$ that is algebraic over $K$, then the derivation $\dd$ extends uniquely to $\dd\colon K'\to\Omega\otimes_{\cal{O}}K'$.
  Let $e$ be the ramification index of $\cal{O}'$ over $\cal{O}$, and let $t'$ be a uniformiser of $\cal{O}'$.
  We set
  \[
    \Omega' = 1/(t')^{e-1}\Omega\otimes_{\cal{O}}\cal{O}'.
  \]
\end{rmenv}

We can easily show, using \hyperref[II.1.6]{(1.6)}, that the triple $(\cal{O}',\dd,\Omega')$ again satisfies \hyperref[II.1.4.1]{(1.4.1)}.

\begin{rmenv}{1.7}
  We will mostly be interested in the following examples.
  Let $X$ be a non-singular complex algebraic curve, and let $x\in X$.
  We choose one of the following:
  \begin{enumerate}
    \item[(1.7.1)]\label{II.1.7.1}
      $\cal{O}=\cal{O}_{x,X}$ (the local ring for the Zariski topology), $\Omega=(\Omega_{X/\CC}^1)_x$, and $\dd=\mbox{the differential}$ ;
    \item[(1.7.2)]\label{II.1.7.2}
      $\cal{O}=\cal{O}_{x,X^\an}$ (the local ring of germs at $x$ of holomorphic functions), $\Omega=(\Omega_{X^\an/\CC}^1)_{(x)}$, and $\dd=\mbox{the differential}$ ; or
    \item[(1.7.3)]\label{II.1.7.3}
      the common completion of \hyperref[II.1.7.1]{(1.7.1)} and \hyperref[II.1.7.2]{(1.7.2)}.
  \end{enumerate}
\end{rmenv}

\oldpage{45}
\begin{rmenv}{1.8}
\label{II.1.8}
  Under the hypotheses of \hyperref[II.1.4]{(1.4)}, let $V$ be a finite-dimensional vector space over $K$, and $V_0$ a lattice in $V$, i.e. a free sub-$\cal{O}$-module of $V$ such that $KV_0=V$.
  For every homomorphism $e\colon\cal{O}^n\to V$, we define the \emph{valuation $v(e)$ of $e$} to be the largest integer $m$ such that $e(\cal{O}^n)\subset\fk{m}^m V_0$.
  If $V_0$ and $V_1$ are two lattices, then there exists an integer $s$ that is independent of $e$ and $n$ and is such that
  \[
  \label{II.1.8.1}
    |v_0(e) - v_1(e)| \leq s.
  \tag{1.8.1}
  \]
\end{rmenv}

\begin{itenv}{Theorem 1.9}
\label{II.1.9}
  \emph{[N. Katz].}
  Under the hypotheses of \hyperref[II.1.4]{(1.4)}, and with the notation of \hyperref[II.1.8]{(1.8)}, let $\nabla$ be a connection \hyperref[II.1.2]{(1.2)} on a vector space $V$ of dimension~$n$ over $K$.
  Then one of the following conditions is satisfied:
  \begin{enumerate}[a)]
    \item For any lattice $V_0$ in $V$, any basis $e\colon K^n\simto V$ of $V$, any differential form with a simple pole $\omega$ and $\tau=\omega^{-1}\in\Omega_K$, the numbers $-v(\nabla_\tau^i e)$ are bounded above ; or
    \item There exists a rational number $r>0$, with denominator at most $n$, such that, for any $V_0$, $e$, and $\tau$ as above, the family of numbers
      \[
        |-v(\nabla_\tau^i e) - ri|
      \]
      is bounded.
  \end{enumerate}
\end{itenv}

Conditions~a) and b) of \hyperref[II.1.9]{(1.9)} are more manageable in a different form:

\begin{itenv}{Lemma 1.9.1}
\label{II.1.9.1}
  Let $V_0$, $\tau$, and $e$ be as in \hyperref[II.1.9]{(1.9)}.
  Then, for any given value of $r$, the bound~b) is equivalent to
  \[
  \label{II.1.9.2}
    |\sup_{j\leq i}(-v\nabla_\tau^i e)-ri| \leq C^{te}.
  \tag{1.9.2}
  \]
  The bound~a) is equivalent to the same bound \hyperref[II.1.9.2]{(1.9.2)} for $r=0$.
\end{itenv}

\begin{proof}
  Going from \hyperref[II.1.9]{(1.9)} to \hyperref[II.1.9.2]{(1.9.2)} is clear, as is the converse for $r=0$.
  So suppose that \hyperref[II.1.9.2]{(1.9.2)} holds true for $r>0$ and some value $C_0$ for the constant.
  We have
  \[
  \label{II.1.9.a}
    -v\nabla_\tau^i e - ri \leq C_0.
  \tag{a}
  \]
  We immediately see that there exists a constant $k$ such that
  \[
    -v\nabla_\tau^n(\nabla_\tau^i e) \leq -v\nabla_\tau^i e + kn.
  \]
\oldpage{46}
  Thus
  \[
    \begin{aligned}
      -C_0 + r(i+n)
      \leq& \sup_{j\leq i+n} -v\nabla_\tau^j e
    \\=& \sup(\sup_{j\leq i} -v\nabla_\tau^j e, -v\nabla_\tau^i e+kn)
    \\\leq& \sup(C_0+ri, -v\nabla_\tau^i e+kn)
    \end{aligned}
  \]
  and, if $-C_0+r(i+n) > C_0+ri$, i.e. if $n>2C_0/r$, then we have
  \[
  \label{II.1.9.b}
    -v\nabla_\tau^i e \geq (-C_0-kn-rn)+ri.
  \tag{b}
  \]

  The inequalities \hyperref[II.1.9.a]{(a)} and \hyperref[II.1.9.b]{(b)} then imply the inequality of the form \hyperref[II.1.9]{(1.9)}:
  \[
    |-v\nabla_\tau^i e - ri| \leq C_0+kn+rn.
  \]
\end{proof}

\begin{itenv}{Lemma 1.9.3}
\label{II.1.9.3}
  Let $(V_0,\tau_0,e_0)$ and $(V_1,\tau_1,e_1)$ be two systems as in \hyperref[II.1.9]{(1.9)}.
  Then
  \[
    |\sup_{j\leq i}(-v_1\nabla_{\tau_1}^j e_1) - \sup_{j\leq i}(-v_0\nabla_{\tau_0}^j e_0)| \leq C^{te}.
  \]
\end{itenv}

\begin{proof}
  It suffices to establish an inequality like \hyperref[II.1.9.3]{(1.9.3)} when we change only one of the data $V_0$, $\tau_0$, or $e$.
  The case where we change only the reference lattice $V_0$ follows from \hyperref[II.1.8.1]{(1.8.1)}.

  We will systematically use the fact that, for $f\in K$, we have, by \hyperref[II.1.4.3]{(1.4.3)},
  \[
  \label{II.1.9.4}
    v(\partial_{\tau_i}f) \geq v(f)
  \tag{1.9.4}
  \]
  (using the notation of \hyperref[II.1.2.3]{(1.2.3)}).

  If $e$ and $f$ are two bases, then $e=fa$ for some $a\in\GL_n(K)$, whence
  \[
    \nabla_\tau^i(e) = \sum_j\binom{i}{j}\nabla_\tau^j(f)\cdot\nabla_\tau^{i-j}a,
  \]
  and, by \hyperref[II.1.9.4]{(1.9.4)},
  \[
    v(\nabla_\tau^i(e)) \geq \inf_{j\leq i}v(\nabla_\tau^j f) + C^{te}
  \]
  whence
  \[
    \sup_{j\leq i}(-v\nabla_\tau^j e) - \sup_{j\leq i}(-v\nabla_\tau^j f) \leq C^{te}.
  \]

  Reversing the roles of $e$ and $f$, we similarly have that
  \[
    \sup_{j\leq i}(-v\nabla_\tau^j f) - \sup_{j\leq i}(-v\nabla_\tau^j e) \leq C^{te}
  \]
  whence the inequality \hyperref[II.1.9.3]{(1.9.3)} for a change of basis.

\oldpage{47}
  If $\tau$ and $\omega$ are vectors as in \hyperref[II.1.9]{(1.9)}, then $\sigma=f\tau$, with $f$ invertible, whence
  \[
    \nabla_\sigma = f\nabla_\tau
    \qquad\mbox{(with $f\in\cal{O}^*$)}
  \]
  and we can show by induction that
  \[
    \nabla_\sigma^i = \sum_{j\leq i}\varphi_j\nabla_\tau^j
    \qquad\mbox{(with $\varphi_j\in\cal{O}$)}.
  \]
  From this we deduce that
  \[
    v\nabla_\sigma^i(e) \geq \inf_{j\leq i}v\nabla_\tau^j(e),
  \]
  whence
  \[
    \sup_{j\leq i}(-v\nabla_\sigma^j(e)) \leq \sup_{j\leq i}(-v\nabla_\tau^j(e)).
  \]
  Reversing the roles of $\sigma$ and $\tau$, we thus conclude that
  \[
  \label{II.1.9.5}
    \sup_{j\leq i}(-v\nabla_\sigma(e)) = \sup_{j\leq i}(-v\nabla_\tau^j(e)).
  \tag{1.9.5}
  \]
\end{proof}

By \hyperref[II.1.9.1]{(1.9.1)} and \hyperref[II.1.9.3]{(1.9.3)} it suffices, to prove \hyperref[II.1.9]{(1.9)}, to prove an upper bound of the form \hyperref[II.1.9.2]{(1.9.2)} for \emph{one} choice of $(V_0,\tau,e)$.

\begin{itenv}{Lemma 1.9.6}
\label{II.1.9.6}
  Under the hypotheses of \hyperref[II.1.9]{(1.9)}, let $e\colon K^n\to V$ be a basis of $V$, $t$ a uniformiser, $\omega$ a differential form presenting a simple pole (i.e. a basis of $t^{-1}\Omega$), set $\tau=\omega^{-1}\in\Omega$, and let $\Gamma=(\Gamma_j^i)$ be the connection matrix in the bases $e$ and $\omega$.
  Let $s$ and $(r_i)_{1\leq i\leq n}$ be rational numbers, and set $r_{i,j}=s+r_i-r_j$, and suppose that
  \[
    -v(\Gamma_j^i) \leq r_{i,j}.
  \]
  Finally, let $\gamma\in \MM_n(k)$ be the matrix whose coefficients are ``$t^{r_{i,j}}\Gamma_j^i\mod m$'', i.e.
  \[
    \gamma_j^i =
    \begin{cases}
      0 & \mbox{if $-v(\Gamma_j^i)<r_{i,j}$ ;}
    \\t^{r_{i,j}}\Gamma_j^i\mod m & \mbox{if $-v(\Gamma_j^i)=r_{i,j}$.}
    \end{cases}
  \]

  Suppose that $s\leq0$, or that $\gamma$ is not nilpotent.
  Then the upper bound \hyperref[II.1.9.2]{(1.9.2)} is satisfied for $r=\sup\{s,0\}$.
\end{itenv}

\begin{proof}
  Let $N$ be an integer such that the $r_iN$ are integers, and set $\cal{O}'=\cal{O}(\sqrt[N]{t})$.
  Let $K'$ be the field of fractions of $K$, $v\colon {K'}^*\to\frac1N\ZZ$ the valuation of $K'$ that extends $v$,
\oldpage{48}
  and $\Lambda$ the diagonal matrix with coefficients being the $t^{-r_i}$.

  On $\cal{O}'$, let $\omega'$ be the basis of $\Omega\otimes K'$ given by the inverse image of $\omega$, let $\tau'$ be the corresponding basis of ${\Omega'}^\vee\otimes K'$, and let $e'=e\Lambda$ be a new basis of $V'=V\otimes K'$.
  In these bases, the connection matrix is
  \[
    \Gamma' = \Lambda^{-1}\Gamma\Lambda + \Lambda^{-1}\partial_{\tau'}\Lambda.
  \]

  The matrix $\Lambda^{-1}\partial_{\tau'}\Lambda$ has coefficients in $\cal{O}'$, and so either
  \begin{enumerate}[(a)]
    \item $s\leq0$, and $\Gamma'$ has coefficients in $\cal{O}'$ ; or
    \item $s>0$, $-v(\Gamma')=s$, and the ``most polar part'' $\gamma$ of $\Gamma'$ is not nilpotent, so that $-v({\Gamma'}^\ell)=\ell s$.
  \end{enumerate}

  By the definition of $\Omega'$ \hyperref[II.1.6]{(1.6)}, $\omega'$ presents a simple pole.
  In case~(a), we thus conclude by induction on $\ell$ that
  \[
    v(\nabla_{\tau'}^\ell,e') \geq 0.
  \]

  We can prove by induction on $m$ that, in the basis $e'$,
  \[
    \nabla_{\tau'}^m = \sum_{0\leq k\leq n}({\Gamma'}^{m-i}+\Delta_i)\partial_\tau^i
  \]
  where $\Delta_i$ is the algebraic sum of the products of at most $m-i-1$ factors $\partial_{\tau'}^\ell\Gamma$.
  In particular,
  \[
    \nabla_\tau^m e' = {\Gamma'}^m+\Delta_m
  \]
  and, in case~(b),
  \[
    -v(\nabla_{\tau'}^m,e') = ms.
  \]

  This satisfies \hyperref[II.1.9.2]{(1.9.2)} on $\cal{O}'$ (for suitable bases), and \hyperref[II.1.9.6]{(1.9.6)} then follows from \hyperref[II.1.9.3]{(1.9.3)}.
\end{proof}

Theorem~\hyperref[II.1.9]{(1.9)} follows from the following proposition and from \hyperref[I.1.3]{(I.1.3)}.

\begin{itenv}{Proposition 1.10}
\label{II.1.10}
  Under the hypotheses of \hyperref[II.1.9]{(1.9)}, let $X$ be a basis of $\Omega^\vee$, $t$ a uniformiser, set $\tau=tX$, and let $v$ be a cyclic vector \hyperref[II.1.2]{(1.2)} of $V$.
  Set
\oldpage{49}
  \[
    \begin{aligned}
      \nabla_X^n v &= \sum_{i<n}s_i\nabla_X^i v
    \\\nabla_\tau^n v &= \sum_{i<n}b_i\nabla_\tau^i v.
    \end{aligned}
  \]
  Then the bound \hyperref[II.1.9.2]{(1.9.2)} holds true for
  \[
    \begin{aligned}
      r
      &= \sup\big\{0,\sup\{-v(b_i)/n-i\}\big\}
    \\&= \sup\big\{0,\sup\{-v(a_i)/n-i\}-1\big\}.
    \end{aligned}
  \]
  The same conclusion holds for a cyclic vector $v$ of $V^\vee$.
\end{itenv}

This proposition gives us a procedure for calculating $r$ for a given vector bundle $V$ with a connection defined by an $n$\textsuperscript{th} order differential equation (cf. \hyperref[I.4.8]{(I.4.8)}).

\begin{proof}
  We have the identities
  \[
    \begin{aligned}
      (t\nabla_X)^n-\sum b_i(t\nabla_X)^i &= t^n\big(\nabla_X^n-\sum a_i\nabla_X^i\big)
    \\(t^{-1}\nabla_\tau)^n-\sum a_i(t^{-1}\nabla_\tau)^i &= t^{-n}\big(\nabla_\tau^n-\sum b_i\nabla_\tau^i\big).
    \end{aligned}
  \]
  From these identities, we see that
  \[
    \begin{aligned}
      a_i &= g_{n,i}+\sum_{j\geq i}g_{j,i}b_j,
      \quad v(g_{j,i})\geq i-n
    \\b_i &= h_{n,i}+\sum_{j\geq i}h_{j,i}a_j,
      \quad v(h_{i,j})\geq n-j
    \end{aligned}
  \]
  and, for $i\geq0$,
  \[
    \sup\{0,\sup_{j\geq i}\{-v(b_j)\}\} = \sup\{0,\sup_{j\geq i}\{-v(a_j)-(n-j)\}\}.
  \]
  The two expressions given for $r$ thus agree.

  If $v\in V$ is a cyclic vector, then the matrix of the connection, in the bases $(\nabla_\tau^i v)_{0\leq i\leq n}$ of $V$ and $\tau$ of $\Omega^\vee$, is
  \oldpage{50}
  \[
    \Gamma =
    \left(
      \def\arraystretch{1.8}
      \begin{array}{rrrrrr}
        0 & 0 & 0 & 0 & \cdots & b_0
      \\1 & 0 & 0 & 0 & \cdots & b_1
      \\0 & 1 & 0 & 0 & & \vdots
      \\\vdots & & \ddots & \ddots & & \vdots
      \\\vdots & & & 1 & 0 & b_{n-2}
      \\0 & \cdots & \cdots & \cdots & 1 & b_{n-1}
      \end{array}
    \right).
  \]
  If $v\in V^\vee$ is a cyclic vector, then the matrix of the connection, in the basis of $V$ given by the dual of $(\nabla_\tau^iv)_{0\leq i\leq n}$ and the basis $\tau$ of $\Omega^\vee\otimes K$, is
  \[
    \Gamma =
    -\left(
      \def\arraystretch{1.8}
      \begin{array}{rrrrrr}
        0 & 1 & 0 & 0 & \cdots & 0
      \\0 & 0 & 1 & 0 & & \vdots
      \\0 & 0 & 0 & 1 & & \vdots
      \\\vdots & \vdots & & \ddots & \ddots & \vdots
      \\\vdots & \vdots & & & 0 & 1
      \\b^0 & b^1 & \cdots & \cdots & b^{n-2} & b^{n-1}
      \end{array}
    \right)
  \]

  It remains only to apply \hyperref[II.1.9.6]{(1.9.6)}.
  For $v\in V$, we take $r_i=-ri$ and $s=r$.
  For $v\in V^\vee$, we take $r_i=ri$ and $s=r$.
  In the first (resp. second) case, if $s=r>0$, then the matrix $\gamma$ is of the type
  \[
    \gamma =
    \left(
      \def\arraystretch{1.8}
      \begin{array}{rrrrrr}
        0 & 0 & 0 & 0 & \cdots & *
      \\1 & 0 & 0 & 0 & \cdots & *
      \\0 & 1 & 0 & 0 & & \vdots
      \\\vdots & & \ddots & \ddots & & \vdots
      \\\vdots & & & 1 & 0 & *
      \\0 & \cdots & \cdots & \cdots & 1 & *
      \end{array}
    \right)
  \]
  with one of the coefficients of the last column being non-zero (resp. of the type given by the transpose of this).
  The coefficients are those of the characteristic polynomial of $\gamma$, which is thus not nilpotent for $s>0$.
\end{proof}

\begin{rmenv}{Definition 1.11}
\label{II.1.11}
  Under the hypotheses of \hyperref[II.1.9]{(1.9)}, we say that the connection $\nabla$ is \emph{regular} if condition~a) of \hyperref[II.1.9]{(1.9)} is satisfied.
\end{rmenv}

\begin{itenv}{Theorem 1.12}
\label{II.1.12}
  \emph{[N. Katz].}
  Under the hypotheses of \hyperref[II.1.9]{(1.9)}, we have the following:
  \begin{enumerate}[(i)]
    \item For the connection $\nabla$ to be regular, it is necessary and sufficient for $V$ to admit
\oldpage{51}
      a basis $e$ such that the matrix of the connection, in this basis, is a matrix of differential forms presenting, at worst, simple poles.
    \item For the connection $\nabla$ to be irregular, and to satisfy a bound of the form \hyperref[II.1.9.2]{(1.9.2)} for $r=a/b>0$, it is necessary and sufficient for $V$ to admit a basis $e$ (after a change of rings from $\cal{O}$ to $\cal{O}'=\cal{O}(\sqrt[b]{t})$, and for the natural valuation, with values in $\ZZ$, of $\cal{O}'$) such that the matrix of the connection, in this basis, presents a pole of order $a+1$, and for the polar part of order $a+1$ of this matrix (which is in $\MM_n(k)$ and determined up to a scalar multiple) to be non-nilpotent.
  \end{enumerate}
\end{itenv}

\begin{proof}
  By extension of scalars, the number $r$ such that $\nabla$ satisfies \hyperref[II.1.9.2]{(1.9.2)} is multiplied by the ramification index.
  This leads us to the case where $b=1$.
  Conditions~(i) and (ii) are then sufficient, by \hyperref[II.1.9.6]{(1.9.6)}.
  Conversely, let $v$ be a cyclic vector \hyperref[II.1.3]{(1.3)}, $t$ a uniformiser, and $\tau\in\Omega^\vee$ of valuation~$1$.
  Then it follows from the proofs of \hyperref[II.1.9.6]{(1.9.6)} and of \hyperref[II.1.10]{(1.10)} that the basis $e_i=t^{ri}\nabla_\tau^i v$ (for $0\leq i\leq\dim V$) satisfies (i) or (ii).
\end{proof}

\begin{itenv}{Proposition 1.13}
\label{II.1.13}
  \begin{enumerate}[(i)]
    \item For every horizontal exact sequence
      \[
        V' \to V \to V'',
      \]
      if the connections on $V'$ and $V''$ are regular, then the connection on $V$ is regular.
    \item If the connections on $V_1$ and $V_2$ are regular, then the natural connections of
      \[
        V_1\otimes V_2,
        \quad \Hom(V_1,V_2),
        \quad V_1^\vee,
        \quad \bigwedge^p V_1,
        \quad \ldots
      \]
      are regular.
    \item If $\cal{O}'$ is a discrete valuation ring with field of fractions $k'$ that is algebraic over the field of fractions $K$ of $\cal{O}$, and if $V'=V\otimes_K K'$, then the connection on $V'$ is regular if and only if the connection on $V$ is regular.
  \end{enumerate}
\end{itenv}

\begin{proof}
  Claim~(iii), already utilised in \hyperref[II.1.12]{(1.12)}, follows for example from the calculation in \hyperref[II.1.10]{(1.10)} and from the fact that the inverse image of a differential form presenting a simple pole again presents a simple pole.

\oldpage{52}
  Claim~(ii) follows immediately from criterion~(i) in \hyperref[II.1.12]{(1.12)}.

  Claim~(i) implies that, for every short exact sequence
  \[
    0 \to V' \to V \to V'' \to 0,
  \]
  $V$ is regular if and only if $V'$ and $V''$ are regular.
  After an eventual extension of scalars, we choose bases $e'$ and $e''$ of $V'$ and $V''$ (respectively) satisfying (i) or (ii) in \hyperref[II.1.12]{(1.12)}.
  Lift $e''$ to a family of vectors $e''_0$ of $V$.
  For large enough $N$, the basis $e'\cup t^{-N}e''_0$ of $V$ will satisfy (i) in \hyperref[II.1.12]{(1.12)} if $e'$ and $e''$ satisfy (i) in \hyperref[II.1.12]{(1.12)}, and will satisfy (ii) in \hyperref[II.1.12]{(1.12)} in the contrary case.
\end{proof}

\begin{rmenv}{1.14}
\label{II.1.14}
  Let $S$ be a Riemann surface, $p\in S$, and $z$ a uniformiser at $p$.
  We denote by $j$ the inclusion of $S^*=S\setminus\{p\}$ in $S$.
  We say that a (holomorphic) vector bundle on $S^*$ is \emph{meromorphic} at $p$ if we have the data of
  \begin{enumerate}[(i)]
    \item a vector bundle $V$ on $S^*$ ; and
    \item an equivalence class of extensions of $V$ to a vector bundle on $S$, with two extensions $V_1$ and $V_2$ being equivalent if there exists an integer $n$ such that $z^nV_1\subset V_2\subset z^{-n}V_1\subset j_*V$.
  \end{enumerate}

  Such a bundle defines a vector space $V_K$ over the field of fractions $K$ of the local ring $\cal{O}_{p,S}$.
  We talk of a \emph{basis} of $V$ to mean a basis that extends to a basis of one of the permissible extensions of $V$.
  It is clear that $V$ admits bases of this type on a neighbourhood of $p$.
  A connection $\nabla$ on $V$ is said to be \emph{meromorphic} at $p$ if its coefficients (in any basis of $V$) are meromorphic at $p$.
  Such a connection defines a connection \hyperref[II.1.2]{(1.2)} on $V_K$ (cf. \hyperref[II.1.7.2]{(1.7.2)}).
  We say that a connection $\nabla$ on $V$ is \emph{regular} at $p$ if it is meromorphic at $0$ and if the induced connection on $V_K$ is regular in the sense of \hyperref[II.1.11]{(1.11)}, i.e. if there exists a basis of $V$ close to $p$ in which the matrix of the connection presents at most a simple pole at $p$ \hyperref[II.1.12]{(1.12)}.
\end{rmenv}

\begin{rmenv}{1.15}
\label{II.1.15}
  Let $D$ be the open unit disc
  \[
    D = \{z : |z|<1\}
  \]
  and let $D^*=D\setminus\{0\}$.
  The group $\pi_1(D^*)$ is cyclic, generated by the loop
\oldpage{53}
  $t\mapsto\lambda e^{2\pi it}$ (for $0\leq t\leq1$).
  The fundamental groupoid is thus the constant group $\ZZ$.
  It acts on any local system on $D^*$.
  Using the dictionary \hyperref[I.2]{(I.2)}, every vector bundle $V$ with connection is thus endowed with an action of the local fundamental group $\ZZ$.
  The generator $T$ of this action is called the \emph{monodromy transformation}.
\end{rmenv}

\begin{rmenv}{1.16}
\label{II.1.16}
  Let $V$ be a vector bundle on $D$, and $\nabla$ a connection on $V|D^*$ that is meromorphic at $0$.
  If $e_1$ (resp. $e_2$) is a basis of $V$ in which $\nabla$ is represented by $\Gamma_1\in\Omega^1(\End(V|D^*))$ (resp. $\Gamma_2$), then the difference $\Gamma_1-\Gamma_2$ is holomorphic at $0$.
  Thus the polar part of $\Gamma$ does not depend on the choice of $e$.

  Suppose that $\Gamma_i$ presents only a simple pole at $0$, and thus has ``polar part'' equal to some element $\gamma_i$ in
  \[
    \HH^0\left( \left(\frac1z\Omega^1/\Omega^1\right)\otimes\shEnd(V) \right).
  \]
  The ``residue'' map $\HH^0((1/z)\Omega^1/\Omega^1)\to\CC$ then associates to $\gamma_i$ an endomorphism of the fibre $V_0$ of $V$ at $0$.
  We call this endomorphism the \emph{residue} of the connection at $0$, and denote it by
  \[
    \Res(\Gamma_i)\in\End(V_0).
  \]
\end{rmenv}

\begin{itenv}{Theorem 1.17}
\label{II.1.17}
  Under the hypotheses of \hyperref[II.1.16]{(1.16)}, the monodromy transformation $T$ extends to an automorphism of $V$ whose fibre at $0$ is given by
  \[
    T_0 = \exp(-2\pi i\Res(\Gamma)).
  \]
\end{itenv}

\begin{proof}
  We can take $V=\cal{O}^n$;
  the differential equation for the horizontal sections is then
  \[
    \partial_z v = -\Gamma v,
  \]
  and the differential equation for a horizontal basis $e\colon\cal{O}^n\to V$ is thus
  \[
  \label{II.1.17.proof.1}
    \partial_z e = -\Gamma\circ e.
  \tag{1}
  \]

  In polar coordinates $(r,\theta)$,
  \[
    \begin{aligned}
      z &= re^{i\theta}
    \\\dd z &= rie^{i\theta}\dd\theta+\dd re^{i\theta},
    \end{aligned}
  \]
  and this equation gives
\oldpage{54}
  \[
    \partial_\theta e = -ir\Gamma\circ e.
  \]
  Set $\Gamma=\frac{\Gamma_0}{z}+\Gamma_1$, where $\Gamma_0$ is constant and $\Gamma_1$ is holomorphic.
  The above equation can then be rewritten as
  \[
    \partial_\theta e = -(ie^{-i\theta}\Gamma_0+ir\Gamma_1)e.
  \]

  The monodromy transformation at $(r,\theta)$ is the value at $(r,\theta+2\pi)$ of the solution of this differential equation which is the identity at $(r,\theta)$.
  As $r\to0$, the aforementioned solution tends to the solution of the limit equation
  \[
  \label{II.1.17.proof.2}
    \partial_\theta e = -ie^{-i\theta}\Gamma_0\circ e.
  \tag{2}
  \]

  We thus deduce that $T$ has a limit value as $z\to0$, for $\theta$ fixed, and that this value depends continuously on $\theta$.
  In particular, $T$ is bounded near $0$, and thus extends to an endomorphism $T$ of $V$ on $D$.
  We conclude that $T$ has a limit value as $z\to0$;
  this value, given by integrating \hyperref[II.1.17.proof.1]{(1)}, depends only on $\Gamma_0$.
  To calculate this limit value, it suffices to calculate it for an arbitrary connection $\Gamma'$ that has the same residue as $\Gamma$.

  For example, we can prove:
  \begin{itenv}{Lemma 1.17.1}
  \label{II.1.17.1}
    Let $\nabla$ be the connection on $\cal{O}^n$ given by the matrix $U\cdot\frac{\dd z}{z}$ for some $U\in\GL_n(\CC)$.
    Then the general solution of the equation $\nabla e=0$ is
    \[
      e = \exp(-\log z\cdot U)f = \todo
    \]
    and thus the monodromy is the automorphism of $\cal{O}^n$ given by the constant matrix $\exp(-2\pi iU)$.
  \end{itenv}

  \begin{itenv}{Corollary 1.17.2}
  \label{II.1.17.2}
    Under the above hypotheses, the automorphism $\exp(-2\pi i\Res(\Gamma))$ of the fibre of $V$ at $0$ is the limit of the conjugates of the monodromy automorphism.
  \end{itenv}

  Note that it is not true in general that $T_0$ is conjugate to $T_x$ for $x$ close to $0$.
  For example, if $\nabla$ is the connection on $\cal{O}^2$ for which
  \[
    \nabla\begin{pmatrix}u\\v\end{pmatrix}
    = \\d\begin{pmatrix}u\\v\end{pmatrix} + \begin{pmatrix}0&0\\0&-1\end{pmatrix}\begin{pmatrix}u\\v\end{pmatrix}\frac{\dd z}{z} + \begin{pmatrix}0&0\\1&0\end{pmatrix}\begin{pmatrix}u\\v\end{pmatrix},
  \]
  then the general horizontal section is
\oldpage{55}
  \[
    \begin{aligned}
      u &= a
    \\v &= az\log z+bz
    \end{aligned}
  \]
  and the monodromy transform is
  \[
    T = \begin{pmatrix}1&2\pi iz\\0&1\end{pmatrix}.
  \]

  However, it follows from \hyperref[II.1.17.2]{1.17.2} that \emph{$T$ and $T_0$ have the same characteristic polynomial}.
  See also \hyperref[II.5.6]{5.6}.
\end{proof}

\begin{rmenv}{1.18}
\label{II.1.18}
  Let $f$ be a multiform function on $D^*$.
  Let $D_1$ be the disc $D^*$ minus the ``cut'' $\RR^+\cap D^*$.
  We say that \emph{$f$ is of moderate growth at $0$} if all the determinations of $f$ on $D_1$ grow as $1/r^n$ for some suitable $n$:
  \[
    f \leq A|z|^{-n}.
  \]
  Here we allow $n$ to vary with the determination.
  It is evident, however, that, for $f$ of moderate growth and of finite determination, there exists some $n$ that works for all the determinations.
  The fact that $f$ is of moderate growth also implies that the function $f(e^{2\pi iz})$ is of at most exponential order in each vertical strip.

  If $f$ is a multiform section of a vector bundle $V$ on $D^*$ that is meromorphic at $0$, then we say that \emph{$f$ is of moderate growth at $0$} if its coordinates, in an arbitrary base of $V$ near $0$, are of moderate growth.
\end{rmenv}

\begin{itenv}{Theorem 1.19}
  Let $V$ be a vector bundle on $D^*$ that is meromorphic at $0$, endowed with a connection $\nabla$.
  Then the following conditions are equivalent:
  \begin{enumerate}[(i)]
    \item $\nabla$ is regular; and
    \item the (multiform) horizontal sections of $V$ are of moderate growth at $0$.
  \end{enumerate}
\end{itenv}

\begin{proof}
  \mbox{(i)$\implies$(ii).}
  Choose, near to $0$, an isomorphism $V\sim\cal{O}^n$, under which
\oldpage{56}
  the differential equation for the horizontal sections is of the form
  \[
    \partial_z v = \Gamma v,
  \]
  where $\Gamma$ has at most a simple pole at $0$.
  We then have, for $|z|\leq\lambda<1$,
  \[
    |\partial_z v| \leq \frac{k}{|z|}|v|
  \]
  and, on $D_1$ \hyperref[II.1.18]{(1.18)}, this inequality integrates, for $|z|\leq\lambda$, to
  \[
    |v| \leq \frac{1}{|z|^k}\sup_{|z|=\lambda}|v|.
  \]

  \mbox{(ii)$\implies$(i).}
  Let $T$ be the monodromy transformation of $V$, and let $U\in\GL_n(\CC)$ be a matrix such that $\exp(2\pi iU)$ is conjugate to $T$.
  Let $V_0$ be the vector bundle $\cal{O}^n$ endowed with the regular connection with matrix
  \[
    \Gamma = \frac{U}{z}.
  \]
  The bundles $V$ and $V_0$ have the same monodromy.
  By the dictionaries in \hyperref[I.1]{I.1} and \hyperref[I.2]{I.2}, they are thus isomorphic as bundles with connections on $D^*$.
  Let
  \[
    \varphi\colon V_0|D^* \to V|D^*
  \]
  be an isomorphism.
  It suffices to prove that $\varphi$ is compatible with the structures on $V_0$ and $V$ of bundles that are meromorphic at $0$;
  this is the case if and only if $\varphi$ is of moderate growth at $0$.
  Let $e$ be a (multiform) horizontal basis of $V_0|D^*$, and let $f$ be a (multiform) horizontal basis of $V|D^*$.
  \[
  \label{II.1.19.proof.i}
    \begin{tikzcd}
      V_0 \rar
      & V
    \\\cal{O}^n \uar["e"] \rar
      & \cal{O}^n \uar[swap,"f"]
    \end{tikzcd}
  \tag{i}
  \]

  The morphism $f$ is, by hypothesis, of moderate growth.
  The morphism $e^{-1}$ has horizontal sections of the regular bundle $V_0^\vee$ as its coordinates, and is thus of moderate growth.
  The morphism $\psi$ that makes \hyperref[II.1.19.proof.i]{diagram~(i)} commute is horizontal with respect to the usual connection on $\cal{O}^n$, and is thus constant.
  The composite $\varphi= f\psi e^{-1}$ is thus of moderate growth.
\end{proof}

\oldpage{57}
\begin{itenv}{Corollary 1.20}
\label{II.1.20}
  Let $V_1$ and $V_2$ be vector bundles on $D^*$ that are meromorphic at $0$, and endowed with regular connections $\nabla_1$ and $\nabla_2$ (respectively).
  Then every horizontal homomorphism $\varphi\colon V_1\to V_2$ is meromorphic at zero.
  In particular, $V_1$ and $V_2$ are isomorphic if and only if they have the same monodromy.
\end{itenv}

\begin{proof}
  Indeed, $\varphi$, thought of as a section of $\shHom(V_1,V_2)$, is horizontal, and thus of moderate growth, since the connection on $\shHom(V_1,V_2)$ is regular.
\end{proof}

\begin{rmenv}{1.21}
\label{II.1.21}
  Let $X$ be a smooth algebraic curve over a field $k$ of characteristic~$0$, and $V$ a vector bundle on $X$ endowed with a connection
  \[
    \nabla\colon V \to \Omega_{X/k}^1(V).
  \]
  Let $\overline{X}$ be the smooth projective curve given by the completion of $X$, and $x_\infty\in\overline{X}\setminus X$ a ``point at infinity'' of $X$.
  The local ring $\cal{O}_{x_\infty}$, endowed with
  \[
    \dd\colon \cal{O}_{x_\infty} \to \Omega_{x_\infty}^1
  \]
  satisfies \hyperref[II.1.4.1]{(1.4.1)}, and $V$ induces a vector space $V_K$ endowed with a connection, in the sense of \hyperref[II.1.2]{(1.2)}, over the field of fractions $K$ of $\cal{O}_{x_\infty}$ (which is equal to the field of functions of $X$ in the case when $X$ is connected).
  We say that the connection on $V$ is \emph{regular at $x_\infty$} if this induced connection on $V_K$ is \emph{regular} in the sense of \hyperref[II.1.10]{(1.10)}.

  If $\overline{X}_1$ is an arbitrary curve that contains $X$ as a dense open subset, and if $S\subset\overline{X}_1\setminus X$, then we say that the connection $\nabla$ is \emph{regular at $S$} if it is regular at all points of the inverse image of $S$ in $\overline{X}$ (which makes sense, since the normalisation of $\overline{X}_1$ can be identified with an open subset of $\overline{X}$).

  Finally, we say that the connection $\nabla$ is \emph{regular} if it is regular at all points at infinity of $X$.
\end{rmenv}

\begin{rmenv}{1.22}
\label{II.1.22}
  If $k=\CC$, then every vector bundle $V$ on $X$ can be extended to a vector bundle on the completed curve $\overline{X}$.
  If $V_1$ and $V_2$ are two extensions of $V$, and if $t$ is a uniformiser at a point $x_\infty\in\overline{X}\setminus X$, then there exists some $N\in\NN$ such that, in a neighbourhood of $x_\infty$, the subsheaves $V_1$ and $V_2$ of the direct image of $V$
\oldpage{58}
  satisfy
  \[
    t^N V_1\subset V_2 \subset t^{-N}V_1.
  \]

  The bundle $V^\an$ is thus canonically endowed with a structure that is meromorphic at every $x_\infty\in\overline{X}\setminus X$.

  If $V$ is endowed with a connection, then we can immediately see \hyperref[II.1.12]{(I.12)} that $(V,\nabla)$ is regular at $x_\infty\in\overline{X}\setminus X$, in the sense of \hyperref[II.1.21]{(1.21)}, if and only if $(V^\an,\nabla)$ is regular at $x_\infty$, in the sense of \hyperref[II.1.14]{(1.14)}.
\end{rmenv}

\medskip
\hrule
\medskip

\emph{[Translator] Theorem~1.23 and Proposition~1.24 have been removed from this edition, due to the following comment from the errata:}
\begin{quote}
  \itshape
  I thank B.~Malgrange for having shown me that the ``theorem'' in (II.1.23) is false.
  We incorrectly suppose, in the proof, that the vector field $\tau$ has no poles.
  The statement of (II.1.23) was used in the proof of the key theorem (II.4.1), and only there.
\end{quote}

\medskip
\hrule
\medskip

% The following are removed by the errata

% \begin{itenv}{Theorem 1.23}
% \label{II.1.23}
%   Consider a commutative diagram
%   \[
%     \begin{tikzcd}
%       X \ar[rr,"j"] \ar[dr,swap,"f"]
%       && \overline{X} \ar[dl,"\overline{f}"]
%     \\&S&
%     \end{tikzcd}
%   \]
%   in which
%   \begin{enumerate}[a)]
%     \item $S$ is a Noetherian scheme of characteristic~$0$, and $j$ is an open immersion of finite type of $S$-schemes;
%     \item $f$ is smooth of pure relative dimension~$1$; and
%     \item $T=\overline{X}\setminus X$ is quasi-finite over $S$.
%   \end{enumerate}

%   Let $V$ be a vector bundle on $X$ endowed with a relative connection $\nabla\colon V\to\Omega_{X/S}^1(V)$.
%   Then the set of points $s\in S$ such that the restriction of $(V,\nabla)$ to the fibre $X_s$ of $X$ at $s$ is regular at $T_s$ is closed in $S$.
% \end{itenv}

% \begin{proof}
%   It is clear, by \hyperref[II.1.12]{(1.12)}, that the set in question in constructable (as is its complement, which satisfies the property that it contains, for any point $\eta$, a neighbourhood of $\eta$ in the closure $\overline{\eta}$).
%   It remains to show that it is stable under specialisation, which can be shown by proving that, if $S$ is the spectrum of a discrete valuation ring, with generic point $\eta$ and closed point $s$, and if $(V_\eta,\nabla)$ on $X_\eta$ is regular at $T_\eta$, then $(V_s,\nabla)$ on $X_s$ is regular at $T_s$.
%   By replacing $\overline{X}$ with its normalisation, if necessary, we can suppose that $\overline{X}$ is flat over $S$ and normal.

%   Let $x\in T_s$.
%   Let $\overline{X}'$ be an affine neighbourhood of $x$ in $\overline{X}$ such that the
% \oldpage{59}
%   restriction of $V$ to $\overline{X}'\cap X_s$ is free, with basis $e_i$, and such that there exists an open affine subset $X''$ of $X$ such that $X''_s=\overline{X}'\cap X_s$ (for there to exist such an $X''$, it suffices to take $\overline{X}$ to be small enough such that $\overline{X}_s\setminus X_s$ is defined by one single equation, for example).
%   We lift $e_i$ to a section $\widetilde{e}_i$ of $V$ over $X''$, and we let $X'$ be the open subset of $X''$ on which $(\widetilde{e}_i)$ is a basis.

%   The hypotheses of \hyperref[II.1.23]{(1.23)} are again satisfied for $X'\hookrightarrow\overline{X}'$, and $V|X'_\eta$ is regular at $\overline{X}'\setminus X'$.
%   To show that $V|X'_s=V|X_s$ is regular at $x$, we can thus restrict to the case where $V$ is free;
%   we can thus suppose that $V$ extends \todo{(or is the extension of?)} to a vector bundle on $\overline{X}$, with basis $(e)$.

%   Let $f$ be a section of $\cal{O}_{\overline{X}}$ that is non-constant on $X_s$ and zero on $\overline{X}\setminus X$.
%   Let $\tau$ be the relative vector field such that $\langle\tau,\dd f/f\rangle=1$.

%   By construction, the vector field $\tau$ induces, on the normalisation $\overline{X}_s^\mathrm{n}$ of $\overline{X}_s$, a vector field that vanishes simply on $\overline{X}_s\setminus X_s$.
%   To show that $V|X_s$ is regular at $x$, it thus suffices to show that there exists some $n$ such that the
%   \[
%     f^n\nabla_\tau^i e_k|\overline{X}_s
%   \]
%   (for $i\geq0$) are all regular.
%   By hypothesis, there exists some $n$ such that the
%   \[
%     f^n\nabla_\tau^i e_k|\overline{X}_\eta
%   \]
%   are regular.
%   The $f^n\nabla_\tau^i e_k$ are thus regular over $\overline{X}_\eta\cup X_s$, and so the complement is of dimension~$2$;
%   since $\overline{X}$ is normal, the $f^n\nabla_\tau^i e_k$ are automatically everywhere regular, and thus in particular over $\overline{X}_s$.
% \end{proof}

% We can similarly prove the following analytic variant of \hyperref[II.1.23]{(1.23)}.

% \begin{itenv}{Proposition 1.24}
%   Consider a commutative diagram
%   \[
%     \begin{tikzcd}
%       X \ar[rr,"j"] \ar[dr,swap,"f"]
%       && \overline{X} \ar[dl,"\overline{f}"]
%     \\&D&
%     \end{tikzcd}
%   \]
%   in which
% \oldpage{60}
%   \begin{enumerate}[a)]
%     \item $D$ is the unit disc;
%     \item $f$ is smooth of pure relative dimension~$1$; and
%     \item $j$ is an open immersion, and $T=\overline{X}\setminus X$ is an analytic subspace that is quasi-finite over $D$.
%   \end{enumerate}

%   Let $V$ be a vector bundle on $X$, extending to a coherent analytic sheaf on $\overline{X}$, and endowed with a relative connection on $X$.
%   For all $\lambda$, the restriction of $V$ to $f^{-1}(\lambda)$ is endowed with a structure that is meromorphic \hyperref{II.1.14}{(1.14)} at all points in the inverse image of $T$ in the Riemann surface given by the normalisation of $\overline{f}^{-1}(\lambda)$.
%   If, for $\lambda\neq0$, $V|f^{-1}(\lambda)$ is regular at these points, then $V|f^{-1}(0)$ has the analogous property.
% \end{itenv}


\section{Growth conditions}
\label{II.2}

\begin{rmenv}{2.1}
\label{II.2.1}
  Let $X^*$ be a separated scheme of finite type over $\CC$.
  By Nagata \cite{20} (see also EGA~II, 2nd edition), $X^*$ can be represented by a dense Zariski open subset of a scheme $X$ that is proper over $\CC$ (here, proper = complete = compact).
  Furthermore, if $X_1$ and $X_2$ are two ``compactifications'' of $X^*$, then there exists a third compactification $X_3$ along with two commutative diagrams
  \[
    \begin{tikzcd}
      X^* \rar[hook] \dar[equal]
      & X_3 \dar
    \\ X^* \rar[hook]
      & X_i
    \end{tikzcd}
  \]
  We can take $X_3$ to be the scheme-theoretic closure of the diagonal image of $X$ in $X_1\times X_2$.

  This makes schemes over $\CC$ much more well behaved at infinity than analytic varieties are.
  Often, an algebraic object or construction with respect to some scheme $X^*$ can be seen as an analogous analytic object or construction, plus a ``growth condition at infinity''.
\end{rmenv}

\oldpage{61}
\begin{rmenv}{2.2}
\label{II.2.2}
  Let $X$ be a separated complex analytic space, and $Y$ a closed analytic subset of $X$.
  Let $X^*=X\setminus Y$, and let $j\colon X^*\hookrightarrow X$ be the inclusion morphism of $X^*$ into $X$.
  In what follows, $X$ is seen as a ``partial compactification'' of $X^*$, with $Y$ being the ``at infinity''.
  We will not lose much generality in supposing that $X^*$ is dense in $X$.
\end{rmenv}

\begin{rmenv}{2.3}
\label{II.2.3}
  Suppose that $X^*$ is smooth, and that $X$ admits an embedding into some $\CC^n$ (or, more generally, into some smooth analytic space).
  If $j_1$ and $j_2$ are embeddings of $X$ into $\CC^{n_i}$, then the two Riemannian structures on $X^*$ induced by $\CC^{n_i}$, say $j_1^*g$ and $j_2^*$, satisfy the following:

  \begin{itenv*}
  \label{II.2.3.*}
    For every compact subset $K$ of $X$, there exist constants $A,B>0$ such that
    \[
      j_1^*g \leq Aj_2^*g \leq Bj_2^*g \leq Bj_1^*g.
    \tag{$*$}
    \]
    on $K\cap X^*$.
  \end{itenv*}

  To see this, we compare $j_i^*g$ with $j_3^*g$, where $j_3$ is the diagonal embedding of $X$ into $\CC^{n_1+n_2}$.
  Locally on $X$, we have $j_3=\alpha\cdot j_i$, where $\alpha\colon\CC^{n+i}\to\CC^{n_1+n_2}$ is a holomorphic section of $\pr_i$, and the claim then follows.

  The compactification $X$ of $X^*$ thus defines an equivalence class under \hyperref[II.2.3.*]{($*$)} of Riemannian structures on $X^*$.

  Now suppose only that $X^*$ is smooth.
  A \emph{Riemannian structure $g$} on $X^*$ is said to be \emph{adapted to $X$} if, for every open subset $U$ of $X$ that admits an embedding into $\CC^n$, the restriction $g|U\cap X^*$ is in the class described above, with respect to $U\cap X^*\hookrightarrow U$.
  This condition is local on $X$.
  Using a partition of unity, we can show that there exist Riemannian structures on $X^*$ that are adapted to $X$;
  these form an equivalence class under \hyperref[II.2.3.*]{($*$)}.
\end{rmenv}

\begin{rmenv}{2.4}
\label{II.2.4}
  We situate ourselves under the hypotheses of \hyperref[II.2.2]{(2.2)}.
  We have multiple ways of defining the distance from a point of $X^*$ to the infinity $Y$.

  \begin{rmenv}{2.4.1}
  \label{II.2.4.1}
    Suppose that $Y$ is defined in $X$ by a finite family of equations $f_i=0$.
    We set
\oldpage{62}
    \[
      d_1(x) = \sum f_i(x)\overline{f_i(x)}.
    \]

    If the functions $d'_1(n)$ and $d''_1(n)$ are obtained by this procedure, then we have:

    \begin{itenv*}
    \label{II.2.4.1.*M}
      For every compact subset $K$ of $X$, there exist constants $A_1,A_2>0$ and $\rho_1,\rho_2>0$ such that, for all $x\in X^*$,
      \[
        \begin{aligned}
          d'_1(x) &\leq A_1d''_1(x)^{\rho_1}
        \\d''_1(x) &\leq A_2d'_1(x)^{\rho_2}.
        \end{aligned}
      \tag{$*$M}
      \]
    \end{itenv*}

    Indeed, let $(f'_i=0)$ and $(f''_i=0)$ be two systems of equations for $Y$.
    It suffices to verify \hyperref[II.2.4.1.*M]{($*$M)} locally on $X$.
    Locally, by the analytic Nullstellensatz, we know that, for large enough $N$, the ${f'_i}^N$ (resp. ${f''_i}^N$) are linear combinations of the $f''_i$ (resp. $f'_i$), and \hyperref[II.2.4.1.*M]{($*$M)} then follows formally.
  \end{rmenv}

  \begin{rmenv}{2.4.2}
  \label{II.2.4.2}
    Suppose that $X$ admits an embedding $j\colon X\hookrightarrow\CC^n$.
    Let $U$ be a relatively compact open subset of $X$.
    In $U$, we set
    \[
      d_2(x) = d(j(x),j(Y\cap U))
    \]
    where $d$ is the Euclidean distance in $\CC^n$.

    We can show, as in \hyperref[II.2.3]{(2.3)}, that, if $d'_2$ and $d''_2$ are obtained by this method, with respect to two different embeddings, then we have:

    \begin{itenv*}
    \label{II.2.4.2.*R}
      For every compact subset $K$ of $U$, there exist constants $A,B>0$ such that, for all $x\in K\cap X^*$,
      \[
        d'_2(x) \leq Ad''_2(x) \leq Bd'_2(x).
      \tag{$*$R}
      \]
    \end{itenv*}

    Furthermore, it follows immediately from the Lojasiewicz inequalities \cite[Th.~1, p.~85]{18} that the ``distances at infinity'' \hyperref[II.2.4.1]{(2.4.1)} and \hyperref[II.2.4.2]{(2.4.2)} are equivalent in the sense of \hyperref[II.2.4.1.*M]{($*$M)}.
  \end{rmenv}
\end{rmenv}

\begin{rmenv}{Definition 2.5}
\label{II.2.5}
  Under the hypotheses of \hyperref[II.2.2]{(2.2)}, a norm $\|x\|$ on $X^*$ is said to be \emph{adapted to $X$} if it is a function from $X^*$ to $\RR^+$ such that, for every open subset $U$ of $X$ on which $Y$ is defined by a finite family of equations $(f_i=0)$, and for every compact subset $K$ of $U$, there exist constants $A_1,A_2>0$ and $\rho_1,\rho_2>0$ such that, for all $x\in K\cap X^*$, we have
  \[
    \begin{aligned}
      (1+\|x\|)^{-1}
      &\leq A_1\left(\sum f_i(x)\overline{f_i(x)}\right)^{\rho_1}
    \\\sum f_i(x)\overline{f_i(x)}
      &\leq A_2(1+\|x\|)^{-\rho2}.
    \end{aligned}
  \]

  \oldpage{63}
  These conditions are local on $X$.
  We can show, using a partition of unity, that there always exist norms on $X^*$ that are adapted to $X$.
  These form an equivalence class under the equivalence relation

  \begin{itenv*}
  \label{II.2.5.*'M}
    For every compact subset $K$ of $X$, there exist constants $A_1,A_2>0$ and $\rho_1,\rho_2>0$ such that, for all $x\in K\cap X^*$,
    \[
      (1+\|x\|_i) \leq A_i(1+\|x\|_j)^{\rho_i}
    \tag{$*'$M}
    \]
    for $i=1,2$.
  \end{itenv*}
\end{rmenv}

\begin{rmenv}{Definition 2.6}
\label{II.2.6}
  A function $f$ on $X^*$ is said to be of \emph{moderate growth along $Y$} if there exists a norm $\|x\|$ on $X^*$ that is adapted to $X$ and such that, for all $x\in X^*$,
  \[
    |f(x)| \leq \|x^*\|.
  \]

  This condition is local on $X$.
\end{rmenv}

\begin{rmenv}{2.7}
\label{II.2.7}
  More precise information about the structure at infinity of $X^*$ is necessary in order to reasonably define what a multiform function with moderate growth at infinity on $X^*$ is.

  The fundamental example is that of the logarithm function.
  Denote by $\widetilde{D}^*$ the universal cover of the punctured disc.
  At an arbitrary point of $D^*$, the set of determinations of $\log\colon\widetilde{D}^*\to\CC$ is not bounded.
  We only have a bound
  \[
    |\log(z)| \leq A(1/|z|)^\varepsilon
  \]
  in a subset of $\widetilde{D}^*$ where the argument $\arg(z)$ of $z$ is bounded.

  The delicate results of Lojasiewicz used below will only be essential in what follows for trivial cases (cf. \hyperref[II.2.20]{(2.20)}).
\end{rmenv}

\begin{rmenv}{2.8}
\label{II.2.8}
  In \cite{17}, Lojasiewicz proves results which are more precise than \hyperref[II.2.8.2]{(2.8.2)} given below.

  \begin{rmenv}{2.8.1}
  \label{II.2.8.1}
    Let $X$ be a separated analytic space.
    In the following, we understand ``semi-analytic triangulation of $X$'' in the following weak sense: a semi-analytic triangulation of $X$ is a set $\scr{T}$ of closed semi-analytic subsets of $X$ (the simplices of the triangulation) such that
    \begin{enumerate}[(a)]
      \item $\scr{T}$ is locally finite and stable under intersection; and
      \item for every $\sigma\in\scr{T}$, there exists a homeomorphism $\gamma$ between $\sigma$ and a simplex
\oldpage{64}
        of the form $\Delta_n$ such that
        \begin{enumerate}[(b1)]
          \item the graph $\Gamma\subset\RR\times X$ of $\gamma$ is semi-analytic, and even semi-algebraic in the first variable; and
          \item $\gamma$ sends the set of faces of $\Delta_n$ to the set of $\tau\in\scr{T}$ contained in $\sigma$.
        \end{enumerate}
    \end{enumerate}
  \end{rmenv}

  \begin{rmenv}{2.8.2}
  \label{II.2.8.2}
    Let $X$ be separated analytic space, and $\scr{F}$ a locally finite set of semi-analytic subsets of $X$.
    Then, locally on $X$, there exists a semi-analytic triangulation $\scr{T}$ of $X$ such that every $F\in\scr{F}$ is the union of the simplices of the triangulation that it contains.
  \end{rmenv}
\end{rmenv}

\begin{rmenv}{Definition 2.9}
\label{II.2.9}
  Under the hypotheses of \hyperref[II.2.2]{(2.2)}, a subset $P$ of a covering $\pi\colon\widetilde{X}^*\to X^*$ of $X^*$ is said to be \emph{vertical} along $Y$ if there exists a finite family of compact semi-analytic subsets $P_i$ of $X$ such that the $P_i\setminus Y$ are simply connected, and lifts $\widetilde{P_i}$ of the $P_i\setminus Y$ on $\widetilde{X}^*$ such that
  \[
    P \subset \bigcup_i\widetilde{P_i}.
  \]

  \begin{rmenv}{2.9.1}
  \label{II.2.9.1}
    If $\scr{T}$ is a semi-analytic triangulation of $X$ that induces a semi-analytic triangulation of $Y$, then a subset $P$ of $\widetilde{X}^*$ is vertical if and only if it is contained in the union of a finite nubmer of lifts of open simplices of $\scr{T}$.
  \end{rmenv}

  \begin{rmenv}{2.9.2}
  \label{II.2.9.2}
    If $X$ is a finite union of open subsets $U_i$, then, for a subset $P$ of $\widetilde{X}^*$ to be vertical, it is necessary and sufficient for $P$ to be a union of subsets $P_i\subset\pi^{-1}(U_i)$ that are vertical along $Y\cap U_i$.
  \end{rmenv}

  \begin{rmenv}{2.9.3}
  \label{II.2.9.3}
    If $U$ is an open subset of $X$, and $P$ is a vertical subset of $\widetilde{X}^*$, then, for every compact subset $K$ of $U$, $P\cap\pi^{-1}(K)$ is vertical in $\pi^{-1}(U)$ along $U\cap Y$.
  \end{rmenv}
\end{rmenv}

\begin{rmenv}{Definition 2.10}
\label{II.2.10}
  Under the hypotheses of \hyperref[II.2.2]{(2.2)}, let $\pi\colon\widetilde{X}^*\to X^*$, and let $f$ be a function on $\widetilde{X}^*$.
  We say that $f$ is \emph{of moderate growth along $Y$} if, for every norm
\oldpage{65}
  $\|x\|$ on $X^*$ that is adapted to $X$, and for every vertical subset $P$ of $\widetilde{X}^*$, there exist $A,N>0$ such that, for all $x\in P$,
  \[
    |f(x)| \leq A(1+\|x\|)^N.
  \]

  This condition is of a local nature on $X$.
\end{rmenv}

\begin{rmenv}{Example 2.11}
\label{II.2.11}
  Let $X$ be the disc, $X^*$ the punctured disc, and $\widetilde{X}^*$ the universal cover of $X^*$.
  The multiform functions on $X^*$ (or functions on $\widetilde{X}^*$) defined by $z\mapsto z^\rho$ (for $\rho\in\CC$), and also $z\mapsto\log z$, are of moderate growth at the origin.
\end{rmenv}

\begin{itenv}{Lemma 2.12}
\label{II.2.12}
  Under the hypotheses of \hyperref[II.2.2]{(2.2)}, let $\cal{V}$ be a coherent analytic sheaf on $X^*$, and $\cal{V}_1$ and $\cal{V}_2$ extensions of $\cal{V}$ to a coherent analytic sheaf on $X$.
  Then the following conditions are equivalent.
  \begin{enumerate}[(i)]
    \item There exists an extension $\cal{V}'$ of $\cal{V}$ on $X$, as well as homomorphisms from $\cal{V}'$ to $\cal{V}_1$ and to $\cal{V}_2$.
    \item There exists an extension $\cal{V}''$ of $\cal{V}$ on $X$, as well as homomorphisms from $\cal{V}_1$ and from $\cal{V}_2$ to $\cal{V}''$.
    \item The two previous conditions are locally satisfied on $Y$.
  \end{enumerate}
\end{itenv}

\begin{proof}
  To prove that (i)$\iff$(ii), we take
  \[
    \cal{V}'' = (\cal{V}_1\oplus \cal{V}_2)/\cal{V}'
  \]
  where
  \[
    \cal{V}' = \cal{V}_1\cap \cal{V}_2.
  \]
  If (i) is locally true, then a global solution is given by the sum of the images of $\cal{V}_1$ and $\cal{V}_2$ in $j_*\cal{V}$, where $j$ is the inclusion of $X^*$ into $X$.
\end{proof}

\begin{rmenv}{2.13}
\label{II.2.13}
  We say that two extensions $\cal{V}_1$ and $\cal{V}_2$ of $\cal{V}$ are \emph{meromorphically equivalent} if the conditions of \hyperref[II.2.2]{(2.2)} are satisfied;
  we say that a coherent analytic sheaf on $X^*$ is \emph{meromorphic along $Y$} if it is locally endowed with an equivalence class of extensions of $\cal{V}$ to a coherent analytic sheaf on $X$.
  If there exists an extension of $\cal{V}$ on $X$ that is, locally on $Y$, meromorphically equivalent to the pre-given extensions, then this extension is unique up to meromorphic equivalence;
  we then say that $\cal{V}$ is \emph{effectively meromorphic along $Y$}.
\oldpage{66}
  I do not know if there exist coherent analytic sheaves on $X^*$ that are meromorphic along $Y$ but not effectively meromorphic along $Y$.

  Let $\cal{V}$ be a coherent analytic sheaf on $X^*$ that is meromorphic along $Y$;
  a \emph{section} $v\in\HH^0(X^*,\cal{V})$ is said to be \emph{meromorphic along $Y$} if, locally on $X$, it is defined by a section of one of the pre-given extensions of $\cal{V}$.
  The information of the sheaf $j_*^\mathrm{mero}\cal{V}$ on $X$ of sections of $\cal{V}$ which are meromorphic along $Y$ is equivalent to the information of the meromorphic structure along $Y$ of $\cal{V}$.
\end{rmenv}

\begin{rmenv}{2.14}
\label{II.2.14}
  Suppose that $X^*$ is reduced, and let $\cal{V}$ be a vector bundle on $X^*$ which is meromorphic along $Y$.
  We will define an equivalence class of ``norms'' on $\cal{V}$.
  The ``norms'' in question will be continuous families of norms on the $\cal{V}_x$ (for $x\in X^*$).
  If $v$ is a continuous section of $\cal{V}$, then $|v|$ is a positive function on $X^*$, which is zero at exactly the points where $v=0$.
  Two norms $|v|_1$ and $|v|_2$ will be said to be equivalent if we have the following:
  \begin{rmenv}{2.14.1}
  \label{II.2.14.1}
    For any norm $\|x\|$ on $X^*$, and any compact subset $K$ of $X$, there exist $A_1,A_2,N_1,N_2>0$ such that
    \[
      \begin{aligned}
        |v|_1 &\leq A_1(1+\|x\|)^{N_1}|v|_2
      \\|v|_2 &\leq A_2(1+\|x\|)^{N_2}|v|_1
      \end{aligned}
    \]
    on $K\cap X^*$.
  \end{rmenv}
\end{rmenv}

\begin{rmenv}{2.15.1}
\label{II.2.15.1}
  If $\cal{V}=\cal{O}^n$, then we set $|v|=\sum|v_i|$.
\end{rmenv}

\begin{rmenv}{2.15.2}
\label{II.2.15.2}
  Let $x\in Y$, and let $\cal{V}_1$ be a pre-given extension of $\cal{V}$ on a neighbourhood of $x$.
  Then there exists an open neighbourhood $U$ of $x$, along with $\omega\colon\cal{V}_1\to\cal{O}^n$, such that $\omega$ is a monomorphism on $U\cap X^*$.
  We set
  \[
    |v|_\omega = |\omega(v)|
  \]
  in $U\cap X^*$.
\end{rmenv}

\begin{rmenv}{2.15.3}
\label{II.2.15.3}
  For $x$ and $\cal{V}_1$ as in \hyperref[II.2.15.1]{(2.15.1)}, there exists a neighbourhood $U$ of $x$, and an epimorphism $\eta\colon\cal{O}^n\to\cal{V}_1$ on $U$.
  We set
  \[
    |v|_\eta = \inf_{\eta(w)=v}|w|
  \]
  in $U\cap X^*$.
\end{rmenv}

\oldpage{67}
\begin{rmenv}{2.15.4}
\label{II.2.15.4}
  We now compare \hyperref[II.2.15.2]{(2.15.2)} and \hyperref[II.2.15.3]{(2.15.3)}.
  Consider $U$ along with the meromorphic homomorphisms
  \[
    \cal{O}^n \xrightarrow{\eta} \cal{V} \xrightarrow{\omega} \cal{O}^m
  \]
  defined on $U\setminus Y$, with $\eta$ an epimorphism and $\omega$ a monomorphism.
  The meromorphic homomorphism $\omega\eta$ is such that its kernel and image are locally direct factors.
  This still holds true in the scheme $\Spec(\cal{O}_{x,X})\setminus Y_x$.
  We thus easily deduce that, for every holomorphic function $f$ on $U$ that vanishes on $Y$, there exists $N>0$, and open neighbourhood $U_1\subset U$ of $x$, and some $\alpha\colon\cal{O}^m\to\cal{O}^n$ such that
  \[
    \eta\alpha\omega = f^N.
  \]
  Let $K$ be a compact subset of $U_1$.
  It is clear that there exists $M>0$ such that
  \[
    \begin{aligned}
      |\omega\eta(v)| &\leq C^{te}(1+\|x\|)^M |v|
    \\|\alpha(v)| &\leq C^{te}(1+\|x\|)^M |v|
    \end{aligned}
  \]
  on $K$, and thus
  \[
  \label{II.2.15.4.1}
    |v|_\omega \leq C^{te}(1+\|x\|)^M |v|_\eta
  \tag{1}
  \]
  \[
  \label{II.2.15.4.2}
    |v|_\eta \leq C^{te}(1+\|x\|)^M |f^N| |v|_\omega
  \tag{2}
  \]
  on $K$.

  We can apply \hyperref[II.2.15.4.2]{(2)} to a finite family of functions $f_i$ that generates an ideal of definition of $Y$.

  By \hyperref[II.2.4]{(2.4)} and \hyperref[II.2.5]{(2.5)}, there exists $M'$ such that
  \[
  \label{II.2.15.4.3}
    \sum_i |f_i^N| \leq C^{te}(1+\|x\|)^{M'}
  \tag{3}
  \]
  on $K$, and it thus follows that, in a small enough neighbourhood of $x$, $|v|_\omega$ and $|v|_\eta$ are equivalent, in the sense of \hyperref[II.2.14.1]{(2.14.1)}.
  The equivalence \hyperref[II.2.14.1]{(2.14.1)} is of a local nature on $X$.
  We can thus prove, by using a partition of unity, the following proposition:
\end{rmenv}

\begin{itenv}{Proposition 2.16}
\label{II.2.16}
  Under the hypotheses of \hyperref[II.2.14]{(2.14)}, there exists exactly one equivalence class \hyperref[II.2.14.1]{(2.14.1)} of norms on $\cal{V}$ which are locally equivalent (in the sense of \hyperref[II.2.14.1]{(2.14.1)}) to the norms \hyperref[II.2.15.2]{(2.15.2)} and \hyperref[II.2.15.3]{(2.15.3)}.
\end{itenv}

We say that the norms whose existence is guaranteed by \hyperref[II.2.16]{(2.16)} are \emph{moderate}.

\oldpage{68}
\begin{rmenv}{Definition 2.17}
\label{II.2.17}
  Under the hypotheses of \hyperref[II.2.14]{(2.14)}, let $|v|$ be a moderate norm on $\cal{V}$, $\pi\colon\widetilde{X}^*\to X^*$ a covering of $X^*$, and $v$ a continuous section of $\pi^*\cal{V}$.
  We say that $v$ is \emph{of moderate growth along $Y$} if $|v|$ is of moderate growth along $Y$ \hyperref[II.2.10]{(2.10)}.
\end{rmenv}

In the particular case where $\pi$ is the identity, we can instead use \hyperref[II.2.6]{Definition~2.6}.
This is the case in the following well-known proposition, which shows that the knowledge of moderate norms on $\cal{V}$ is equivalent to the knowledge of the meromorphic structure of $\cal{V}$ along $Y$.

\begin{itenv}{Proposition 2.18}
\label{II.2.18}
  Under the hypotheses of \hyperref[II.2.14]{(2.14)}, for a holomorphic section $v$ of $\cal{V}$ over $X^*$ to be meromorphic along $Y$, it is necessary and sufficient for it to be of moderate growth along $Y$.
\end{itenv}

\begin{proof}
  The claim is local on $X$, and we can reduce, by \hyperref[II.2.16]{(2.16)} and \hyperref[II.2.15.2]{(2.15.2)} to the classical case where $\cal{V}=\cal{O}$.
\end{proof}

\begin{itenv}{Proposition 2.19}
\label{II.2.19}
  Consider a commutative diagram of separated analytic spaces
  \[
    \begin{tikzcd}
      X_1^* \rar[hook] \dar
      & X_1 \dar["f"]
      & Y_1 \lar[hook'] \dar
    \\X_2^* \rar[hook]
      & X_2
      & Y_2 \lar[hook']
    \end{tikzcd}
  \]
  where $Y_i$ is closed in $X_i$, and $X_i^*=X_i\setminus Y_i$, and thus $Y_1=f^{-1}(Y_2)$.

  Consider the hypotheses:
  \begin{enumerate}[(a)]
    \item There exists a subset $K$ of $X_1$ that is proper over $X_2$ and such that $f(K_1)\supset\overline{X_2^*}$.
    \item $f$ is proper and induces an isomorphism $X_1^*\simto X_2^*$.
  \end{enumerate}

  It is clear that \mbox{(b)$\implies$(a)}.
  We have
  \begin{enumerate}
    \item[{\rm(i\textsubscript{a})}] If $\|x\|$ is a norm on $X_2^*$ adapted to $X_2$, then $\|f(x)\|$ is a norm on $X_1^*$ adapted to $X_1$.
\oldpage{69}
    \item[{\rm(i\textsubscript{b})}] Conversely, if {\rm(a)} is satisfied, and if $\|x\|$ is a function on $X_2^*$ such that $\|f(x)\|$ is a norm on $X_1^*$, then $\|x\|$ is a norm on $X_2^*$.
    \item[{\rm(i\textsubscript{c})}] In particular, if {\rm(b)} is satisfied, then the norms on $X_1^*=X_2^*$ adapted to $X_1$ or $X_2$ agree.
  \end{enumerate}

  Let $\pi_2\colon\widetilde{X_2}^*\to X_2^*$ be a covering, and $\pi_1\colon\widetilde{X_1}^*\to X_1^*$ its inverse image over $X_1^*$.
  \begin{enumerate}
    \item[{\rm(ii\textsubscript{a})}] If $P$ is a subset of $\widetilde{X_1}^*$ which is vertical along $Y_1$, then $f(P)$ is vertical along $Y_2$.
    \item[{\rm(ii\textsubscript{b})}] Conversely, if {\rm(a)} is satisfied, then every vertical subset of $\widetilde{X_2}^*$ is the image of a vertical subset of $\widetilde{X_1}^*$.
    \item[{\rm(ii\textsubscript{c})}] In particular, if {\rm(b)} is satisfied, then the subsets of $\widetilde{X_1}^*=\widetilde{X_2}^*$ which are vertical along $Y_1$ or $Y_2$ agree.
  \end{enumerate}

  Let $\cal{V}_2$ be a vector bundle on $X_2^*$ that is meromorphic along $Y_2$, and let $\cal{V}_1$ be its inverse image.
  The inverse images of the pre-given extensions of $\cal{V}_2$ define a meromorphic structure along $Y_1$ on $\cal{V}_1$, and we have
  \begin{enumerate}
    \item[{\rm(iii\textsubscript{a})}] The inverse image of a moderate norm on $\cal{V}_2$ is a moderate norm on $\cal{V}_1$.
    \item[{\rm(iii\textsubscript{b})}] Conversely, if {\rm(a)} is satisfied, then a norm on $\cal{V}_2$ is moderate if its inverse image is.
    \item[{\rm(iii\textsubscript{c})}] In particular, if {\rm(b)} is satisfied, then the moderate norms on $\cal{V}_1=\cal{V}_2$ adapted to $X_1$ or $X_2$ agree.
  \end{enumerate}
\end{itenv}

\begin{proof}
  We trivially have that \mbox{(i\textsubscript{a})~+~(i\textsubscript{b})$\implies$(i\textsubscript{c})}, \mbox{(ii\textsubscript{a})~+~(ii\textsubscript{b})$\implies$(ii\textsubscript{c})}, \mbox{(iii\textsubscript{a})~+~(iii\textsubscript{b})$\implies$(iii\textsubscript{c})}, and almost trivially that \mbox{(i\textsubscript{a})$\implies$(i\textsubscript{b})} and \mbox{(i\textsubscript{a})~+~(iii\textsubscript{a})$\implies$(iii\textsubscript{b})}.

  If $Y_2$ is defined by equations $f_i=0$, then $Y_1$ is defined by the inverse image of the $f_i$, and (i\textsubscript{a}) follows from \hyperref[II.2.5]{Definition~2.5} (along with \hyperref[II.2.4.1]{(2.4.1)}).

  By \hyperref[II.2.15.2]{(2.15.2)} and \hyperref[II.2.16]{(2.16)}, we can reduce to proving (iii\textsubscript{a}) in the trivial case where $\cal{V}=\cal{O}$.

\oldpage{70}
  Finally, if $\scr{T}_2$ is a semi-analytic triangulation of the pair $(X_2,Y_2)$, then the $f^{-1}(\sigma)$ (for $\sigma\in\scr{T}_2$) form a locally finite set of semi-analytic subsets of $X_1$, and, by \hyperref[II.2.8.2]{2.8.2}, there exists, locally on $X_1$, a semi-analytic triangulation $\scr{T}_1$ of the pair $(X_1,Y_1)$ such that
  \[
    \forall\,\sigma\in\scr{T}_1
    \quad
    \exists\,\tau\in\scr{T}_2
    \quad
    \mbox{such that $f(\sigma)\subset\tau$}.
  \]
  Claims {\rm(ii\textsubscript{a})} and {\rm(ii\textsubscript{b})} thus follow immediately.
\end{proof}

\begin{rmenv}{Remarks 2.20}
  \begin{enumerate}[(i)]
    \item Let $X=D^{n+m}$ and $X^*=(D^*)^n\times D^m$;
      then $Y$ is a normal crossing divisor in $X$.
      On the universal cover $\widetilde{X}^*$ of $X^*$, the functions $\arg(z_i)$ (for $1\leq i\leq n$) are defined.
      It is clear that a subset $P$ of $\widetilde{X}^*$ is vertical along $Y$ if and only if its image in $X$ is relatively compact and if the functions $\arg(z_i)$ (for $1\leq i\leq n$) are bounded on $P$.
    \item Hironaka's resolution of singularities, along with (ii\textsubscript{c}) of \hyperref[II.2.19]{(2.19)}, allows us, in the general case, to make explicit the idea of vertical subsets, starting from the particular case of~(i).
  \end{enumerate}
\end{rmenv}

\begin{rmenv}{2.21}
\label{II.2.21}
  \hyperref[II.2.19]{Proposition~2.19} allows us to pursue the programme of \hyperref[II.2.1]{(2.1)}.
  So let $X^*$ be a separated scheme of finite type over $\CC$, and $X$ a proper scheme over $\CC$ that contains $X^*$ as a Zariski open subset.
  If $\scr{F}$ is a coherent algebraic sheaf on $X^*$, then we know (EGA~I, 9.4.7) that $\scr{F}$ can be extended to a coherent algebraic sheaf $\scr{F}_1$ on $X$.
  The various sheaves $\scr{F}_1^\an$ define on $\scr{F}^\an$ (effectively) the same meromorphic structure along $Y=X\setminus X^*$.
\end{rmenv}

We further immediately deduce, from \cite{GAGA}, the following:

\begin{itenv}{Proposition 2.22}
\label{II.2.22}
  Under the hypotheses of \hyperref[II.2.21]{(2.21)}, the functor $\scr{F}\mapsto\scr{F}^\an$ induces an equivalence between the category of coherent algebraic sheaves on $X^*$ and the category of coherent analytic sheaves on $(X^*)^\an$ that are effectively meromorphic along $Y$.
\end{itenv}

\begin{rmenv}{Definition 2.23}
\label{II.2.23}
  Let $X^*$ be a separated scheme of finite type over $\CC$, and let $X$ be as in \hyperref[II.2.21]{(2.21)}.
  Let $\pi\colon\widetilde{X}^*\to(X^*)^\an$ be a covering of $(X^*)^\an$, and $\cal{V}$ an
\oldpage{71}
  algebraic vector bundle on $X^*$.
  \begin{enumerate}[(i)]
    \item A \emph{norm on $X^*$} is defined to be a norm on $(X^*)^\an$ that is adapted to $X^\an$ \hyperref[II.2.5]{(2.5)}.
    \item A \emph{vertical subset of $\widetilde{X}^*$} is defined to be a subset of $\widetilde{X}^*$ that is vertical along $Y=X\setminus X^*$ \hyperref[II.2.9]{(2.9)}.
    \item A \emph{moderate norm on $\cal{V}$} is defined to be a moderate norm on $\cal{V}^\an$ with respect to the meromorphic structure at infinity of $\cal{V}^\an$ \hyperref[II.2.16]{(2.16)}.
    \item A continuous section $v$ of $\pi^*\cal{V}$ is said to be of \emph{moderate growth} if it is of moderate growth along $Y$ \hyperref[II.2.17]{(2.17)}.
  \end{enumerate}
\end{rmenv}

By \hyperref[II.2.19]{(2.19)}, these definitions do not depend on the choice of compactification $X$ of $X^*$.

We can also immediately deduce, from \hyperref[II.2.18]{(2.18)} and \cite{GAGA} (as in \hyperref[II.2.22]{(2.22)}), the following:

\begin{itenv}{Proposition 2.24}
\label{II.2.24}
  Let $X$ be a separated scheme that is reduced and of finite type over $\CC$, and let $\cal{V}$ be an algebraic vector bundle on $X$.
  Then a holomorphic section $v$ of $\cal{V}^\an$ is algebraic if and only if it is of moderate growth.
\end{itenv}

\begin{rmenv}{Problem 2.25}
\label{II.2.25}
  Let $X=G/K$ be a Hermitian symmetric domain (with $G$ a real Lie group, and $K$ a compact maximal subgroup), and $\Gamma$ an arithmetic subgroup of $G$.
  The quotient $\Gamma\backslash G/K$ is then naturally a quasi-projective algebraic variety \cite{2}.
  Is it true that a subset $P$ of $G/K$ is vertical \hyperref[II.2.23]{(2.23)} if and only if it is contained in the union of a finite number of Siegel domains?
\end{rmenv}



\section{Logarithmic poles}
\label{II.3}

\oldpage{72}

This section brings together some constructions that are ``local at infinity\todo'' of which we will later have use.

\begin{rmenv}{Definition 3.1}
\label{II.3.1}
  Let $Y$ be a normal crossing divisor in a complex-analytic variety $X$, and let $j$ be the inclusion of $X^*=X\setminus Y$ into $X$.
  We define the \emph{logarithmic de Rham complex} of $X$ along $Y$ to be the smallest sub-complex $\Omega_X^\bullet\langle Y\rangle$ of $j_*\Omega_{X^*}^\bullet$ that containing $\Omega_X^\bullet$ that is stable under the exterior product, and such that $\dd f/f$ is a local section of $\Omega_X^1\langle Y\rangle$ for all local sections $f$ of $j_*\Omega_X^\bullet$ that are meromorphic along $Y$.

  A section of $j_*\Omega_{X^*}^p$ is said to present a \emph{logarithmic pole} along $Y$ if it is a section of $\Omega_X^p\langle Y\rangle$.
\end{rmenv}

\begin{itenv}{Proposition 3.2}
\label{II.3.2}
  Under the hypotheses of \hyperref[II.3.1]{(3.1)},
  \begin{enumerate}[(i)]
    \item For a section $\alpha$ of $j_*\Omega_{X^*}^p$ to present a logarithmic pole along $Y$, it is necessary and sufficient for $\alpha$ and $\dd\alpha$ to present at worst simple poles along $Y$.
    \item The sheaf $\Omega_X^1\langle Y\rangle$ is locally free, and
      \[
        \Omega_X^p\langle Y\rangle = \bigwedge^p \Omega_X^1\langle Y\rangle.
      \]
    \item If the pair $(X,Y)$ is a product $(X,Y)=(X_1,Y_1)\times(X_2,Y_2)$, i.e. if
      \[
        X = X_1\times X_2
        \quad\mbox{and}\quad
        Y = X_1\times Y_2 \cup X_2\times Y_1,
      \]
      then the isomorphism between $\Omega_{X^*}^\bullet$ and the external tensor product $\Omega_{X_1^*}^\bullet\boxtimes\Omega_{X_2^*}^\bullet$ (defined by $\pr_1^*\Omega_{X_1^*}^\bullet\otimes\pr_2^*\Omega_{X_2^*}^\bullet$) induces an isomorphism
      \[
        \Omega_{X_1}^\bullet\langle Y_1\rangle \boxtimes \Omega_{X_2}^\bullet\langle Y_2\rangle
        \simto \Omega_X^\bullet\langle Y\rangle.
      \]
    \item Let $Y_i$ be a normal crossing divisor in $X_i$ (for $i=1,2$), and $f\colon X_1\to X_2$ a morphism such that $f^{-1}(Y_2)=Y_1$.
      Then the morphism $f^*\colon f^*((j_2)_*\Omega_{X_2^*}^\bullet)\to (j_1)_*\Omega_{X_1^*}^\bullet$ induces an ``inverse image'' morphism
      \[
        f^*\colon f^*\Omega_{X_2}^\bullet\langle Y_2\rangle \to \Omega_{X_1}^1\langle Y_1\rangle.
      \]
  \end{enumerate}
\end{itenv}

\oldpage{73}
\begin{proof}
  Claim~(iv) is trivial, given the definition.
  Let $D$ be the open unit disc, and $D^*=D\setminus\{0\}$.
  To prove (i), (ii), and (iii), we can assume that $X$ is the polydisc $D^n$, and that $X^*=(D^*)^k\times D^{n-k}$, and $Y=\bigcup_{1\leq i\leq k}Y_i$, where $Y_i=\pr_i^{-1}(0)$.
  Under these hypotheses, we have

  \begin{itenv}{Lemma 3.2.1}
  \label{II.3.2.1}
    The sheaf $\Omega_X^1\langle Y\rangle$ is free, with basis $(\dd z_i/z_i)_{1\leq i\leq k}\cup(\dd z_j)_{k<j\leq n}$.
  \end{itenv}

  \begin{proof}
    Indeed, every section of $j_*\cal{O}_{X^*}^*$ that is meromorphic along $Y$ can be written locally as $f=g\cdot\prod_{i=1}^k z_i^{k_i}$ with $g$ invertible, and
    \[
      \dd f/f = \dd g/g + \sum_{i=1}^k k_i\dd z_i/z_i
    \]
    is a linear combination of the proposed basis vectors, which are clearly independent.
  \end{proof}

  From this lemma, we immediately deduce (ii), (iii), and the necessity of the condition in~(i).

  Let $\alpha$ be a section of $j_*\Omega_{X^*}^p$ satisfying the condition of~(i).
  To prove that $\alpha$ is a section of $\Omega_{X^*}^p\langle Y\rangle$, it suffices (since this sheaf is locally free) to prove this outside of a set of complex codimension~$\geq2$.
  This allows us to suppose that the hypotheses of \hyperref[II.3.2.1]{(3.2.1)} are satisfied, with $k=1$.
  The form $\alpha$ can thus be written in the form $\alpha=\alpha_1+\alpha_2\wedge\dd z_1/z_1$ in exactly one way, with the forms $\alpha_1$ and $\alpha_2$ being such that they do not contain any $\dd z_1$ term.

  The hypotheses imply that
  \begin{itemize}
    \item $\alpha_2$ is holomorphic;
    \item $z_1\alpha_1$ is holomorphic; and
    \item $z_1\dd\alpha = z_1\dd\alpha_1+\dd\alpha_2\wedge\dd z_1$ is holomorphic.
  \end{itemize}

  From this,
  \[
    \dd z_1\wedge\alpha_1 = \dd(z_1\wedge\alpha_1) + z_1\dd\alpha - \dd\alpha_2\wedge\dd z_1
  \]
  is holomorphic, and thus so too is $\alpha_1$, which proves~(i).
\end{proof}

\begin{rmenv}{Variants 3.3}
\label{II.3.3}
  \begin{rmenv}{3.3.1}
  \label{II.3.3.1}
    Let $f\colon X\to S$ be a smooth morphism of schemes of characteristic~$0$, or a smooth morphism of analytic spaces, and let $Y$ be a relative normal crossing divisor in $X$.
    \hyperref[II.3.1]{Definition~3.1} still makes sense, and defines a sub-complex $\Omega_{X/S}^\bullet\langle Y\rangle$ of $j_*\Omega_{X^*/S}^\bullet$ (where $j$ is the inclusion $j\colon X^*=X\setminus Y\to X$).
    \hyperref[II.3.2]{Proposition~3.2} still holds true, mutatis mutandis.
    Forming the complex $\Omega_{X/S}^\bullet\langle Y\rangle$ is compatible with any base change, and with \'{e}tale localisation on $X$.
  \end{rmenv}

  \begin{rmenv}{3.3.2}
  \label{II.3.3.2}
    Let $f\colon X\to S$ be a morphism of smooth analytic spaces, $0$ a point of $S$, and $Y$ a normal crossing divisor in $X$.
    Let $S^*=S\setminus\{0\}$, and $X^*=X\setminus Y$, and let $j$ be the inclusion of $X^*$ into $X$.
    Suppose that
    \begin{enumerate}[(a)]
      \item $\dim(S)=1$;
      \item $f|f^{-1}(S^*)$ is smooth, and $Y\cap f^{-1}(S^*)$ is a relative normal crossing divisor in $f^{-1}(S^*)$; and
      \item $Y\supset f^{-1}(0)$.
    \end{enumerate}

    We can then define the complex $\Omega_{X/S}^\bullet\langle Y\rangle$ as the image of $j_*\Omega_{X^*}^\bullet$ in $\Omega_X^\bullet\langle Y\rangle$.

    Locally, close to $0$ and $f^{-1}(0)$, we can find coordinate systems $(z_i)_{0\leq i\leq n}$ on $X$, and $z$ on $S$, such that $z(0)=0$, such that $z\circ f = \prod_{i=0}^k z_i^{e_i}$ (for $k\leq n$ and $e_i>0$), and such that $Y$ can be described by the equation $\prod_{i=0}^l z_i=0$ (for $k\leq l\leq n$).
    In such a coordinate system, the sheaf $\Omega_{X/S}^1\langle Y\rangle$ is free, with basis given by $(\dd z_i/z_i)_{1\leq i\leq l}\cup(\dd z_j)_{l<j\leq n}$.
    In $\Omega_{X/S}^1\langle Y\rangle$, we have the equation
    \[
      \dd f/f = \sum_{i=0}^k e_i\dd z_i/z_i = 0.
    \]
    We thus deduce that $\Omega_{X/S}^1\langle Y\rangle$ is \emph{locally free}, that
    \[
    \label{II.3.3.2.1}
      \bigwedge^p \Omega_{X/S}^1\langle Y\rangle
      \simto \Omega_{X/S}^p\langle Y\rangle,
    \tag{3.3.2.1}
    \]
    and that the sequence
  \oldpage{75}
    \[
    \label{II.3.3.2.2}
      0
      \to f^*\Omega_S^1\langle0\rangle
      \xrightarrow{f^*} \Omega_X^1\langle Y\rangle
      \to \Omega_{X/S}^1\langle Y\rangle
      \to 0
    \tag{3.3.2.2}
    \]
    is exact and \emph{locally split}.
    This will play a key role in \hyperref[II.7]{(II.7)}, in the following form:

    \begin{itenv}{Lemma 3.3.2.3}
    \label{II.3.3.2.3}
      Every vector field $v_0$ on $S$ that vanishes at $0$ can be locally lifted to a vector field $v$ on $X$ satisfying
      \[
        \big\langle v,\Omega_X^1\langle Y\rangle\big\rangle \subset \cal{O}_X.
      \]
    \end{itenv}

    Indeed, the transpose of the direct monomorphism $f^*$ in \hyperref[II.3.3.2.2]{(3.3.2.2)} is an epimorphism.
  \end{rmenv}

  \begin{rmenv}{3.3.3}
  \label{II.3.3.3}
    The reader can translate \hyperref[II.3.3.2]{(3.3.2)} into the setting of a morphism $f\colon X\to S$ of schemes of finite type over $\CC$ that satisfies the conditions analogous to \hyperref[II.3.3.2]{(3.3.2.a,b,c)}.
  \end{rmenv}
\end{rmenv}

\begin{rmenv}{3.4}
\label{II.3.4}
  Let $Y$ be a normal crossing divisor in $S$.
  Locally on $X$, we can write $Y$ as a sum of smooth divisors $Y_i$.
  We denote by $Y^p$ (resp. $\widetilde{Y}^p$) the union (resp. disjoint sum) of the $p$-fold intersections (??\todo??) of the $Y_i$;
  the $Y^p$, thus locally defined, glue to give a subspace $Y^p$ of $X$, and the $\widetilde{Y}^p$ glue to give the normalised variety of $Y^p$.
  We have $\widetilde{Y}^0=Y^0=X$, and we set $\widetilde{Y}=Y^1$.
  Let $a\colon Y^p\to X$ be the projection.

  If, to each point $y\in\widetilde{Y}^p$, we associate the set of \todo, then we define a local system $E_p$ on $\widetilde{Y}^p$ of sets with $p$~elements.

  Denote by $\epsilon^p$ the rank-$1$ local system
  \[
    \epsilon^p = \bigwedge^p \underline{\CC}^{E_p}
  \]
  on $\widetilde{Y}^p$.
  We have that $(\epsilon^p)^{\otimes2}\simeq\underline{\CC}$.
  If $Y$ is the sum of smooth divisors $(Y_i)_{i\in I}$, then the choice of a total order on $I$ trivialises the $\epsilon^p$.
\end{rmenv}

\begin{rmenv}{3.5}
\label{II.3.5}
  Denote by $W_n(\Omega_X^\bullet\langle Y\rangle)$ the smallest sub-$\cal{O}$-module of $\Omega_X^\bullet\langle Y\rangle$ that is stable under the exterior product with the local sections of $\Omega_X^\bullet$ and that contains the products
  \[
    \dd f_1/f_1\wedge\ldots\wedge\dd f_k/f_k
  \]
\oldpage{76}
  for $k\leq n$, and local sections $f_i$ of $j_*\cal{O}_{X^*}^*$ that are meromorphic along $Y$.
  Then the $W_n$ form an \emph{increasing} filtration of $\Omega_X^\bullet\langle Y\rangle$ by sub-complexes, called the \emph{weight filtration}.
  We have that
  \[
  \label{II.3.5.1}
    W_n(\Omega_X^\bullet\langle Y\rangle) \wedge W_m(\Omega_X^\bullet\langle Y\rangle)
    \subset W_{n+m}(\Omega_X^\bullet\langle Y\rangle).
  \tag{3.5.1}
  \]

  Locally on $X$, we can write $Y$ as a finite sum of smooth divisors $(Y_i)_{i\in I}$ defined by equations $t_i=0$.
  Let $q$ be an injection of $[1,n]$ into $I$, let $e(q)$ be the corresponding section $e_{q(1)}\wedge\ldots\wedge e_{q(n)}$ of $\epsilon^n$ on the component $Y_q=\bigcap_{1\leq i\leq n}Y_{q(i)}$ of $\widetilde{Y}^n$, and let $a_q\colon Y_q\to X$ be the projection.

  The map $\rho_0$ from $\Omega_X^p$ to $W^n/W^{n-1}(\Omega_X^{p+n}\langle Y\rangle)$ given by
  \[
  \label{II.3.5.2}
    \alpha
    \longmapsto \dd t_{(q_1)}/t_{q(1)}\wedge\ldots\wedge\dd t_{q(n)}/t_{q(n)}\wedge\alpha
  \tag{3.5.2}
  \]
  does not depend on the choice of the $t_i$, since, if $t'_i$ are a different choice, then the $\dd t_i/t_i-\dd t'_i/t'_i=\dd(t_i/t'_i)/(t_i/t'_i)$ are holomorphic, and $\rho_0(\alpha)-\rho'_0(\alpha)\in W^{n-1}(\Omega_X^{p+n}\langle Y\rangle)$.
  Similarly, $\rho_0(t_{q(i)}\cdot\beta)=0$, and $\rho_0(\dd t_{q(i)}\wedge\beta)=0$, and so $\rho_0$ factors through
  \[
    \rho_1\colon (a_q)_*\Omega_{Y_q}^p \to W^n/W^{n-1}(\Omega_X^{p+n}\langle Y\rangle).
  \]
  The trivialisation $e(q)$ of $\epsilon^n|Y_q$ identifies $\rho_1$ with
  \[
    \rho_2\colon (a_q)_*\Omega_{Y_q}^\bullet(\epsilon^n)[-n] \to \Gr_n^W(\Omega_X^\bullet\langle Y\rangle).
  \]
  Finally, the sum of the morphisms $\rho_2$ for the different $q$ defines a morphism of complexes
  \[
  \label{II.3.5.3}
    \rho\colon a_*\Omega_{\widetilde{Y}^n}^\bullet(\epsilon)[-n] \to \Gr_n^W(\Omega_X^\bullet\langle Y\rangle).
  \tag{3.5.3}
  \]
  This morphism, defined locally by \hyperref[II.3.5.2]{(3.5.2)}, glues to give a morphism of complexes on $X$.
\end{rmenv}

\begin{itenv}{Proposition 3.6}
\label{II.3.6}
  The morphisms \hyperref[II.3.5.3]{(3.5.3)} are isomorphisms.
\end{itenv}

\begin{proof}
  If the pair $(X,Y)$ is a product $(X,Y)=(X_1,Y_1)\times(X_2,Y_2)$, i.e. if
  \[
    X = X_1\times X_2
    \quad\mbox{and}\quad
    Y = X_1\times Y_2 \cup X_2\times Y_1,
  \]
\oldpage{77}
  then the weight filtration on $\Omega_X^\bullet\langle Y\rangle$ is the external tensor product, via \hyperref[II.3.2]{(3.2.iii)}, of the weight filtrations on the $\Omega_{X_i}^\bullet\langle Y_i\rangle$.
  We thus have
  \[
  \label{II.3.6.1}
    \Gr^W(\Omega_{X_1}^\bullet\langle Y\rangle) \boxtimes \Gr^W(\Omega_{X_2}^\bullet\langle Y_2\rangle)
    \simto \Gr^W(\Omega_X^\bullet\langle Y\rangle).
  \tag{3.6.1}
  \]

  The isomorphisms
  \[
    \begin{aligned}
      \widetilde{Y}^n
      &= \coprod_{p+q=n} \widetilde{Y}^p\boxtimes\widetilde{Y}^q
    \\\epsilon^n
      &= \coprod_{p+q=n} \epsilon^p\boxtimes\epsilon^q
    \end{aligned}
  \]
  induce an isomorphism
  \[
  \label{II.3.6.2}
    \sum_p a_*\Omega_{\widetilde{Y}^p}^\bullet(\epsilon^p)[-p]
    \boxtimes
    \sum_q a_*\Omega_{\widetilde{Y}^q}^\bullet(\epsilon^q)[-q]
    \simto
    \sum_n a_*\Omega_{\widetilde{Y}^n}^\bullet(\epsilon^n)[-n].
  \tag{3.6.2}
  \]
  Further, via \hyperref[II.3.6.1]{(3.6.1)} and \hyperref[II.3.6.2]{(3.6.2)}, we have
  \[
  \label{II.3.6.3}
    \rho_1\boxtimes\rho_2 = \rho.
  \tag{3.6.3}
  \]

  For $\rho$ to be an isomorphism, it is thus sufficient that the $\rho_i$ be isomorphisms.
  Since the problem is local on $X$, this allows us to restrict to the trivial case where $\dim(X)=1$.
\end{proof}

\begin{rmenv}{3.7}
\label{II.3.7}
  The isomorphism inverse to $\rho$ is called the \emph{Poincar\'{e} residue}
  \[
  \label{II.3.7.1}
    \Res\colon \Gr_n^W(\Omega_X^p\langle Y\rangle) \to \Omega_{\widetilde{Y}^n}^p(\epsilon^n)[-n].
  \tag{3.7.1}
  \]
  We will only need the case where $p=1$, which gives
  \[
  \label{II.3.7.2}
    \Res\colon \Omega_X^1\langle Y\rangle \to \cal{O}_{\widetilde{Y}}.
  \tag{3.7.2}
  \]

  If $\cal{V}$ is a vector bundle on $X$, then the morphism \hyperref[II.3.7.2]{(3.7.2)} extends by linearity to
  \[
  \label{II.3.7.3}
    \Res\colon \Omega_X^1\langle Y\rangle(\cal{V}) \to \cal{O}_{\widetilde{Y}}\otimes\cal{V}.
  \tag{3.7.3}
  \]
  For each smooth component $Y_i$ of $Y$, this gives
\oldpage{78}
  \[
  \label{II.3.7.4}
    \Res_{Y_i}\colon \Omega_X^1\langle Y\rangle(\cal{V}) \to \cal{V}|Y_i.
  \tag{3.7.4}
  \]
\end{rmenv}

\begin{rmenv}{3.8}
\label{II.3.8}
  Under the hypotheses of \hyperref[II.3.1]{(3.1)}, let $\cal{V}_0$ be a vector bundle on $X*$, endowed with an integrable connection $\nabla$.
  Suppose that $\cal{V}_0$ is given as the restriction to $X^*$ of a vector bundle $\cal{V}$ on $X$.
  Locally on $X$, the choice of a basis $e$ of $\cal{V}$ allows us to define the ``connection matrix''
  \[
  \label{II.3.8.1}
    \Gamma \in j_*\Omega_{X^*}^1(\shEnd(\cal{V})).
  \tag{3.8.1}
  \]
  A change of basis $e\mapsto e'$ modifies $\Gamma$ by addition of a section of $\Omega_X^1(\shEnd(\cal{V}))$ \hyperref[I.3.1.3]{(I.3.1.3)}.
  Thus the ``polar part of $\Gamma$''
  \[
  \label{II.3.8.2}
    \dot{\Gamma} \in j_*\Omega_{X^*}^1\big(\shEnd(\cal{V}_0))\big/\Omega_X^1(\shEnd(\cal{V})\big)
  \tag{3.8.2}
  \]
  depends only on $\cal{V}$ and on $\nabla$.
  We say that \emph{the connection $\nabla$ has at worst a logarithmic pole along $Y$} if, in every local basis of $\cal{V}$, the connection forms present at worst a logarithmic pole along $Y$.
  In this case, \emph{the residue of the connection $\Gamma$} along a local component $Y_i$ of $Y$ is defined \hyperref[II.3.7.4]{(3.7.4)}
  \[
  \label{II.3.8.3}
    \Res_{Y_i}(\Gamma) \in \shEnd(\cal{V}|Y_i)
  \tag{3.8.3}
  \]
  and it depends only on $\cal{V}$ and on $\nabla$.
  More globally, if $i\colon\widetilde{Y}\to X$ is the projection of the normalisation of $Y$ onto $X$, then the residue of the connection along $Y$ is an endomorphism of $i^*\cal{V}$
  \[
  \label{II.3.8.4}
    \Res_Y(\Gamma) \in \shEnd(i^*\cal{V}).
  \tag{3.8.4}
  \]
\end{rmenv}

\begin{rmenv}{3.9}
\label{II.3.9}
  We place ourselves under the hypotheses of \hyperref[II.3.8]{(3.8)}, and make the following additional hypotheses:
  \begin{enumerate}[a)]
    \item $Y$ is the sum of smooth divisors $(Y_i)_{1\leq i\leq n}$ (which is always the case locally).
      For $P\subset[1,n]$, we set $Y_P=\bigcap_{i\in P}Y_i$ and $Y'_P=Y_P\setminus\bigcup_{i\not\in P}Y_i$.
    \item The connection on $\cal{V}$ has at worst a logarithmic pole along $Y$.
  \end{enumerate}

  The dual of the vector bundle $\Omega_X^1\langle Y\rangle$ is the bundle $T_X^1\langle-Y\rangle$ of vector fields $v$ on $X$ that satisfy
\oldpage{79}
  \begin{rmenv}{3.9.1}
    For $P\subset[1,n]$, $v|Y_P$ is tangent to $Y_P$.
  \end{rmenv}

  If a vector field $v$ satisfies \hyperref[II.3.9.1]{(3.9.1)}, and if $g$ is a section of $\cal{V}$, then $\nabla_v(g)$ is again a regular section of $\cal{V}$.
  Its restriction to $Y_P$ depends only on $g|Y_P$ and on the image of $v$ in $T_X^1\langle-Y\rangle\otimes\cal{O}_{Y_P}$.
  If $s$ is a local section of the evident epimorphism from $T_X^1\langle-Y\rangle\otimes\cal{O}_{Y_P}$ to its image in the tangent bundle to $Y_P$, then $\nabla_{s(v)}(g)$ defines a connection ${}_s\nabla$ on $\cal{V}|Y'_P$.
  There is a Lie bracket defined, by passing to the quotient, on $T_X^1\langle-Y\rangle\otimes\cal{O}_{Y_P}$.
  The connection ${}_s\nabla$ is integrable if $s$ commutes with the bracket;
  it presents at worst a logarithmic pole along $Y_P\cap\bigcup_{i\not\in P}Y_i$.
\end{rmenv}

An easy calculation shows the following:

\begin{itenv}{Proposition 3.10}
\label{II.3.10}
  Under the previous hypotheses, and with the above notation,
  \begin{enumerate}[(i)]
    \item $[\Res_{Y_i}(\Gamma),\Res_{Y_j}(\Gamma)]=0$ on $Y_i\cap Y_j$; and
    \item if $i\in P$, then ${}_s\nabla\Res_{Y_i}(\Gamma)=0$ on $Y'_P$.
  \end{enumerate}
\end{itenv}

We deduce from \hyperref[II.3.10]{(3.10.ii)}, for $P=\{i\}$, that the characteristic polynomial of $\Res_{Y_i}(\Gamma)$ is constant on $Y_i$.

We can also deduce \hyperref[II.3.10]{(3.10)} from the following proposition, which can be proven similarly to \hyperref[II.1.17]{(1.17)}:

\begin{itenv}{Proposition 3.11}
\label{II.3.11}
  Let $\cal{V}$ be a vector bundle on $X=D^n$, let $Y=\{0\}\times D^{n-1}$, and $X^*=X\setminus Y$, and let $\Gamma$ be an integrable connection on $\cal{V}|X^*$ presenting a logarithmic pole along $Y$.
  Let $T$ be the monodromy transformation of $\cal{V}|X^*$ defined by the positive generator of $\pi_1(X^*)\cong\pi_1(D^*)\cong\ZZ$ (cf. \hyperref[II.1.15]{(1.15)}).
  Then the horizontal automorphism $T$ of $\cal{V}|X^*$ can be extended to an automorphism of $\cal{V}$, again denoted by $T$, and
  \[
    T|Y = \exp(-2\pi i\Res_Y(\Gamma)).
  \]
\end{itenv}

\begin{rmenv}{3.12}
\label{II.3.12}
  Let $X$ be a complex-analytic variety, $Y$ a normal crossing divisor in $X$, and $j\colon X^*=X\setminus Y\to X$ the inclusion.
  For a vector bundle $\cal{V}$ on $X$, we denote by $j_*^\mathrm{m}j^*\cal{V}$ the sheaf of sections of $\cal{V}$ over $X^*$ that are meromorphic along $Y$.

\oldpage{80}
  Locally on $X$, $Y$ is the union of smooth divisors $Y_i$, and we define the \emph{pole-order filtration $P$} of $j_*^\mathrm{m}j^*\cal{O}=j_*^\mathrm{m}\cal{O}_{X^*}$ by the equation
  \[
  \label{II.3.12.1}
    P^k(j_*^\mathrm{m}j^*\cal{O})
    = \sum_{\underline{n}\in A_k}\cal{O}(\sum(n_i+1)Y_i)
  \tag{3.12.1}
  \]
  where
  \[
    A_k = \big\{
      (n_i) \mid \mbox{$\sum_i n_i\leq-k$ and $n_i\geq0$ for all $i$}
    \big\}.
  \]

  This construction can be made global, and endows $j_*^\mathrm{m}\cal{O}_{X^*}$ with an exhaustive filtration such that $P^k=0$ for $k>0$.

  Let $\cal{V}$ be a vector bundle on $X$, and $\Gamma$ an integrable connection on $\cal{V}|X^*$ presenting a logarithmic pole along $Y$.
  We define a filtration $P$, again called the \emph{pole-order filtration}, of the complex $j_*^\mathrm{m}j^*\Omega_X^\bullet(\cal{V})=j_*^\mathrm{m}\Omega_{X^*}^\bullet(\cal{V})$ by
  \[
  \label{II.3.12.2}
    P^k\big(j_*^\mathrm{m}\Omega_{X^*}^p(\cal{V})\big)
    = P^{k-p}(j_*^\mathrm{m}\cal{O}_{X^*})\otimes\Omega_X^p\otimes\cal{V}.
  \tag{3.12.2}
  \]

  From the fact that $\Gamma$ presents at worst logarithmic poles along $Y$, we deduce that
  \begin{enumerate}[a)]
    \item the filtration $P$ in \hyperref[II.3.12.2]{(3.12.2)} is compatible with the differentials; and
    \item $\Omega_X^\bullet\langle Y\rangle(\cal{V})$ is a sub-complex of $j_*^\mathrm{m}\Omega_{X^*}^\bullet(\cal{V})$.
  \end{enumerate}

  Furthermore,
  \begin{enumerate}
    \item[{\rm c)}] the operators $\dd$ are $\cal{O}_X$-linear on the complexes $\Gr_p^n(j_*^\mathrm{m}\Omega_{X^*}^\bullet(\cal{V}))$; and
    \item[{\rm d)}] the filtration $P$ induces the \emph{Hodge filtration} $F$ on $\Omega_X^\bullet\langle Y\rangle(\cal{V})$ by the stupid truncations $\sigma_{\geq p}$, whence we have a morphism of filtered complexes
    \[
    \label{II.3.12.3}
      \big(\Omega_X^\bullet\langle Y\rangle(\cal{V}), F\big)
      \to
      \big(j_*^\mathrm{m}\Omega_{X^*}^\bullet(\cal{V}), P\big).
    \tag{3.12.3}
    \]
  \end{enumerate}
\end{rmenv}

\begin{itenv}{Proposition 3.13}
\label{II.3.13}
  With the hypotheses and notation of \hyperref[II.3.12]{(3.12)}, if the residues of the connection $\Gamma$ along all the local components of $Y$ do not admit any strictly positive integer as an eigenvalue, then
  \begin{enumerate}[(i)]
    \item the morphism of complexes \hyperref[II.3.12.3]{(3.12.3)} is a quasi-isomorphism; and
    \item more precisely, it induces a quasi-isomorphism
\oldpage{81}
      \[
      \label{II.3.13.1}
        \Gr_F\big(\Omega_X^\bullet\langle Y\rangle(\cal{V})\big)
        \to \Gr_P\big(j_*^\mathrm{m}\Omega_{X^*}^\bullet(\cal{V})\big).
      \tag{3.13.1}
      \]
  \end{enumerate}
\end{itenv}

\begin{proof}
  It suffices to prove (ii), which also implies that, for each $p$, $\Gr_P^p(j_*^\mathrm{m}\Omega_{X^*}^\bullet(\cal{V}))[p]$ is a resolution of $\Omega_X^p\langle Y\rangle(\cal{V})$.

  \begin{itemize}
    \item[] \textbf{First reduction (Extensions).}
      If $\cal{V}$ is an extension of bundles with connections satisfying \hyperref[II.3.13]{(3.13)}, as in
      \[
        0 \to \cal{V}' \to \cal{V} \to \cal{V}'' \to 0
      \]
      then the lines of the diagram
      \[
        \begin{tikzcd}
          0 \rar
          & \Gr_F\Omega_X^\bullet\langle Y\rangle(\cal{V}') \rar \dar
          & \Gr_F\Omega_X^\bullet\langle Y\rangle(\cal{V}) \rar \dar
          & \Gr_F\Omega_X^\bullet\langle Y\rangle(\cal{V}'') \rar \dar
          & 0
        \\0 \rar
          & \Gr_Pj_*^\mathrm{m}\Omega_{X^*}^\bullet(\cal{V}') \rar
          & \Gr_Pj_*^\mathrm{m}\Omega_{X^*}^\bullet(\cal{V}) \rar
          & \Gr_Pj_*^\mathrm{m}\Omega_{X^*}^\bullet(\cal{V}'') \rar
          & 0
        \end{tikzcd}
      \]
      are exact.
      For \hyperref[II.3.13.1]{(3.13.1)} to be a quasi-isomorphism, it thus suffices to prove the analogous claim for $\cal{V}'$ and $\cal{V}''$.
    \item[] \textbf{Second reduction (Products).}
      Suppose that $(X,Y)$ is the product of $(X_1,Y_1)$ and $(X_2,Y_2)$, and that $\cal{V}$ is the external tensor product $\cal{V}=\cal{V}_1\boxtimes\cal{V}_2$ of bundles $\cal{V}_i$ with connections satisfying the hypotheses of \hyperref[II.3.13]{(3.13)} on $(X_i,Y_i)$.

      The isomorphism \hyperref[II.3.2]{(3.2.iii)} identifies the Hodge filtration of $\Omega_X^\bullet\langle Y\rangle$ with the external tensor product of the Hodge filtrations of the $\Omega_{X_i}^\bullet\langle Y_i\rangle$, whence the evident isomorphism
      \[
      \label{II.3.13.2}
        \Gr_F\big(\Omega_{X_1}^\bullet\langle Y_1\rangle(\cal{V}_1)\big)
        \boxtimes \Gr_F\big(\Omega_{X_2}^\bullet\langle Y_2\rangle(\cal{V}_2)\big)
        \simto \Gr_F\big(\Omega_X^\bullet\langle Y\rangle(\cal{V})\big).
      \tag{3.13.2}
      \]

      The evident isomorphism
      \[
        j_*^\mathrm{m}\Omega_{X_1^*}^\bullet
        \boxtimes j_*^\mathrm{m}\Omega_{X_2^*}^\bullet
        \simto j_*^\mathrm{m}\Omega_{X^*}^\bullet
      \]
\oldpage{82}
      identifies the filtration $P$ of $j_*^\mathrm{m}\Omega_{X^*}^\bullet$ with the external tensor product of the filtrations $P$ of the $j_*^\mathrm{m}\Omega_{X_i^*}^\bullet$.
      We thus have
      \[
      \label{II.3.13.3}
        \Gr_P\big(j_*^\mathrm{m}\Omega_{X_1^*}^\bullet(\cal{V}_1)\big)
        \boxtimes \Gr_P\big(j_*^\mathrm{m}\Omega_{X_2^*}^\bullet(\cal{V}_2)\big)
        \simto \Gr_P\big(j_*^\mathrm{m}\Omega_{X^*}^\bullet(\cal{V})\big).
      \tag{3.13.3}
      \]
      The morphism \hyperref[II.3.13.1]{(3.13.1)} can be identified, via \hyperref[II.3.13.2]{(3.13.2)} and \hyperref[II.3.13.3]{(3.13.3)}, with the external tensor product of the analogous morphisms for $\cal{V}_1$ and $\cal{V}_2$.
      Since the complexes in question have $\cal{O}_X$-linear differentials \hyperref[II.3.12]{(3.12.d)}, to prove that \hyperref[II.3.13.3] is a quasi-isomorphism, it suffices to prove the analogous claim for $\cal{V}_1$ and $\cal{V}_2$.
    \item[] \textbf{Case of constant coefficients.}
      We first prove \hyperref[II.3.13]{(3.13.ii)} in the case where the following conditions are satisfied:
      \begin{itemize}
        \item[(3.13.4)]\label{II.3.13.4}
          $X$ is the open polydisc $D^n$;
        \item[(3.13.5)]\label{II.3.13.5}
          $Y=\bigcup_{1\leq i\leq k}Y_i$, with $Y_i=\pr_i^{-1}(0)$; and
        \item[(3.13.6)]\label{II.3.13.6}
          $\cal{V}$ is the constant vector bundle defined by a vector space $V$, and the connection is of the form
          \[
            \Gamma = \sum_i\Gamma_i\frac{\dd z_i}{z_i}
          \]
          with $\Gamma_i\in\End(V)$ and $\Gamma_i=0$ for $i>k$.
      \end{itemize}

      Since the connection is integrable, the $\Gamma_i$ pairwise commute, and there exists a finite filtration $G$ of $V$ that is stable under the $\Gamma_i$, and such that $\dim\Gr_G^l(V)\leq1$.
      By the first reduction, we can assume that $V=\CC$, in which case the $\Gamma_i$ can be identified with a scalar $\gamma_i$.
      The bundle $\cal{V}$ with connection is then the external tensor product of the bundles $(\cal{O},\gamma_i\dd z/z)$ on $D$.
      The second reduction allows us to assume that $n=1$.
      If $k=0$, i.e. if $Y=\varnothing$, then $\Omega_X^\bullet\langle Y\rangle=j_*^\mathrm{m}\Omega_{X^*}^\bullet$ and $F=P$.
      If $k=1$, i.e. if $Y=\{0\}$, then
      \begin{enumerate}[a)]
        \item $P^i(j_*^\mathrm{m}\Omega_{X^*}^\bullet(\cal{V}))=0$ for $i>-1$;
\oldpage{83}
        \item $P^{-1}(j_*^\mathrm{m}\Omega_{X^*}^p(\cal{V}))$ is equal to $0$ if $p=0$, and to $\Omega_X^1\langle Y\rangle(\cal{V})$ if $p=1$;
        \item $\Gr_P^0(j_*^\mathrm{m}\Omega_{X^*}^\bullet(\cal{V}))$ is the complex
          \[
            \frac{1}{z}\cal{O}
            \xrightarrow{\partial_z+\gamma} \frac{1}{z^2}\cal{O}\bigg/\frac{1}{z}\cal{O}
          \]
          and if $\gamma-1\neq0$ then $\Coker(\dd)=0$ and $\Ker(\dd)=\cal{V}=\Omega_X^0\langle Y\rangle(\cal{V})$; and
        \item $\Gr_P^{-n}(j_*^\mathrm{m}\Omega_{X^*}^\bullet(\cal{V}))$, for $n>0$, is the complex
          \[
            \frac{1}{z^{n+1}}\cal{O}\bigg/\frac{1}{z^{n}}\cal{O}
            \xrightarrow{\partial_z+\gamma} \frac{1}{z^{n+2}}\cal{O}\bigg/\frac{1}{z^{n+1}}\cal{O}.
          \]
      \end{enumerate}
      This satisfies \hyperref[II.3.13]{(3.13.ii)} case by case.
    \item[] \textbf{General case.}
      Since the problem is local, we can suppose that conditions \hyperref[II.3.13.4]{(3.13.4)} and \hyperref[II.3.13.5]{(3.13.5)} are satisfied, and it suffices to prove that the germ at $0$ of \hyperref[II.3.13.1]{(3.13.1)} is a quasi-isomorphism.

      For $0<|t|\leq1$, let $\cal{V}_t$ be the bundle with connection given by the inverse image of $\cal{V}$ under the homothety $H_t$ with ratio $t$.
      As $t\to0$, the $\cal{V}_t$ ``tend'' to the constant vector bundle $\cal{V}_0$ defined by the fibre $V_0$ of $\cal{V}$ at $0$, endowed with a connection satisfying \hyperref[II.3.13.4]{(3.13.4)}, \hyperref[II.3.13.5]{(3.13.5)}, and \hyperref[II.3.13.6]{(3.13.6)}.

      More precisely, let $H$ and $i_t$ be the morphisms
      \[
        \begin{aligned}
          H\colon &D^n\times D \to D^n
          \colon &(x,t) \mapsto t\cdot x
        \\i_t\colon &D^n \to D^n\times D
          \colon &x\mapsto (x,t).
        \end{aligned}
      \]
      Then $H_t=H\circ i_t$.
      The inverse image of the connection $\nabla$ on $\cal{V}$ is $\nabla_1$ on $H^*\cal{V}|H^{-1}(X^*)$.
      The corresponding relative connection (relative to $\pr_2$) extends to $H^*(\cal{V})|X^*\times D$.
      If we set $\cal{V}_t=i_t^*H^*\cal{V}$, then, for $t\neq0$, we have an isomorphism of bundles with connections
      \[
      \label{II.3.13.7}
        \cal{V}_t \cong H_t^*(\cal{V}).
      \tag{3.13.7}
      \]
      For $t=0$, we have an isomorphism of vector bundles
\oldpage{84}
      \[
      \label{II.3.13.8}
        \cal{V}_0 = H_0^*(\cal{V}) = \cal{O}_X\otimes_{\CC}V_0
      \tag{3.13.8}
      \]
      and the connection on $\cal{V}_0|X^*$ satisfies \hyperref[II.3.13.4]{(3.13.4)}, \hyperref[II.3.13.5]{(3.13.5)}, and \hyperref[II.3.13.6]{(3.13.6)}.

      The relative version of \hyperref[II.3.12]{(3.12)} gives a morphism of filtered complexes
      \[
      \label{II.3.13.9}
        \varphi\colon
        \big(
          \Omega_{X\times D/D}^\bullet\langle Y\times D\rangle(H^*\cal{V}),F
        \big)
        \to
        \big(
          j_*^\mathrm{m}\Omega_{X^*\times D}^\bullet(H^*\cal{V}),P
        \big).
      \tag{3.13.9}
      \]
      The associated graded complexes are flat over $D$ (via $\pr_2$), and their homogeneous graded components are coherent, and the differentials are $\cal{O}_{X\times D}$-linear.
      We already know that $i_0^*\Gr(\varphi)$ is a quasi-isomorphism.
      It thus follows that $i_t^*\Gr^p(\varphi)$ (the arrow \hyperref[II.3.13.1]{(3.13.1)} for $\cal{V}_t$) is a quasi-isomorphism near to $0$, for $t$ small enough.
      Since the $\cal{V}_t$ are isomorphic to one another, close to $0$, for $t\neq0$ \hyperref[II.3.13.7]{(3.13.7)}, $i_1^*\Gr^p(\varphi)$ is a quasi-isomorphism near to $0$, which proves \hyperref[II.3.13].
  \end{itemize}
\end{proof}

\begin{itenv}{Corollary 3.14}
\label{II.3.14}
  Let $X$ be a smooth scheme over $\CC$, $Y$ a normal crossing divisor in $X$, $j$ the inclusion of $X^*=X\setminus Y$ into $X$, $\cal{V}$ a vector bundle on $X$, and $\Gamma$ an integrable connection on $\cal{V}|X^*$ that presents a logarithmic pole along $Y$.
  Suppose that the residues of the connection along $Y$ do not admit any strictly positive integer as an eigenvalue.
  Then
  \begin{enumerate}[(i)]
    \item the homomorphism of complexes
      \[
        i\colon \Omega_X^\bullet\langle Y\rangle(\cal{V}) \to j_*\Omega_{X^*}^\bullet(\cal{V})
      \]
      induces an isomorphism on the cohomology sheaves (for the Zariski topology); and
    \item more precisely, $i$ is injective, and there exists an exhaustive increasing filtration of the complex $\Coker(i)$ whose successive quotients are acyclic complexes whose differentials are linear.
  \end{enumerate}
\end{itenv}

\begin{proof}
  The filtration $P$ of \hyperref[II.3.12]{(3.12)} has an evident algebraic analogue, which also satisfies conditions~a) to d) of \hyperref[II.3.12]{(3.12)}.
  The corollary follows from the statement, which is more precise than (ii), saying that the complexes
  \[
    G^i = \Gr^i_P\big(j_*\Omega_{X^*}^\bullet(\cal{V})\big/\Omega_X^\bullet\langle Y\rangle(\cal{V})\big)
  \]
  are acyclic.
  These complexes have $\cal{O}_X$-linear differentials, and, by \hyperref[II.3.13]{(3.13)},
\oldpage{85}
  the $(G^i)^\an$ are acyclic.
  By the flatness of $\cal{O}_{X^\an}$ over $\cal{O}_X$, the $G^i$ are thus acyclic, which finishes the proof.
\end{proof}

\begin{itenv}{Corollary 3.15}
\label{II.3.15}
  Under the hypotheses of \hyperref[II.3.14]{(3.14)}, we have
  \[
    \mathbb{H}^k\big(X,\Omega_X^\bullet\langle Y\rangle(\cal{V})\big)
    \simto \mathbb{H}^k\big(X^*,\Omega_{X^*}^\bullet(\cal{V})\big).
  \]
\end{itenv}

\begin{proof}
  Since the morphism $j$ is affine, we have
  \[
    \RR^kj_*\Omega_{X^*}^p(\cal{V})=0
    \quad\mbox{for $k>0$}
  \]
  and so
  \[
    \mathbb{H}^\bullet\big(X,j_*\Omega_{X^*}^\bullet(\cal{V})\big)
    \simto \mathbb{H}^\bullet\big(X^*,\Omega_{X^*}^\bullet(\cal{V})\big).
  \]
  Also, by \hyperref[II.3.14]{(3.14.i)}, we have
  \[
    \mathbb{H}^\bullet\big(X,\Omega_X^\bullet\langle Y\rangle(\cal{V})\big)
    \simto \mathbb{H}^\bullet\big(X,j_*\Omega_{X^*}^\bullet(\cal{V})\big)
  \]
  whence the corollary.
\end{proof}

\begin{rmenv}{Remark 3.16}
\label{II.3.16}
  It is easy to generalise \hyperref[II.3.13]{(3.13)} and \hyperref[II.3.14]{(3.14)} to the relative situation, where $f\colon X\to S$ is a smooth morphism (with $S$ an analytic space, or scheme of characteristic~$0$), and $Y$ is a relative normal crossing divisor.
\end{rmenv}


\section{Regularity in dimension \texorpdfstring{$n$}{n}}
\label{II.4}

\medskip
\hrule
\medskip

\emph{[Translator] The proof of \hyperref[II.4.1]{Theorem~4.1} has been replaced with the proof given in the errata, which also cites the following:}
\begin{quote}
  N.~Katz.
  The regularity theorem in algebraic geometry.
  \emph{Actes du Congr\`{e}s intern. math.} \textbf{1} (1970), 437--443.
\end{quote}
\emph{Note that pages 87 and 88 are thus missing from this translation.}

\medskip
\hrule
\medskip

\begin{itenv}{Theorem~4.1}
\label{II.4.1}
  Let $X$ be a complex-analytic space, $Y$ a closed analytic subset of $X$ such that $X^*=X\setminus Y$ is smooth, $X'$ the normalisation of $X$, and $Y'$ the inverse image of $Y$ in $X'$.
  Let $\cal{V}$ be a vector bundle on $X^*$ that is meromorphic along $Y$, and let $\nabla$ be a connection on $\cal{V}$.
  Then the following conditions are equivalent:
  \begin{enumerate}[(i)]
    \item there exists an open subset $U$ of $Y'$ that contains a point of each codimension~$1$ component of $Y'$, and an isomorphism $\varphi$ from a neighbourhood of $U$ in $X'$ to $U\times D$ (where $D$ is the unit disc) that induces the identity map from $U$ to $U\times\{0\}$, such that, for all $u\in U$, the restriction of $\varphi^*\cal{V}$ to $\{u\}\times D$ is regular at~$0$;
\oldpage{86}
    \item for every map $\varphi\colon D\to X$ with $\varphi^{-1}(Y)=\{0\}$, the inverse image of $V$ under $\varphi$ is regular; and
    \item the (multiform) horizontal sections of $\cal{V}$ are of moderate growth along $Y$.
  \end{enumerate}

  By (i), these conditions hold ``in codimension~$1$ at infinity'' on the normalisation of $X$.
  If $Y$ is a smooth normal crossing divisor in $X$, then the above three conditions are also equivalent to the following:
  \begin{enumerate}
    \item[(iv)] for all $y\in Y$, there exists an open neighbourhood $U$ of $y$, and a basis $e\colon\cal{O}^d\to\cal{V}$ of $\cal{V}$ on $U\setminus Y$ that is meromorphic along $Y$, such that the connection matrix (which is a matrix of differential forms) presents at worst a logarithmic pole along $Y$.
  \end{enumerate}
\end{itenv}

\begin{proof}
  We will use constructions \hyperref[II.5.1]{(5.1)} to \hyperref[II.5.5]{(5.5)}; the reader can verify the lack of circularity.

  The theorem is local along $Y$, which allows us to suppose that
  \begin{enumerate}[a)]
    \item the meromorphic structure of $\cal{V}$ along $Y$ is defined by a coherent extension $\widetilde{\cal{V}}$ of the vector bundle $\cal{V}$ on $X$; and
    \item there exists a resolution of singularities $\pi\colon X_1\to X$ such that $X_1$ is smooth, $\pi$ is proper, $\pi^{-1}(X^*)\simto X^*$, and such that $Y_1=X_1\setminus\pi^{-1}(X^*)$ is a normal crossing divisor in $X_1$.
  \end{enumerate}

  We can think of $\cal{V}$ as a vector bundle with integrable connection on $X_1^*=\pi^{-1}(X^*)$, and $\widetilde{\cal{V}}$ as its canonical extension on $X_1$.
  The vector bundle $\cal{V}_0=\pi_*\widetilde{\cal{V}}$ is a coherent extension of $\cal{V}$ on $X$.
  We will show that conditions~(i), (ii), and (iii) are all equivalent to the following:
  \begin{enumerate}
    \item[\textit{(v)}] \textit{The extensions $\widetilde{\cal{V}}$ and $\cal{V}_0$ of $\cal{V}$ are meromorphically equivalent.}
  \end{enumerate}

  First we prove that \mbox{(i)$\implies$(v)}.
  If $X$ is of dimension~$1$, then this is a consequence of \hyperref[II.1.20]{(1.20)}.
  Let $u$ be the identity map from $\cal{V}_0|X^*$ to $\widetilde{\cal{V}}|X^*$.
  We need to show that, for every open subset $W\subset X$, for every linear form $w\in\HH^0(W,\cal{V}^\vee)$, and for every local section $e\in\HH^0(W,\cal{V}_0)$, the function $f=\langle w,u(e)\rangle$ is meromorphic along $Y\cap W$.

  Suppose that $X$ is smooth, that $Y$ is a smooth divisor in $X$, and that $X_1=X$.
  The fact that $f$ is meromorphic follows from the already discussed dimension~$1$ case, and from two applications of the following lemma (once to show that $f$ is meromorphic along $U$, and once again to show that $f$ is meromorphic along $Y$).

  \begin{itenv}{Lemma 4.1.1}
  \label{II.4.1.1}
    Let $f$ be an analytic function on $D^{m+1}\setminus(\{0\}\times D^m)$.
    Suppose that there exists a non-empty open subset $U$ of $D^m$ such that, for all $u\in U$, $f|(D^*\times\{u\})$ is meromorphic at $0$.
    Then there exists some $n$ such that $f$ has a pole of order at most $n$ along $\{0\}\times D^m$.
  \end{itenv}

  \begin{proof}
    Let $F_n\subset D^m$ be the set of $u$ such that $f|D^*\times\{u\}$ presents at worst a pole of order~$n$ at $0$.
    Let
    \[
      A_k(u) = \oint f(z,u)z^k\dd z.
    \]
    Then $A_k(u)$ is holomorphic, and $F_n$ is defined by the equations $A_k(u)=0$ for $k\geq n$.
    By hypothesis, the union of the closed subsets $F_n$ has an interior point.
    By Baire, there exists some $n$ such that $F_n$ has an interior point, and $A_k(u)$ is thus zero on an open subset (and thus everywhere) for all $k\geq n$.
    But then $f$ has at worst a pole of order~$n$ along $\{0\}\times D^m$.
  \end{proof}

  To pass from here to the case where $X$ is normal, we note that the above conditions are then satisfied outside of a subset $Z$ of $Y$ of codimension~$\geq2$ in $X$.
  We conclude by noting that a function $f$ on $X\setminus Y$ which is meromorphic along $Y$ outside of $Z$ is meromorphic along $Y$.
  Indeed, the proof of \hyperref[II.4.1.1]{(4.1.1)} shows that, locally on $Y$, the product of $f$ with a high enough power $g^k$ of a function that vanishes on $Y$ is holomorphic on $X\setminus Z$, and this product extends to a holomorphic function on $X$.

  In the general case, we note that condition~(i) (resp.~(v)) is equivalent to condition~(i) (resp.~(v)) on the normalisation of $X$.

  It is trivial that \mbox{(ii)$\implies$(i)}, and it follows from \hyperref[II.1.19]{(1.19)} that \mbox{(iii)$\implies$(i)}.

  Under the hypotheses of (iv), and for $X_1=X$, it is clear that \mbox{(v)$\implies$(iv)};
  by \hyperref[II.5.5]{(5.5.i)}, \mbox{(v)$\implies$(iii)};
  since the inverse image of a differential form that presents at worst a simple pole also presents at worst a simple pole, \mbox{(iv)$\implies$(ii)}.
  Under these hypotheses, claims~(i) to (v) are thus equivalent.

  It follows from \hyperref[II.2.19]{(2.19)} and from the above that \mbox{(v)$\implies$(iii)}.
  To prove that \mbox{(v)$\implies$(ii)}, we can either use \hyperref[II.1.19]{(1.19)} and \hyperref[II.2.19]{(2.19)} to show that \mbox{(iii)$\implies$(ii)}, or we can note that condition~(ii) for $\cal{V}$ on $X^*\subset X$ is equivalent to condition~(ii) for $\pi^{-1}\cal{V}$ on $X^*\subset X_1$.
  Conditions~(i), (ii), (iii), and (v) are thus equivalent.
  In particular, condition~(v) is independent of the choice of $X_1$;
  we thus deduce that, under the hypotheses of (iv), \mbox{(iv)$\iff$(v)}, and this finishes the proof.
\end{proof}

We note that the above proof already contains the essential part of the proof of \hyperref[II.5.7]{(5.7)} and \hyperref[II.5.9]{(5.9)} (the existence theorem).

\oldpage{89}
\begin{rmenv}{Definition 4.2}
\label{II.4.2}
  Under the hypotheses of \hyperref[II.4.1]{(4.1)}, we say that $(\cal{V},\nabla)$ is \emph{regular along $Y$} if any of the equivalent conditions of \hyperref[II.4.1]{(4.1)} are satisfied.
\end{rmenv}

\begin{itenv}{Proposition 4.3}
\label{II.4.3}
  With the hypotheses and notation of \hyperref[II.2.19]{(2.19)}, let $\cal{V}$ be a vector bundle on $X_2^*$ that is meromorphic along $Y_2$, endowed with an integrable connection.
  Then
  \begin{enumerate}[(a)]
    \item if $\cal{V}$ is regular, then $f^*\cal{V}$ is regular; and
    \item if condition~\hyperref[II.2.19]{(2.19.a)} is satisfied, and if $f^*\cal{V}$ is regular, then $\cal{V}$ is regular.
  \end{enumerate}
\end{itenv}

\begin{proof}
  By \hyperref[II.2.19]{(2.19)}, this is clear from \hyperref[II.4.1]{(4.1.iii)}.
\end{proof}

\begin{itenv}{Proposition 4.4}
\label{II.4.4}
  Let $\cal{V}$ be a vector bundle on a smooth separated complex-algebraic variety $X$.
  Let $\overline{X}$ be a compactification of $X$, so that $\cal{V}^\an$ is meromorphic along $Y=\overline{X}\setminus X$.
  Let $\nabla$ be a connection on $\cal{V}^\an$.
  Then the following conditions are equivalent:
  \begin{enumerate}[(i)]
    \item $\cal{V}^\an$ is regular along $Y$; and
    \item for every smooth algebraic curve $C$ on $X$ (and locally closed in $X$), $\cal{V}|C$ is regular \hyperref[II.1.21]{(1.21)}.
  \end{enumerate}

  If $\overline{X}$ is normal, then the above two conditions are also equivalent to the following:
  \begin{enumerate}
    \item[(iii)] $\nabla$ is algebraic, and, for every generic point $\eta$ of a codimension~$1$ component of $Y$, there exists an algebraic vector field $v$ on a neighbourhood of $\eta$, with $v$ transversal to $Y$ (so that the triple $(\cal{O}_\eta,\cal{O}_\eta,\partial_v)$ satisfies \hyperref[II.1.4.1]{(1.4.1)}), such that
\oldpage{90}
    $\cal{V}$ induces, over the field of fractions $K$ of $\cal{O}_\eta$, endowed with $\partial_v$, a vector space with regular connection, in the sense of \hyperref[II.1.11]{(1.11)}; and
    \item[(iii')] \emph{idem.} for every field $v$ of this type.
  \end{enumerate}
\end{itenv}

\begin{proof}
  We have \mbox{(ii)$\implies$\hyperref[II.4.1]{(4.1.i)}$\implies$\hyperref[II.4.1]{(4.1.ii)}$\implies$(ii)}.
  Also, \hyperref[II.4.1]{(4.1.iv)} implies that $\nabla$ is meromorphic in codimension~$1$ on $\overline{X}$, and thus meromorphic, and thus algebraic (by GAGA).
  We thus have
  \[
    \mbox{(iii')}
    \implies \mbox{(iii)}
    \implies \mbox{\hyperref[II.4.1]{(4.1.i)}}
    \implies \mbox{\hyperref[II.4.1]{(4.1.iv)}}
    \implies \mbox{(iii')}.
  \]
\end{proof}

\begin{rmenv}{Definition 4.5}
\label{II.4.5}
  Under the hypotheses of \hyperref[II.4.4]{(4.4)}, we say that $(\cal{V},\nabla)$ is \emph{regular} if any of the equivalent conditions of \hyperref[II.4.4]{(4.4)} are satisfied.
\end{rmenv}

If $(\cal{V},\nabla)$ is a vector bundle with integrable algebraic connection on $X$, then it is clear, by \hyperref[II.4.4]{(4.4.ii)}, that the regularity of $\nabla$ is a purely algebraic condition, independent of the choice of compactification.
We can, in many different ways, define regularity when $X$ is a smooth scheme of finite type over a field $k$ of characteristic~$0$.
For example, we can take \hyperref[II.4.4]{(4.4.ii)} or \hyperref[II.4.4]{(4.4.iii)} as a definition.
We will restrict ourselves in what follows to the case where $k=\CC$.
By the Lefschetz principle, this does not reduce the level of generality.

\begin{itenv}{Proposition 4.6}
\label{II.4.6}
  Let $X$ be a (smooth) complex-algebraic variety.
  \begin{enumerate}[(i)]
    \item If $\cal{V}'\to\cal{V}\to\cal{V}''$ is a horizontal exact sequence of vector bundles with integrable connections on $X$, and if $\cal{V}'$ and $\cal{V}''$ are regular, then $\cal{V}$ is regular.
    \item If $\cal{V}_1$ and $\cal{V}_2$ are vector bundles with regular integrable connections on $X$, then $\cal{V}_1\otimes\cal{V}_2$, $\shHom(\cal{V}_1,\cal{V}_2)$, $\cal{V}_1^\vee$, and $\bigwedge^p\cal{V}_1$ are all regular.
    \item Let $f\colon X\to Y$ be a morphism of smooth schemes over $\CC$, and $\cal{V}$ a vector bundle with integrable connection on $Y$.
      If $\cal{V}$ is regular, then $f^*\cal{V}$ is regular.
      Conversely, if $f^*\cal{V}$ is regular and $f$ is dominant, then $\cal{V}$ is regular.
  \end{enumerate}
\end{itenv}

\begin{proof}
  Claims~(i) and (ii) follow immediately from the definition, by \hyperref[II.4.4]{(4.4.ii)}
\oldpage{91}
  and \hyperref[II.1.13]{(1.13)}.
  It is clear, by \hyperref[II.4.4]{(4.4.iii)}, that regularity, being satisfied in codimension~$1$ at infinity, is a \emph{birational notion}.
  This allows us to replace ``$f$ is dominant'' in (iii) by ``$f$ is surjective''.
  We then apply \hyperref[II.4.4]{(4.4.ii)} and \hyperref[II.1.13]{(1.13.iii)} by noting that, for $f$ surjective, for every curve $C$ on $Y$, there exists a curve $C'$ on $X$ such that $\overline{f(C')}\supset C$.
\end{proof}



\section{Existence theorem}
\label{II.5}










\todo










\section{Comparison theorem}
\label{II.6}










\todo










\section{Regularity theorem}
\label{II.7}










\todo













\chapter{Applications and examples}
\label{III}


\section{Functions in the Nilsson class}
\label{III.1}

\begin{rmenv}{1.1}
\label{III.1.1}
\oldpage{122}
  Let $X$ be a non-singular complex algebraic variety that is connected and endowed with a base point $x_0$.
  We denote by $\widetilde{X}$ the universal cover of $(X,x_0)$, and by $\widetilde{x}_0$ the base point of $\widetilde{X}$.
  Suppose that we have a given complex representation $W_0$ of finite dimension~$d$ of $\pi_1(X,x_0)$ endowed with a cyclic vector $w_0$.
  We denote by $W$ the corresponding local system \hyperref[I.1.4]{(I.1.4)}, and by $\scr{W}$ the algebraic vector bundle with regular integrable connection endowed with an isomorphism of $\pi_1(X,x_0)$-representations \hyperref[II.5.7]{(II.5.7)}
  \[
    \scr{W}_{x_0} \simeq W.
  \]
  Finally, we denote by $w$ the multiform horizontal section of $\scr{W}^\an$ with base determination $w_0$.
\end{rmenv}

\begin{rmenv}{Definition 1.2}
\label{III.1.2}
  A section of an \emph{algebraic} vector bundle $\cal{V}$ on $X$ is said to be \emph{in the Nilsson class} if it is a multiform holomorphic section of finite determination that is of moderate growth at infinity \hyperref[II.2.23]{(II.2.23.iv)}.

  If $\cal{V}=\cal{O}$, then we speak of \emph{functions in the Nilsson class}.
\end{rmenv}

The two following theorems will be proven simultaneously in \hyperref[III.1.5proof]{(1.5)}.
The first says that, for a function \emph{of finite determination}, various variants of the ``moderate growth at infinity'' condition are equivalent.

\begin{itenv}{Theorem 1.3}
\label{III.1.3}
  Let $s$ be a multiform holomorphic section of finite determination of an algebraic vector bundle on $X$.
  Then the following conditions are equivalent:
  \begin{enumerate}[(i)]
    \item $s$ is in the Nilsson class; and
    \item the restriction of $s$ to every (locally closed) smooth algebraic curve along $X$ is in the Nilsson class.
  \end{enumerate}
\oldpage{123}
  If $X$ is a Zariski open subset of a compact normal variety $\overline{X}$, then the two conditions above are also equivalent to the following:
  \begin{enumerate}
    \item[{\rm(iii)}] every irreducible component of $\overline{X}\setminus X$ of codimension~$1$ in $\overline{X}$ contains a non-empty open subset $U$ along which $s$ is of moderate growth.
  \end{enumerate}
\end{itenv}

In the above, we do not lose any generality in supposing, in (iii), that $X\cup U\subset\overline{X}$ is smooth, and that $U$ is a smooth divisor there.
Unlike (i), conditions (ii) and (iii) do not make any reference to the theory of Lojasiewicz.
It follows from (iii) that, if $\codim(\overline{X}\setminus X)\geq2$, then a function of finite determination is automatically in the Nilsson class.

For $X$ of dimension~$1$, the following theorem is due to Plemelj \cite{23}.

\begin{itenv}{Theorem 1.4}
\label{III.1.4}
  Let $\cal{V}$ be an algebraic vector bundle on $X$.
  The ``evaluation at $w$'' function, that sends each (algebraic) $f\in\Hom(\scr{W},\cal{V})$ to the section $f(w)$ of $\cal{V}^\an$ gives a bijection between $\Hom(\scr{W},\cal{V})$ and the set of sections of $\cal{V}$ in the Nilsson class that are of monodromy subordinate to $(W_0,w_0)$.
\end{itenv}

\begin{rmenv}{1.5}
\label{III.1.5proof}
  Here we prove \hyperref[III.1.3]{(1.3)} and \hyperref[III.1.4]{(1.4)}.
  We have already seen (in \hyperref[I.6.11]{(I.6.11)}) that the function $E_w\colon f\mapsto f(w)$ identifies $\Hom(\scr{W}^\an,\cal{V}^\an)$ with the set of multiform holomorphic sections of $\cal{V}^\an$ of finite determination with monodromy subordinate to $(W_0,w_0)$.
  It thus remains to prove that $f$ is algebraic if and only if $f(w)$ satisfies \hyperref[III.1.3]{(1.3.i)} (resp. \hyperref[III.1.3]{(1.3.ii)}, resp. \hyperref[III.1.3]{(1.3.iii)}).

  By \hyperref[II.4.1]{(II.4.1.iii)}, the ``section'' $w$ of $\scr{W}$ is in the Nilsson class, so that, if $f$ is algebraic, then $f(w)$ satisfies \hyperref[III.1.3]{(1.3.i)}.
  Trivially, \mbox{\hyperref[III.1.3]{(1.3.i)}$\implies$\hyperref[III.1.3]{(1.3.ii)}} and \mbox{\hyperref[III.1.3]{(1.3.i)}$\implies$\hyperref[III.1.3]{(1.3.iii)}}.

  Let $e\colon\cal{O}^d\to\scr{W}$ be a multiform basis of $\scr{W}$ consisting of determinations of $w$.
  Since $\scr{W}$ is regular, $e^{-1}$ is of moderate growth at infinity.
  For $f\colon\scr{W}^\an\to\cal{V}^\an$, the $f(e_i)$ are determinations of $f(w)$.
  From this, and from the equation $f=fee^{-1}$, we deduce that
\oldpage{124}
  \begin{enumerate}[a)]
    \item if $f(w)$ satisfies \hyperref[III.1.3]{(1.3.ii)}, then the restriction of $f\in\HH^0(\shHom(\scr{W},\cal{V})^\an)$ to any curve is of moderate growth; and
    \item if $f(w)$ satisfies \hyperref[III.1.3]{(1.3.iii)}, then $f$ is of moderate growth near a non-empty open subset of each irreducible component of $\overline{X}\setminus X$ of codimension~$1$.
  \end{enumerate}
\end{rmenv}

By \hyperref[II.4.1.1]{(II.4.1.1)}, under each of these hypotheses, $f$ is algebraic.

\begin{itenv}{Corollary 1.5}
\label{III.1.5}
  Under the hypotheses of \hyperref[III.1.4]{(1.4)}, if $X$ is affine, with coordinate ring $A$, and if $\cal{V}$ is of rank~$m$, then the set of sections of $\cal{V}$ in the Nilsson class that are of monodromy subordinate to $(W_0,w_0)$ is a \emph{projective} $A$-module of rank~$dm$.
\end{itenv}

\begin{rmenv}{Remark 1.6}
\label{III.1.6}
  A \emph{meromorphic function in the Nilsson class} is, by definition, a section of some sheaf $\cal{O}(D)$, for $D$ a sufficiently positive divisor, in the Nilsson class (where $\cal{O}(D)$ is the sheaf of meromorphic functions $f$ such that $\operatorname{div}(f)\geq-D$).
  It follows from \hyperref[III.1.5]{(1.5)} that the set of meromorphic functions in the Nilsson class that are of monodromy subordinate to $(W_0,w_0)$ is a vector space of dimension~$d$ over the field of rational functions on $X$.
\end{rmenv}

\begin{rmenv}{1.7}
\label{III.1.7}
  Let $f\colon X\to S$ be a smooth morphism, with $S$ smooth.
  By \hyperref[III.1.4]{(1.4)}, the set of relative differential $p$-forms on $X$ that are in the Nilsson class and of monodromy subordinate to $(W_0,w_0)$ can be identified with the space
  \[
    \HH^0\Big(
      S,\Ker\big(
      \dd\colon f_*\Omega_{X/S}^p(\scr{W}^\vee)\to f_*\Omega_{X/S}^{p+1}(\scr{W}^\vee)
      \big)
    \Big).
  \]

  Let $U'$ be a dense Zariski open subset of $S$ such that, over $U'$, $f$ is locally $C^\infty$-trivial.
  The \emph{homology} groups $\HH_p(X_S^\an,W)$ then form a local system $\scr{H}$ on $U'$.

  We denote by $\langle-,-\rangle$ the pairing of sheaves on a small enough dense Zariski open subset $U\subset U'$ \hyperref[II.6.13]{(II.6.13)}
  \[
    \begin{aligned}
      \scr{H} \otimes \Ker\big(\dd\colon f_*\Omega_{X/S}^p(\scr{W}^\vee)\to f_*\Omega_{X/S}^{p+1}(\scr{W}^\vee)\big)
      &\to \scr{H} \otimes \RR^p f_* \Omega_{X/S}^\bullet(\scr{W}^\vee)
    \\&\to \scr{H} \otimes \RR^p f_*^\an \scr{W}^\vee \otimes \cal{O}_S^\an
    \\&\to \cal{O}_S^\an.
    \end{aligned}
  \]

\oldpage{125}
  We define a \emph{period} of a closed relative $p$-form $\alpha$ in the Nilsson class of the type considered above to be any multiform function on $U$ of the form $\langle h,\alpha\rangle$, with $h$ a multiform (horizontal) section of $\scr{H}$.
  A period is thus a multiform function of finite determination that is of monodromy subordinate to $\scr{H}$.
  \hyperref[III.1.4]{Theorem~1.4} and \hyperref[II.6.13]{Theorem~II.6.13} thus give us the theorem essentially equivalent to that of \hyperref[II.7.4]{(II.7.4)}.
\end{rmenv}

\begin{itenv}{Theorem 1.8}
\label{III.1.8}
  Under the hypotheses of \hyperref[III.1.7]{(1.7)}, the periods of a closed relative differential $p$-form on $X$ in the Nilsson class are functions in the Nilsson class on a suitable dense Zariski open subset of $S$.
\end{itenv}


\section{The monodromy theorem (by Brieskorn)}
\label{III.2}

The proof of the monodromy theorem given in this section is due to Brieskorn \cite{5}.

\begin{rmenv}{2.1}
\label{III.2.1}
  Let $S$ be a smooth algebraic curve over $\CC$, induced by a smooth projective curve $\overline{S}$ by removing a finite set $T$ of points.
  For $t\in T$, the \emph{local monodromy group at $t$}, or the \emph{local fundamental group of $S$ at $t$}, is the fundamental group of $D\setminus\{t\}$, where $D$ is a small disc centred at $t$.
  This group is canonically isomorphic to $\ZZ$, and we call its canonical generator the \emph{monodromy transformation}.

  If $\cal{V}$ is a local system of $\CC$-vector spaces on $S$, then the local monodromy group at $t$ acts on $V|(D\setminus\{t\})$.
  If $\cal{V}$ is the complexification of a local system of $\ZZ$-modules of finite type, then the characteristic polynomial of the monodromy transformation has integer coefficients.

  Recall that a linear substitution is said to be quasi-unipotent if one of its powers is unipotent.
  A local system of $\CC$-vector spaces on $S$ is said to be \emph{quasi-unipotent} (resp. \emph{unipotent}) \emph{at infinity} if, for all $t\in T$, the corresponding monodromy transformation is quasi-unipotent (resp. unipotent).
\end{rmenv}

\oldpage{126}
\begin{rmenv}{Example 2.2}
\label{III.2.2}
  Let $X=\SL_2(\mathbb{R})/\SO_2(\mathbb{R})$ be the Poincar\'{e} half plane, and $\Gamma$ a torsion-free discrete subgroup of $\SL_2(\mathbb{R})$ such that $\Gamma\backslash\SL_2(\mathbb{R})$ is of finite volume.
  We then know that $\Gamma\backslash X$ is an algebraic curve, with fundamental group $\Gamma$.
  Each finite-dimensional complex representation $\rho$ of $\Gamma$ thus defines a local system $V_\rho$ on $\Gamma\backslash X$ (and conversely).
  For $V_\rho$ to be unipotent at infinity, it is necessary and sufficiently for $\rho(\gamma)$ to be unipotent for every element $\gamma$ of $\Gamma$ that is unipotent in $\SL_2(\mathbb{R})$.
\end{rmenv}

\begin{itenv}{Theorem 2.3}
\label{III.2.3}
  Let $S$ be as in \hyperref[III.2.1]{(2.1)}, let $i$ be an integer, and let $f\colon X\to S$ be a smooth morphism.
  Suppose that $\RR^if_*\CC$ is a local system (i.e. that it is locally constant) \hyperref[II.6.13]{(II.6.13)}.
  Then $\RR^if_*\CC$ is quasi-unipotent at infinity.
\end{itenv}

The proof relies on \hyperref[II.7.4]{(II.7.4)} and on the following theorem of Gelfond (\cite{6} or \cite{2}):

\begin{itenv*}
\label{III.2.3.*}
  {\rm\textbf{($*$)}}
  If $\alpha$ and $\exp(2\pi i\alpha)$ are algebraic numbers, then $\alpha$ is rational.
\end{itenv*}

An immediate corollary of \hyperref[III.2.3.*]{($*$)} is:

\begin{itenv*}
\label{III.2.3.**}
  {\rm\textbf{($**$)}}
  If $N$ is a matrix with entries in a subfield $K$ of $\CC$, and if, for every embedding $\sigma K\to\CC$, the characteristic polynomial of $\exp(2\pi i\sigma(N))$ has integer coefficients, then $\exp(2\pi iN)$ is quasi-unipotent.
\end{itenv*}

Indeed, let $\alpha$ be an eigenvalue of $N$ in an extension $K'$ of $K$.
For every embedding $\sigma$ of $K'$ into $\CC$, $\exp(2\pi i\sigma(\alpha))$ is algebraic.
We thus deduce first of all that $\alpha$ is algebraic, since otherwise $\sigma(\alpha)$ could take any non-algebraic value.
Then \hyperref[III.2.3.*]{($*$)} implies that $\alpha$ is rational, so that the eigenvalues $\exp(2\pi i\alpha)$ of $\exp(2\pi iN)$ are roots of unity.

\begin{proof}[Proof of {\hyperref[III.2.3]{Theorem 2.3}}]
  By shrinking $S$ if necessary, we can assume that $\scr{H}=\RR^if_*\Omega_{X/S}^\bullet$ is locally free.

  Let $K$ be a subfield of $\CC$ such that $f$, $X$, $S$, $\overline{S}$, and the points of $T$ are all definable over $K$, i.e. all come from extension of scalars $\sigma_0\colon K\to\CC$ of
\oldpage{127}
  \[
    f_0\colon X_0\to S_0
    \quad\mbox{and}\quad
    T_0\subset\overline{S}_0(K).
  \]

  The Gauss--Manin connection on $\scr{H}_0=\RR^if_*\Omega_{X_0/S_0}^\bullet$ is regular \hyperref[II.7.4]{(II.7.4)}.
  There thus exists an extension of $\scr{H}'_0$ to a vector bundle on $\overline{S}_0$ such that the connection has at worst a simple pole at each $t\in T_0$.
  Let $N_t$ be the matrix of the residue of the connection at $t\in T_0$ in a basis of $(\scr{H}'_0)_t$.

  For every embedding $\sigma\colon K\to\CC$, $f_0$ defines, by extension of scalars,
  \[
    f_{(\sigma)}\colon X_{(\sigma)}\to S_{(\sigma)}
  \]
  and $\scr{H}_{(\sigma)}=\RR^i(f_{(\sigma)})_*\Omega_{X_{(\sigma)}/S_{(\sigma)}}^\bullet$ is induced by extension of scalars from $\scr{H}_0$.
  By \hyperref[II.1.17.1]{(II.1.17.1)}, $\exp(2\pi i\sigma/N_t)$ has the same characteristic polynomial as the local monodromy transformation at $t$ acting on $\RR^i(f_{(\sigma)})_*\CC$.
  Thus $\exp(2\pi i\sigma/N_t)$ has a characteristic polynomial with integer coefficients, and, by \hyperref[III.2.3.**]{($**$)}, $\exp(2\pi iN_t)$ is quasi-unipotent, whence \hyperref[III.2.3]{Theorem~2.3}.
\end{proof}




%% Bibliography %%

\nocite{*}

\begin{thebibliography}{26}

  \bibitem{1}
  {Atiyah,~M. and Hodge,~W.L.}
  \newblock Integrals of the second kind on an algebraic variety.
  \newblock {\em Ann. of Math.} \textbf{62} (1955), 56--91.

  \bibitem{2}
  {Baily,~W.L. and Borel,~A.}
  \newblock Compactification of arithmetic quotients of bounded domains.
  \newblock {\em Ann. of Math.} \textbf{84} (1966), 442--528.

  \bibitem{3}
  {Baker,~A.}
  \newblock Linear forms in the logarithms of algebraic numbers II.
  \newblock {\em Mathematika} \textbf{14} (1967), 102--107.

  \bibitem{4}
  {Berthelot,~P.}
  \newblock Cohomologie $p$-cristalline des sch\'{e}mas.
  \newblock {\em CR Acad. Sci. Paris} \textbf{269} (1969), 297--300, 357--360, and 397--400.

  \bibitem{5}
  {Brieskorn,~E.}
  \newblock Die monodromie der isolierten singularit\"{a}ten von hyperfl\"{a}chen.
  \newblock {\em Manuscripta math.} \textbf{2} (1970), 103--161.

  \bibitem{6}
  {Gelfond,~A.}
  \newblock Sur le septi\`{e}me probl\`{e}me de D.~Hilbert.
  \newblock {Doklady Akad. Nauk. URSS} \textbf{2} (1934), 4--6.

  \bibitem{7}
  {Godement,~R.}
  \newblock {\em Topologie alg\'{e}brique et th\'{e}orie des faisceaux}.
  \newblock Hermann, 1958.

  \bibitem{8}
  {Griffiths,~P.A.}
  \newblock ``Some results on Moduli and Periods of Integrals on Algebraic Manifolds III''.
  \newblock (Mimeographed notes from Princeton).

  \bibitem{9}
  {Grothendieck,~A.}
  \newblock On the De Rham cohomology of algebraic varieties.
  \newblock {\em Publ. Math. IHES} \textbf{29} (1966), 95--103.

  \bibitem{10}
  {Grothendieck,~A. (notes by Coates,~I. and Jussila,~O.)}
  \newblock ``Crystals and the De Rham cohomology of schemes'', in: {\em dix expos\'{e}s sur la cohomologie des sch\'{e}mas}.
  \newblock North Holl. Publ. Co., 1968.

  \bibitem{11}
  {Gunning,~R.}
  \newblock {\em Lectures on Riemann surfaces}.
  \newblock Princeton Math. Notes, 1966.

  \bibitem{12}
  {Hironaka,~H.}
  \newblock Resolution of singularities of an algebraic variety over a field of characteristic zero, I and II.
  \newblock {\em Ann. of Math.} \textbf{79} (1964).

  \bibitem{13}
  {Ince,~E.L.}
  \newblock {\em Ordinary differential equations}.
  \newblock Dover, 1956.

  \bibitem{14}
  {Katz,~N.}
  \newblock Nilpotent connections and the monodromy theorem. Applications of a result of Turrittin.
  \newblock {\em Publ. Math. IHES} \textbf{39} (1970), 175--232.

  \bibitem{15}
  {Katz,~N. and Oda,~T.}
  \newblock On the differentiation of De Rham cohomology classes with respect to parameters.
  \newblock {\em J. Math. Kyoto Univ.} \textbf{8} (1968), 199--213.

  \bibitem{16}
  {Leray,~J.}
  \newblock Un compl\'{e}ment au th\'{e}or\`{e}me de N.~Nilsson sur les int\'{e}grales de formes diff\'{e}rentielles \`{a} support singulier alg\'{e}brique.
  \newblock {\em Bull. Soc. Math. France} \textbf{95} (1967), 313--374.

  \bibitem{17}
  {Lojasiewicz,~S.}
  \newblock Triangulation of semi-analytic sets.
  \newblock {\em Annali della Scuola Normale Sup. di Pisa Ser III} \textbf{18} (1964), 449--474.

  \bibitem{18}
  {Lojasiewicz,~S.}
  \newblock (Mimeographed notes by IHES).

  \bibitem{19}
  {Manin,~Y.}
  \newblock Moduli Fuchsiani.
  \newblock {\em Annali della Scuola Normale Sup. di Pisa Ser III} \textbf{19} (1965), 113--126.

  \bibitem{20}
  {Nagata,~M.}
  \newblock Embedding of an abstract variety in a complete variety.
  \newblock {\em J. Math. Kyoto} \textbf{2} (1962), 1--10.

  \bibitem{21}
  {Nagata,~M.}
  \newblock A generalization of the embedding problem.
  \newblock {\em J. Math. Kyoto} \textbf{3} (1963), 89--102.

  \bibitem{22}
  {Nilsson,~N.}
  \newblock Some growth and ramification properties of certain integrals on algebraic manifolds.
  \newblock {\em Arkiv f\"{o}r Math.} \textbf{5} (1963--65), 527--540.

  \bibitem{23}
  {Plemelj,~J.}
  \newblock {\em Monatsch. Math. Phys.} \textbf{19} (1908), 211.

  \bibitem[GAGA]{GAGA}
  {Serre,~J.P.}
  \newblock Géométrie algébrique et géométrie analytique.
  \newblock {\em Ann. Inst. Fourier. Grenoble} \textbf{6} (1956).

  \bibitem[25]{25}
  {Turrittin,~H.L.}
  \newblock Convergent solutions of ordinary homogeneous differential equations in the neighbourhood of a singular point.
  \newblock {\em Acta Math.} \textbf{93} (1955), 27--66.

  \bibitem[26]{26}
  {Turrittin,~H.L.}
  \newblock ``Asymptotic expansions of solutions of systems of ordinary linear differential equations containing a parameter'', in: {\em Contributions to the theory of nonlinear oscillations.}
  \newblock Princeton, 1952.

  \bibitem[27]{27}
  {G\'{e}rard,~R.}
  \newblock {\em Th\'{e}orie de Fuchs sur une vari\'{e}t\'{e} analytic complexe}.
  \newblock Thesis, Strasbourg (1968).

\end{thebibliography}

\end{document}
