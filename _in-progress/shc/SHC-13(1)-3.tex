\documentclass{article}

\title{Regular deformations}
\author{Adrien Douady}
\date{14\textsuperscript{th} of November, 1960}

\usepackage{amssymb,amsmath}

\usepackage{hyperref}
\usepackage{xcolor}
\hypersetup{colorlinks,linkcolor={red!50!black},citecolor={blue!50!black},urlcolor={blue!80!black}}
\usepackage[nameinlink]{cleveref}
\usepackage{enumerate}
\usepackage{tikz-cd}

\usepackage[mathscr]{eucal}
%% Fancy fonts --- feel free to remove! %%
\usepackage{fouriernc}


\usepackage{fancyhdr}
\usepackage{lastpage}
\usepackage{xstring}
\makeatletter
\ifx\pdfmdfivesum\undefined
  \let\pdfmdfivesum\mdfivesum
\fi
\edef\filesum{\pdfmdfivesum file {\jobname}}
\pagestyle{fancy}
\makeatletter
\let\runauthor\@author
\let\runtitle\@title
\makeatother
\fancyhf{}
\lhead{\footnotesize\runtitle}
\lfoot{\footnotesize Version: \texttt{\StrMid{\filesum}{1}{8}}}
\cfoot{\small\thepage\ of \pageref*{LastPage}}


\crefname{section}{\S\!}{\S\S\!}
\crefname{equation}{}{}
\renewcommand{\thesection}{\Roman{section}}
\renewcommand{\thesubsection}{\arabic{subsection}}


%% Theorem environments %%

\usepackage{amsthm}

\theoremstyle{plain}

  \newtheorem{innercustomtheorem}{Theorem}
  \crefname{innercustomtheorem}{Theorem}{Theorems}
  \newenvironment{theorem}[1]
    {\renewcommand\theinnercustomtheorem{#1}\innercustomtheorem}
    {\endinnercustomtheorem}

  \newtheorem{innercustomproposition}{Proposition}
  \crefname{innercustomproposition}{Proposition}{Propositions}
  \newenvironment{proposition}[1]
    {\renewcommand\theinnercustomproposition{#1}\innercustomproposition}
    {\endinnercustomproposition}

  \newtheorem*{proposition*}{Proposition}

\theoremstyle{definition}
  \newtheorem*{definition*}{Definition}


%% Shortcuts %%

\newcommand{\RR}{\mathbb{R}}
\newcommand{\CC}{\mathbb{C}}
\newcommand{\DD}{\mathrm{D}}
\newcommand{\HH}{\mathrm{H}}

\renewcommand{\geq}{\geqslant}
\renewcommand{\leq}{\leqslant}

\newcommand{\oldpage}[1]{\marginpar{\footnotesize$\Big\vert$ \textit{p.~#1}}}


%% Document %%

% \usepackage{embedall}
\begin{document}

\maketitle
\thispagestyle{fancy}

\renewcommand{\abstractname}{Translator's note.}

\begin{abstract}
  \renewcommand*{\thefootnote}{\fnsymbol{footnote}}
  \emph{This text is one of a series\footnote{\url{https://thosgood.com/translations/}} of translations of various papers into English.}
  \emph{The translator takes full responsibility for any errors introduced in the passage from one language to another, and claims no rights to any of the mathematical content herein.}

  \medskip
  
  \emph{What follows is a translation of the French seminar talk:}

  \medskip\noindent
  \textsc{Douady, A.}
  ``D\'{e}formations r\'{e}guli\`{e}res''.
  \emph{S\'{e}minaire Henri Cartan}, Volume~\textbf{13 (1)} (1960--1961), Talk no.~3.
  {\url{http://www.numdam.org/item/SHC_1960-1961__13_1_A2_0/}}
\end{abstract}

\setcounter{footnote}{0}

\tableofcontents
\bigskip


%% Content %%

\oldpage{3-01}
All throughout this talk, $B$ is a $\mathscr{C}^\infty$ manifold (resp. $\RR$-analytic, resp. $\CC$-analytic); $\pi\colon V\to B$ denotes a proper mixed manifold; $b_0$ is a point of $B$; and $V_0=\pi^{-1}(b_0)$ is thus a compact $\CC$-analytic manifold.


\section{The map \texorpdfstring{$\widetilde{\rho}$}{p}}
\label{I}

Let $\widetilde{\Theta}$ (resp. $\widetilde{\Pi}$) be the sheaf of germs of vertical holomorphic (resp. locally projectable holomorphic) vector fields on $V$.
The quotient sheaf $\widetilde{\Lambda}=\widetilde{\Pi}/\widetilde{\Theta}$ is exactly the inverse image under $\pi$ of the sheaf $\widetilde{T}$ of germs of $\mathscr{C}^\infty$ fields (resp. \ldots) of tangent vectors on $B$.

For every open subset $U$ of $B$, set $V_U=\pi^{-1}(U)$.
The exact sequence
\[
  0 \to \widetilde{\Theta} \to \widetilde{\Pi} \to \widetilde{\Lambda} \to 0
\]
of sheaves on $V_U$ gives rise to a homomorphism
\[
  \widetilde{\rho}_U\colon
  \HH^0(U;\widetilde{T})
  \xrightarrow{\pi_*} \HH^0(V_U;\widetilde{\Lambda})
  \xrightarrow{\delta} \HH^1(V_U;\widetilde{\Theta}).
\]
Let $\mathrm{R}^1\pi_*\widetilde{\Theta}$ be the sheaf on $B$ defined by the presheaf $U\mapsto\HH^1(V_U;\widetilde{\Theta})$.
Then $\widetilde{\rho}$ becomes a homomorphism of sheaves on $B$:
\[
  \widetilde{\rho}\colon \widetilde{T} \to \mathrm{R}^1\pi_*\widetilde{\Theta}.
\]
In particular, we have a homomorphism
\[
  \widetilde{\rho}_0\colon
  \widetilde{T}_0
  \to \mathrm{R}^1\pi_*\widetilde{\Theta}
  = \HH^1(V_0;\widetilde{\Theta})
\]
where $\widetilde{T}_0$ is the vector space of germs at $b_0$ of fields of tangent vectors to $B$.
Finally, we have a commutative diagram
\oldpage{3-02}
\[
  \begin{tikzcd}
    \widetilde{T}_0 \rar["\widetilde{\rho}_0"] \dar[swap,"\varepsilon"]
    & \HH^1(V_0;\widetilde{\Theta}) \dar["\varepsilon"]
  \\T_0 \rar[swap,"\rho_0"]
    & \HH^1(V_0;\Theta_0)
  \end{tikzcd}
\]
where $\rho_0$ is the Spencer--Kodaira map \cite{2}.

\begin{theorem}{1}
\label{theorem1}
  For the proper mixed manifold $\pi\colon V\to B$ to be locally trivial in a neighbourhood of the point $b_0\in B$, it is necessary and sufficient for the map $\widetilde{\rho}_0\colon\widetilde{T}_0\to\HH^1(V_0;\widetilde{\Theta})$ to be zero.
\end{theorem}

\begin{proof}
  \begin{itemize}
    \item[\emph{(Necessary).}]
      If $\pi\colon V\to B$ is locally trivial at $b_0$, then, for every open subset $U$ of $B$ over which $V$ is trivial, we have $\widetilde{\Pi}=\widetilde{\Lambda}\oplus\widetilde{\Theta}$ on $V_U$, and so $\delta\colon\HH^0(V_U;\widetilde{\Lambda})\to\HH^0(V_U;\widetilde{\Theta})$ is zero.
    \item[\emph{(Sufficient).}]
      Let $(\eta_1,\ldots,\eta_p)$ be $\mathscr{C}^\infty$ vector fields (resp. \ldots) on a neighbourhood of $b_0$ in $B$, such that $(\eta_1(b_0),\ldots,\eta_p(b_0))$ forms a basis of the tangent space $T_0$ to $B$ at $b_0$.
      It then follows from the hypothesis that the map
      \[
        \HH^0(V_0;\widetilde{\Pi}) \to \HH^0(V_0;\widetilde{\Lambda})
      \]
      is surjective.

      So let $(\xi_1,\ldots,\xi_p)$ be projectable holomorphic vector fields on a neighbourhood of $V_0$ in $V$, that project to $(\eta_1,\ldots,\eta_p)$.
      Let $f$ be the map defined on a neighbourhood of $\{0\}\times V_0$ in $\RR^p\times V_0$ (resp. $\CC^p\times V_0$) by
      \[
        f(t_1,\ldots,t_p,y) = e^{\xi_1}(t_1,e^{\xi_2}(\ldots,e^{\xi_p}(t_p,y)\ldots)).
      \]
      It follows from the proposition stated in \cite[\S III.2]{1} that $f$ induces an isomorphism of mixed manifolds from $U\times V_0$ to $\pi^{-1}(f_1(U))$ over $f_1$, where $U$ is a sufficiently small cubical neighbourhood of $0$ in $\RR^p$, and $f_1$ is the map from $U$ to $B$ defined by
      \[
        f_1(t_1,\ldots,t_p) = e^{\eta_1}(t_1,e^{\eta_2}(\ldots,e^{\eta_p}(t_p,b_0)\ldots)),
      \]
\oldpage{3-03}
      which proves the theorem.
  \end{itemize}
\end{proof}


\section{The regular case}
\label{II}

For all $b\in B$, set $V_b=\pi^{-1}(b)$.
Consider the family $\{\HH^1(V_b;\Theta_b)\}_{b\in B}$ of finite-dimensional $\CC$-vector spaces, and, for all $b\in B$,  the map
\[
  \varepsilon_b\colon \HH^1(V_b;\widetilde{\Theta}) \to \HH^1(V_b;\Theta_b).
\]

For every open subset $U\subset B$, we have a map
\[
  \widetilde{\varepsilon}_U\colon \HH^1(V_U;\widetilde{\Theta}) \to \prod_{b\in U}\HH^1(V_b;\Theta_B)
\]
that defines, by varying $U$, a homomorphism from the sheaf $\mathrm{R}^1\pi_*\widetilde{\Theta}$ to the sheaf $\Phi$ on $B$ defined by $\Phi(U)=\prod_{b\in U}\HH^1(V_b;\Theta_b)$.

\begin{definition*}
  We say that the proper mixed manifold $\pi\colon V\to B$ is \emph{regular} if
  \begin{enumerate}
    \item the dimension of $\HH^1(V_b;\Theta_b)$ does not depend on the point $b\in B$ ; and
    \item we can endow $E=\bigcup_{b\in B}\HH^1(V_b;\Theta_b)$ with the structure of a $\mathscr{C}^\infty$ vector bundle (resp. \ldots) such that $\widetilde{\varepsilon}$ is an isomorphism from the sheaf $\mathrm{R}^1\pi_*\widetilde{\Theta}$ to the sheaf of germs of $\mathscr{C}^\infty$ sections (resp. \ldots) of the bundle $E$.
  \end{enumerate}
\end{definition*}

In fact, Kodaira and Spencer have shown \cite{2} that, by identifying the $\HH^1$ spaces with spaces of harmonic forms, condition~2 is a consequence of condition~1.

Then \cref{theorem1} has the following corollary:

\begin{proposition}{1}
\label{proposition1}
  For the proper mixed variety $\pi\colon V\to B$ to be locally trivial, it is necessary and sufficient for it to be regular and, for all $b\in B$, for the Spencer--Kodaira map
  \[
    \rho_b\colon T_b \to \HH^1(V_b;\Theta_b)
  \]
  to be zero.
\end{proposition}

Indeed, since $\widetilde{\varepsilon}$ is injective, this condition implies that the map
\oldpage{3-04}
\[
  \widetilde{\rho}_b\colon \widetilde{T}_b \to \HH^1(V_b;\widetilde{\Theta})
\]
is zero for all $b$.

At the end of this talk, we will construct a counter-example which shows that it is necessary to assume that the mixed variety is regular.


\section{An example of non-regular deformation: Hopf varieties}
\label{III}

\subsection{Hopf varieties}
\label{III.1}





%% Bibliography %%

\nocite{*}

\begin{thebibliography}{2}

  \bibitem{1}
  {\sc Douady, A.}
  \newblock Vari\'{e}t\'{e}s et espaces mixtes
  \newblock {\em S\'{e}minaire H. Cartan} \textbf{13} (1960--61), Talk no.~2.

  \bibitem{2}
  {\sc Kodaira, K. and Spencer, D.}
  \newblock On deformation of complex analytic structures, I.
  \newblock {\em Annals of Math.} \textbf{67} (1958), 328--401.

\end{thebibliography}


\end{document}
