\documentclass{article}

\title{Regular deformations}
\author{Adrien Douady}
\date{14\textsuperscript{th} of November, 1960}

\usepackage{amssymb,amsmath}

\usepackage{hyperref}
\usepackage{xcolor}
\hypersetup{colorlinks,linkcolor={red!50!black},citecolor={blue!50!black},urlcolor={blue!80!black}}
\usepackage[nameinlink]{cleveref}
\usepackage{enumerate}
\usepackage{tikz-cd}

\usepackage[mathscr]{eucal}
%% Fancy fonts --- feel free to remove! %%
\usepackage{fouriernc}


\usepackage{fancyhdr}
\usepackage{lastpage}
\usepackage{xstring}
\makeatletter
\ifx\pdfmdfivesum\undefined
  \let\pdfmdfivesum\mdfivesum
\fi
\edef\filesum{\pdfmdfivesum file {\jobname}}
\pagestyle{fancy}
\makeatletter
\let\runauthor\@author
\let\runtitle\@title
\makeatother
\fancyhf{}
\lhead{\footnotesize\runtitle}
\lfoot{\footnotesize Version: \texttt{\StrMid{\filesum}{1}{8}}}
\cfoot{\small\thepage\ of \pageref*{LastPage}}


\crefname{section}{\S\!}{\S\S\!}
\crefname{equation}{}{}
\renewcommand{\thesection}{\Roman{section}}
\renewcommand{\thesubsection}{\arabic{subsection}}


%% Theorem environments %%

\usepackage{amsthm}

\theoremstyle{plain}
  \newtheorem*{proposition*}{Proposition}
\theoremstyle{definition}
  \newtheorem*{definition*}{Definition}


%% Shortcuts %%

\newcommand{\RR}{\mathbb{R}}
\newcommand{\CC}{\mathbb{C}}
\newcommand{\DD}{\mathrm{D}}
\newcommand{\HH}{\mathrm{H}}

\renewcommand{\geq}{\geqslant}
\renewcommand{\leq}{\leqslant}

\newcommand{\oldpage}[1]{\marginpar{\footnotesize$\Big\vert$ \textit{p.~#1}}}


%% Document %%

% \usepackage{embedall}
\begin{document}

\maketitle
\thispagestyle{fancy}

\renewcommand{\abstractname}{Translator's note.}

\begin{abstract}
  \renewcommand*{\thefootnote}{\fnsymbol{footnote}}
  \emph{This text is one of a series\footnote{\url{https://thosgood.com/translations/}} of translations of various papers into English.}
  \emph{The translator takes full responsibility for any errors introduced in the passage from one language to another, and claims no rights to any of the mathematical content herein.}

  \medskip
  
  \emph{What follows is a translation of the French seminar talk:}

  \medskip\noindent
  \textsc{Douady, A.}
  ``D\'{e}formations r\'{e}guli\`{e}res''.
  \emph{S\'{e}minaire Henri Cartan}, Volume~\textbf{13 (1)} (1960--1961), Talk no.~3.
  {\url{http://www.numdam.org/item/SHC_1960-1961__13_1_A2_0/}}
\end{abstract}

\setcounter{footnote}{0}

\tableofcontents
\bigskip


%% Content %%

\oldpage{3-01}
All throughout this talk, $B$ is a $\mathscr{C}^\infty$ manifold (resp. $\RR$-analytic, resp. $\CC$-analytic); $\pi\colon V\to B$ denotes a proper mixed manifold; $b_0$ is a point of $B$; and $V_0=\pi^{-1}(b_0)$ is thus a compact $\CC$-analytic manifold.


\section{The map \texorpdfstring{$\widetilde{\rho}$}{p}}
\label{I}

Let $\widetilde{\Theta}$ (resp. $\widetilde{\Pi}$) be the sheaf of germs of vertical holomorphic (resp. locally projectable holomorphic) vector fields on $V$.
The quotient sheaf $\widetilde{\Lambda}=\widetilde{\Pi}/\widetilde{\Theta}$ is exactly the inverse image under $\pi$ of the sheaf $\widetilde{T}$ of germs of $\mathscr{C}^\infty$ fields (resp. \ldots) of tangent vectors on $B$.

For every open subset $U$ of $B$, set $V_U=\pi^{-1}(U)$.
The exact sequence
\[
  0 \to \widetilde{\Theta} \to \widetilde{\Pi} \to \widetilde{\Lambda} \to 0
\]
of sheaves on $V_U$ gives rise to a homomorphism
\[
  \widetilde{\rho}_U\colon
  \HH^0(U;\widetilde{T})
  \xrightarrow{\pi_*} \HH^0(V_U;\widetilde{\Lambda})
  \xrightarrow{\delta} \HH^1(V_U;\widetilde{\Theta}).
\]
Let $\mathrm{R}^1\pi_*\widetilde{\Theta}$ be the sheaf on $B$ defined by the presheaf $U\mapsto\HH^1(V_U;\widetilde{\Theta})$.
Then $\widetilde{\rho}$ becomes a homomorphism of sheaves on $B$:
\[
  \widetilde{\rho}\colon \widetilde{T} \to \mathrm{R}^1\pi_*\widetilde{\Theta}.
\]
In particular, we have a homomorphism
\[
  \widetilde{\rho}_0\colon
  \widetilde{T}_0
  \to \mathrm{R}^1\pi_*\widetilde{\Theta}
  = \HH^1(V_0;\widetilde{\Theta})
\]
where $\widetilde{T}_0$ is the vector space of germs at $b_0$ of fields of tangent vectors to $B$.
Finally, we have a commutative diagram
\oldpage{3-02}
\[
  \begin{tikzcd}
    \widetilde{T}_0 \rar["\widetilde{\rho}_0"] \dar[swap,"\varepsilon"]
    & \HH^1(V_0;\widetilde{\Theta}) \dar["\varepsilon"]
  \\T_0 \rar[swap,"\rho_0"]
    & \HH^1(V_0;\Theta_0)
  \end{tikzcd}
\]
where $\rho_0$ is the Spencer--Kodaira map \cite{2}.



%% Bibliography %%

\nocite{*}

\begin{thebibliography}{2}

  \bibitem{1}
  {\sc Douady, A.}
  \newblock Vari\'{e}t\'{e}s et espaces mixtes
  \newblock {\em S\'{e}minaire H. Cartan} \textbf{13} (1960--61), Talk no.~2.

  \bibitem{2}
  {\sc Kodaira, K. and Spencer, D.}
  \newblock On deformation of complex analytic structures, I.
  \newblock {\em Annals of Math.} \textbf{67} (1958), 328--401.

\end{thebibliography}


\end{document}
