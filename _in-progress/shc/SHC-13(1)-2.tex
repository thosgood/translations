\documentclass{article}

\title{Mixed varieties and mixed spaces}
\author{Adrien Douady}
\date{7\textsuperscript{th} of November, 1960}

\usepackage{amssymb,amsmath}

\usepackage{hyperref}
\usepackage{xcolor}
\hypersetup{colorlinks,linkcolor={red!50!black},citecolor={blue!50!black},urlcolor={blue!80!black}}
\usepackage[nameinlink]{cleveref}
\usepackage{enumerate}
\usepackage{tikz-cd}

\usepackage[mathscr]{eucal}
%% Fancy fonts --- feel free to remove! %%
\usepackage{fouriernc}


\usepackage{fancyhdr}
\usepackage{lastpage}
\usepackage{xstring}
\makeatletter
\ifx\pdfmdfivesum\undefined
  \let\pdfmdfivesum\mdfivesum
\fi
\edef\filesum{\pdfmdfivesum file {\jobname}}
\pagestyle{fancy}
\makeatletter
\let\runauthor\@author
\let\runtitle\@title
\makeatother
\fancyhf{}
\lhead{\footnotesize\runtitle}
\lfoot{\footnotesize Version: \texttt{\StrMid{\filesum}{1}{8}}}
\cfoot{\small\thepage\ of \pageref*{LastPage}}


\crefname{section}{\S\!}{\S\S\!}
\crefname{equation}{}{}
\renewcommand{\thesection}{\Roman{section}}
\renewcommand{\thesubsection}{\arabic{subsection}}


%% Theorem environments %%

\usepackage{amsthm}

\theoremstyle{plain}
  \newtheorem*{proposition}{Proposition}
\theoremstyle{definition}
  \newtheorem*{definition}{Definition}


%% Shortcuts %%

\newcommand{\RR}{\mathbb{R}}
\newcommand{\CC}{\mathbb{C}}

\renewcommand{\geq}{\geqslant}
\renewcommand{\leq}{\leqslant}

\newcommand{\oldpage}[1]{\marginpar{\footnotesize$\Big\vert$ \textit{p.~#1}}}


%% Document %%

% \usepackage{embedall}
\begin{document}

\maketitle
\thispagestyle{fancy}

\renewcommand{\abstractname}{Translator's note.}

\begin{abstract}
  \renewcommand*{\thefootnote}{\fnsymbol{footnote}}
  \emph{This text is one of a series\footnote{\url{https://thosgood.com/translations/}} of translations of various papers into English.}
  \emph{The translator takes full responsibility for any errors introduced in the passage from one language to another, and claims no rights to any of the mathematical content herein.}

  \medskip
  
  \emph{What follows is a translation of the French seminar talk:}

  \medskip\noindent
  \textsc{Douady, A.}
  ``Vari\'{e}t\'{e}s et espaces mixtes''.
  \emph{S\'{e}minaire Henri Cartan}, Volume~\textbf{13 (1)} (1960--1961), Talk no.~2.
  {\url{http://www.numdam.org/item/SHC_1960-1961__13_1_A1_0/}}
\end{abstract}

\setcounter{footnote}{0}

\tableofcontents
\bigskip


%% Content %%

\oldpage{2-01}
\section{Category of models}
\label{I}

Let $B$ be a topological space.
We define the category $\mathscr{S}_B^n$ in the following manner: the objects of $\mathscr{S}_B^n$ are the open subsets of $B\times\CC^n$, and a morphism $f\colon U\to U'$ from an open subset $U\subset B\times\CC^n$ to an open subset $U'\subset B\times\CC^n$ is a continuous map $f\colon U\to U'$ satisfying the following two conditions:
\begin{enumerate}
  \item the diagram
    \[
      \begin{tikzcd}
        U \ar[rr,"f"] \ar[dr,swap,"\pi_1"]
        && U' \ar[dl,"\pi_1"]
      \\&B&
      \end{tikzcd}
    \]
    commutes, where $\pi_1$ denotes the projection of $B\times\CC^n$ to $B$ ; and
  \item for all $x\in B$, the map $f_x\colon U_x\to U'_x$ is holomorphic, where
    \[
      U_x = \{z\in\CC^n \mid (x,z)\in U\}
    \]
    (and similarly for $U'$).
\end{enumerate}

If $B$ is endowed with the structure of a $\mathscr{C}^\infty$ manifold (resp. an $\RR$-analytic manifold, resp. $\CC$-analytic manifold), then we obtain a category $\mathscr{C}^\infty\mathscr{S}_B$ (resp. $\RR\mathscr{S}_B$, resp. $\CC\mathscr{S}_B$) by requiring the morphisms to be $\mathscr{C}^\infty$ (resp. $\RR$-analytic, resp. $\CC$-analytic).

More generally, if $f_1\colon B\to B'$ is a continuous map from one topological space to another, then a morphism of $\mathscr{S}_{f_1}$ is defined to be a continuous map $f$ from an object $U$ of $\mathscr{S}_B$ to an object $U'$ of $\mathscr{S}_{B'}$ such that
\begin{enumerate}
  \item the diagram
    \[
      \begin{tikzcd}
        U \rar["f"] \dar[swap,"\pi_1"]
        & U' \dar["\pi_1"]
      \\B \rar[swap,"f_1"]
        & B'
      \end{tikzcd}
    \]
    commutes ; and
  \item $f_x\colon U_x\to U'_{f_1(x)}$ is holomorphic for all $x\in B$.
\end{enumerate}

\oldpage{2-02}
If $f_1$ is a $\mathscr{C}^\infty$ map from one $\mathscr{C}^\infty$ manifold to another, then $f$ will be a morphism of $\mathscr{C}^\infty\mathscr{S}_{f_1}$ if, further, it is a $\mathscr{C}^\infty$ map (resp. \ldots).
We thus obtain, for every category of topological spaces, a fibred category $\mathscr{S}^n$ (resp. $\mathscr{C}^\infty\mathscr{S}^n$, resp. \ldots).


\section{The definition of mixed spaces and mixed varieties}
\label{II}

\subsection{First definition}
\label{II.1}

Let $B$ and $V$ be separated spaces, and let $\pi\colon V\to B$ be a continuous map.
The structure of a \emph{mixed space} over $B$ is defined on $V$ by a system of charts $\varphi_i\colon U_i\to V$, where the $(U_i)$ are objects of $\mathscr{S}_B^n$;
for each $i$, $\varphi_i$ is a homeomorphism from $U_i$ to an open subset of $V$ such that the diagram
\[
  \begin{tikzcd}
    U_i \ar[rr,"\varphi_i"] \ar[dr,swap,"\pi_1"]
    && V \ar[dl,"\pi"]
  \\&B&
  \end{tikzcd}
\]
commutes;
finally, for all $i$ and all $j$, the ``change of chart'' $\varphi_j^{-1}\circ\varphi_i$ is an isomorphism of $\mathscr{S}_B$ from an open subset of $U_i$ to an open subset of $U_j$.

The structure thus defined is that of a \emph{$(\mathscr{C}^0,\CC)$-mixed space}.
If $B$ is a $\CC$-analytic space, and if the change of chart maps are all $\CC$-analytic, then we have a \emph{$\CC$-analytic mixed space}.
In this case, $V$ itself is a $\CC$-analytic space, and the fibres $V_x=\pi^{-1}\{x\}$ are $\CC$-analytic sub-manifolds.

If $B$ is a $\mathscr{C}^\infty$ manifold (resp. $\RR$-analytic, resp. $\CC$-analytic), and if the change of chart maps are all $\mathscr{C}^\infty$ (resp. \ldots), then we have a \emph{$(\mathscr{C}^\infty,\CC)$-mixed manifold} (resp. $(\RR,\CC)$, resp. $(\CC,\CC)$).
In this case, $V$ itself is a manifold.
Note that the notion of a $(\CC,\CC)$-mixed manifold, or a $\CC$-analytic mixed manifold, reduces to simply having a $\CC$-analytic manifold $V$ endowed with a projection $\pi\colon V\to B$ onto another $\CC$-analytic manifold such that $\pi$ is of maximal rank at every point.

Let $\pi\colon V\to B$ and $\pi'\colon V'\to B'$ be mixed spaces, and let $f_1\colon B\to B'$ be a continuous (resp. \ldots) map.
Then a morphism from $V$ to $V'$ over $f_1$ is defined to be a continuous map $f\colon V\to V'$ such that the diagram
\[
  \begin{tikzcd}
    V \rar["f"] \dar[swap,"\pi"]
    & V' \dar["\pi'"]
  \\B \rar["f_1"]
    & B'
  \end{tikzcd}
\]
\oldpage{2-03}
commutes, and such that, for any charts $\varphi_i\colon U_i\to V$ and $\varphi'_j\colon U'_j\to V'$, the map ${\varphi'_j}^{-1}\circ f\circ\varphi_i$ is a morphism of $\mathscr{S}_{f_1}$ (resp. \ldots) from an open subset of $U_i$ to $U_j$.


\subsection{An equivalent definition}
\label{II.2}

We now give another way of defining mixed spaces, equivalent to the above.



%% Bibliography %%

\nocite{*}

\begin{thebibliography}{2}

  \bibitem{1}
  {\sc Cartan, H.}
  \newblock Un th\'{e}or\`{e}me de finitude.
  \newblock {\em S\'{e}minaire H. Cartan} \textbf{6} (1953--54), Talk no.~17.

  \bibitem{2}
  {\sc Kodaira, K. and Spencer, D.}
  \newblock On deformation of complex analytic structures, I.
  \newblock {\em Annals of Math.} \textbf{67} (1958), 328--401.

\end{thebibliography}


\end{document}
