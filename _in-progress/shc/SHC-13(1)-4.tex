\documentclass{article}

\usepackage[margin=1.6in]{geometry}

\title{The primary obstruction to deformation}
\author{Adrien Douady}
\date{21\textsuperscript{st} of November, 1960}

\newcommand{\doctype}{French seminar talk}
\newcommand{\origcit}{%
  \textsc{Douady, A.}
  ``Obstruction primaire \`{a} la d\'{e}formation''.
  \emph{S\'{e}minaire Henri Cartan}, Volume~\textbf{13 (1)} (1960--1961), Talk no.~4.
  {\url{http://www.numdam.org/item/SHC_1960-1961__13_1_A3_0}}%
}


\usepackage{amssymb,amsmath}

\usepackage{hyperref}
\usepackage{xcolor}
\hypersetup{colorlinks,linkcolor={red!50!black},citecolor={blue!50!black},urlcolor={blue!80!black}}
\usepackage{enumerate}
\usepackage{tikz-cd}

\usepackage[mathscr]{eucal}
%% Fancy fonts --- feel free to remove! %%
\usepackage{fouriernc}


\usepackage{fancyhdr}
\usepackage{lastpage}
\usepackage{xstring}
\makeatletter
\ifx\pdfmdfivesum\undefined
  \let\pdfmdfivesum\mdfivesum
\fi
\edef\filesum{\pdfmdfivesum file {\jobname}}
\pagestyle{fancy}
\makeatletter
\let\runauthor\@author
\let\runtitle\@title
\makeatother
\fancyhf{}
\lhead{\footnotesize\runtitle}
\lfoot{\footnotesize Version: \texttt{\StrMid{\filesum}{1}{8}}}
\cfoot{\small\thepage\ of \pageref*{LastPage}}


\renewcommand{\thesection}{\Roman{section}}
\renewcommand{\thesubsection}{\arabic{subsection}}


%% Theorem environments %%

\usepackage{amsthm}

\newenvironment{itenv}[1]
  {\phantomsection\par\medskip\noindent\textbf{#1.}\itshape}
  {\medskip}

\newenvironment{rmenv}[1]
  {\phantomsection\par\medskip\noindent\textbf{#1.}\rmfamily}
  {\medskip}


%% Shortcuts %%

\newcommand{\scr}[1]{{\mathscr{#1}}}

\newcommand{\ZZ}{\mathbb{Z}}
\newcommand{\RR}{\mathbb{R}}
\newcommand{\CC}{\mathbb{C}}
\newcommand{\DD}{\mathrm{D}}
\newcommand{\HH}{\mathrm{H}}
\newcommand{\dd}{\mathrm{d}}

\DeclareMathOperator{\Hom}{Hom}

\renewcommand{\geq}{\geqslant}
\renewcommand{\leq}{\leqslant}

\newcommand{\oldpage}[1]{\marginpar{\footnotesize$\Big\vert$ \textit{p.~#1}}}


%% Document %%

\usepackage{embedall}
\begin{document}

\maketitle
\thispagestyle{fancy}

\renewcommand{\abstractname}{Translator's note.}

\begin{abstract}
  \renewcommand*{\thefootnote}{\fnsymbol{footnote}}
  \emph{This text is one of a series\footnote{\url{https://thosgood.com/translations}} of translations of various papers into English.}
  \emph{The translator takes full responsibility for any errors introduced in the passage from one language to another, and claims no rights to any of the mathematical content herein.}

  \medskip
  
  \emph{What follows is a translation of the \doctype:}

  \medskip\noindent
  \origcit
\end{abstract}

\setcounter{footnote}{0}

\tableofcontents
\bigskip


%% Content %%

\section*{Introduction}
\label{introduction}

\oldpage{4-01}
Let $V_0$ be a compact complex-analytic manifold, and let $\Theta$ be the sheaf of germs of holomorphic fields of tangent vectors.
We ask the following question: given an element $a\in\HH^1(V_0,\Theta)$, does there exists a deformation of $V_0$, with a non-singular base (i.e. a fibred mixed manifold $\pi\colon V\to B$, with $b_0\in B$, along with an isomorphism $V_0\xrightarrow{\cong}\pi^{-1}(b_0)$), such that $a$ is the image, under the map $\rho$ defined in [Talk no.~2], of a vector $v$ that is tangent to $B$ at $b_0$?
An element $a\in\HH^1(V_0,\Theta)$ for which the answer is positive is called a \emph{deformation vector}.
We will give a necessary condition for $a$ to be a deformation vector;
this condition is written $[a\smile a]=0$.
We will then give an example where this condition is not satisfied.


\section{Exact sequences of sheaves of algebras}
\label{I}

Let $K$ be a commutative ring, and let $\Phi$, $\Phi_1$, and $\Phi_2$ be sheaves of $K$-modules on some space $X$, and suppose that we have some given homomorphism $\Phi_1\otimes\Phi_2\to\Phi$, written as a product.
We define, for any cover $\scr{U}$ of $X$, the \emph{cup product}
\[
  \smile\colon C^p(X,\scr{U};\Phi_1) \otimes C^q(X,\scr{U};\Phi_2)
  \to C^{p+q}(X,\scr{U};\Phi)
\]
by the formula
\[
  (\alpha\smile\beta)_{i_0,\ldots,i_{p+q}}
  = \alpha_{i_0,\ldots,i_p}\cdot\beta_{i_p,\ldots,i_{p+q}}.
\]
We have the relation
\[
  \dd(\alpha\smile\beta) = \dd\alpha\smile\beta + (-1)^p\alpha\smile\dd\beta.
\]
This induces a cup product on the cohomology of the cover $\scr{U}$, and, by passing to the inductive limit over open covers, a cup product
\[
  \smile\colon \HH^p(X;\Phi_1) \otimes \HH^q(X;\Phi_2)
  \to \HH^{p+q}(X;\Phi).
\]

\oldpage{4-02}
\begin{rmenv}{Definition}
  A \emph{sheaf of algebras} on $X$ is a sheaf of modules $\Phi$ on $X$ endowed with a product $\Phi\otimes\Phi\to\Phi$ (which we do not assume to be either commutative nor associative).
\end{rmenv}

If $f\colon\Phi\to\Psi$ is a homomorphism of sheaves of algebras, then the kernel $\Phi'$ of $f$ is a sheaf of two-sided ideals of $\Phi$, i.e. we have products $\Phi'\otimes\Phi\to\Phi'$ and $\Phi\otimes\Phi'\to\Phi'$ such that the two diagrams
\[
  \begin{tikzcd}
    \Phi'\otimes\Phi \rar \dar
    & \Phi' \dar
  \\\Phi\otimes\Phi \rar
    & \Phi
  \end{tikzcd}
  \qquad
  \begin{tikzcd}
    \Phi\otimes\Phi' \rar \dar
    & \Phi' \dar
  \\\Phi\otimes\Phi \rar
    & \Phi
  \end{tikzcd}
\]
both commute.

\begin{itenv}{Proposition 1}
\label{proposition1}
  Let $0\to\Phi'\to\Phi\to\Phi''\to0$ be an exact sequence of sheaves of algebras on $X$;
  let $a\in\HH^p(X;\Phi'')$.
  Then $\delta a\in\HH^{p+1}(X;\Phi')$, and, for any class $b\in\HH^q(X;\Phi')$, we have
  \[
    \delta a\smile b=0.\qedhere
  \]
\end{itenv}

\begin{proof}
  Let $\scr{U}$ be a cover of $X$ such that $a$ and $b$ are represented by cocycles $\alpha$ and $\beta$ (respectively), and such that $\alpha$ lifts to a cochain $\eta\in C^p(X,\scr{U};\Phi)$.
  Then $\delta\eta$ is a cocycle in $C^{p+1}(X,\scr{U};\Phi')$ whose class in $\HH^{p+1}(X;\Phi')$ is, by definition, $\delta a$, and $\delta a\smile b$ is the class of $\delta\eta\smile\beta$.
  But $\delta(\eta\smile\beta)=\delta\eta\smile\beta$, and $\eta\smile\beta$ is a cochain in $C^{p+q}(X,\scr{U};\Phi')$, since $\Phi'$ is a sheaf of ideals.
  So the cocycle $\delta\eta\smile\beta$ is cohomologous to $0$ in $\HH^{p+q+1}(X;\Phi')$, which proves the proposition.
\end{proof}


\section{The primary obstruction}
\label{II}

Let $V_0$ be a complex-analytic manifold, and $\Theta_0$ the sheaf of germs of holomorphic fields of tangent vectors.
Then $\Theta_0$ is a sheaf of Lie algebras, and, if $a,b\in\HH^\bullet(V_0,\Theta_0)$, then we denote by $[a\smile b]$ the cup product defined by the bracket $[-,-]\colon\Theta_0\otimes\Theta_0\to\Theta_0$.
It satisfies
\[
  [b\smile a] = (-1)^{pq+1}[a\smile b]
\]
for $a\in\HH^p(V_0,\Theta_0)$ and $b\in\HH^q(V_0,\Theta_0)$.

\oldpage{4-03}
\begin{itenv}{Theorem 1}
\label{theorem1}
  Let
\end{itenv}




%% Bibliography %%

\nocite{*}

\begin{thebibliography}{5}

  \bibitem{1}
  {\sc Grothendieck, A.}
  \newblock {\em A general theory of fibre spaces with structure sheaf}.
  \newblock University of Kansas, Department of Mathematics (1955).

  \bibitem{2}
  {\sc Haefliger, A.}
  \newblock Structures feuillet\'{e}es et cohomologie \`{a} valeur dans un faisceau de groupo\"{i}des.
  \newblock {\em Commont. Math. Helvet.} \textbf{32} (1957/58), 248--239.

  \bibitem{3}
  {\sc Kodaira, K. and Spencer, D.}
  \newblock On deformation of complex analytic structures, I.
  \newblock {\em Annals of Math.} \textbf{67} (1958), 328--401.

  \bibitem{4}
  {\sc Kodaira, K. and Nirenberg, L. and Spencer, D.C.}
  \newblock On the existence of deformations of complex analytic structures.
  \newblock {\em Annals of Math.} \textbf{68} (1958), 450--459.

  \bibitem{5}
  {\sc Kuranishi, M.}
  \newblock On the locally complete families of complex analytic structures.
  \newblock (To appear in the Annals of Mathematics).

\end{thebibliography}


\end{document}
