\documentclass{article}

\title{Primary obstruction to deformation}
\author{Adrien Douady}
\date{21\textsuperscript{st} of November, 1960}

\newcommand{\doctype}{French seminar talk}
\newcommand{\origcit}{%
  \textsc{Douady, A.}
  ``Obstruction primaire \`{a} la d\'{e}formation''.
  \emph{S\'{e}minaire Henri Cartan}, Volume~\textbf{13 (1)} (1960--1961), Talk no.~4.
  {\url{http://www.numdam.org/item/SHC_1960-1961__13_1_A3_0}}%
}


\usepackage{amssymb,amsmath}

\usepackage{hyperref}
\usepackage{xcolor}
\hypersetup{colorlinks,linkcolor={red!50!black},citecolor={blue!50!black},urlcolor={blue!80!black}}
\usepackage{enumerate}
\usepackage{tikz-cd}

\usepackage[mathscr]{eucal}
%% Fancy fonts --- feel free to remove! %%
\usepackage{fouriernc}


\usepackage{fancyhdr}
\usepackage{lastpage}
\usepackage{xstring}
\makeatletter
\ifx\pdfmdfivesum\undefined
  \let\pdfmdfivesum\mdfivesum
\fi
\edef\filesum{\pdfmdfivesum file {\jobname}}
\pagestyle{fancy}
\makeatletter
\let\runauthor\@author
\let\runtitle\@title
\makeatother
\fancyhf{}
\lhead{\footnotesize\runtitle}
\lfoot{\footnotesize Version: \texttt{\StrMid{\filesum}{1}{8}}}
\cfoot{\small\thepage\ of \pageref*{LastPage}}


\renewcommand{\thesection}{\Roman{section}}
\renewcommand{\thesubsection}{\arabic{subsection}}


%% Theorem environments %%

\usepackage{amsthm}

\theoremstyle{plain}

  \newtheorem{innertheorem}{Theorem}
  \newenvironment{theorem}[1]
    {\renewcommand\theinnertheorem{#1}\innertheorem}
    {\endinnertheorem}

  \newtheorem{innerproposition}{Proposition}
  \newenvironment{proposition}[1]
    {\renewcommand\theinnerproposition{#1}\innerproposition}
    {\endinnerproposition}

  \newtheorem{innerlemma}{Lemma}
  \newenvironment{lemma}[1]
    {\renewcommand\theinnerlemma{#1}\innerlemma}
    {\endinnerlemma}

  \newtheorem*{proposition*}{Proposition}

\theoremstyle{definition}
  \newtheorem*{definition*}{Definition}


%% Shortcuts %%

\newcommand{\ZZ}{\mathbb{Z}}
\newcommand{\RR}{\mathbb{R}}
\newcommand{\CC}{\mathbb{C}}
\newcommand{\DD}{\mathrm{D}}
\newcommand{\HH}{\mathrm{H}}

\DeclareMathOperator{\Hom}{Hom}

\renewcommand{\geq}{\geqslant}
\renewcommand{\leq}{\leqslant}

\newcommand{\oldpage}[1]{\marginpar{\footnotesize$\Big\vert$ \textit{p.~#1}}}


%% Document %%

\usepackage{embedall}
\begin{document}

\maketitle
\thispagestyle{fancy}

\renewcommand{\abstractname}{Translator's note.}

\begin{abstract}
  \renewcommand*{\thefootnote}{\fnsymbol{footnote}}
  \emph{This text is one of a series\footnote{\url{https://thosgood.com/translations}} of translations of various papers into English.}
  \emph{The translator takes full responsibility for any errors introduced in the passage from one language to another, and claims no rights to any of the mathematical content herein.}

  \medskip
  
  \emph{What follows is a translation of the \doctype:}

  \medskip\noindent
  \origcit
\end{abstract}

\setcounter{footnote}{0}

\tableofcontents
\bigskip


%% Content %%

\section*{Introduction}
\label{introduction}

\oldpage{4-01}
Let $V_0$ be a compact complex-analytic manifold, and let $\Theta$ be the sheaf of germs of holomorphic fields of tangent vectors.




%% Bibliography %%

\nocite{*}

\begin{thebibliography}{5}

  \bibitem{3}
  {\sc Kodaira, K. and Spencer, D.}
  \newblock On deformation of complex analytic structures, I.
  \newblock {\em Annals of Math.} \textbf{67} (1958), 328--401.

\end{thebibliography}


\end{document}
