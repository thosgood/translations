\documentclass{article}

\title{The work of Koszul}
\author{Henri Cartan}
\date{December 1948, March 1949, and May 1949}

\usepackage{amssymb,amsmath}

\usepackage{hyperref}
\usepackage{xcolor}
\hypersetup{colorlinks,linkcolor={red!50!black},citecolor={blue!50!black},urlcolor={blue!80!black}}
\usepackage[nameinlink]{cleveref}
\usepackage{enumerate}

\usepackage{mathrsfs}
%% Fancy fonts --- feel free to remove! %%
\usepackage{Baskervaldx}
\usepackage{mathpazo}


\usepackage{fancyhdr}
\usepackage{lastpage}
\usepackage{xstring}
\makeatletter
\ifx\pdfmdfivesum\undefined
  \let\pdfmdfivesum\mdfivesum
\fi
\edef\filesum{\pdfmdfivesum file {\jobname}}
\pagestyle{fancy}
\makeatletter
\let\runauthor\@author
\let\runtitle\@title
\makeatother
\fancyhf{}
\lhead{\footnotesize\runtitle}
\rhead{\footnotesize Version: \texttt{\StrMid{\filesum}{1}{8}}}
\cfoot{\small\thepage\ of \pageref*{LastPage}}


\crefname{section}{\S\!}{\S\S\!}
\crefname{equation}{}{}


%% Theorem environments %%

\usepackage{amsthm}


%% Shortcuts %%

\newcommand{\llp}{\mathbin{\llcorner}}
\newcommand{\lrp}{\mathbin{\lrcorner}}
\newcommand{\dd}{\mathrm{d}}
\newcommand{\RR}{\mathbb{R}}

\renewcommand{\geq}{\geqslant}
\renewcommand{\leq}{\leqslant}

\DeclareMathOperator{\HH}{H}

\newcommand{\todo}{\textbf{ !TODO! }}
\newcommand{\oldpage}[1]{\marginpar{\footnotesize$\Big\vert$ \textit{p.~#1}}}


%% Document %%

\usepackage{embedall}
\begin{document}

\maketitle
\thispagestyle{fancy}

\renewcommand{\abstractname}{Translator's note.}

\begin{abstract}
  \renewcommand*{\thefootnote}{\fnsymbol{footnote}}
  \emph{This text is one of a series\footnote{\url{https://thosgood.com/translations/}} of translations of various papers into English.}
  \emph{The translator takes full responsibility for any errors introduced in the passage from one language to another, and claims no rights to any of the mathematical content herein.}
  
  \emph{What follows is a translation of the French seminar talks:}

  \medskip\noindent
  \textsc{Cartan, H.}
  ``Les travaux de Koszul, I''.
  \emph{S\'{e}minaire Bourbaki}, Volume~\textbf{1} (1952), Talk no.~1, 7--12.
  {\url{http://www.numdam.org/book-part/SB_1948-1951__1__7_0/}}

  \medskip\noindent
  \textsc{Cartan, H.}
  ``Les travaux de Koszul, II''.
  \emph{S\'{e}minaire Bourbaki}, Volume~\textbf{1} (1952), Talk no.~8, 45--52.
  {\url{http://www.numdam.org/book-part/SB_1948-1951__1__7_0/}}

  \medskip\noindent
  \textsc{Cartan, H.}
  ``Les travaux de Koszul, III''.
  \emph{S\'{e}minaire Bourbaki}, Volume~\textbf{1} (1952), Talk no.~12, 71--74.
  {\url{http://www.numdam.org/book-part/SB_1948-1951__1__7_0/}}
\end{abstract}

\setcounter{footnote}{0}

\tableofcontents
\bigskip


%% Content %%

\part{}
\label{I}

\oldpage{7}
The homology and cohomology of a \emph{compact} Lie group can be directly studied via the Lie algebra of the group (cf. {\sc Chevalley and Eilenberg}, Cohomology theory of Lie groups and Lie algebras, \emph{Trans. Amer. Math. Soc}~\textbf{63} (1948), 85--124).
Proceeding like so, we can study Lie algebras over an arbitrary field (most often of characteristic zero);
the compactness hypotheses are then replaced with \emph{semi-simplicity} hypotheses.
The current conference aims to explain certain algebraic tools that are useful for this study, and to prove the first results obtained (notably the theorem on the third Betti number).
In a later conference, we will introduce notions concerning a sub-algebra of a Lie algebra, and its corresponding ``homogeneous space''.


\section{General notions}
\label{I.1}

Denoting an operator on a set $A$ (that is, a transformation from $A$ to $A$) by an arbitrary letter, such as $T$, we write $T\cdot x$ to mean the transformation of $x\in A$ under $T$, and $TU$ to mean the composition of operators $T$ and $U$, so that $TU\cdot x = T\cdot (U\cdot x)$.


\section{Notions concerning the exterior algebra}
\label{I.2}

{(cf. {\sc Bourbaki}, \emph{Alg\`{e}bre}, Chap.~III).}

Let $E$ be a vector space over a commutative field $K$ (or, equivalently, a unital module over a commutative ring).
Let $\Lambda(E)$ be the exterior algebra of $E$, given by the direct sum of the $\Lambda^p(E)$ (with $\Lambda^0(E)=K$ and $\Lambda^1(E)=E$).
We use lowercase \emph{Latin} letters to denote elements of $E$, and \emph{Greek} for those of $\Lambda(E)$.
We write $a\wedge b$ for the product;
in particular, for a product of elements of degree~$1$, we write $x_1\wedge x_2\wedge\ldots\wedge x_p$.

The bilinear form that defines the duality between $E$ and its \emph{dual} $E'$ is denoted by $\langle x,x'\rangle$.
The \emph{interior product} of some $\alpha\in\Lambda^p(E)$ with some $x'\in E$ is an element of $\Lambda^{p-1}(E)$, denoted by $a\llp x'$, and defined by
\[
  (x_1\wedge x_2\wedge\ldots\wedge x_p) \llp x'
  = \sum_{i=1}^p (-1)^{i+1} \langle x_i,x' \rangle x_1\wedge\ldots\wedge\widehat{x_i}\wedge\ldots\wedge x_p
\]
(where the hat $\;\widehat{\,}\;$ over $x_i$ indicates that the $x_i$ term should be omitted).
The operator on $\Lambda(E)$ given by $\alpha\mapsto\alpha\llp x'$ is denoted by $i(x')$;
we have $i(x')i(x')=0$.
We define $i(\alpha')$ for $\alpha'\in\Lambda(E)'$ by
\[
  i(x'_1\wedge\ldots\wedge x'_p)
  = i(x'_p)\ldots i(x'_1).
\]
We write
\oldpage{8}
$\alpha\llp\alpha'$ to mean $i(\alpha')\cdot\alpha$.
We similarly define $i(\alpha)$, which acts on $\Lambda(E')$;
we denote $i(\alpha)\cdot\alpha'$ by $\alpha\lrp\alpha'$.
We have $i(\alpha\wedge\beta)=i(\beta)i(\alpha)$.
The scalar components of $\alpha\llp\alpha'$ and of $\alpha\lrp\alpha'$ are equal;
we denote this scalar component by $\langle\alpha,\alpha'\rangle$: this ``scalar product'' extends $\langle x,x'\rangle$ and defines the duality between $\Lambda(E)$ and $\Lambda(E')$.
We have $\langle\alpha,\alpha'\rangle=0$ if $\alpha$ and $\alpha'$ are homogeneous of different degrees, and
\[
  \langle x_1\wedge\ldots\wedge x_p,x'_1\wedge\ldots\wedge x'_p\rangle = \det(\langle x_i,x'_j\rangle).
\]

Let $e(\beta)$ be the exterior multiplication $\alpha\mapsto\beta\wedge\alpha$, and $e(\beta')$ the exterior multiplication $\alpha'\mapsto\beta'\wedge\alpha'$.
The equation
\[
  \langle \beta\wedge\alpha, \alpha' \rangle
  = \langle \alpha, \beta\lrp\alpha' \rangle
\]
tells us that $e(\beta)$ and $i(\beta)$ are \emph{transpose} to one another (the former acts on $\Lambda(E)$, and the latter on $\Lambda(E')$).
Similarly, $e(\beta')$ and $i(\beta')$ are transpose.


\section{Notions concerning endomorphisms of algebras in general}
\label{I.3}

We now study a \emph{graded} algebra $\Lambda$.
We denote by $\alpha\mapsto\overline{\alpha}$ the automorphism that sends a homogeneous element $\alpha$ of degree~$n$ to the element $(-1)^n\alpha$.
An endomorphism $\theta$ (of the vector structure) is said to be a \emph{derivation} if
\[
  \begin{aligned}
    \theta\cdot\overline{\alpha}
    &= \overline{\theta\cdot\alpha}
  \\\theta\cdot(\alpha\beta)
    &= (\theta\cdot\alpha)\beta + \alpha(\theta\cdot\beta),
  \end{aligned}
\]
and an \emph{antiderivation} if
\[
  \begin{aligned}
    \theta\cdot\overline{\alpha}
    &= -\overline{\theta\cdot\alpha}
  \\\theta\cdot(\alpha\beta)
    &= (\theta\cdot\alpha)\beta + \overline{\alpha}(\theta\cdot\beta).
  \end{aligned}
\]

If $\theta$ an antiderivation, then $\theta\theta$ is a derivation;
if $\theta_1$ and $\theta_2$ are antiderivations, then  $\theta_1\theta_2+\theta_2\theta_1$ is a derivation.

The ``bracket'' $[\theta_1,\theta_2]$ of two operators is, by definition, $\theta_1\theta_2-\theta_2\theta_1$.
The bracket of two derivations is again a derivation;
the bracket of a derivation and an antiderivation is an antiderivation.

If $\Lambda$ is generated by its degree~$0$ and degree~$1$ elements, then every derivation (resp. antiderivation) that is zero on the degree~$0$ and degree~$1$ elements is identically zero.

If $\Lambda$ is the exterior algebra $\Lambda(E)$ of a vector space $E$, then $i(x')$ (for $x'\in E'$) is an antiderivation.
If $\theta$ is a derivation of $\Lambda(E)$, then $e(x)\theta$ is an antiderivation.


\section{Notions concerning differentiable manifolds}
\label{I.4}

For simplicity, we will restrict our study to that of infinitely differentiable manifolds;
all the ``functions'' that we consider will be infinitely differentiable.

\oldpage{9}
At each point $M$ of the manifold $V$, we have a duality between the space $E(m)$ of \emph{tangent vectors} (at $M$) and the space $E'(M)$ of differentials of real-valued functions (to $\RR$) at the point $M$.
They are both $n$-dimensional vector spaces over the field $\RR$ of real numbers (where $n$ is the dimension of $V$).
A \emph{vector field} $X$ is a function that, to each point $M$ of $V$, associates a tangent vector at $M$;
vector fields form a module $E$ (over the ring of real-valued functions) whose dual $E'$ is the module of degree~$1$ differential forms.
We denote by $\langle X,\omega\rangle$ the bilinear form defining this duality.
The differential $\dd f$ of a real-valued function is a differential form (an element of $E'$).
The algebra of ``exterior differential forms'' can be identified with the exterior algebra $\Lambda(E')$ (where $E'$ is considered as a module over the ring of real-valued functions);
the operator $\dd$ (\emph{exterior differentiation}) is characterised by the following three properties:
\begin{enumerate}[1)]
  \item for a function $f$ (an element of $\Lambda^0(E')$), $\dd f$ is the differential of $f$ ;
  \item $\dd\dd=0$ ; and
  \item $\dd$ is an \emph{antiderivation}.
\end{enumerate}

Every vector field $X$ defines an \emph{infinitesimal} transformation, which we denote by $\theta(X)$, and which acts on $\Lambda(E)$ and $\Lambda(E')$.
We first define $\theta(X)$ on $\Lambda^0(E)=\Lambda^0(E')$ by setting $\theta(X)\cdot f=\langle X,\dd f\rangle$.
There then exists a kernel of the automorphism group of $V$, depending on a real parameter $t$, say $M\mapsto\varphi(M,t)$, such that, for every function $f$,
\[
  \frac{\partial}{\partial t} f(\varphi(M,t))
  = \theta(X)\cdot f(\varphi(M,t)).
\]
This group acts on $\Lambda(E)$ and $\Lambda(E')$, and, by differentiating with respect to $t$ at $t=0$, we recover the operator $\theta(X)$.
It can be calculated using the following rules (which don't need explicit knowledge of the automorphism group):
\begin{itemize}
  \item $\theta(X)$ \emph{commutes} with $\dd$ (on $\Lambda(E')$); to take $\theta(X)$ of a \emph{product} (either exterior, interior, or scalar), we apply the classical formula for taking the derivative of a product; and, in particular, on $\Lambda(E)$ and $\Lambda(E')$, $\theta(X)$ is a \emph{derivation}.
  \item If we apply $\theta(X)$ to a vector field $Y$, then we obtain $\theta(X)\cdot Y$; we have the fundamental formula
    \[
    \label{equation-I.1}
      \theta(\theta(X)\cdot Y) = \theta(X)\theta(Y) - \theta(Y)\theta(X),
    \tag{1}
    \]
    which leads us to denote by $[X,Y]$ the vector field $\theta(X)\cdot Y$, and \cref{equation-I.1} then gives the \emph{Jacobi identity}
\oldpage{10}
  \item Finally, we have the ``fundamental formula of the calculation of variations'':
    on the space $\Lambda(E')$ of exterior differential forms,
    \[
    \label{equation-I.2}
      \theta(X) = i(X)\dd + \dd\cdot i(X),
    \tag{2}
    \]
    where $i(X)$ denotes, as in \cref{I.2}, the interior product.
    (Proof: both sides of the equation are derivations that commute with $\dd$ and that are equal on functions).
\end{itemize}


\section{Lie groups}
\label{I.5}

Let $V$ denote the manifold of a Lie group $G$, and denote by $\alpha$ the subspace of $E$ given by vector fields that are \emph{invariant under left-translations} by $G$;
we denote by $\alpha'$ the subspace of $E'$ given by \emph{left-invariant} differential forms (of degree~$1$).
Then $\alpha$ and $\alpha'$ are $n$-dimensional vector spaces over the field $\RR$ of reals, and are in duality.
The exterior algebra $\Lambda(\alpha')$ can be identified with the sub-algebra of $\Lambda(E')$ given by \emph{left-invariant} exterior differential forms, and it is \emph{stable} under the operator $\dd$ of exterior differentiation.
(In particular, $\Lambda^0(\alpha')$ is the field of constant functions, identified with $\RR$).

If $X\in\alpha$, then the automorphism group of $V$ defined by $X$ (cf. \cref{I.4}) is the group of \emph{right}-translations by elements of the subgroup of $G$;
the orbit of the identity element.
Thus $\Lambda(\alpha)$ and $\Lambda(\alpha')$ are \emph{stable} under $\theta(X)$ if $X\in\alpha$.
In particular, if $X$ and $Y$ are in $\alpha$, then $[X,Y]$ is in $\alpha$.
The elements upon which $\theta(X)$ acts trivially (for \emph{all} $X\in\alpha$) are those that are simultaneously invariant under left- and right-translations, and are simply called \emph{invariant elements}.

Since $\theta(X)$ is zero on the scalars (constant functions), we have:
\[
\label{equation-I.3}
  \langle \theta(X)\cdot Y,\omega \rangle + \langle Y,\theta(X)\cdot\omega \rangle
  = 0
  \qquad\mbox{for $Y\in\alpha$ and $\omega\in\alpha'$.}
\tag{3}
\]
In other words, $\theta(X)$ (acting on forms) is the \emph{transpose} of $-\theta(X)$ (acting on vector fields).

The vector space $\alpha$, endowed with the structure defined by the map $(X,Y)\mapsto[X,Y]$ from $\alpha\times\alpha$ to $\alpha$, is the \emph{Lie algebra} of the group $G$.


\section{Lie algebras}
\label{I.6}

We start with an abstract Lie algebra $\alpha$ (under the classical definition), taken over a field $K$ that is, for now, arbitrary.
Let $n$ be the dimension of the vector space $\alpha$ over $K$.
We will construct everything from the structure of $\alpha$, by taking the relations established above (in the case of the field $\RR$) as our definitions.

\oldpage{11}
Let $\alpha'$ be the dual vector space of $\alpha$.
From now on, we denote elements of $\alpha$ by $x,y,\ldots$;
the elements of $\alpha'$ by $x',y',\ldots$;
the elements of $\Lambda(\alpha)$ by $\alpha,\beta,\ldots$;
and the elements of $\Lambda(\alpha')$ by $\alpha',\beta',\ldots$.
We \emph{define} $\theta(x)$, for $x\in\alpha$, by
\[
  \begin{aligned}
    \theta(x)\cdot y
    &= [x,y];
  \\\langle y,\theta(x)\cdot x'\rangle
    &= -\langle[x,y],x'\rangle.
  \end{aligned}
\]
With $\theta(x)$ defined on $\alpha$ and $\alpha'$, we extend it to $\Lambda(\alpha)$ and $\Lambda(\alpha')$ by imposing the condition that $\theta(x)$ be a \emph{derivation}.
The $\theta(x)$ that acts on $\Lambda(\alpha)$ is the \emph{transpose} of the $-\theta(x)$ that acts on $\Lambda(\alpha')$.
The elements of $\Lambda(\alpha)$ (resp. of $\Lambda(\alpha')$) for which $\theta(x)$ acts trivially (for \emph{all} $x\in\alpha$) are called \emph{invariant} (or \emph{bi-invariant}) elements.
The invariant elements of $\Lambda(\alpha)$ form a \emph{sub-algebra} $\mathfrak{J}$, and those of $\Lambda(\alpha')$ form a sub-algebra $\mathfrak{J}'$.

Equation~\cref{equation-I.2} leads us to define an endomorphism $\delta$ of $\Lambda(\alpha')$ that is zero on the scalars, and such that
\[
\label{equation-I.4}
  \theta(x) = i(x)\delta + \delta i(x).
\tag{4}
\]
Furthermore, such an operator is unique, commutes with the $\theta(x)$, and satisfies $\delta\delta=0$;
it is an \emph{antiderivation}, characterised by
\[
  \langle x\wedge y,\delta x'\rangle
  = -\langle[x,y],x'\rangle,
\]
and it maps $\Lambda^p(\alpha')$ to $\Lambda^{p+1}(\alpha')$.

We define, on $\Lambda(\alpha)$, the endomorphism $\partial$ that is the \emph{transpose} of $-\delta$ by
\[
  \langle\partial\alpha,\alpha'\rangle
  = -\langle\alpha,\delta\alpha'\rangle.
\]
Then $\partial$ commutes with the $\theta(x)$, satisfies $\partial\partial=0$, maps $\Lambda^p(\alpha)$ to $\Lambda^{p+1}(\alpha)$, and is \emph{zero on $\alpha$};
finally, we have that
\[
  \begin{aligned}
    \partial(x\wedge y)
    &= [x,y]
  \\\theta(x)
    &= e(x)\partial + \partial e(x),
  \end{aligned}
\]
whence, by induction, we have the explicit formula
\[
  \partial(x_1\wedge x_2\wedge\ldots\wedge x_p)
  = \sum_{i<j} (-1)^{i+j+1} [x_i,x_j] x_1\wedge\ldots\wedge\widehat{x_i}\wedge\ldots\wedge x_p.
\]

The operator $\delta$ on $\Lambda(\alpha')$ (the \emph{algebra of cochains}) defines a \emph{cohomology algebra}, denoted by $\HH(\alpha')$.
The operator $\partial$ on $\Lambda(\alpha)$ (the \emph{algebra of chains}) defines a \emph{homology group}, denoted by $\HH(\alpha)$;
there is not, in general, a multiplication in $\HH(\alpha)$, since $\partial$ is \emph{not} an antiderivation.

We have that $\HH(\alpha)$ and $\HH(\alpha')$ are naturally in duality;
for each degree~$p$, $\HH^p(\alpha)$ and $\HH^p(\alpha')$ are in duality, and thus of the same dimension.
If $\alpha$ is the Lie algebra of a \emph{compact connected group}, then $\HH(\alpha')$ can be identified with the cohomology algebra of the (topological) space of the group, by the de Rham theorem.
This proves that
\oldpage{12}
any two compact connected groups that are locally isomorphic have the same Betti number.
In all cases, the common dimension of $\HH^p(\alpha)$ and $\HH^p(\alpha')$ is called the \emph{$p$-th Betti number} of the Lie algebra $\alpha$.

\medskip
\textbf{Unimodularity.}
We say that $\alpha$ is \emph{unimodular} if $\theta(x)$ (for \emph{all} $x\in\alpha$), considered as an endomorphism of $\alpha$ (the \emph{adjoint} representation) has \emph{zero trace};
an equivalent condition is that the chain $\omega$ of degree~$n$ is a \emph{cycle};
another equivalent condition is that $\omega$ is \emph{invariant}.

If $\alpha$ is unimodular, then $\alpha\mapsto\omega\llp\alpha'$ defines an isomorphism from $\HH^p(\alpha')$ to $\HH^{n-p}(\alpha)$;
this corresponds to the ``Poincar\'{e} duality theorem'' for the Betti numbers of a manifold.

\medskip
\textbf{Expansions in a basis.}
We can express $\delta$ and $\partial$ in terms of a basis $(x_k)$ of $\alpha$ and the dual basis $(x'_k)$ of $\alpha'$ (when $K$ is of characteristic $\neq2$):
\[
  \begin{aligned}
    2\delta
    &= \sum_k e(x'_k)\theta(x_k)
    \qquad\mbox{always,}
  \\2\partial
    &= \sum_k i(x'_k)\theta(x_k)
    \qquad\mbox{if $\alpha$ is \emph{unimodular}.}
  \end{aligned}
\]

\medskip
\textbf{Semi-simple Lie algebras.}
We say that a Lie algebra is \emph{semi-simple} if it has no radical, or, equivalently (at least if $K$ is of characteristic~$0$), if all its representations are completely reducible.
If $\alpha/\mathfrak{c}$ is semi-simple (where $\mathfrak{c}$ is the centre of $\alpha$), then we have a canonical isomorphism from $\HH(\alpha')$ to $\mathfrak{J}'$, and from $\HH(\alpha)$ to $\mathfrak{J}$ (and thus a multiplicative structure on $\HH(\alpha)$ in this case).

If $\alpha$ is semi-simple, then its centre is trivial, the Betti numbers in dimensions~$1$ and $2$ are zero, and the Betti number in dimension~$3$ is equal to the dimension of the vector space of \emph{invariant quadratic forms} (where an invariant quadratic form is a bilinear map $f(x,y)$ such that $f(y,x)=f(x,y)$ and $f(\theta(z)\cdot x,y)+f(x,\theta(z)\cdot y)=0$).

\textbf{Corollary.}
If $K$ is maximal and quasi-real, and $\alpha$ is \emph{simple}, then the third Betti number is equal to $1$.



\part{}
\label{II}

\oldpage{45}
\section{Reminder of notation}
\label{II.1}



%% Bibliography %%

\nocite{*}

\end{document}
