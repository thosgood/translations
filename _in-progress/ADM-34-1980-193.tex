\documentclass{article}

\usepackage[margin=1.6in]{geometry}

\title{Multialgebraic categories}
\author{Yves Diers}
\date{}

\newcommand{\doctype}{French paper}
\newcommand{\origcit}{%
  \textsc{Diers, Y}
  ``Cat\'{e}gories Multialg\'{e}briques''.
  \emph{Archiv der Mathematik} \textbf{34} (1980), 193--209.%
}


\usepackage{amssymb,amsmath}

\usepackage{hyperref}
\usepackage{xcolor}
\hypersetup{colorlinks,linkcolor={red!50!black},citecolor={blue!50!black},urlcolor={blue!80!black}}
\usepackage{enumerate}
\usepackage{tikz-cd}

\usepackage[only,sslash]{stmaryrd} % For the GIT quotient double slash command

\usepackage{mathrsfs}
%% Fancy fonts --- feel free to remove! %%
\usepackage{fouriernc}


\usepackage{fancyhdr}
\usepackage{lastpage}
\usepackage{xstring}
\makeatletter
\ifx\pdfmdfivesum\undefined
  \let\pdfmdfivesum\mdfivesum
\fi
\edef\filesum{\pdfmdfivesum file {\jobname}}
\pagestyle{fancy}
\makeatletter
\let\runauthor\@author
\let\runtitle\@title
\makeatother
\fancyhf{}
\lhead{\footnotesize\runtitle}
\lfoot{\footnotesize Version: \texttt{\StrMid{\filesum}{1}{8}}}
\cfoot{\small\thepage\ of \pageref*{LastPage}}




%% Theorem environments %%

\usepackage{amsthm}

\newenvironment{itenv}[1]
  {\phantomsection\par\medskip\noindent\textbf{#1.}\itshape}
  {\medskip}

\newenvironment{rmenv}[1]
  {\phantomsection\par\medskip\noindent\textbf{#1.}\rmfamily}
  {\medskip}


%% Shortcuts %%

\renewcommand{\geq}{\geqslant}
\renewcommand{\leq}{\leqslant}

\newcommand{\todo}{\textbf{ !TODO! }}

\newcommand{\oldpage}[1]{\marginpar{\footnotesize$\Big\vert$ \textit{p.~#1}}}


%% Document %%

\usepackage{embedall}
\begin{document}

\maketitle
\thispagestyle{fancy}

\renewcommand{\abstractname}{Translator's note.}

\begin{abstract}
  \renewcommand*{\thefootnote}{\fnsymbol{footnote}}
  \emph{This text is one of a series\footnote{\url{https://thosgood.com/translations}} of translations of various papers into English.}
  \emph{The translator takes full responsibility for any errors introduced in the passage from one language to another, and claims no rights to any of the mathematical content herein.}

  \medskip
  
  \emph{What follows is a translation of the \doctype:}

  \medskip\noindent
  \origcit
\end{abstract}

\setcounter{footnote}{0}
\setcounter{section}{-1}

\tableofcontents
\bigskip


%% Content %%

\section{Introduction}
\label{0}

...







%% Bibliography %%

\nocite{*}

\begin{thebibliography}{8}

  \bibitem[0]{0}
  {\sc Barr, M.}
  \newblock {\em Exact Categories and Categories of Sheaves.}
  \newblock Berlin--Heidelberg, LNM \textbf{236} (1971).

  \bibitem[1]{1}
  {\sc Benabou, J.}
  \newblock {\em Structures alg\'{e}briques dans les cat\'{e}gories.}
  \newblock Thesis, Universit\'{e} de Paris (1966).

  \bibitem[2]{2}
  {\sc Diers, Y.}
  \newblock Familles Universelles de Morphismes.
  \newblock {\em Ann. Soc. Sci. Bruxelles} \textbf{93} (1979).

  \bibitem[3]{3}
  {\sc Diers, Y.}
  \newblock Multimonads and Multimonadic Categories.
  \newblock {\em J. Pure Appl. Algebra} \textbf{17} (1980), 153--170.

  \bibitem[4]{4}
  {\sc Diers, Y.}
  \newblock Type de densit\'{e} d'un sous-cat\'{e}gorie pleine.
  \newblock {\em Ann. Soc. Bruxelles} \textbf{90} (1976), 25--47.

  \bibitem[5]{5}
  {\sc Gabriel, P. and Ulmer, F.}
  \newblock {\em Lokal Pr\"{a}sentierbare Kategorien.}
  \newblock Berlin--Heidelberg, LNM \textbf{221} (1971).

  \bibitem[6]{6}
  {\sc Lawvere, F.W.}
  \newblock Functorial Semantics of Algebraic Theories.
  \newblock {\em Proc. National Acad. Sci.} \textbf{50} (1963), 869--873.

  \bibitem[7]{7}
  {\sc Schubert, H.}
  \newblock {\em Categories.}
  \newblock Berlin--Heidelberg (1972).

\end{thebibliography}

\end{document}
