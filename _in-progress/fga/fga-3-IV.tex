\documentclass{article}

\usepackage[margin=1.6in]{geometry}

\title{Techniques of descent and existence theorems in algebraic geometry\\IV. \emph{Hilbert schemes}}
\author{A. Grothendieck}
\date{May 1961}

\newcommand{\doctype}{French seminar talk}
\newcommand{\origcit}{%
  \textsc{Grothendieck, A.}
  Technique de descente et th\'{e}or\`{e}mes d'existence en g\'{e}om\'{e}trie alg\'{e}brique. IV. Les sch\'{e}mas de Hilbert.
  \emph{S\'{e}minaire Bourbaki}, Volume~\textbf{13} (1960--61), Talk no.~221.%
}


\usepackage{amssymb,amsmath}

\usepackage{hyperref}
\usepackage{xcolor}
\hypersetup{colorlinks,linkcolor={blue!50!black},citecolor={blue!50!black},urlcolor={blue!80!black}}
\usepackage{enumerate}
\usepackage{tikz-cd}
\usepackage{graphicx}

\usepackage{mathrsfs}
%% Fancy fonts --- feel free to remove! %%
\usepackage{fouriernc}


\usepackage{fancyhdr}
\usepackage{lastpage}
\usepackage{xstring}
\pagestyle{fancy}
\fancypagestyle{plain}{}
\fancyhf{}
\lhead{\footnotesize\nouppercase\leftmark}
\cfoot{\small\thepage\ of \pageref*{LastPage}}
% Git commit hash for server builds
\newif\ifserver
\serverfalse
\lfoot{\footnotesize\ifserver{Git commit: \href{https://github.com/thosgood/translations/commit/GitCommitHashVariable}{GitCommitHashVariable}}\fi}


\renewcommand{\thepart}{\Alph{part}}
\renewcommand{\thesection}{\arabic{section}}
\renewcommand{\thesubsection}{(\alph{subsection})}

\usepackage{chngcntr}
\counterwithin*{section}{part}


%% Theorem environments %%

\usepackage{amsthm}

\newenvironment{itenv}[1]
  {\phantomsection\par\medskip\noindent\textbf{#1.}\itshape}
  {\par\medskip}

\newenvironment{rmenv}[1]
  {\phantomsection\par\medskip\noindent\textbf{#1.}\rmfamily}
  {\par\medskip}


%% Shortcuts %%

\newcommand{\scr}[1]{{\mathscr{#1}}}
\renewcommand{\cal}[1]{{\mathcal{#1}}}
\newcommand{\fk}[1]{{\mathfrak{#1}}}

\renewcommand{\geq}{\geqslant}
\renewcommand{\leq}{\leqslant}

\DeclareMathOperator{\id}{id}
\DeclareMathOperator{\Hom}{Hom}
\DeclareMathOperator{\shHom}{\underline{\Hom}}
\DeclareMathOperator{\Tor}{Tor}
\DeclareMathOperator{\shTor}{\underline{\Tor}}

\newcommand{\todo}{\textbf{ !TODO! }}
\newcommand{\oldpage}[1]{\marginpar{\footnotesize$\Big\vert$ \textit{p.~#1}}}


%% Document %%

\usepackage{embedall}
\begin{document}

\maketitle
\thispagestyle{fancy}

\renewcommand{\abstractname}{Translator's note.}

\begin{abstract}
  \renewcommand*{\thefootnote}{\fnsymbol{footnote}}
  \emph{This text is one of a series\footnote{\url{https://thosgood.com/translations}} of translations of various papers into English.}
  \emph{The translator takes full responsibility for any errors introduced in the passage from one language to another, and claims no rights to any of the mathematical content herein.}

  \medskip
  
  \emph{What follows is a translation of the \doctype:}

  \medskip\noindent
  \origcit
\end{abstract}

\setcounter{footnote}{0}

\setcounter{tocdepth}{1}
\tableofcontents


%% Content %%

\subsubsection*{}

\emph{[Trans.] We have made changes throughout the text following the errata (\emph{S\'{e}minaire Bourbaki} \textbf{14}, 1961--62, Compl\'{e}ment); we preface them with ``[Comp.]''.}
\medskip

\todo additif

\todo errata


\section*{Introduction}
\oldpage{221-01}

The techniques described in \cite[I and II]{2} were, for the most part, independent of any projective hypotheses on the schemes in question.
Unfortunately, they have not as of yet allowed us to solve the existence problems posed in \cite[II]{2}.
In the current expos\'{e}, and the following, we will solve these problems by imposing projective hypotheses.
The techniques used are typically projective, and practically make no use of results from \cite[I and II]{2}.
Here we will construct ``Hilbert schemes'', which are meant to replace the use of Chow coordinates, as was mentioned in \cite[II, \S2]{2}.
In the next expos\'{e}, the theory of passing to the quotient in schemes, developed in \cite[III]{2}, combined with the theory of Hilbert schemes, will allow us, for example, to construct Picard schemes (defined in \cite[II, \S3]{2}) under rather general conditions.

In summary, we can say that we now have a more or less satisfying technique of projective constructions, apart from the fact that we are still missing\footnote{See the addendum at the end of this expos\'{e}.} a theory of passing to the quotient by groups such as the projective group, acting ``without fixed points'' (cf. \cite[III, \S8]{2}).
The situation even seems slightly better in analytic geometry (if we restrict to the study of ``projective'' analytic spaces over a given analytic space), since, for analytic spaces, the difficulty of passing to the quotient by a group acting nicely disappears.
Also, in algebraic geometry, as well as in analytic geometry, it remains to develop a technique of construction that works without any projective hypotheses.


\section{Bounded sets of sheaves: invariance properties}
\label{1}

Let $k$ be a field, and $X$ a $k$-prescheme (which we take to be of finite type, for simplicity).
For every extension $K/k$, we obtain a $K$-prescheme $X_K=X\otimes_k K$.
If $\scr{F}$ is a coherent sheaf on $X_K$, and if $K'$ is an extension of $K$, then $\scr{F}\otimes_K K'=\scr{F}_{K'}$ is a quasi-coherent sheaf on $X_K\otimes_KK'=X_{K'}$.
So, if $K$ and $K'$ are arbitrary extensions of $k$, and $\scr{F}$ a quasi-coherent sheaf on $X_K$ and $\scr{F}'$
\oldpage{221-02}
a quasi-coherent sheaf on $X_{K'}$, then we say that $\scr{F}$ and $\scr{F}'$ are \emph{equivalent} if there exists an extension $K''/k$ along with $k$-homomorphisms $K\to K''$ and $K'\to K''$ such that $\scr{F}_{K''}$ and $\scr{F}'_{K''}$ are isomorphic on $X_{K''}$ (\todo? is this right?).
This defines an equivalence relation, and we are interested in the equivalence classes of sheaves under this relation, and of sets of such equivalence classes.
Note that, if $X_0$ is of finite type over $k$, then every class of coherent sheaves can be defined by a coherent sheaf on $X_K$, where $K$ is some extension of $k$ \emph{of finite type}.
We can thus, in the definition of classes of coherent sheaves, restrict ourselves to \emph{algebraically closed} extensions of $k$, and we can also limit ourselves to a fixed algebraically closed extension $\Omega$ of $k$, of infinite transcendence degree;
two coherent sheaves $\scr{F}$ and $\scr{F}'$ on $X_\Omega$ are then equivalent if and only if there exists a $K$-automorphism $\sigma$ of $\Omega$ such that $\scr{F}\otimes_K(\Omega,\sigma)$ is isomorphic to $\scr{F}'$.
Note that there is a bijective correspondence between classes of coherent sheaves under the first definition and under the second.

Let $E$ and $E'$ be two sets of classes of coherent sheaves on $X$.
Consider the classes of all sheaves of the form $\scr{F}\otimes\scr{F}'$, where $\scr{F}$ and $\scr{F}'$ are coherent sheaves on the \emph{same} $X_K$, with the class of $\scr{F}$ being in $E$ and the class of $\scr{F}'$ being in $E'$.
We thus define a set of classes of coherent sheaves that we denote by $E\otimes E'$.
We can similarly define $\shTor_i(E,E')$, etc.






%% Bibliography %%

\nocite{*}
\begin{thebibliography}{8}

  \bibitem{1}
  {Borel, A. and Serre, J.-P.}
  \newblock Le th\'{e}or\`{e}me de Riemann--Roch.
  \newblock {\em Bull. Soc. math. France} \textbf{86} (1958), 97--136.

  \bibitem{2}
  {Grothendieck, A.}
  \newblock Technique de descente et th\'{e}or\`{e}mes d'existence en g\'{e}om\'{e}trie alg\'{e}brique, I, II, III
  \newblock {\em S\'{e}minaire Bourbaki} \textbf{12} (1959--60), Talk no.~190; \textbf{12} (1959--60), Talk no.~195; \textbf{13} (1960--61), Talk no.~212.

  \bibitem{3}
  {Grothendieck, A.}
  \newblock {\em S\'{e}minaire de G\'{e}om\'{e}trie alg\'{e}brique, I, II, III, IV.}
  \newblock Paris, Institut des Hautes \'{E}tudes Scientifiques (1960--61).

  \bibitem{4}
  {Grothendieck, A.}
  \newblock Technique de construction en g\'{e}om\'{e}trie analytique, IV, V.
  \newblock {\em S\'{e}minaire Cartan} \textbf{13} (1960--61), Talks no.~11 and 12.

  \bibitem{5}
  {Grothendieck, A. and Dieudonn\'{e}, J.}
  \newblock El\'{e}ments de g\'{e}om\'{e}trie alg\'{e}brique, I: Le langage de sch\'{e}mas.
  \newblock {\em Publications math\'{e}matiques de l'Institut des Hautes \'{E}tudes Scientifiques} \textbf{4} (1960); II, III, IV (to appear).

  \bibitem{6}
  {Kodaira, K.}
  \newblock Characteristic linear systems of complete continuous systems.
  \newblock {\em Amer. J. of Math.} \textbf{78} (1956), 716--744.

  \bibitem{7}
  {Serre, J.-P.}
  \newblock Exemples de vari\'{e}t\'{e}s projectives en caract\'{e}ristique $p$ non relevables en caract\'{e}ristique $0$.
  \newblock {\em Proc. Nat. Acad. Sc. U.S.A.} \textbf{47} (1961), 108--109.

  \bibitem{8}
  {Zappa, G.}
  \newblock Sull'esistenza, sopra la superficie algebriche, di sistemi continui completi infiniti, la cui curva e a serie caratteristica incompleta.
  \newblock {\em Pont. Acad. Sc. Acta} \textbf{9} (1945), 91--93.

\end{thebibliography}

\end{document}
