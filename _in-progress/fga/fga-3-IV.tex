\documentclass{article}

\usepackage[margin=1.6in]{geometry}

\title{Techniques of descent and existence theorems in algebraic geometry\\IV. \emph{Hilbert schemes}}
\author{A. Grothendieck}
\date{May 1961}

\newcommand{\doctype}{French seminar talk}
\newcommand{\origcit}{%
  \textsc{Grothendieck, A.}
  Technique de descente et th\'{e}or\`{e}mes d'existence en g\'{e}om\'{e}trie alg\'{e}brique. IV. Les sch\'{e}mas de Hilbert.
  \emph{S\'{e}minaire Bourbaki}, Volume~\textbf{13} (1960--61), Talk no.~221.%
}


\usepackage{amssymb,amsmath}

\usepackage{hyperref}
\usepackage{xcolor}
\hypersetup{colorlinks,linkcolor={red!50!black},citecolor={blue!50!black},urlcolor={blue!80!black}}
\usepackage{enumerate}
\usepackage{tikz-cd}
\usepackage{graphicx}

\usepackage{mathrsfs}
%% Fancy fonts --- feel free to remove! %%
\usepackage{fouriernc}


\usepackage{fancyhdr}
\usepackage{lastpage}
\usepackage{xstring}
\makeatletter
\ifx\pdfmdfivesum\undefined
  \let\pdfmdfivesum\mdfivesum
\fi
\edef\filesum{\pdfmdfivesum file {\jobname}}
\pagestyle{fancy}
\makeatletter
\let\runauthor\@author
\let\runtitle\@title
\makeatother
\fancyhf{}
\lhead{\footnotesize\runtitle}
\lfoot{\footnotesize Version: \texttt{\StrMid{\filesum}{1}{8}}}
\cfoot{\small\thepage\ of \pageref*{LastPage}}


\renewcommand{\thepart}{\Alph{part}}
\renewcommand{\thesection}{\arabic{section}}
\renewcommand{\thesubsection}{(\alph{subsection})}

\usepackage{chngcntr}
\counterwithin*{section}{part}


%% Theorem environments %%

\usepackage{amsthm}

\newenvironment{itenv}[1]
  {\phantomsection\par\medskip\noindent\textbf{#1.}\itshape}
  {\medskip}

\newenvironment{rmenv}[1]
  {\phantomsection\par\medskip\noindent\textbf{#1.}\rmfamily}
  {\medskip}


%% Shortcuts %%

\newcommand{\scr}[1]{{\mathscr{#1}}}
\renewcommand{\cal}[1]{{\mathcal{#1}}}
\newcommand{\fk}[1]{{\mathfrak{#1}}}

\renewcommand{\geq}{\geqslant}
\renewcommand{\leq}{\leqslant}

\DeclareMathOperator{\id}{id}
\DeclareMathOperator{\Hom}{Hom}
\DeclareMathOperator{\shHom}{\underline{\Hom}}
\DeclareMathOperator{\Tor}{Tor}
\DeclareMathOperator{\shTor}{\underline{\Tor}}

\newcommand{\todo}{\textbf{ !TODO! }}
\newcommand{\oldpage}[1]{\marginpar{\footnotesize$\Big\vert$ \textit{p.~#1}}}


%% Document %%

\usepackage{embedall}
\begin{document}

\maketitle
\thispagestyle{fancy}

\renewcommand{\abstractname}{Translator's note.}

\begin{abstract}
  \renewcommand*{\thefootnote}{\fnsymbol{footnote}}
  \emph{This text is one of a series\footnote{\url{https://thosgood.com/translations}} of translations of various papers into English.}
  \emph{The translator takes full responsibility for any errors introduced in the passage from one language to another, and claims no rights to any of the mathematical content herein.}

  \medskip
  
  \emph{What follows is a translation of the \doctype:}

  \medskip\noindent
  \origcit
\end{abstract}

\setcounter{footnote}{0}

\setcounter{tocdepth}{1}
\tableofcontents


%% Content %%

\subsubsection*{}

\emph{[Trans.] We have made changes throughout the text following the errata (\emph{S\'{e}minaire Bourbaki} \textbf{14}, 1961--62, Compl\'{e}ment); we preface them with ``[Comp.]''.}
\medskip

\todo additif

\todo errata


\section*{Introduction}
\oldpage{221-01}






%% Bibliography %%

\nocite{*}
\begin{thebibliography}{8}

  \bibitem{1}
  {\sc Borel, A. and Serre, J.-P.}
  \newblock Le th\'{e}or\`{e}me de Riemann--Roch.
  \newblock {\em Bull. Soc. math. France} \textbf{86} (1958), 97--136.

  \bibitem{2}
  {\sc Grothendieck, A.}
  \newblock Technique de descente et th\'{e}or\`{e}mes d'existence en g\'{e}om\'{e}trie alg\'{e}brique, I, II, III
  \newblock {\em S\'{e}minaire Bourbaki} \textbf{12} (1959--60), Talk no.~190; \textbf{12} (1959--60), Talk no.~195; \textbf{13} (1960--61), Talk no.~212.

  \bibitem{3}
  {\sc Grothendieck, A.}
  \newblock {\em S\'{e}minaire de G\'{e}om\'{e}trie alg\'{e}brique, I, II, III, IV.}
  \newblock Paris, Institut des Hautes \'{E}tudes Scientifiques (1960--61).

  \bibitem{4}
  {\sc Grothendieck, A.}
  \newblock Technique de construction en g\'{e}om\'{e}trie analytique, IV, V.
  \newblock {\em S\'{e}minaire Cartan} \textbf{13} (1960--61), Talks no.~11 and 12.

  \bibitem{5}
  {\sc Grothendieck, A. and Dieudonn\'{e}, J.}
  \newblock El\'{e}ments de g\'{e}om\'{e}trie alg\'{e}brique, I: Le langage de sch\'{e}mas.
  \newblock {\em Publications math\'{e}matiques de l'Institut des Hautes \'{E}tudes Scientifiques} \textbf{4} (1960); II, III, IV (to appear).

  \bibitem{6}
  {\sc Kodaira, K.}
  \newblock Characteristic linear systems of complete continuous systems.
  \newblock {\em Amer. J. of Math.} \textbf{78} (1956), 716--744.

  \bibitem{7}
  {\sc Serre, J.-P.}
  \newblock Exemples de vari\'{e}t\'{e}s projectives en caract\'{e}ristique $p$ non relevables en caract\'{e}ristique $0$.
  \newblock {\em Proc. Nat. Acad. Sc. U.S.A.} \textbf{47} (1961), 108--109.

  \bibitem{8}
  {\sc Zappa, G.}
  \newblock Sull'esistenza, sopra la superficie algebriche, di sistemi continui completi infiniti, la cui curva e a serie caratteristica incompleta.
  \newblock {\em Pont. Acad. Sc. Acta} \textbf{9} (1945), 91--93.

\end{thebibliography}

\end{document}
