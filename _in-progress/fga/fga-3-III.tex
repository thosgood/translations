\documentclass{article}

\usepackage[margin=1.6in]{geometry}

\title{Techniques of descent and existence theorems in algebraic geometry\\III. \emph{Quotient preschemes}}
\author{A. Grothendieck}
\date{February 1961}

\newcommand{\doctype}{French seminar talk}
\newcommand{\origcit}{%
  \textsc{Grothendieck, A.}
  Technique de descente et th\'{e}or\`{e}mes d'existence en g\'{e}om\'{e}trie alg\'{e}brique. III. Pr\'{e}schemas quotients.
  \emph{S\'{e}minaire Bourbaki}, Volume~\textbf{13} (1960--61), Talk no.~212.%
}


\usepackage{amssymb,amsmath}

\usepackage{hyperref}
\usepackage{xcolor}
\hypersetup{colorlinks,linkcolor={red!50!black},citecolor={blue!50!black},urlcolor={blue!80!black}}
\usepackage{enumerate}
\usepackage{tikz-cd}
\usepackage{graphicx}

\usepackage{mathrsfs}
%% Fancy fonts --- feel free to remove! %%
\usepackage{fouriernc}


\usepackage{fancyhdr}
\usepackage{lastpage}
\usepackage{xstring}
\makeatletter
\ifx\pdfmdfivesum\undefined
  \let\pdfmdfivesum\mdfivesum
\fi
\edef\filesum{\pdfmdfivesum file {\jobname}}
\pagestyle{fancy}
\makeatletter
\let\runauthor\@author
\let\runtitle\@title
\makeatother
\fancyhf{}
\lhead{\footnotesize\runtitle}
\lfoot{\footnotesize Version: \texttt{\StrMid{\filesum}{1}{8}}}
\cfoot{\small\thepage\ of \pageref*{LastPage}}


\renewcommand{\thepart}{\Alph{part}}
\renewcommand{\thesection}{\arabic{section}}
\renewcommand{\thesubsection}{(\alph{subsection})}

\usepackage{chngcntr}
\counterwithin*{section}{part}


%% Theorem environments %%

\usepackage{amsthm}

\newenvironment{itenv}[1]
  {\phantomsection\par\medskip\noindent\textbf{#1.}\itshape}
  {\medskip}

\newenvironment{rmenv}[1]
  {\phantomsection\par\medskip\noindent\textbf{#1.}\rmfamily}
  {\medskip}


%% Shortcuts %%

\newcommand{\scr}[1]{{\mathscr{#1}}}
\renewcommand{\cal}[1]{{\mathcal{#1}}}
\newcommand{\fk}[1]{{\mathfrak{#1}}}
\newcommand{\kres}{\mathfrak{K}}
\newcommand{\simto}{\xrightarrow{\raisebox{-0.7ex}[0ex][0ex]{$\sim$}}}
\newcommand{\simfrom}{\xleftarrow{\raisebox{-0.7ex}[0ex][0ex]{$\sim$}}}

\newcommand{\Set}{\mathsf{Set}}

\renewcommand{\geq}{\geqslant}
\renewcommand{\leq}{\leqslant}

\DeclareMathOperator{\id}{id}
\DeclareMathOperator{\Hom}{Hom}
\DeclareMathOperator{\shHom}{\underline{\Hom}}
\DeclareMathOperator{\Aut}{Aut}
\DeclareMathOperator{\shAut}{\underline{\Aut}}
\DeclareMathOperator{\HH}{H}
\DeclareMathOperator{\RR}{R}
\DeclareMathOperator{\GL}{GL}
\DeclareMathOperator{\Ga}{G_a}
\DeclareMathOperator{\Gm}{G_m}
\DeclareMathOperator{\SL}{SL}
\DeclareMathOperator{\Sp}{Sp}
\DeclareMathOperator{\Spec}{Spec}
\DeclareMathOperator{\Pro}{Pro}
\DeclareMathOperator{\Ext}{Ext}

\newcommand{\todo}{\textbf{ !TODO! }}
\newcommand{\oldpage}[1]{\marginpar{\footnotesize$\Big\vert$ \textit{p.~#1}}}


%% Document %%

\usepackage{embedall}
\begin{document}

\maketitle
\thispagestyle{fancy}

\renewcommand{\abstractname}{Translator's note.}

\begin{abstract}
  \renewcommand*{\thefootnote}{\fnsymbol{footnote}}
  \emph{This text is one of a series\footnote{\url{https://thosgood.com/translations}} of translations of various papers into English.}
  \emph{The translator takes full responsibility for any errors introduced in the passage from one language to another, and claims no rights to any of the mathematical content herein.}

  \medskip
  
  \emph{What follows is a translation of the \doctype:}

  \medskip\noindent
  \origcit
\end{abstract}

\setcounter{footnote}{0}

\setcounter{tocdepth}{1}
\tableofcontents


%% Content %%

\subsubsection*{}

\emph{[Trans.] We have made changes throughout the text following the errata (\emph{S\'{e}minaire Bourbaki} \textbf{14}, 1961--62, Compl\'{e}ment); we preface them with ``[Comp.]'' (except for small corrections, which we insert silently).}
\medskip


\section*{Introduction}
\oldpage{212-01}

The problems discussed in the current expos\'{e} differ from those discussed in the two previous ones, in that we try to represent certain covariant, no longer contravariant, functors of varying schemes.
The procedure of passing to the quotient is, however, essential in many questions of construction in algebraic geometry, including those from expos\'{e}s~I and II (\cite{1}, \cite{2}).
Indeed, the question of \emph{effectiveness of a descent data} on a $T$-prescheme $X$, with respect to a faithfully flat and quasi-compact morphism $T\to S$, is equivalent to the question of existence of a quotient of $X$ (satisfying reasonable properties that we examine below) by the flat equivalence relation on $X$ defined by the descent data;
the questions raised in \cite[A.2.c]{1} can probably be answered at the same time as the questions posed in \hyperref[2]{\S2} of this current expos\'{e}.
Similarly, the \emph{Picard scheme} (for the definition, see \cite[C.3]{2}) of an $S$-scheme $X$ can be defined in many ways, such as as a quotient of certain other schemes (with positive divisors, or immersions into a projective) by flat equivalence relations, with the definition and construction of these auxiliary schemes being also more simple: they are basically schemes of the type $\Hom_S(X,Y)$, and variants defined in \cite[C.2]{2}, and their construction will be the object of the following expos\'{e} (under suitable hypotheses of projectivity).
Thus, combining the results of the current expos\'{e} with those of the following, we will obtain the construction of Picard schemes, under suitable hypotheses.

The problem of passing to the quotient in preschemes again offers unresolved questions.
The most important is mentioned in \hyperref[8]{\S8}.
It currently remains as the only obstacle to the construction of \emph{schemes of modules over the integers for curves of arbitrary degree}, \emph{polarised abelian varieties}, etc.
That is to say, it's solution deserves the efforts of specialists of algebraic groups.


\oldpage{212-02}
\section{Equivalence relations, effective equivalence relations}




%% Bibliography %%

\nocite{*}
\begin{thebibliography}{3}

  \bibitem{1}
  {\sc Grothendieck, A.}
  \newblock Technique de descente et th\'{e}or\`{e}mes d'existence en g\'{e}om\'{e}trie alg\'{e}brique, I: G\'{e}n\'{e}ralit\'{e}s, Descente pas morphismes fid\`{e}lement plats.
  \newblock {\em S\'{e}minaire Bourbaki} \textbf{12} (1959--60), Talk no.~190.

  \bibitem{2}
  {\sc Grothendieck, A.}
  \newblock Technique de descente et th\'{e}or\`{e}mes d'existence en g\'{e}om\'{e}trie alg\'{e}brique, II: Le th\'{e}or\`{e}me d'existence et th\'{e}orie formelle des modules.
  \newblock {\em S\'{e}minaire Bourbaki} \textbf{12} (1959--60), Talk no.~195.

  \bibitem{3}
  {\sc Grothendieck, A. and Dieudonn\'{e}, J.}
  \newblock El\'{e}ments de g\'{e}om\'{e}trie alg\'{e}brique, I: Le langage de sch\'{e}mas.
  \newblock {\em Publications math\'{e}matiques de l'Institut des Hautes Etudes Scientifiques} \textbf{4} (1960).

\end{thebibliography}

\end{document}
