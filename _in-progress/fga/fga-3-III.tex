\documentclass{article}

\usepackage[margin=1.6in]{geometry}

\title{Techniques of descent and existence theorems in algebraic geometry\\III. \emph{Quotient preschemes}}
\author{A. Grothendieck}
\date{February 1961}

\newcommand{\doctype}{French seminar talk}
\newcommand{\origcit}{%
  \textsc{Grothendieck, A.}
  Technique de descente et th\'{e}or\`{e}mes d'existence en g\'{e}om\'{e}trie alg\'{e}brique. III. Pr\'{e}schemas quotients.
  \emph{S\'{e}minaire Bourbaki}, Volume~\textbf{13} (1960--61), Talk no.~212.%
}


\usepackage{amssymb,amsmath}

\usepackage{hyperref}
\usepackage{xcolor}
\hypersetup{colorlinks,linkcolor={red!50!black},citecolor={blue!50!black},urlcolor={blue!80!black}}
\usepackage{enumerate}
\usepackage{tikz-cd}
\usepackage{graphicx}

\usepackage{mathrsfs}
%% Fancy fonts --- feel free to remove! %%
\usepackage{fouriernc}


\usepackage{fancyhdr}
\usepackage{lastpage}
\usepackage{xstring}
\makeatletter
\ifx\pdfmdfivesum\undefined
  \let\pdfmdfivesum\mdfivesum
\fi
\edef\filesum{\pdfmdfivesum file {\jobname}}
\pagestyle{fancy}
\makeatletter
\let\runauthor\@author
\let\runtitle\@title
\makeatother
\fancyhf{}
\lhead{\footnotesize\runtitle}
\lfoot{\footnotesize Version: \texttt{\StrMid{\filesum}{1}{8}}}
\cfoot{\small\thepage\ of \pageref*{LastPage}}


\renewcommand{\thepart}{\Alph{part}}
\renewcommand{\thesection}{\arabic{section}}
\renewcommand{\thesubsection}{(\alph{subsection})}

\usepackage{chngcntr}
\counterwithin*{section}{part}


%% Theorem environments %%

\usepackage{amsthm}

\newenvironment{itenv}[1]
  {\phantomsection\par\medskip\noindent\textbf{#1.}\itshape}
  {\medskip}

\newenvironment{rmenv}[1]
  {\phantomsection\par\medskip\noindent\textbf{#1.}\rmfamily}
  {\medskip}


%% Shortcuts %%

\newcommand{\scr}[1]{{\mathscr{#1}}}
\renewcommand{\cal}[1]{{\mathcal{#1}}}
\newcommand{\fk}[1]{{\mathfrak{#1}}}
\newcommand{\kres}{\mathfrak{K}}
\newcommand{\simto}{\xrightarrow{\raisebox{-0.7ex}[0ex][0ex]{$\sim$}}}
\newcommand{\simfrom}{\xleftarrow{\raisebox{-0.7ex}[0ex][0ex]{$\sim$}}}

\newcommand{\Set}{\mathsf{Set}}
\newcommand{\pr}{\mathrm{pr}}

\renewcommand{\geq}{\geqslant}
\renewcommand{\leq}{\leqslant}

\DeclareMathOperator{\id}{id}
\DeclareMathOperator{\Hom}{Hom}

\newcommand{\todo}{\textbf{ !TODO! }}
\newcommand{\oldpage}[1]{\marginpar{\footnotesize$\Big\vert$ \textit{p.~#1}}}


%% Document %%

\usepackage{embedall}
\begin{document}

\maketitle
\thispagestyle{fancy}

\renewcommand{\abstractname}{Translator's note.}

\begin{abstract}
  \renewcommand*{\thefootnote}{\fnsymbol{footnote}}
  \emph{This text is one of a series\footnote{\url{https://thosgood.com/translations}} of translations of various papers into English.}
  \emph{The translator takes full responsibility for any errors introduced in the passage from one language to another, and claims no rights to any of the mathematical content herein.}

  \medskip
  
  \emph{What follows is a translation of the \doctype:}

  \medskip\noindent
  \origcit
\end{abstract}

\setcounter{footnote}{0}

\setcounter{tocdepth}{1}
\tableofcontents


%% Content %%

\subsubsection*{}

\emph{[Trans.] We have made changes throughout the text following the errata (\emph{S\'{e}minaire Bourbaki} \textbf{14}, 1961--62, Compl\'{e}ment); we preface them with ``[Comp.]'' (except for small corrections, which we insert silently).}
\medskip


\section*{Introduction}
\oldpage{212-01}

The problems discussed in the current expos\'{e} differ from those discussed in the two previous ones, in that we try to represent certain covariant, no longer contravariant, functors of varying schemes.
The procedure of passing to the quotient is, however, essential in many questions of construction in algebraic geometry, including those from expos\'{e}s~I and II (\cite{1}, \cite{2}).
Indeed, the question of \emph{effectiveness of a descent data} on a $T$-prescheme $X$, with respect to a faithfully flat and quasi-compact morphism $T\to S$, is equivalent to the question of existence of a quotient of $X$ (satisfying reasonable properties that we examine below) by the flat equivalence relation on $X$ defined by the descent data;
the questions raised in \cite[A.2.c]{1} can probably be answered at the same time as the questions posed in \hyperref[2]{\S2} of this current expos\'{e}.
Similarly, the \emph{Picard scheme} (for the definition, see \cite[C.3]{2}) of an $S$-scheme $X$ can be defined in many ways, such as as a quotient of certain other schemes (with positive divisors, or immersions into a projective) by flat equivalence relations, with the definition and construction of these auxiliary schemes being also more simple: they are basically schemes of the type $\Hom_S(X,Y)$, and variants defined in \cite[C.2]{2}, and their construction will be the object of the following expos\'{e} (under suitable hypotheses of projectivity).
Thus, combining the results of the current expos\'{e} with those of the following, we will obtain the construction of Picard schemes, under suitable hypotheses.

The problem of passing to the quotient in preschemes again offers unresolved questions.
The most important is mentioned in \hyperref[8]{\S8}.
It currently remains as the only obstacle to the construction of \emph{schemes of modules over the integers for curves of arbitrary degree}, \emph{polarised abelian varieties}, etc.
That is to say, it's solution deserves the efforts of specialists of algebraic groups.


\oldpage{212-02}
\section{Equivalence relations, effective equivalence relations}

Let $\cal{C}$ be a category, and $X$ an object of $\cal{C}$.
A pair of morphisms
\[
  p_1,p_2\colon R\rightrightarrows X,
\]
is said to be an ``\emph{equivalence pair}'' in $\cal{C}$, with \emph{target} $X$ and \emph{source} $R$, if, for every object $T$ of $\cal{C}$, the corresponding maps
\[
  p_1(T),p_2(T)\colon R(T)\rightrightarrows X(T)
\]
(where we set $Y(T)=\Hom(T,Y)$ for any object $Y$ of $\cal{C}$) define a map
\[
  R(T)\to X(T)\times X(T)
\]
that induces a bijection from $R(T)$ to the graph of an equivalence relation on the set $X(T)$.
We introduce an evident equivalence relation on equivalence pairs with target $X$, and call an equivalence class an \emph{equivalence $\cal{C}$-relation} on $X$, or simply an equivalence relation if no confusion may arise.

If $X\times X$ exists, then the data of an equivalence relation on $X$ is equivalent to the data of a sub-object $R$ of $X\times X$ such that, for every object $T$ of $\cal{C}$, the subset of $(X\times X)(T)=X(T)\times X(T)$ that corresponds to $R(T)$ is the graph of an equivalence relation on $X(T)$.
Denoting the morphisms from $R$ to $X$ induced by the projections $\pr_1$ and $\pr_2$ by $p_1$ and $p_2$ (respectively), the above condition says that $(p_1,p_2)$ is an equivalence pair.
We can also express the axioms of a set-theoretical equivalence relation for the $R(T)$ in the $X(T)$ diagrammatically in $\cal{C}$ (under the assumption that both $X\times X$ and the fibre product $(R,p_2)\times_X(R,p_1)$ exist), following the general principle of \cite[A.1]{2}.
We will not need this.

Every time that we have a pair of morphisms $(p_1,p_2)$ with the same source $R$ and the same target $X$, we can define the \emph{cokernel} of the pair as an object $Y$ of $\cal{C}$ that represents the contravariant (in $Z$) functor
\[
  \Hom_{p_1,p_2}(X,Z)
\]
which denotes the set of morphisms $u$ from $X$ to $Z$ such that $up_1=up_2$.
If $Y$ exists, then it is determined up to unique isomorphism.
We will denote it by $Y/(p_1,p_2)$, or, by an abuse of notation, $Y/R$, with the latter mostly being used when
\oldpage{212-03}
$(p_1,p_2)$ is an equivalence pair: it is then common to identify, in notation, the equivalence relation defined by the pair with the one defined by $R$.
Note that, if we consider $Y$ as a quotient of $X$, then it depends only on the equivalence relation defined by the pair $(p_1,p_2)$

We now start with a morphism
\[
  f\colon X\to Y
\]
which allows us to consider $X$ as an ``object over $Y$'', and we suppose that the fibre product
\[
  \cal{R}(f) = X\times_Y X
\]
exists.
Let $p_1$ and $p_2$ be its projections.
Then $(p_1,p_2)$ is an equivalence pair, and is said to be \emph{associated} with the morphism $f$.
It thus defines an equivalence relation, and is said to be \emph{associated} with the morphism $f$.

We say that a pair of morphisms $(p_1,p_2)$ with target $X$, and source $R$, is an \emph{effective equivalence pair} if
\begin{enumerate}[(i)]
  \item the cokernel $Y=X/(p_1,p_2)$ exists ;
  \item the fibre product $X\times_Y X$ exists ; and
  \item the morphism $R\to X\times_Y X$ with components $p_1$ and $p_2$ is an isomorphism.
\end{enumerate}
Then the pair $(p_1,p_2)$ is indeed an equivalence pair.
We also say that the equivalence relation that it defines is an \emph{effective equivalence relation}.

We say that a morphism $f\colon X\to Y$ is an \emph{effective epimorphism} if
\begin{enumerate}[(i)]
  \item the fibre product $R=X\times_Y X$ exists ;
  \item the quotient $X/(p_1,p_2)$ exists, where $p_1$ and $p_2$ are the projections from $R$ to $X$ ; and
  \item the morphism $X/(p_1,p_2)\to Y$ induced by $f$ is an isomorphism.
\end{enumerate}
Then $f$ is indeed an epimorphism, and even a strict epimorphism (cf. \cite[A.2.3]{1}), with the converse being true if the fibre product $X\times_Y X$ exists.
We also say that the quotient object of $X$ defined by the epimorphism $f$ is an \emph{effective quotient} of $X$.

The above definitions imply the following ``\emph{Galois correspondence}'':

\oldpage{212-04}
\begin{itenv}{Proposition 1.1}
\label{proposition1.1}
  There is a bijective correspondence, respecting the natural orders, between the set of effective equivalence relations $R$ on $X$ and the set of effective quotients $Y$ of $X$, with such an $R$ corresponding to the effective quotient $X/R$, and such a $Y$ corresponding to the effective equivalence relation defined by the canonical projection $X\to Y$ (which is defined by the fibre product $X\times_Y X$ endowed with its two projections).
\end{itenv}

In very nice categories (sets, sheaves of sets, etc.), every quotient is effective, and every equivalence relation is effective.
This is no longer true in categories such as the category of preschemes over a given prescheme $S$, not even if $S$ is the spectrum of field, nor even if we restrict to finite schemes over $S$.
The question of effectiveness, and even (in the case of non-finite preschemes over $S$) the question of existence of quotients, more often than not turn out to be delicate.


\section{Example: finite preschemes over \texorpdfstring{$S$}{S}}
\label{2}





%% Bibliography %%

\nocite{*}
\begin{thebibliography}{3}

  \bibitem{1}
  {\sc Grothendieck, A.}
  \newblock Technique de descente et th\'{e}or\`{e}mes d'existence en g\'{e}om\'{e}trie alg\'{e}brique, I: G\'{e}n\'{e}ralit\'{e}s, Descente pas morphismes fid\`{e}lement plats.
  \newblock {\em S\'{e}minaire Bourbaki} \textbf{12} (1959--60), Talk no.~190.

  \bibitem{2}
  {\sc Grothendieck, A.}
  \newblock Technique de descente et th\'{e}or\`{e}mes d'existence en g\'{e}om\'{e}trie alg\'{e}brique, II: Le th\'{e}or\`{e}me d'existence et th\'{e}orie formelle des modules.
  \newblock {\em S\'{e}minaire Bourbaki} \textbf{12} (1959--60), Talk no.~195.

  \bibitem{3}
  {\sc Grothendieck, A. and Dieudonn\'{e}, J.}
  \newblock El\'{e}ments de g\'{e}om\'{e}trie alg\'{e}brique, I: Le langage de sch\'{e}mas.
  \newblock {\em Publications math\'{e}matiques de l'Institut des Hautes Etudes Scientifiques} \textbf{4} (1960).

\end{thebibliography}

\end{document}
