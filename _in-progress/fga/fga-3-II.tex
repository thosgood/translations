\documentclass{article}

\title{Techniques of descent and existence theorems in algebraic geometry\\II. \emph{The existence theorem and the formal theory of modules}}
\author{A. Grothendieck}
\date{February 1960}

\usepackage{amssymb,amsmath}

\usepackage{hyperref}
\usepackage{xcolor}
\hypersetup{colorlinks,linkcolor={red!50!black},citecolor={blue!50!black},urlcolor={blue!80!black}}
\usepackage[nameinlink]{cleveref}
\usepackage{enumerate}
\usepackage{tikz-cd}
\usepackage{graphicx}

\usepackage{mathrsfs}
%% Fancy fonts --- feel free to remove! %%
\usepackage{fouriernc}


\usepackage{fancyhdr}
\usepackage{lastpage}
\usepackage{xstring}
\makeatletter
\ifx\pdfmdfivesum\undefined
  \let\pdfmdfivesum\mdfivesum
\fi
\edef\filesum{\pdfmdfivesum file {\jobname}}
\pagestyle{fancy}
\makeatletter
\let\runauthor\@author
\let\runtitle\@title
\makeatother
\fancyhf{}
\lhead{\footnotesize\runtitle}
\lfoot{\footnotesize Version: \texttt{\StrMid{\filesum}{1}{8}}}
\cfoot{\small\thepage\ of \pageref*{LastPage}}


\crefname{section}{}{}
\crefname{equation}{}{}
\renewcommand{\thepart}{\Alph{part}}
\renewcommand{\thesection}{\arabic{section}}
\renewcommand{\thesubsection}{(\alph{subsection})}


%% Theorem environments %%

\usepackage{amsthm}

\theoremstyle{plain}

\newtheorem{innercustomtheorem}{Theorem}
\crefname{innercustomtheorem}{Theorem}{Theorems}
\newenvironment{theorem}[1]
  {\renewcommand\theinnercustomtheorem{#1}\innercustomtheorem}
  {\endinnercustomtheorem}

\newtheorem{innercustomproposition}{Proposition}
\crefname{innercustomproposition}{Proposition}{Propositions}
\newenvironment{proposition}[1]
  {\renewcommand\theinnercustomproposition{#1}\innercustomproposition}
  {\endinnercustomproposition}

\newtheorem{innercustomlemma}{Lemma}
\crefname{innercustomlemma}{Lemma}{Lemmas}
\newenvironment{lemma}[1]
  {\renewcommand\theinnercustomlemma{#1}\innercustomlemma}
  {\endinnercustomlemma}

\newtheorem{innercustomcorollary}{Corollary}
\crefname{innercustomcorollary}{Corollary}{Corollaries}
\newenvironment{corollary}[1]
  {\renewcommand\theinnercustomcorollary{#1}\innercustomcorollary}
  {\endinnercustomcorollary}

\newtheorem*{corollary*}{Corollary}
\newtheorem*{lemma*}{Lemma}


\theoremstyle{definition}

\newtheorem{innercustomdefinition}{Definition}
\crefname{innercustomdefinition}{Definition}{Definitions}
\newenvironment{definition}[1]
  {\renewcommand\theinnercustomdefinition{#1}\innercustomdefinition}
  {\endinnercustomdefinition}

\newtheorem{innercustomexample}{Example}
\crefname{innercustomexample}{Example}{Examples}
\newenvironment{example}[1]
  {\renewcommand\theinnercustomexample{#1}\innercustomexample}
  {\endinnercustomexample}

\newtheorem*{remark*}{Remark}
\newtheorem*{remarks*}{Remarks}
\newtheorem*{example*}{Example}

%% Shortcuts %%

\newcommand{\sh}[1]{{\mathscr{#1}}}
\newcommand{\cat}[1]{{\mathcal{#1}}}
\newcommand{\fk}[1]{{\mathfrak{#1}}}
\newcommand{\kres}{k}

\newcommand{\Set}{\mathsf{Set}}

\renewcommand{\geq}{\geqslant}
\renewcommand{\leq}{\leqslant}

\DeclareMathOperator{\id}{id}
\DeclareMathOperator{\Hom}{Hom}
\DeclareMathOperator{\shHom}{\underline{\Hom}}
\DeclareMathOperator{\Aut}{Aut}
\DeclareMathOperator{\HH}{H}
\DeclareMathOperator{\RR}{R}
\DeclareMathOperator{\GL}{GL}
\DeclareMathOperator{\Ga}{G_a}
\DeclareMathOperator{\Gm}{G_m}
\DeclareMathOperator{\SL}{SL}
\DeclareMathOperator{\Sp}{Sp}
\DeclareMathOperator{\Spec}{Spec}

\newcommand{\todo}{\textbf{ !TODO! }}
\newcommand{\oldpage}[1]{\marginpar{\footnotesize$\Big\vert$ \textit{p.~#1}}}


%% Document %%

\usepackage{embedall}
\begin{document}

\maketitle
\thispagestyle{fancy}

\renewcommand{\abstractname}{Translator's note.}

\begin{abstract}
  \renewcommand*{\thefootnote}{\fnsymbol{footnote}}
  \emph{This text is one of a series\footnote{\url{https://thosgood.com/translations/}} of translations of various papers into English.}
  \emph{The translator takes full responsibility for any errors introduced in the passage from one language to another, and claims no rights to any of the mathematical content herein.}

  \medskip
  
  \emph{What follows is a translation of the French seminar talk:}

  \medskip\noindent
  \textsc{Grothendieck, A.}
  Technique de descente et th\'{e}or\`{e}mes d'existence en g\'{e}om\'{e}trie alg\'{e}brique. I. Le th\'{e}or\`{e}me d'existence et th\'{e}orie formelle des modules.
  \emph{S\'{e}minaire Bourbaki}, Volume~\textbf{12} (1959--60), Talk no.~195.
\end{abstract}

\setcounter{footnote}{0}

\setcounter{tocdepth}{1}
\tableofcontents


%% Content %%

\subsubsection*{}

\emph{[Trans.] We have made changes throughout the text following the errata (\emph{S\'{e}minaire Bourbaki} \textbf{14}, 1961--62, Compl\'{e}ment); we preface them with ``[Comp.]''.}
\medskip

\todo{errata}

\oldpage{195-01}
\part{Representable and pro-representable functors}
\label{A}

\section{Representable functors}
\label{A.1}

Let $\cat{C}$ be a category.
For all $X\in\cat{C}$, let $h_X$ be the contravariant functor from $\cat{C}$ to the category $\Set$ of sets given by
\[
  \begin{aligned}
    h_X\colon \cat{C} &\to \Set
  \\Y&\mapsto \Hom(Y,X).
  \end{aligned}
\]
If we have a morphism $X\to X'$ in $\cat{C}$, then this evidently induces a morphism $h_X\to h_{X'}$ of functors;
$h_X$ is a covariant functor in $X$, i.e. we have defined a \emph{canonical covariant functor}
\[
  h\colon \cat{C} \to \Hom(\cat{C}^o,\Set)
\]
from $\cat{C}$ to the category of covariant functors from the dual $\cat{C}^o$ of $\cat{C}$ to the category of sets.
We then recall:

\begin{proposition}{1}
  This functor $h$ is \emph{faithfully flat};
  in other words, for every pair $X,X'$ of objects of $\cat{C}$, the natural map
  \[
    \Hom(X,X') \to \Hom(h_X,h_{X'})
  \]
  is \emph{bijective}.
\end{proposition}


%% Bibliography %%

\nocite{*}
\begin{thebibliography}{4}

  \bibitem{1}
  {\sc Grothendieck, A.}
  \newblock Sur quelques points d'alg\'{e}bre homologique.
  \newblock {\em Tohoku math. J.} {\bf 9} (1957), 119--221.

  \bibitem{2}
  {\sc Grothendieck, A.}
  \newblock G\'{e}om\'{e}trie formelle et g\'{e}om\'{e}trie alg\'{e}brique.
  \newblock {\em S\'{e}minaire Bourbaki} \textbf{11} (1958--59), Talk no.~182.

  \bibitem{3}
  {\sc Grothendieck, A.}
  \newblock Technique de descente et th\'{e}or\`{e}mes d'existence en g\'{e}om\'{e}trie alg\'{e}brique, I: G\'{e}n\'{e}ralit\'{e}s, Descente pas morphismes fid\`{e}lement plats.
  \newblock {\em S\'{e}minaire Bourbaki} \textbf{12} (1959--60), Talk no.~190.

  \bibitem{4}
  {\sc Serre, J.-P.}
  \newblock Corps locaux et isog\'{e}nies.
  \newblock {\em S\'{e}minaire Bourbaki} \textbf{11} (1958--59), Talk no.~185.

\end{thebibliography}

\end{document}
