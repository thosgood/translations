\documentclass{article}

\usepackage[margin=1.6in]{geometry}

\title{On modifications and exceptional analytic sets}
\author{Hans Grauert}
\date{}

\newcommand{\doctype}{German paper}
\newcommand{\origcit}{%
  \textsc{Grauert, H.}
  ``\"{U}ber Modifikationen und exzeptionelle analytische Mengen''.
  \emph{Math. Ann.}, Volume~\textbf{146} (1962), 331--368.
  {\url{http://eudml.org/doc/160940}}%
}


\usepackage{amssymb,amsmath}

\usepackage{hyperref}
\usepackage{xcolor}
\hypersetup{colorlinks,linkcolor={blue!50!black},citecolor={blue!50!black},urlcolor={blue!80!black}}
\usepackage{enumerate}

\usepackage{mathrsfs}
%% Fancy fonts --- feel free to remove! %%
\usepackage{fouriernc}


\usepackage{fancyhdr}
\usepackage{lastpage}
\usepackage{xstring}
\makeatletter
\ifx\pdfmdfivesum\undefined
  \let\pdfmdfivesum\mdfivesum
\fi
\edef\filesum{\pdfmdfivesum file {\jobname}}
\pagestyle{fancy}
\makeatletter
\let\runauthor\@author
\let\runtitle\@title
\makeatother
\fancyhf{}
\lhead{\footnotesize\runtitle}
\lfoot{\footnotesize Version: \texttt{\StrMid{\filesum}{1}{8}}}
\cfoot{\small\thepage\ of \pageref*{LastPage}}



\usepackage{titlesec}
\titleformat{\section}[block]
  {\bfseries\Large\filcenter}
  {\S\thesection}
  {1em}
  {}
\titleformat{\subsection}[hang]
  {\large}
  {\bfseries\arabic{subsection}}
  {1em}
  {}


%% Theorem environments %%

\usepackage{amsthm}

\theoremstyle{plain}

\newenvironment{itenv}[1]
  {\phantomsection\par\medskip\noindent\textbf{#1.}\itshape}
  {\medskip}

\newenvironment{rmenv}[1]
  {\phantomsection\par\medskip\noindent\textbf{#1.}\rmfamily}
  {\medskip}


%% Shortcuts %%

\newcommand{\scr}[1]{{\mathscr{#1}}}
\renewcommand{\cal}[1]{{\mathcal{#1}}}

\newcommand{\CC}{\mathbb{C}}
\newcommand{\PP}{\mathbb{P}}
\newcommand{\sh}{\mathcal}
\newcommand{\fk}{\mathfrak}

\newcommand{\unsure}[1]{\textbf{TODO: \underline{#1}}}
\newcommand{\todo}{\textbf{ !TODO! }}
\newcommand{\oldpage}[1]{\marginpar{\footnotesize$\Big\vert$ \textit{p.~#1}}}


%% Document %%

\usepackage{embedall}
\begin{document}

\maketitle
\thispagestyle{fancy}

\renewcommand{\abstractname}{Translator's note.}

\begin{abstract}
  \renewcommand*{\thefootnote}{\fnsymbol{footnote}}
  \emph{This text is one of a series\footnote{\url{https://thosgood.com/translations}} of translations of various papers into English.}
  \emph{The translator takes full responsibility for any errors introduced in the passage from one language to another, and claims no rights to any of the mathematical content herein.}

  \medskip

  \emph{What follows is a translation of the \doctype:}

  \medskip\noindent
  \origcit
\end{abstract}

\setcounter{footnote}{0}

\setcounter{tocdepth}{1}
\tableofcontents


%% Content %%

\subsubsection*{}

\emph{The author was supported by US Air Force Grant No.~AF-EOAR-61-50.}

\bigskip

\oldpage{331}
The term ``modification'' first appeared in a 1951 publication \cite{1} by H.~Behnke and K.~Stein.
The authors used it to refer to a process that allows a given complex space to be modified.
If $X$ is a complex space, and $N\subset X$ a low-dimensional analytic set, then $N$ is replaced by another set $N'$ such that the complex structure on $X\setminus N$ can be extended to the entire space $X'=(X\setminus N)\cup N'$.
The newly obtained complex space $X'$ is then called a \emph{modification} of $X$.

As already demonstrated in \cite{1}, modifications can be very pathological.
The interest therefore turned towards special classes of modifications.
In \cite{12}, H.~Hopf considered so-called ``$\sigma$-processes'' on $n$-dimensional complex manifolds $M$.
These modifications made it possible to replace any point $x\in M$ with a complex projective space $\PP^{n-1}$ of dimension $n-1$.
The result is a new singularity-free complex manifold $M'$.
There are more general modifications that modify the manifold $M$ at only one point $x\in M$, but the space thus obtained can then contain singular points, i.e. is just a complex space.

This present work deals with the following question.
Let $X$ be a complex space, and $A\subset X$ a complex-analytic set.
Then when does there exist a modification $Y$ of $X$ where $A$ is replaced by a point $y$, and such that $X\setminus A=Y\setminus y$?

If such a $Y$ exists, then $A$ is said to be an \emph{exceptional analytic set} in $X$, and we say that $A$ can be ``\emph{collapsed}'' to a point.

In general, such a $Y$ does not exist.
If $X$ is a complex space, and $A\subset X$ is a compact connected analytic set, then, from a set-theoretic point of view, $A$ can of course always be replaced by a point $y_0$.
Then $Y=(X\setminus A)\cup y_0$ has a canonical topological structure, $Y\setminus y_0=X\setminus A$ has a complex structure $\fk{S}$, and the identity $X\setminus A\to X\setminus y_0$ can be extended to a continuous map $\lambda\colon X\to Y$.
Then $\lambda$ maps $X\setminus A$ topologically (and even biholomorphically) to $Y$, and sends $A$ to $y_0$.
If $A$ can now be collapsed to a point, then $\fk{S}$ can be extended to the entire space $Y$, and $\lambda$ becomes a holomorphic map $X\to Y$.

\oldpage{332}
We now give an overview of the present work.
In \hyperref[1]{\S1} we study the concepts of \emph{pseudoconvexity} and \emph{holomorphic convexity} on complex spaces.
The reduction theory of Remmert then leads, in \hyperref[2]{\S2}, to the first general theorem concerning exceptional analytic sets $A\subset X$.
In order to simplify the somewhat strong assumption in this theorem, we consider, in \hyperref[3]{\S3}, a coherent analytic sheaf $\fk{m}$ of germs of holomorphic functions that vanish on $A$, so that $A$ is exactly the zero set of $\fk{m}$.
Using $\fk{m}$, we then endow $A$ with a normal bundle $N_\fk{m}$.
The structure of $N_\fk{m}$ is then critical: $A$ is exceptional if $N$ is weakly negative.
We use the word ``negative'' here in the sense of Kodaira's definition; our result shows that it can be defined in a purely algebraic way in the world of algebraic geometry.
Also in \hyperref[3]{\S3}, we obtain simple criteria for positive (negative) line bundles and characterising projective algebraic spaces.
The well-known theorem of Kodaira (that every Hodge manifold $X$ is projective algebraic) is generalised to the case where $X$ is a normal complex space.
Then \hyperref[4]{\S4} deals with the complex structure of neighbourhoods of analytic sets $A\subset X$, which can be collapsed to a point.
The main result of this section is that the neighbourhoods of (special) exceptional analytic sets $A\subset X$ and $A'\subset X'$ are analytically equivalent if (\unsure{and only if??}) they are equivalent in a formal sense.
This means that the complex structure can be ``calculated'', which makes it possible to solve one of Hirzebruch's problems \cite{11}, and to transfer the propositions of Enriques and Kodaira from algebraic geometry to complex analysis.
--- It should also be mentioned that, using the main results of \hyperref[4]{\S4}, we construct a complex space $X$ with the following properties:
\begin{enumerate}[1)]
  \item $X$ is connected, compact, and of dimension~$2$;
  \item $X$ is normal, and has only one non-regular points;
  \item there exist two analytically and algebraically independent meromorphic functions on $X$; and
  \item $X$ is not an algebraic variety (neither in the projective sense nor the more general sense of Weil).%
  \footnote{
    Some of the results of the present work were discovered in 1959, and published in \cite{7}.
    There are, however, some errors in \cite{7}.
    In Theorem~1, it should, of course, read ``\ldots{} such that $G$ is strongly pseudoconvex and $A$ is the maximal compact analytic subset of $G$''.
    Furthermore, the criterion in Theorem~2 is only sufficient.
    See \hyperref[3.8]{\S3.8}.
    Theorem~3 is only proven in the present work in the case where the normal bundle $N(A)$ is weakly negative.
    --- The author has already presented, several times, previously, the example of the complex space $X$, and, since then, Hironaka has found more interesting examples of complex spaces of this type.
  }
\end{enumerate}

In contrast, as is well known, Kodaira and Chow \cite{4} have shown that every compact, $2$-dimensional complex manifold with two independent meromorphic functions is projective algebraic.


\section{Complex spaces, pseudoconvexity}
\label{1}

\subsection{}
\label{1.1}

Complex spaces are defined as in \cite{10}.
We always assume that they are reduced: their local rings contain no nilpotent elements.
If $X$ is a complex space, $U=U(x)$ a neighbourhood, $A\subset G\subset \CC^n$ an analytic set in a domain $G$ of the space $\CC^n$ of the $n$-dimensional complex numbers, and $\tau$ a biholomorphic map $U\to A$, then $(U,\tau,A)$ is called a chart in $X$, and $\tau$ a biholomorphic embedding of $U$ in $G$.

We always denote by $\scr{O}=\scr{O}(X)$ the sheaf of germs of locally holomorphic functions on $X$.
If $A\subset X$ is an analytic subset, then we denote by $\fk{m}=\fk{m}(A)\subset\scr{O}$ the sheaf of germs of locally holomorphic functions that vanish on $A$.
By a theorem of Cartan, $\fk{m}$ is coherent.
For every subsheaf $\scr{I}\subset\scr{O}$, let $\scr{I}^k$ be the sheaf consisting of germs $f_x=f_{1x}\cdot\ldots\cdot f_{kx}$, where $f_{1x},\ldots,f_{kx}\in\scr{I}_x$ for $x\in X$, $k=1,2,\ldots$.

Now\footnote{
  A subscript $x$ always denotes the stalk of the sheaf at the point $x$.
  If $s$ is a section, then $s_x$ denotes its value at $x$.
  Holomorphic functions and sections in $\scr{O}$ are always considered to be the same thing.
  --- If $F$ is a complex-analytic vector bundle, then $\underline{F}$ always denotes the sheaf of germs of locally holomorphic sections in $F$.
}
let $x\in X$, $\fk{m}=\fk{m}(x)$, and $d(x)=\dim_\CC\fk{m}_x/\fk{m}_x^2$.
If $\psi\colon X\to\CC^n$ is a holomorphic map, then $\psi$ defines, at each point $x\in X$, a homomorphism $\psi_x^*\colon\scr{O}_z(\CC^n)\to\scr{O}_x(X)$.
This homomorphism maps the maximal ideal $\fk{m}_z\subset\scr{O}_z(\CC^n)$ to the maximal ideal $\fk{m}_x\subset\scr{O}_x(X)$.
If the induced map $\fk{m}_z/\fk{m}_z^2\to\fk{m}_x/\fk{m}_x^2$ is surjective, then we say that $\psi$ is a \emph{regular map} at $x$.
In the case where $X$ is a complex manifold, we see that $\fk{m}_x/\fk{m}_x^2$ is exactly the covariant tangent space of $X$.
Then $\psi$ is regular at $x\in X$ if and only if the Jacobian matrix of $\psi$ at $x$ has rank equal to $\dim_x X$.

We say that a map $\psi\colon X\to\CC^n$ is \emph{biholomorphic} if it is a bijection that is regular at every point $x\in X$.

\begin{enumerate}[(1)]
  \itshape
  \item Let $x$ be a point of a complex space $X$.
    Then there exists a neighbourhood $U=U(x)$ and a chart $(U,\tau,A)$ with $A\subset G$ and $\dim G=d(x)$.
    If $(U,\tau,A)$ is any such chart, and $\psi\colon U\to\CC^n$ is a regular holomorphic map, then there exists an open neighbourhood $V=V(z)$ of $z=\tau(x)$ in $G$, and a biholomorphic map $\hat{\psi}\colon V\to\CC^n$ such that $\psi|W=\hat{\psi}\circ\tau$ (where $W=\tau^{-1}(V)$).\footnote{This statement (1) and its proof were communicated to me by A.~Andreotti.}
    \label{(1)}
\end{enumerate}

Of course, $\psi|W$ is then also biholomorphic.

\begin{proof}
To prove \hyperref[(1)]{(1)}, we may assume that $X$ is an analytic set in a domain $D\subset\CC^m$.
Let $\hat{\fk{m}}_x$ be the maximal ideal in $\scr{O}_x(\CC^m)$, and $\fk{i}_x\subset\scr{O}_x(\CC^m)$ the ideal of germs of holomorphic functions that vanish on $X\subset D$.
Let $r$ be the dimension of the image $\scr{F}$ of $\fk{i}_x$ under the natural homomorphism $\lambda\colon\fk{i}_x\to\hat{\fk{m}}_x/\hat{\fk{m}}_x^2$.
Clearly $m=r+d(x)$.
Let $f_1,\ldots,f_r$ be functions that are holomorphic on a neighbourhood of $x$, with $f_{vx}\in\fk{i}_x$, so that the elements $\lambda(f_{vx})$ for $v=1,\ldots,r$ span the complex vector space $\scr{F}$.
Then the rank of the Jacobian matrix of $(f_1,\ldots,f_r)$ in $X$ is equal to $r$.
Then, in a neighbourhood $W=W(x)$, the following properties apply:
\oldpage{334}
\begin{enumerate}[1)]
  \item The functions $f_1,\ldots,f_r$ are holomorphic on $W$, and vanish on $X\cap W$;
  \item $\hat{G}=\{z\in W\mid \mbox{$f_v(z)=0$ for $v=1,2\ldots,r$}\}$ is a $d(x)$-dimensional analytic subset of $W$ that contains no singularities, and which is mapped to a domain in $\CC^{d(x)}$ under some biholomorphic map $\tau$.
\end{enumerate}

Now let $A=\tau(X\cap W)$ and $U'=W\cap X$, and we obtain a chart satisfying the required properties.

To prove the second claim of \hyperref[(1)]{(1)}, let $(U,\tau,A)$ be a chart with $A\subset G$ and $\dim G=d(x)$.
We may assume that $U=A$ and that $\tau$ is the identity.
Then $\lambda(\fk{i}_x)=0$, since $r=0$.
If $f_1,\ldots,f_n$ are holomorphic functions on $U$ that define a holomorphic map $\psi\colon U\to\CC^n$, and if $\hat{f}_1,\ldots,\hat{f}_n$ are holomorphic continuations in an open neighbourhood of $x$ in $G$, then the rank of the Jacobian matrix of $(\hat{f}_1,\ldots,\hat{f}_n)$ at $x$ is equal to $d(x)=\dim G$.
There thus exists a neighbourhood $W=W(x)$ in which the $\hat{f}_1,\ldots,\hat{f}_n$ are holomorphic and give a biholomorphic map $\psi\colon W\to\CC^n$.
\end{proof}

By the definition of a complex space, for every point $x\in X$ there is a non-empty system of charts $(U,\tau,A)$ such that $x\in U$.
As we will show in this section, it is thus possible to transfer the concept of strictly plurisubharmonic functions to the setting of complex spaces.


%% Bibliography %%

\nocite{*}

\begin{thebibliography}{20}

  \bibitem{1}
  {\sc Behnke, H. and Stein, K.}
  \newblock Modifikationen komplexer Mannigfaltigkeiten und Riemannscher Gebiete.
  \newblock {\em Math. Ann.} {\bf 124} (1951), 1--16.

  \bibitem{2}
  {\sc Cartan, H.}
  \newblock ``Quotients of complex spaces''.
  \newblock {\em Contributions to Function Theory}, 1--16.
  \newblock Tata Inst. Fund. Res. Bombay 1960.

  \bibitem{3}
  {\sc Cartan, H. and Eilenberg, S.}
  \newblock ``Homological Algebra''.
  \newblock Princeton University Press 1956.

  \bibitem{4}
  {\sc Chow, W.L. and Kodaira, K.}
  \newblock On analytic surfaces with two independent meromorphic functions.
  \newblock {\em Proc. Nat. Acad. Sci. U.S.A.} {\bf 38} (1952), 319--325.

  \bibitem{5}
  {\sc Frenkel, J.}
  \newblock Cohomologie non ab\'{e}linne et espaces fib\'{e}s.
  \newblock {\em Bull. soc. math. France} {\bf 85} (1957), 135--218.

  \bibitem{6}
  {\sc Grauert, H.}
  \newblock On Levi's problem and the imbedding of real-analytic manifolds.
  \newblock {\em Ann. Math.} {\bf 68} (1958), 460--472.

  \bibitem{7}
  {\sc Grauert, H.}
  \newblock ``On point modifications''.
  \newblock {\em Contributions to Function Theory}, 139--142.
  \newblock Tata Inst. Fund. Res. Bombay 1960.

  \bibitem{8}
  {\sc Grauert, H.}
  \newblock Ein Theorem der analytischen Garbentheorie und die Modulr\"{a}ume komplexer Strukturen.
  \newblock {\em Pub. Math.} {\bf 5} (1960), 233--292.

  \bibitem{9}
  {\sc Grauert, H. and Remmert, R.}
  \newblock Plurisubharmonische Funktionen in komplexen R\"{a}umen.
  \newblock {\em Math. Z.} {\bf 65} (1956), 175--194.

  \bibitem{10}
  {\sc Grauert, H. and Remmert, R.}
  \newblock Komplexe R\"{a}ume.
  \newblock {\em Math. Ann.} {\bf 136} (1958), 245--318.

  \bibitem{11}
  {\sc Hirzebruch, F.}
  \newblock Some problems on differentiable and complex manifolds.
  \newblock {\em Ann. Math.} {\bf 60} (1954), 213--236.

  \bibitem{12}
  {\sc Hopf, H.}
  \newblock Schlichte Abbildungen und lokale Modifikation 4-dimensionaler komplexer Mannigfaltigkeiten.
  \newblock {\em Comment. Math. Helv.} {\bf 29} (1955), 132--156.

  \bibitem{13}
  {\sc Kodaira, K.}
  \newblock On K\"{a}hler Varieties of restricted type.
  \newblock {\em Ann. Math.} {\bf 60} (1954), 28--48.

  \bibitem{14}
  {\sc Koopman, B.O. and Brown, A.B.}
  \newblock On the covering of analytic loci by complexes.
  \newblock {\em Trans. Am. Math. Soc.} {\bf 34} (1931), 231--251.

  \bibitem{15}
  {\sc Nakano, S.}
  \newblock On complex analytic vector bundles.
  \newblock {\em J. Math. Soc. Japan} {\bf 7} (1955), 1--12.

  \bibitem{16}
  {\sc Narasimhan, R.}
  \newblock The Levi problem for complex spaces.
  \newblock {\em Math. Ann.} {\bf 142} (1961), 355-365.

  \bibitem{17}
  {\sc Remmert, R.}
  \newblock Sur les espaces analytiques holomorphiquement s\'{e}perables et holomorphiquement convexes.
  \newblock {\em C. R. Acad. Sci. (Paris)} {\bf 243} (1956), 118--121.

  \bibitem{18}
  {\sc Remmert, R. and Stein, K.}
  \newblock \"{U}ber die wesentlichen Singularit\"{a}ten analytischer Mengen.
  \newblock {\em Math. Ann.} {\bf 126} (1953), 263--306.

  \bibitem{19}
  {\sc Weil, A.}
  \newblock ``Introduction \`{a} l'\'{e}tude des vari\'{e}t\'{e}s kaehleri\'{e}nnes''.
  \newblock Paris: Hermann 1958.

\end{thebibliography}

\end{document}
