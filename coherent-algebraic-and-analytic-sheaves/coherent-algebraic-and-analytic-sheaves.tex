\documentclass[11pt]{article}

\usepackage{amssymb,amsmath}
\usepackage{hyperref}
\usepackage{cleveref}
\usepackage{enumerate}
\usepackage{mathrsfs}
%% Fancy fonts --- feel free to remove! %%
% \usepackage{ebgaramond-maths}
\usepackage{Baskervaldx}


%% Theorem environments %%

\usepackage{amsthm}

\theoremstyle{plain}

\newtheorem{innercustomtheorem}{Theorem}
\crefname{innercustomtheorem}{Theorem}{Theorems}
\newenvironment{theorem}[1]
  {\renewcommand\theinnercustomtheorem{#1}\innercustomtheorem}
  {\endinnercustomtheorem}

\newtheorem{innercustomproposition}{Proposition}
\crefname{innercustomproposition}{Proposition}{Propositions}
\newenvironment{proposition}[1]
  {\renewcommand\theinnercustomproposition{#1}\innercustomproposition}
  {\endinnercustomproposition}

\newtheorem{innercustomlemma}{Lemma}
\crefname{innercustomlemma}{Lemma}{Lemmas}
\newenvironment{lemma}[1]
  {\renewcommand\theinnercustomlemma{#1}\innercustomlemma}
  {\endinnercustomlemma}

\newtheorem{innercustomcorollary}{Corollary}
\crefname{innercustomcorollary}{Corollary}{Corollaries}
\newenvironment{corollary}[1]
  {\renewcommand\theinnercustomcorollary{#1}\innercustomcorollary}
  {\endinnercustomcorollary}


\theoremstyle{definition}

\newtheorem*{remark}{Remark}

\newtheorem{innercustomdefinition}{Definition}
\crefname{innercustomdefinition}{Definition}{Definitions}
\newenvironment{definition}[1]
  {\renewcommand\theinnercustomdefinition{#1}\innercustomdefinition}
  {\endinnercustomdefinition}


%% Shortcuts %%

\newcommand{\sh}{\mathscr}
\newcommand{\cat}{\mathcal}
\usepackage{aurical}
\newcommand{\shHom}{\sh{H}\textup{\kern-2.2pt{\Fontauri\slshape om}}}
\newcommand{\HH}{\mathrm{H}}

\renewcommand{\geq}{\geqslant}
\renewcommand{\leq}{\leqslant}

\newcommand{\todo}{\textbf{ !TODO! }}
\newcommand{\oldpage}[1]{\marginpar{\textit{p.~#1}}}


%% Document %%

\usepackage{embedall}
\begin{document}

\renewcommand{\abstractname}{Translator's note.}

\title{On coherent algebraic and analytic sheaves}
\author{A. Grothendieck}
\date{4\textsuperscript{th} and 11\textsuperscript{th} of February, 1957}
\maketitle

\begin{abstract}
  What follows is a translation of the French paper:

  \smallskip\noindent
  \textsc{Grothendieck, A.}. Sur les faisceaux algébriques et les faisceaux analytiques cohérents. \emph{Séminaire Henri Cartan}, Volume 9 (1956-1957), Talk no.~2, pp.~1–16. {\footnotesize\url{http://www.numdam.org/item/SHC_1956-1957__9__A2_0/}}

  \smallskip
  This translation is one of a series: {\footnotesize\url{https://github.com/thosgood/translations}.}
\end{abstract}

\tableofcontents


%% Content %%

\bigskip
The aim of this exposé is to generalise certain theorems of Serre.
\oldpage{2-01}
It makes fundamental use of the techniques of Serre \cite{1,2,3}.


\section{Generalities on coherent algebraic sheaves}

Let $X$ be a topological space endowed with a sheaf of rings $\sh{O}$.
A sheaf of $\sh{O}$-modules $\sh{A}$ (or simply an $\sh{O}$-module) is said to be \emph{of finite type} if, on every small-enough open subset, it is isomorphic to a quotient of $\sh{O}^n$ (for some finite integer $n\geq0$), and \emph{coherent} if it is of finite type and if, for every homomorphism $\sh{O}^m\to\sh{A}$ on an open subset $U$ of $X$, the kernel is of finite type.
If $0\to\sh{A}'\to\sh{A}\to\sh{A}''\to0$ is an exact sequence of $\sh{O}$-modules, and if two of the modules are coherent, then so too is the third;
the kernel, cokernel, image, and coimage of a homomorphism of coherent $\sh{O}$-modules is a coherent $\sh{O}$-module.
If $\sh{A}$ and $\sh{B}$ are coherent $\sh{O}$-modules, then so too is the sheaf $\shHom_\sh{O}(\sh{A},\sh{B})$ of germs of homomorphisms from $\sh{A}$ to $\sh{B}$.
If $\sh{O}$ itself is coherent, then coherent $\sh{O}$-modules are exactly the $\sh{O}$-modules that, on small-enough open subsets, are isomorphic to the cokernel of some homomorphism $\sh{O}^m\to\sh{O}^n$.
For all of this, and other elementary properties, see \cite[chapitre~1, paragraphe~2]{1}.

Let $X$ be an algebraic set (over an algebraically closed field $k$, to illustrate the idea; but the results of this exposé still hold true for schemes, and even for general arithmetic schemes...).
We denote by $\sh{O}_X$ the sheaf of local rings of $X$, with its sections over an open subset $U\subset X$ being the regular functions on $U$.
This is a sheaf of rings, and even of $k$-algebras.

\begin{theorem}{1}
\label{theorem1}
\begin{enumerate}[a)]
  \item $\sh{O}_X$ is a coherent sheaf of rings.
  \item If $X$ is affine with coordinate ring $A(X)$, then, for every coherent $\sh{O}$-module $\sh{A}$ on $X$, the localised modules $A_X$ \todo(?) are generated by the canonical image of $\Gamma(X,\sh{A})$.
    Furthermore, $\Gamma(X,\sh{A})$ is an $A(X)$-module of finite type, and every $A(X)$-module of finite type comes from an essentially unique coherent $\sh{O}$-module.
    (Recall that $\Gamma(X,\sh{A})$ denotes the module of sections of $\sh{A}$ over $X$).
  \item Under the conditions of b), we have that $\HH^i(X,\sh{A})=0$ for $i>0$.
\oldpage{2-02}
\end{enumerate}
\end{theorem}

\begin{proof}
  For the proofs, which are very elementary, see \cite[chapitre~2, paragraphes~2,3,4]{1}, or an exposé of Cartier in the 1957 \emph{Séminaire Grothendieck}.
\end{proof}


\section{A dévissage theorem}

\begin{definition}{1}
\label{definition1}
  Let $\cat{C}$ be an abelian category, and $\cat{C}'$ a subclass of $\cat{C}$.
  We say that $\cat{C}'$ is an \emph{exact subcategory} if, for every exact sequence $0\to\sh{A}'\to\sh{A}\to\sh{A}''\to0$ in $\cat{C}$ with two (non-zero) terms in $\cat{C}'$, the third term is also in $\cat{C}'$, and if every direct factor of any $\sh{A}\in\cat{C}'$ is also in $\cat{C}'$.
\end{definition}

\begin{theorem}{2}
\label{theorem2}
  Let $X$ be an algebraic set;
  suppose that, for every irreducible subset $Y$ of $X$, we are given a coherent $\sh{O}_Y$-module $\sh{F}_Y$ on $Y$ that has support equal to $Y$.
  Let $K(X)$ be the abelian category of coherent algebraic sheaves on $X$.
  Then every \emph{exact} subcategory $\cat{K}$ of $K(X)$ containing the $\sh{F}_Y$ is identical to $K(X)$.
\end{theorem}

\begin{proof}
  
\end{proof}


%% Bibliography %%

\nocite{*}
\bibliographystyle{acm}
\bibliography{\jobname}

\end{document}
