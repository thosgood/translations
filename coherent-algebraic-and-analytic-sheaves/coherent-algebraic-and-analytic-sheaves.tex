\documentclass[11pt]{article}

\usepackage{amssymb,amsmath}
\usepackage{hyperref}
\usepackage[nameinlink]{cleveref}
\usepackage{enumerate}
\usepackage{mathrsfs}
%% Fancy fonts --- feel free to remove! %%
% \usepackage{ebgaramond-maths}
\usepackage{Baskervaldx}


%% Theorem environments %%

\usepackage{amsthm}

\theoremstyle{plain}

\newtheorem{innercustomtheorem}{Theorem}
\crefname{innercustomtheorem}{Theorem}{Theorems}
\newenvironment{theorem}[1]
  {\renewcommand\theinnercustomtheorem{#1}\innercustomtheorem}
  {\endinnercustomtheorem}

\newtheorem{innercustomproposition}{Proposition}
\crefname{innercustomproposition}{Proposition}{Propositions}
\newenvironment{proposition}[1]
  {\renewcommand\theinnercustomproposition{#1}\innercustomproposition}
  {\endinnercustomproposition}

\newtheorem{innercustomlemma}{Lemma}
\crefname{innercustomlemma}{Lemma}{Lemmas}
\newenvironment{lemma}[1]
  {\renewcommand\theinnercustomlemma{#1}\innercustomlemma}
  {\endinnercustomlemma}

\newtheorem{innercustomcorollary}{Corollary}
\crefname{innercustomcorollary}{Corollary}{Corollaries}
\newenvironment{corollary}[1]
  {\renewcommand\theinnercustomcorollary{#1}\innercustomcorollary}
  {\endinnercustomcorollary}


\theoremstyle{definition}

\newtheorem*{remark}{Remark}

\newtheorem{innercustomdefinition}{Definition}
\crefname{innercustomdefinition}{Definition}{Definitions}
\newenvironment{definition}[1]
  {\renewcommand\theinnercustomdefinition{#1}\innercustomdefinition}
  {\endinnercustomdefinition}


%% Shortcuts %%

\newcommand{\sh}{\mathscr}
\newcommand{\cat}{\mathcal}
\usepackage{aurical}
\newcommand{\shHom}{\sh{H}\textup{\kern-2.2pt{\Fontauri\slshape om}}}
\newcommand{\HH}{\mathrm{H}}
\newcommand{\EE}{\mathrm{E}}
\newcommand{\RR}{\mathrm{R}}
\newcommand{\supp}{\operatorname{supp}}

\renewcommand{\geq}{\geqslant}
\renewcommand{\leq}{\leqslant}

\newcommand{\todo}{\textbf{ !TODO! }}
\newcommand{\oldpage}[1]{\marginpar{\textit{p.~#1}}}


%% Document %%

\usepackage{embedall}
\begin{document}

\renewcommand{\abstractname}{Translator's note.}

\title{On coherent algebraic and analytic sheaves}
\author{A. Grothendieck}
\date{4\textsuperscript{th} and 11\textsuperscript{th} of February, 1957}
\maketitle

\begin{abstract}
  What follows is a translation of the French paper:

  \smallskip\noindent
  \textsc{Grothendieck, A.}. Sur les faisceaux algébriques et les faisceaux analytiques cohérents. \emph{Séminaire Henri Cartan}, Volume 9 (1956-1957), Talk no.~2, pp.~1–16. {\footnotesize\url{http://www.numdam.org/item/SHC_1956-1957__9__A2_0/}}

  \smallskip
  This translation is one of a series: {\footnotesize\url{https://github.com/thosgood/translations}.}
\end{abstract}

\tableofcontents


%% Content %%

\bigskip
The aim of this exposé is to generalise certain theorems of Serre.
\oldpage{2-01}
It makes fundamental use of the techniques of Serre \cite{1,2,3}.


\section{Generalities on coherent algebraic sheaves}

Let $X$ be a topological space endowed with a sheaf of rings $\sh{O}$.
A sheaf of $\sh{O}$-modules $\sh{A}$ (or simply an $\sh{O}$-module) is said to be \emph{of finite type} if, on every small-enough open subset, it is isomorphic to a quotient of $\sh{O}^n$ (for some finite integer $n\geq0$), and \emph{coherent} if it is of finite type and if, for every homomorphism $\sh{O}^m\to\sh{A}$ on an open subset $U$ of $X$, the kernel is of finite type.
If $0\to\sh{A}'\to\sh{A}\to\sh{A}''\to0$ is an exact sequence of $\sh{O}$-modules, and if two of the modules are coherent, then so too is the third;
the kernel, cokernel, image, and coimage of a homomorphism of coherent $\sh{O}$-modules is a coherent $\sh{O}$-module.
If $\sh{A}$ and $\sh{B}$ are coherent $\sh{O}$-modules, then so too is the sheaf $\shHom_\sh{O}(\sh{A},\sh{B})$ of germs of homomorphisms from $\sh{A}$ to $\sh{B}$.
If $\sh{O}$ itself is coherent, then coherent $\sh{O}$-modules are exactly the $\sh{O}$-modules that, on small-enough open subsets, are isomorphic to the cokernel of some homomorphism $\sh{O}^m\to\sh{O}^n$.
For all of this, and other elementary properties, see \cite[chapitre~1, paragraphe~2]{1}.

Let $X$ be an algebraic set (over an algebraically closed field $k$, to illustrate the idea; but the results of this exposé still hold true for schemes, and even for general arithmetic schemes...).
We denote by $\sh{O}_X$ the sheaf of local rings of $X$, with its sections over an open subset $U\subset X$ being the regular functions on $U$.
This is a sheaf of rings, and even of $k$-algebras.

\begin{theorem}{1}
\label{theorem1}
\begin{enumerate}[a)]
  \item $\sh{O}_X$ is a coherent sheaf of rings.
  \item If $X$ is affine with coordinate ring $A(X)$, then, for every coherent $\sh{O}$-module $\sh{A}$ on $X$, the stalks $\sh{A}_x$ are generated by the canonical image of $\Gamma(X,\sh{A})$.
    Furthermore, $\Gamma(X,\sh{A})$ is an $A(X)$-module of finite type, and every $A(X)$-module of finite type comes from an essentially unique coherent $\sh{O}$-module.
    (Recall that $\Gamma(X,\sh{A})$ denotes the module of sections of $\sh{A}$ over $X$).
  \item Under the conditions of b), we have that $\HH^i(X,\sh{A})=0$ for $i>0$.
\oldpage{2-02}
\end{enumerate}
\end{theorem}

\begin{proof}
  For the proofs, which are very elementary, see \cite[chapitre~2, paragraphes~2,3,4]{1}, or an exposé of Cartier in the 1957 \emph{Séminaire Grothendieck}.
\end{proof}


\section{A dévissage theorem}

\begin{definition}{1}
\label{definition1}
  Let $\cat{C}$ be an abelian category, and $\cat{C}'$ a subclass of $\cat{C}$.
  We say that $\cat{C}'$ is an \emph{exact subcategory} if, for every exact sequence $0\to\sh{A}'\to\sh{A}\to\sh{A}''\to0$ in $\cat{C}$ with two (non-zero) terms in $\cat{C}'$, the third term is also in $\cat{C}'$, and if every direct factor of any $\sh{A}\in\cat{C}'$ is also in $\cat{C}'$.
\end{definition}

\begin{theorem}{2}
\label{theorem2}
  Let $X$ be an algebraic set;
  suppose that, for every irreducible subset $Y$ of $X$, we are given a coherent $\sh{O}_Y$-module $\sh{F}_Y$ on $Y$ that has support equal to $Y$.
  Let $K(X)$ be the abelian category of coherent algebraic sheaves on $X$.
  Then every \emph{exact} subcategory $\cat{K}$ of $K(X)$ containing the $\sh{F}_Y$ is identical to $K(X)$.
\end{theorem}

\begin{proof}
  The proof is done by induction on $n=\dim X$, with the case $n=0$ being immediate, by the second condition of \cref{definition1}.
  So suppose that $n>0$, and that the theorem is true in dimension $<n$.
  We can consider $K(Y)$ as a subcategory of $K(X)$ (where $Y$ is some given closed subset of $X$), and then $\cat{K}\cap K(Y)$ is a subcategory of $K(Y)$ satisfying the conditions of \cref{theorem2}, and so, if $\dim Y<n$, then the induction hypothesis implies that $K(Y)=K(Y)\cap\cat{K}$, i.e. $K(Y)\subset\cat{K}$.
\end{proof}

\begin{lemma}{1}
\label{lemma1}
  Let $Y$ be a closed subset of $X$, and $\sh{A}$ a coherent $\sh{O}_X$-module such that $\supp\sh{A}\subset Y$.
  Let $\sh{I}_Y$ be the sheaf of ideals of $\sh{O}_X$ defined by $Y$.
  Then there exists an integer $k$ such that $\sh{I}_Y^k\sh{A}=0$.
\end{lemma}

\begin{proof}
  By ``compactness'' reasons, we can restrict to the case where $X$ is affine, and then apply part \emph{b)} of \cref{theorem1}, noting that, if $\sh{A}$ is defined by the $A(X)$-module $M=\Gamma(X,\sh{A})$, then the ideal of support $\sh{A}$ \todo(?) is the intersection of the minimal prime ideals associated to the annihilator of $M$, whence the result.
\end{proof}

\begin{corollary}{—}
  Under the above conditions, $\sh{A}$ admits a composition series with each $\sh{A}_i/\sh{A}_{i+1}\in K(Y)$.
\end{corollary}

\begin{proof}
  \todo?
\end{proof}

This implies that $\sh{A}_i/\sh{A}_{i+1}$ is annihilated by $\sh{I}_Y$;
we take $\sh{A}_i=\sh{I}_Y^i\sh{A}$.
In the case where $\dim Y<n$, by induction on the length of
\oldpage{2-03}
this composition series, using \cref{definition1} and the fact that $K(Y)\subset\cat{K}$, we see that, \emph{if $\dim\supp\sh{A}<n$, then $\sh{A}\in\cat{K}$.}

Suppose first of all that $X$ is irreducible.
For \todo\textbf{\emph{(not $\sh{A}$...)}} $\sh{A}\in K(X)$, let $T(A)$ be the torsion submodule of $A$ (whose stalks are the torsion submodules of $\sh{A}_X$).

\begin{lemma}{2}
\label{lemma2}
  If $\sh{A}\in K(X)$, then the torsion submodule $T(A)$ is also in $K(X)$, and $A=T(A)$ if and only if $\supp\sh{A}\neq X$.
\end{lemma}

\begin{proof}
  We can immediately restrict to the case where $X$ is affine, where it is evident, by the interpretation of coherent $\sh{O}$-modules as $A(X)$-modules of finite type.
\end{proof}

Using the exact sequence $0\to T(A)\to A\to A_0\to 0$\todo\emph{\textbf{($A$ or $\sh{A}$ or what??)}}, and that $T(A)\in\cat{K}$, we see that $A\in\cat{K}$ if and only if $A_0\in\cat{K}$.

Let $\sh{R}$ be the sheaf of fields over $X$ given by the fields of fractions of the $\sh{O}_{X,x}$, i.e. the sheaf of germs of rational functions, which is a constant sheaf, and we have an injective homomorphism $A_0\to A_0\otimes_{\sh{O}_X}\sh{R}$ \todo\emph{\textbf{again.... what is $A$?}}

\todo\emph{\textbf{finish}}

\begin{lemma}{3}
\label{lemma3}
  On any irreducible algebraic set $X$, every locally constant sheaf is constant.
\end{lemma}

\begin{proof}
  This is an easy consequence of the fact that every open subset of $X$ is connected (consider a maximal open subset where the sheaf in question is constant!).
\end{proof}

We will thus identify \todo

\todo\emph{\textbf{finish all this}}


\section{Complements on sheaf cohomology}

Let $X$ be a topological space, and write $\cat{C}^X$ to denote the category of abelian sheaves on $X$.
We define, in the usual manner, injective sheaves, and we can prove the existence, for all $\sh{A}\in\cat{C}^X$, of a resolution $C(\sh{A})$ of $\sh{A}$ by injective sheaves, which allows us to develop the theory of right-derived functors.
In particular, consider the left-exact functor $\Gamma(X,\sh{A})$
\oldpage{2-05}
from $\cat{C}^X$ to the category $\cat{C}$ of abelian groups;
its derived functors are denoted $\HH^i(X,\sh{A})$.
So
\[
  \HH^i(X,\sh{A}) = \HH^i\big(\Gamma(X,C(\sh{A}))\big).
\]
The $\HH^i(X,\sh{A})$ form a ``cohomological functor'' in $\sh{A}$ that is zero for $i<0$, and satisfies
\[
  \HH^0(X,\sh{A}) = \Gamma(X,\sh{A}).
\]

If $f\colon X\to Y$ is a continuous map from $X$ to a space $Y$, then we can define, for any abelian sheaf $\sh{B}$ on $Y$, the abelian sheaf $f^{-1}(\sh{B})$ on $X$, which we call the \emph{inverse image of $\sh{B}$}, as well as the canonical homomorphism
\[
  \HH^0(Y,\sh{B}) \to \HH^0(X,f^{-1}(\sh{B}))
\]
which extends uniquely to give functorial, compatible (with the cobords \todo(?)) homomorphisms
\[
  \HH^i(Y,\sh{B}) \to \HH^i(X,f^{-1}(\sh{B})).
\]

Now let $\sh{A}$ be an abelian sheaf on $X$, and define its \emph{direct image} $f_*(\sh{A})$ to be the abelian sheaf on $Y$ whose sections over any open subset $V$ are the sections of $\sh{A}$ over $f^{-1}(V)$.
Clearly $f_*$ is a covariant additive left-exact functor from $\cat{C}^X$ to $\cat{C}^Y$, and, if $\Gamma_X$ (resp. $\Gamma_Y$) denotes the ``sections'' functor on $\cat{C}^X$ (resp. $\cat{C}^Y$), then, by definition
\[
  \Gamma_X = \Gamma_Y\circ f_*.
\]
Furthermore, it is trivial to show that $f_*$ sends injective sheaves to injective sheaves.
From this, we easily obtain the \emph{Leray spectral sequence of the continuous map $f$}, i.e. there is a cohomological spectral sequence starting with
\[
  \EE_2^{p,q} = \HH^p(Y,\RR^qf_*(\sh{A}))
\]
that abuts to $\HH^\bullet(X,\sh{A})$, where the $\RR^qf_*(\sh{A})$ are the sheaves on $Y$ given by taking the right-derived functors of the functor $f_*\colon\cat{C}^X\to\cat{C}^Y$, i.e. $\RR^qf_*(\sh{A}) = \HH^q(f_*C(\sh{A}))$.
We immediately see that $\RR^qf_*(\sh{A})$ is \emph{the sheaf on $Y$ associated to the presheaf $V\mapsto\HH^q(f^{-1}(V),\sh{A})$}.

From the Leray spectral sequence, we get homomorphisms
\[
\label{equation1}
  \HH^p(Y,f_*(\sh{A})) \to \HH^p(X,\sh{A})
  \tag{1}
\]
whose direct definition is evident (noting that we have a natural homomorphism $f^{-1}(f_*(\sh{A}))\to A$).
Furthermore, \emph{if $\RR^qf_*(\sh{A})=0$ for $q>0$, then the}
\oldpage{2-06}
\emph{homomorphisms in \cref{equation1} are isomorphisms}.
This follows immediately from the spectral sequence, or, even more simply, from the fact that $f_*(C(\sh{A}))$ is an injective resolution of $f_*(\sh{A})$.

For the results of this section, see the 1957 \emph{Séminaire Grothendieck}.


\section{Supplementary results about algebraic sheaves on projective space}


%% Bibliography %%

\nocite{*}
\bibliographystyle{acm}
\bibliography{\jobname}

\end{document}
