\documentclass{article}

\usepackage[margin=1.6in]{geometry}

\title{Notes on the Riemann $\zeta$ function, 2}
\author{Michel Balazard, Eric Saias, and Marc Yor}
\date{}

\newcommand{\doctype}{French paper}
\newcommand{\origcit}{%
  \textsc{Balazard, M., Saias, E., and Yor, M.}
  ``Notes sur la fonction $\zeta$ de Riemann, 2''.
  \emph{Advances in Mathematics}, Volume~\textbf{143} (1999), 284--287.%
}


\usepackage{amssymb,amsmath}

\usepackage{hyperref}
\usepackage{xcolor}
\hypersetup{colorlinks,linkcolor={blue!50!black},citecolor={blue!50!black},urlcolor={blue!80!black}}
\usepackage{enumerate}
\usepackage{tikz}

\usepackage{mathrsfs}
%% Fancy fonts --- feel free to remove! %%
\usepackage{fouriernc}


\usepackage{fancyhdr}
\usepackage{lastpage}
\usepackage{xstring}
\makeatletter
\ifx\pdfmdfivesum\undefined
  \let\pdfmdfivesum\mdfivesum
\fi
\edef\filesum{\pdfmdfivesum file {\jobname}}
\pagestyle{fancy}
\makeatletter
\let\runauthor\@author
\let\runtitle\@title
\makeatother
\fancyhf{}
\lhead{\footnotesize\runtitle}
\lfoot{\footnotesize Version: \texttt{\StrMid{\filesum}{1}{8}}}
\cfoot{\small\thepage\ of \pageref*{LastPage}}


%% Theorem environments %%

\usepackage{amsthm}

\newenvironment{itenv}[1]
  {\phantomsection\par\medskip\noindent\textbf{#1.}\itshape}
  {\medskip}

\newenvironment{rmenv}[1]
  {\phantomsection\par\medskip\noindent\textbf{#1.}\rmfamily}
  {\medskip}


%% Shortcuts %%

\newcommand{\dd}{\operatorname{d}\!}

\renewcommand{\geq}{\geqslant}
\renewcommand{\leq}{\leqslant}

\newcommand{\oldpage}[1]{\marginpar{\footnotesize$\Big\vert$ \textit{p.~#1}}}


%% Document %%

\usepackage{embedall}
\begin{document}

\maketitle
\thispagestyle{fancy}

\renewcommand{\abstractname}{Translator's note.}

\begin{abstract}
  \renewcommand*{\thefootnote}{\fnsymbol{footnote}}
  \emph{This text is one of a series\footnote{\url{https://thosgood.com/translations}} of translations of various papers into English.}
  \emph{The translator takes full responsibility for any errors introduced in the passage from one language to another, and claims no rights to any of the mathematical content herein.}

  \medskip
  
  \emph{What follows is a translation of the \doctype:}

  \medskip\noindent
  \origcit
\end{abstract}

\setcounter{footnote}{0}

\bigskip


%% Content %%

\bigskip

\oldpage{284}
We denote by $\sum_{\Re\rho>1/2}$ a sum over the possible zeros of $\zeta(s)$ with real part greater than $\frac12$, where the zeros of multiplicity~$m$ are counted $m$ times.
The goal of this note is the proof of the following result.

\begin{itenv}{Theorem}
  We have
  \[
  \label{1}
    \frac{1}{2\pi}\int_{\Re(s)=1/2} \frac{\log|\zeta(s)|}{|s|^2}|\dd s|
    = \sum_{\Re\rho>1/2} \log\left|\frac{\rho}{1-\rho}\right|.
  \tag{1}
  \]
  In particular, the Riemann hypothesis is true if and only if
  \[
    \int_{\Re(s)=1/2} \frac{\log|\zeta(s)|}{|s|^2}|\dd s| = 0.
  \]
\end{itenv}

\begin{proof}
  This proof consists of two steps.

  \begin{itemize}
    \item[First step.]
      We start by stating some properties satisfied by a generic function $f$ in the Hardy space $H^p(\mathbf{D})$, where $\mathbf{D}=\{z\in\mathbb{C}:|z|<1\}$, and $p$ is a positive real number.
      We denote by $f^*$ the function defined almost everywhere on the trigonometric circle $\partial\mathbf{D}=\{z\in\mathbb{C}:|z|=1\}$ by $f^*(e^{i\theta})=\lim_{r\to1^-}f(re^{i\theta})$.
      We use the letter $z$ to denote an element of the trigonometric disc $\mathbf{D}$, and write
      \[
        s = s(z) = \frac12+\frac{1+z}{2(1-z)} = \frac{1}{1-z}.
      \]
\oldpage{285}
      This formula defines a conformal representation of the disc $\mathbf{D}$ in the semi-plane $\Re(s)>1/2$.

      By Jensen's formula (see, for example, \cite[Theorem~3.61]{4}), we have, for $f(0)\neq0$ and $r<1$,
      \[
      \label{2}
        \frac{1}{2\pi}\int_{-\pi}^\pi \log|f(re^{i\theta})|\dd\theta
        = \log|f(0)| + \sum_{\substack{|\alpha|<r\\f(\alpha)=0}} \log\frac{r}{|a|}
      \tag{2}
      \]
      where, in the sum, the zeros of multiplicity~$m$ are counted $m$ times.
      Denote by
      \[
        \exp\left\{
          -\int_{-\pi}^\pi \frac{e^{i\theta}+z}{e^{i\theta}-z}\dd\mu(\theta)
        \right\}
      \]
      the singular interior factor of $f$.
      As $r$ tends to $1$, \hyperref[2]{Equation~(2)} becomes (cf. \cite[p.~68]{2})
      \[
      \label{3}
        \frac{1}{2\pi}\int_{-\pi}^\pi \log|f(re^{i\theta})|\dd\theta
        = \log|f(0)| + \sum_{\substack{|\alpha|<1\\f(\alpha)=0}} \log\frac{1}{|a|} + \int_{-\pi}^\pi\dd\mu(\theta).
      \tag{3}
      \]
      This formula is a consequence of the factorisation theorem for functions in $H^p$;
      it is stated in \cite{2} for $p=1$, but also holds for all positive values of $p$.

    \item[Second step.]
      Now consider the function
      \[
        f(z) = (s-1)\zeta(s)
      \]
      (where $s=1/(1-z)$).
      The elementary properties of the Riemann $\zeta$ function (see, for example, \cite{5}) allow us to show that, on one hand, $f$ belongs to the Hardy space $H^{1/3}(\mathbf{D})$, and, on the other hand, that the measure $\mu$ associated to the singular interior factor of $f$ is zero (for this latter point, it suffices to reuse the argument developed by Bercovici and Foias for the interior factor of the functions $(\theta-\theta^s)\zeta(s)(s+1/2)/s$, found in the proof of \cite[Proposition~2.1]{1}).
      We can equally show that
      \[
        \begin{aligned}
          \int_{-\pi}^\pi \log|f^*(e^{i\theta})|\dd\theta
          &= \int_{\Re(s)=1/2} \frac{\log|\zeta(s)|}{|s|^2}|\dd s|,
        \\\log|f(0)|
          &= 0,
        \\\sum_{\substack{|\alpha|<1\\f(\alpha)=0}} \log\frac{1}{|\alpha|}
          &= \sum_{\Re\rho>1/2} \log\left|\frac{\rho}{1-\rho}\right|.
        \end{aligned}
      \]
      \oldpage{286}
      With all this information, our result follows from \hyperref[3]{Equation~(3)}.
  \end{itemize}
\end{proof}

We finish with some remarks.
There are statements related to ours in the works \cite{7,6} of Wang and Volchkov.
It is even possible that Jensen himself was aware of \hyperref[1]{Equation~(1)} (the reader can consult the article \cite{3} where Jensen informs Mittag–Leffler of his discovery of \hyperref[2]{Equation~(2)}).
It seem interesting, however, to present things as we have done here, and this is for the following three reasons:
\begin{enumerate}[(a)]
  \item \hyperref[1]{Equation~(1)} is simpler than those that appear in \cite{7,6};
  \item we show here that, to establish \hyperref[1]{Equation~(1)}, it is natural to place ourselves in the framework of Hardy spaces;
  \item the form of the integral in \hyperref[1]{Equation~(1)} allows us to interpret this result via Brownian motion, as we show below.
\end{enumerate}

Denote by $Z=X+iY$ the planar Brownian motion from $0$ (or from $1$), and by $Z_{T_{1/2}}=\frac12+iY_{T_{1/2}}$ its first point of impact on the critical line $\Re s=1/2$, where $T_{1/2}:=\inf\{t:X_t=1/2\}$.
We know that $Y_{T_{1/2}}$ follows a Cauchy law with parameter~$1/2$.
In other words, the law of $Y_{T_{1/2}}$ has density $1/2\pi(1/4+t^2)$.
Thus the second part of the theorem can be stated in the following manner: the Riemann hypothesis is true if and only if
\[
  \mathbb{E}[\log|\zeta(Z_{T_{1/2}})] = 0.
\]


\subsection*{Thanks}

We thank Luis B\'{a}ez-Duarte, Michel Delasneri, Catherine Donati, Laurent Habsieger, Aleksandar Ivi\'{c}, and Alain Plagne for useful conversations.


%% Bibliography %%

\nocite{*}

\begin{thebibliography}{7}

  \bibitem{1}
  {\sc Bercovici, H. and Foias, C.}
  \newblock A real variable restatement of Riemann's hypothesis.
  \newblock {\em Israel J. Math.} \textbf{48} (1984), 57--68.

  \bibitem{2}
  {\sc Hoffman, K.}
  \newblock {\em Banach Spaces of Analytic Functions.}
  \newblock Dover, New York (1988).

  \bibitem{3}
  {\sc Jensen, J.L.W.V.}
  \newblock Sur un nouvel et important th\'{e}or\`{e}me de la th\'{e}orie des fonctions.
  \newblock {\em Acta Math.} \textbf{22} (1898--1899), 359--364.

  \bibitem{4}
  {\sc Titchmarsh, E.C.}
  \newblock {\em The Theory of Functions.}
  \newblock 2nd ed., Oxford Science Publications, 1939.

  \bibitem{5}
  {\sc Titchmarsh, E.C.}
  \newblock {\em The Theory of the Riemann Zeta-Function} (revised by D.R.~Heath-Brown).
  \newblock Clarendon, Oxford (1986).

  \bibitem{6}
  {\sc Volchkov, V.V.}
  \newblock On an equality equivalent to the Riemann hypothesis.
  \newblock {\em Ukrainian Math. J.} \textbf{47} (1995), 491--493.

  \bibitem{7}
  {\sc Wang, F.T.}
  \newblock A note on the Riemann zeta-function.
  \newblock {\em Bull. Amer. Math. Soc.} \textbf{52} (1946), 319-321.

\end{thebibliography}



\end{document}
