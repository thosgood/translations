\documentclass{article}

\title{Hodge Theory I}
\author{Pierre Deligne}
\date{}

\newcommand{\doctype}{French paper}
\newcommand{\origcit}{%
  \textsc{Deligne, P.}
  ``Th\'{e}orie de Hodge I''.
  \emph{Actes du Congr\`{e}s intern. math.}, Volume~\textbf{1} (1970), 425--430.
  \url{https://publications.ias.edu/deligne/paper/359}%
}


\usepackage{amssymb,amsmath}

\usepackage{hyperref}
\usepackage{xcolor}
\hypersetup{colorlinks,linkcolor={red!50!black},citecolor={blue!50!black},urlcolor={blue!80!black}}
\usepackage{enumerate}
\usepackage{booktabs}
\usepackage{tikz-cd}

\usepackage{mathrsfs}
%% Fancy fonts --- feel free to remove! %%
\usepackage{fouriernc}


\usepackage{fancyhdr}
\usepackage{lastpage}
\usepackage{xstring}
\makeatletter
\ifx\pdfmdfivesum\undefined
  \let\pdfmdfivesum\mdfivesum
\fi
\edef\filesum{\pdfmdfivesum file {\jobname}}
\pagestyle{fancy}
\makeatletter
\let\runauthor\@author
\let\runtitle\@title
\makeatother
\fancyhf{}
\lhead{\footnotesize\runtitle}
\lfoot{\footnotesize Version: \texttt{\StrMid{\filesum}{1}{8}}}
\cfoot{\small\thepage\ of \pageref*{LastPage}}




%% Theorem environments %%

\usepackage{amsthm}

\theoremstyle{plain}

\theoremstyle{definition}

  \newtheorem{innerdefinition}{Definition}
  \newenvironment{definition}[1]
    {\renewcommand\theinnerdefinition{#1}\innerdefinition}
    {\endinnerdefinition}

  \newtheorem{innerprinciple}{Principle}
  \newenvironment{principle}[1]
    {\renewcommand\theinnerprinciple{#1}\innerprinciple}
    {\endinnerprinciple}

  \newtheorem{innerenv}{}
  \newenvironment{env}[1]
    {\renewcommand\theinnerenv{#1}\innerenv}
    {\endinnerenv}

  \newtheorem*{translation*}{Translation}
  \newtheorem*{scholium*}{Scholium}
  \newtheorem*{principle*}{Principle}


%% Shortcuts %%

\newcommand{\ZZ}{\mathbb{Z}}
\newcommand{\QQ}{\mathbb{Q}}
\newcommand{\CC}{\mathbb{C}}
\newcommand{\PP}{\mathbb{P}}
\newcommand{\Id}{\mathrm{Id}}

\DeclareMathOperator{\Gr}{Gr}
\DeclareMathOperator{\Ker}{Ker}
\DeclareMathOperator{\Gal}{Gal}
\DeclareMathOperator{\Spec}{Spec}
\DeclareMathOperator{\Hom}{Hom}

\renewcommand{\geq}{\geqslant}
\renewcommand{\leq}{\leqslant}

\newcommand{\oldpage}[1]{\marginpar{\footnotesize$\Big\vert$ \textit{p.~#1}}}


%% Document %%

\usepackage{embedall}
\begin{document}

\maketitle
\thispagestyle{fancy}

\renewcommand{\abstractname}{Translator's note.}

\begin{abstract}
  \renewcommand*{\thefootnote}{\fnsymbol{footnote}}
  \emph{This text is one of a series\footnote{\url{https://thosgood.com/translations}} of translations of various papers into English.}
  \emph{The translator takes full responsibility for any errors introduced in the passage from one language to another, and claims no rights to any of the mathematical content herein.}
  
  \emph{What follows is a translation of the \doctype:}

  \medskip\noindent
  \origcit
\end{abstract}

\setcounter{footnote}{0}

% \tableofcontents
\bigskip


%% Content %%

\oldpage{425}
We intend to give a heuristic dictionary between statements in $l$-adic cohomology and statements in Hodge theory.
This dictionary has, as its most notable sources sources, \cite{3} and the conjectural theory of Grothendieck motives \cite{2}.
Up until now, it has mainly served to formulate conjectures in Hodge theory, and it has sometimes even suggested a proof.


\section{}
\label{1}

\begin{definition}{1.1}
  A \emph{mixed Hodge structure $H$} consists of
  \begin{enumerate}[(a)]
    \item a $\ZZ$-module $H_\ZZ$ of finite type (the ``\emph{integer lattice}'');
    \item a finite increasing filtration $W$ of $H_\QQ = H_\ZZ\otimes\QQ$ (the ``\emph{weight filtration}'');
    \item a finite decreasing filtration $F$ of $H_\CC = H_\ZZ\otimes\CC$ (the ``\emph{Hodge filtration}'').
  \end{enumerate}
  This data is subject to the following condition:
  there exists a (unique) bigradation of $\Gr_W(H_\CC)$ by subspaces $H^{p,q}$ such that
  \begin{enumerate}[(i)]
    \item $\Gr_W^n(H_\CC) = \bigoplus_{p+q=n}H^{p,q}$
    \item the filtration $F$ induces on $\Gr_W(H_\CC)$ the filtration
      \[
        \Gr_W(F)^p = \bigoplus_{p'\geq p} H^{p',q'}
      \]
    \item $\overline{H^{pq}}=H^{qp}$.
  \end{enumerate}
\end{definition}

A \emph{morphism} $f\colon H\to H'$ is a homomorphism $f_\ZZ\colon H_\ZZ\to H'_\ZZ$ such that $f_\QQ\colon H_\QQ\to H'_\QQ$ and $f_\CC\colon H_\CC\to H'_\CC$ are compatible with the filtrations $W$ and $F$ (respectively).

The \emph{Hodge numbers} of $H$ are the integers
\[
  h^{pq} = \dim H^{pq} = h^{qp}.
\tag{1.2}
\]

We say that $H$ is pure \emph{of weight~$n$} if $h^{pq}=0$ for $p+q\neq n$ (i.e. if $\Gr_W^i(H)=0$ for $i\neq n$).
We also say that $H$ is a \emph{Hodge structure of weight~$n$}.

The \emph{Tate Hodge structure} $\ZZ(1)$ is the Hodge structure of weight~$-2$, purely of type~$(-1,-1)$, for which $\ZZ(1)_\CC=\CC$ and $\ZZ(1)_\ZZ = 2\pi i\ZZ = \Ker(\exp\colon\CC\to\CC^*)\subset\CC$.
We set $\ZZ(n)=\ZZ(1)^{\otimes n}$.

We can show that mixed Hodge structures form an abelian category.
If $f\colon H\to H'$ is a morphism, then $f_\QQ$ and $f_\CC$ are strictly compatible with the filtrations $W$ and $F$ (cf. \cite[2.3.5]{1}).


\section{}
\label{2}

Let $A$ be a normal integral ring of finite type over $\ZZ$, with field of fractions $K$,
\oldpage{426}
and $\overline{K}$ an algebraic closure of $K$.
Let $K_{nr}$ be the largest sub-extension of $\overline{K}$ that is unramified at each prime ideal of $A$.
We know that, or we set,
\[
  \pi_1(\Spec(A),\overline{K}) = \Gal(K_{nr}/K).
\]

For every closed point $x$ of $\Spec(A)$, defined by some maximal ideal $m_x$ of $A$, the residue field $k_x=A/m_x$ is finite;
the point $x$ defines a conjugation class of ``Frobenius substitutions'' $\varphi_x\in\pi_1(\Spec(A),\overline{K})$.
We set $q_x=\#k_x$ and $F_x=\varphi_x^{-1}$.

Let $K$ be a field of finite type over the prime field of characteristic~$p$, let $\overline{K}$ be an algebraic closure of $K$, let $l$ be a prime number $\neq p$, and let $H$ be a $\ZZ_l$- (or a $\QQ_l$-) module of finite type endowed with a continuous action $\rho$ of $\Gal(\overline{K}/K)$.
We will still suppose in what follows that there exists an $A$ as above, with $l$ invertible in $A$, and such that $\rho$ factors through $\pi_1(\Spec(A),\overline{K}) = \Gal(K_{nr}/K)$.
We say that $H$ is \emph{pure of weight~$n$} if, for every closed point $x$ of an non-empty open subset of $\Spec(A)$, the eigenvalues $\alpha$ of $F_x$ acting on $H$ are algebraic integers whose complex conjugates are all of absolute value $|\alpha|=q_x^{n/2}$.

\begin{principle}{2.1}
\label{principle-2.1}
  If the Galois module $H$ ``comes from algebraic geometry'', then there exists a (unique) increasing filtration $W$ on $H_{\QQ_l}=H\otimes_{\ZZ_l}\QQ_l$ (the ``\emph{weight filtration}'') that is Galois invariant and such that $\Gr_n^W(H)$ is pure of weight~$n$.
\end{principle}

We can also further suppose that $\Gr_n^W(H)$ is semi-simple.

When we have a resolution of singularities, we can often give a conjectural definition of $W$, whose validity follows from the Weil conjectures \cite{5} (cf. \hyperref[6]{\S6}).

Let $\mu$ be the subgroup of $\overline{K}^*$ given by the roots of unity.
The \emph{Tate module $\ZZ_l(1)$}, defined by
\[
  \ZZ_l(1) = \Hom(\QQ_l/\ZZ_l,\mu)
\]
is pure of weight~$-2$.
We set $\ZZ_l(n)=\ZZ_l(1)^\otimes n$.

It is trivial that every morphism $f\colon H\to H'$ is strictly compatible with the weight filtration.

\hyperref[principle-2.1]{Principle~2.1} agrees with the fact that every extension of $\mathbb{G}_m$ (``weight~$-2$'') by an abelian variety (``weight~$-1>-2$'') is trivial.


\section{}
\label{3}

\begin{translation*}
  The Galois modules that appear in $l$-adic cohomology have, as analogues, over $\CC$, mixed Hodge structures.
  We further have the dictionary
  
  \bigskip\noindent
  \begin{tabular}{p{0.45\linewidth}|p{0.45\linewidth}}
    \toprule
    pure module of weight $n$
    & Hodge structure of weight $n$
  \\weight filtration
    & weight filtration
  \\Galois-compatible homomorphism
    & morphism
  \\Tate module $\ZZ_l(1)$
    & Tate Hodge structure $\ZZ(1)$
  \\\bottomrule
  \end{tabular}
\end{translation*}


\section{}
\label{4}

Let $X$ be a complex algebraic variety (i.e. a scheme of finite type over $\CC$ that we assume to be separated).
Then there exists a subfield $K$ of $\CC$, of finite type over $\QQ$, such that $X$ can be defined over $K$ (i.e. it comes from an extension of scalars of $K$ to $\CC$ applied to a $K$-scheme $X'$).
Let $\overline{K}$ be the algebraic closure of $K$ in $\CC$.
The Galois group $\Gal(\overline{K}/K)$ then acts on the $l$-adic cohomology groups $H^\bullet(X,\ZZ_l)$;
we have
\[
  H^\bullet(X(\CC),\ZZ)\otimes\ZZ_l
  = H^\bullet(X,\ZZ_l)
  = H^\bullet(X'_{\overline{K}},\ZZ_l).
\]

\oldpage{427}
By \hyperref[3]{\S3}, we should expect for the cohomology groups $H^n(X(\CC),\ZZ)$ to carry natural mixed Hodge structures.
This is what we can prove (see \cite[3.2.5]{1} for the case where $X$ is smooth; the proof is algebraic, using classical Hodge theory \cite{6}).
For $X$ projective and smooth, the Weil conjectures imply that $H^n(X,\ZZ_l)$ is pure of weight~$n$, while classical Hodge theory endows $H^n(X,\ZZ)$ with a Hodge structure of weight~$n$.
For every morphism $f\colon X\to Y$, and for $K$ large enough, $f^\bullet\colon H^\bullet(Y,\ZZ_l)\to H^\bullet(X,\ZZ_l)$ Galois-commutes (by structure transport);
similarly, $f^\bullet\colon H^\bullet(Y,\ZZ)\to H^\bullet(X,\ZZ)$ is a morphism of mixed Hodge structures.
For $X$ smooth, the cohomology class $Z$ in $H^{2n}(X,\ZZ_l(n))$ of an algebraic cycle of codimension~$n$ defined over $K$ is Galois invariant, i.e. it defines
\[
  c(Z) \in \Hom_{\Gal}(\ZZ_l(-n),H^{2n}(X,\ZZ_l)).
\]
Similarly, the cohomology class $c(Z)\in H^{2n}(X(\CC),\ZZ)$ is purely of type~$(n,n)$, i.e. it corresponds to
\[
  c(Z) \in \Hom_{\mathrm{H.M.}}(\ZZ(-n),H^{2n}(X(\CC),\ZZ)).
\]


\section{}
\label{5}

If $f\colon H\to H'$ is a Galois-compatible morphism between $\QQ_l$-vector spaces of different weights, then $f=0$.
Similarly, if $f\colon H\to H'$ is a morphism of pure mixed Hodge structures of different weights, then $f$ is torsion.
A more useful remark is

\begin{scholium*}
  Let $H$ and $H'$ be Hodge structures of weight~$n$ and $n'$ (respectively), with $n>n'$.
  Let $f\colon H_\QQ\to H'_\QQ$ be a homomorphism such that $f\colon H_\CC\to H'_\CC$ respects $F$.
  Then $f=0$.
\end{scholium*}


\section{}
\label{6}

Let $X$ be a smooth projective variety over $\CC$, let $D=\sum_1^n D_i$ a normal crossing divisor in $X$, with $D_i$ all smooth divisors, and let $j$ be the inclusion of $U=X\setminus D$ into $X$.
For $Q\subset[1,n]$, we set $D_q=\bigcap_{i\in Q}D_i$.

In $l$-adic cohomology, we canonically have
\[
  R^q j_* \ZZ_l = \bigoplus_{\#Q=q} \ZZ_l(-q)_{D_Q}
\tag{6.1}
\]
and the Leray spectral sequence for $j$ is of the form
\[
\label{6.2}
  E_2^{pq}
  = \bigoplus_{\#Q=q} H^p(D_Q,\QQ_l)\otimes\ZZ_l(-q)
  \Rightarrow H^{p+q}(U,\QQ_l).
\tag{6.2}
\]

\emph{By the Weil conjectures} \cite{5}, $H^p(D_Q,\QQ_l)$ is pure of weight~$p$, so that $E_2^{pq}$ is pure of weight $p+2q$.
As a quotient of a sub-object of $E_2^{pq}$, $E_r^{pq}$ is also pure of weight $p+2q$.
By \hyperref[5]{\S5}, $d_r=0$ for $r\geq3$, since the weights $p+2q$ and $p+2q-r+2$ of $E_r^{pq}$ and $E_r^{p+q,q-r+1}$ (respectively) are different.
Thus $E_3^{pq}=E_\infty^{pq}$.
Up to renumbering, the weight filtration of $H^\bullet(U,\QQ_l)$ is the abutment of \hyperref[6.2]{(6.2)}:
\[
\label{6.3}
  \Gr_n^W(H^k(U,\QQ_l)) = E_3^{2k-n,n-k}.
\tag{6.3}
\]


\section{}
\label{7}

In integer cohomology, for the usual topology, the Leray spectral sequence for $j$ is of the form
\[
\label{7.1}
  'E_2^{pq}
  = \bigoplus_{\#Q=q} H^p(D_Q,\ZZ)
  \Rightarrow H^{p+q}(U,\ZZ).
\tag{7.1}
\]

\oldpage{428}
Since each $D_Q$ is a non-singular projective variety, $'E_2^{pq}$ is endowed with a Hodge structure of weight~$p$.
We set $E_2^{pq}='E_2^{pq}\otimes\ZZ(-q)$ (a Hodge structure of weight $p+2q$).
As an abelian group, $E_2^{pq}='E_2^{pq}$;
it is interesting to consider \hyperref[7.1]{(7.1)} as a spectral sequence with initial page $E_2^{pq}$.
By~\hyperref[3]{\S3}, we should expect for $d_2\colon E_2^{pq}\to E_2^{p+2,q-1}$ to be a morphism of Hodge structures.
We prove this by thinking of $d_2$ as a Gysin morphism.
Then $E_3^{pq}$ is endowed with a Hodge structure of weight $p+2q$.
By~\hyperref[3]{\S3}, we expect that, \emph{modulo torsion}, the spectral sequence\footnote{\emph{[Trans.] The original refers to (6.4), but this seems to be a typo.}} \hyperref[6.2]{(6.2)} degenerates at the $E_3$ page (i.e. $E_3=E_\infty$), and that the vanishing of the $d_r$ (for $r\geq3$) is an application of \hyperref[5]{\S5}.
This programme was successfully completed in \cite[§3.2]{1}.
There, we \emph{define} the weight filtration of $H^\bullet(U,\QQ)$ as the abutment of \hyperref[7.1]{(7.1)}, up to renumbering \hyperref[6.3]{(6.3)}.

In fact, to endow the cohomology groups $H^\bullet$ with a mixed Hodge structure, the key point has always been, up until now, to find a spectral sequence $E$ abutting to $H^\bullet$ such that the $l$-adic analogue of $E_2^{pq}$ be conjecturally pure (of weight $p+2q$);
$E_2^{pq}$ should then carry a natural Hodge structure (of weight $p+2q$), and the filtration $W$ is the abutment of $E$.


\section{}
\label{8}

Let $\Spec(V)$ be the spectrum of a Henselian discrete valuation ring (a \emph{Henselian trait}) with field of fractions $K$, and residue field $k$ that is of finite type over the prime field of characteristic $p$.
Let $\overline{K}$ be an algebraic closure of $K$, and let $H$ be a vector space of finite dimension over $\QQ_l$ (for $l\neq p$), on which $\Gal(\overline{K}/K)$ acts continuously.
By Grothendieck, we know (\cite[Appendix]{4}) that a subgroup of finite index of the inertia group $I$ acts unipotently.
By replacing $V$ with a finite extension, we arrive to the case where the action of all of $I$ is unipotent (the \emph{semi-stable} case);
it then factors as the largest pro-$l$-group $I_l$ that is a quotient of $I$, and canonically isomorphic to $\ZZ_l(1)$.

\begin{principle}{8.1}
  In the semi-stable case, if the Galois module $H$ ``comes from algebraic geometry'', then there exists a (unique) increasing filtration $W$ of $H$ (the ``\emph{weight filtration}'') such that $I$ acts trivially on $\Gr_n^W(H)$, and such that $\Gr_n^W(H)$, as a Galois module under $\Gal(\overline{k}/k)\simeq\Gal(\overline{K}/K)/I$ is pure of weight~$n$.
\end{principle}

We can compare this with \hyperref[principle-2.1]{Principle~2.1} and with the appendix of \cite{4}.

If we have a resolution of the singularities, then we can sometimes give a conjectural definition of $W$, whose validity follows from the Weil conjectures.
With the help of the resolution and of Weil, it is sometimes easy to show that, in any case, $H$ splits into pure Galois modules (under $\Gal(\overline{k}/k)$).

Suppose that $H$ is semi-stable.
For $T\in I_t$, we define $\log T$ by the \emph{finite} sum $-\sum_{n>0}(\Id-T)^n/n$.
The map $(T,x)\mapsto\log T(x)$ can be identified with a homomorphism
\[
\label{8.2}
  M\colon \ZZ_l(1)\otimes H \to H.
\tag{8.2}
\]
Since $\ZZ_l(1)$ is of weight~$-2$, we necessarily have (cf. \hyperref[5]{\S5})
\[
\label{8.3}
  M(\ZZ_l(1)\otimes W_n(H)) \subset W_{n-2}(H),
\tag{8.3}
\]
and $M$ induces
\[
\label{8.4}
  \Gr(M)\colon \ZZ_l(1)\otimes\Gr_n^W(H) \to \Gr_{n-2}^W(H).
\tag{8.4}
\]

\oldpage{429}
\begin{env}{8.5}
\label{8.5}
  If $X$ is a non-singular projective variety over an algebraically closed field $k_0$, then we define
  \[
    L\colon \ZZ_l(-1)\otimes H^\bullet(X,\ZZ_l) \to H^\bullet(X,\ZZ_l)
  \]
  as being the cup product with the cohomology class with a hyperplane section.
  We note that there is a formal analogy between $L$ and $M$;
  in the same way that $M$ is defined by an action of $\ZZ_l(1)$, we can think of $L$ as being defined by an action of $\ZZ_l(-1)$;
  $L$ increases the degree by~$2$, and $\Gr M$ \hyperref[8.4]{(8.4)} decreases it by~$2$.
\end{env}


\section{}
\label{9}

Let $D$ be the unit disc, $D^*=D\setminus\{0\}$, and $X$
\[
  \begin{tikzcd}[column sep=tiny]
    X \ar[rr,hook] \ar[dr,swap,"f"]
    && \PP^r(\CC)\times D \ar[dl,"\mathrm{pr}_2"]
  \\&D&
  \end{tikzcd}
\]
a family of projective varieties parameterised by $D$, with $f$ proper, and $f|D^*$ smooth.
Keeping the notation of \hyperref[8]{\S8}, and recalling that, in the analogy between Henselian traits and small neighbourhoods of $0$ in the complex line, we have the following dictionary (note that the spectrum of the ring of germs at $0$ of holomorphic functions is a Henselian trait):

\begin{env}{9.1}
  \phantom{}
  \begin{tabular}{p{0.45\linewidth}|p{0.45\linewidth}}
    \toprule
    $D$
    & $\Spec(V)$
  \\$D^*$
    & $\Spec(K)$
  \\a universal covering $\widetilde{D^*}$ of $D^*$
    & $\Spec(\overline{K})$
  \\the fundamental group $\pi_1(D^*)$
    & the inertia group $I$
  \\(with $\pi_1(D^*)=\ZZ\simeq\ZZ(1)_\ZZ$)
    & (with $I_l=\ZZ_l(1)$)
  \\$X$
    & a projective scheme $X$ over $\Spec(V)$
  \\$X^*=f^{-1}(D^*)$
    & $X_K$
  \\$\widetilde{X}=X\times_D\widetilde{D^*}$
    & $X_{\overline{K}}$
  \\the local system $R^if_*\ZZ|D^*$
    & the Galois module $H^i(X_{\overline{K}},\ZZ_l$
  \\$H^i(\widetilde{X},\ZZ)$
    & $H^i(X_{\overline{K}},\ZZ_l)$
  \end{tabular}
\end{env}

Note that $\widetilde{X}$ is homotopically equivalent to each of the fibres $X_t=f^{-1}(t)$ (for $t\in D^*$): $H^i(X_{\overline{K}},\ZZ_l)$ is again analogous to $H^i(X_t,\ZZ)$, and the transformation of the monodromy $T$ corresponds to the action of $I$.

Here, again, we know that a subgroup of finite index of $\pi_1(D^*)$ acts unipotently on $H^i(\widetilde{X},\QQ)=H^i(X_t,\QQ)$.
We place ourselves in the semi-stable case, where all of $\pi_1(D^*)$ acts unipotently (this reduces to replacing $D$ by a finite covering), and let $T$ be the action of the canonical generator of $\pi_1(D^*)$.

By \hyperref[3]{\S3} and \hyperref[8]{\S8}, we expect for $H^i(\widetilde{X},\QQ)\simeq H^i(X_t,\QQ)$ to be endowed with an increasing filtration $W$, for $\Gr_n^W(H^i(\widetilde{X},\QQ))$ to be endowed with a Hodge structure of weight~$n$, for $\log T(W_n)\subset W_{n-2}$, and for $\log T$ to induce a morphism of Hodge structures
\[
  M_n\colon \ZZ(-1)\otimes\Gr_n^W(H^i) \to \Gr_{n-2}^W(H^i).
\]
We would further like for \hyperref[8.2]{(8.2)}, and not just \hyperref[8.3]{(8.3)} and \hyperref[8.4]{(8.4)}, to have an analogue.

We have in fact managed to define, for each vector $u$ of the tangent space to $D$ at $\{0\}$, a mixed Hodge structure $\mathscr{H}_u$ on $H^i(\widetilde{X},\ZZ)$.
The filtration $W$ and the Hodge structures on the $\Gr_n^W(H^i)$ are independent of $u$, and the dependence on $u$ of $\mathscr{H}_u$ can be expressed
\oldpage{430}
simply in terms of $T$.
Analogously to \hyperref[8.2]{(8.2)}, we find that, for any $u$, $\log T$ induces a homomorphism of mixed Hodge structures
\[
  M\colon \ZZ(1)\otimes H^i(\widetilde{X},\ZZ) \to H^i(\widetilde{X},\ZZ).
\]

Finally, the analogy in \hyperref[8.5]{8.5} is not misleading (but here, the fact that $f|D^*$ is assumed to be proper and smooth is probably essential).
We can prove that
\[
  (\log T)^k\colon \Gr_{n+k}^W(H^n(\widetilde{X},\QQ)) \to \Gr_{n-k}^W(H^n(\widetilde{X},\QQ))
\]
is an isomorphism for all $k$ (cf. \cite[IV~6, Corollary to Theorem~5]{6}).
This characterises the filtration $W$.
Up until the present, we have only had an analogue of the positivity theorem of Hodge (cf. \cite[IV~7, Corollary to Theorem~7]{6}) in very particular cases.
We hope that the mixed structures $\mathscr{H}_u$ determine the asymptotic behaviour, for $t\to0$, of the family of pure structures $H^i(X,\ZZ)$ (for $t\in D^*$).


%% Bibliography %%

\nocite{*}

\begin{thebibliography}{6}

  \bibitem{1}
  {\sc Deligne, P.}
  \newblock Th\'{e}orie de Hodge.
  \newblock (To appear in {\em Publ. Math. I.H.E.S.} \textbf{40}).

  \bibitem{2}
  {\sc Demazure, M.}
  \newblock Motifs des vari\'{e}t\'{e}s alg\'{e}briques.
  \newblock {\em S\'{e}m. Bourbaki} (1969--70), Talk no.~365.

  \bibitem{3}
  {\sc Serre, J.-P.}
  \newblock Analogues k\"{a}hl\'{e}riens de certaines conjectures de Weil.
  \newblock {\em Ann. of Math.} \textbf{71} (1960), 392--394.

  \bibitem{4}
  {\sc Serre, J.-P. and Tate, J.}
  \newblock Good reduction of abelian varieties.
  \newblock {\em Ann. of Math.} \textbf{88} (1968), 392--517.

  \bibitem{5}
  {\sc Weil, A.}
  \newblock Number of solutions of equations in finite fields.
  \newblock {\em Bull. Amer. Math. Soc.} \textbf{55} (1949), 497--508.

  \bibitem{6}
  {\sc Weil, A.}
  \newblock {\em Introduction \`{a} l'\'{e}tude des vari\'{e}t\'{e}s k\"{a}hl\'{e}riennes.}
  \newblock {Act. Sci. et Ind.} \textbf{1267}, Hermann (1958).

\end{thebibliography}


\end{document}
