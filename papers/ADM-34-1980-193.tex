\documentclass{article}

\usepackage[margin=1.6in]{geometry}

\title{Multialgebraic categories}
\author{Yves Diers}
\date{}

\newcommand{\doctype}{French paper}
\newcommand{\origcit}{%
  \textsc{Diers, Y}
  ``Cat\'{e}gories Multialg\'{e}briques''.
  \emph{Archiv der Mathematik} \textbf{34} (1980), 193--209.
  DOI: \href{https://doi.org/10.1007/BF01224953}{\texttt{10.1007/BF01224953}}%
}


\usepackage{amssymb,amsmath}

\usepackage{hyperref}
\usepackage{xcolor}
\hypersetup{colorlinks,linkcolor={blue!50!black},citecolor={blue!50!black},urlcolor={blue!80!black}}
\usepackage{enumerate}
\usepackage{tikz-cd}
\usepackage{longtable}

\usepackage{mathrsfs}
%% Fancy fonts --- feel free to remove! %%
\usepackage{fouriernc}


\usepackage{fancyhdr}
\usepackage{lastpage}
\usepackage{xstring}
\pagestyle{fancy}
\fancypagestyle{plain}{}
\fancyhf{}
\lhead{\footnotesize\nouppercase\leftmark}
\cfoot{\small\thepage\ of \pageref*{LastPage}}
% Git commit hash for server builds
\newif\ifserver
\serverfalse
\lfoot{\footnotesize\ifserver{Git commit: \href{https://github.com/thosgood/translations/commit/GitCommitHashVariable}{GitCommitHashVariable}}\fi}




%% Theorem environments %%

\usepackage{amsthm}

\newenvironment{itenv}[1]
  {\phantomsection\par\medskip\noindent\textbf{#1.}\itshape}
  {\medskip}

\newenvironment{rmenv}[1]
  {\phantomsection\par\medskip\noindent\textbf{#1.}\rmfamily}
  {\medskip}


%% Shortcuts %%

\newcommand{\bb}[1]{{\mathbb{#1}}}

\newcommand{\op}{{\mathrm{op}}}
\newcommand{\st}{{\mathrm{st}}}
\newcommand{\Set}{\mathbb{E}\mathrm{ns}}
\newcommand{\MulAlg}{\mathbb{M}\mathrm{ulAlg}}
\newcommand{\Dom}{\mathbb{D}\mathrm{om}}
\newcommand{\Locc}{\mathbb{L}\mathrm{occ}}
\newcommand{\Orth}{\mathbb{O}\mathrm{rth}}
\newcommand{\Euc}{\mathbb{E}\mathrm{uc}}
\newcommand{\PHilb}{\mathbb{P}\mathrm{Hilb}}
\newcommand{\TotOrd}{\mathbb{T}\mathrm{otOrd}}

\DeclareMathOperator{\Part}{Part}
\DeclareMathOperator{\Hom}{Hom}

\renewcommand{\geq}{\geqslant}
\renewcommand{\leq}{\leqslant}

\newcommand{\oldpage}[1]{\marginpar{\footnotesize$\Big\vert$ \textit{p.~#1}}}


%% Document %%

\usepackage{embedall}
\begin{document}

\maketitle
\thispagestyle{fancy}

\renewcommand{\abstractname}{Translator's note.}

\begin{abstract}
  \renewcommand*{\thefootnote}{\fnsymbol{footnote}}
  \emph{This text is one of a series\footnote{\url{https://thosgood.com/translations}} of translations of various papers into English.}
  \emph{The translator takes full responsibility for any errors introduced in the passage from one language to another, and claims no rights to any of the mathematical content herein.}

  \medskip
  
  \emph{What follows is a translation of the \doctype:}

  \medskip\noindent
  \origcit
\end{abstract}

\setcounter{footnote}{0}
\setcounter{section}{-1}

\setcounter{tocdepth}{1}
\tableofcontents
\bigskip


%% Content %%

\section{Introduction}
\label{0}

\oldpage{193}
We revise the notions of algebraic theories, algebras, categories, and algebraic functors introduced by F.W.~Lawvere, in such a way that the essential theorems can be generalised to apply to non-algebraic situations, such as that of fields, local rings, totally ordered sets, metric spaces, normed vector spaces, pre-Hilbert spaces, etc.

A multialgebraic category is a category of functors that are multicontinuous for finite multiproducts, defined over a small category with finite multiproducts with values in $\Set$.
We show that multialgebraic categories have filtered colimits, connected limits, and cokernels for coequalisable pairs of morphisms, and that their equivalence relations are effective, and their regular epimorphisms are universal, and that they have regular universal factorisations.
They are equipped with a structure-forgetful functor with values in $\Set$, which admits a left multiadjoint, reflects isomorphisms, and preserves filtered colimits, connected limits, and regular epimorphisms.
We give two characterisations of multialgebraic categories, and we show that they are equivalent to multimonadic categories of finite rank over $\Set$.
Proper morphisms of multialgebraic theories determine proper multialgebraic functors.
These functors possess a left adjoint.
For example, the inclusion functors of the category of commutative fields into the category of integral domains and into the category of commutative local rings are both proper multialgebraic.

We use the notation and results of \cite{2} and \cite{3}.

\oldpage{194}
\section{Multialgebraic theories and multialgebras}

\begin{rmenv}{1.0 Definition \cite{2}}
\label{1.0}
  A \emph{multiproduct} of a small family $(X_i)_{i\in I}$ of objects of a category $\bb{A}$ is a small family $(\gamma_{ij}\colon Y_j\to X_i)_{(i,j)\in I\times J}$ of morphisms in $\bb{A}$ such that, for every family $(f_i\colon Y\to X_i)_{i\in I}$ of morphisms in $\bb{A}$, there exists a unique pair $(j,f)$ consisting of $j\in J$ and a morphism $f\colon Y\to Y_j$ such that $\gamma_{ij}f=f_i$ for all $i\in I$.

  We say that the $Y_j$ \emph{belong} to the multiproduct of objects $(X_i)_{i\in I}$.
  The multiproduct is said to be \emph{finite} if $I$ is finite.
  The category $\bb{A}$ is said to have \emph{finite multiproducts} if every finite family of objects of $\bb{A}$ has a multiproduct.
\end{rmenv}

\begin{rmenv}{1.1 Definitions}
\label{1.1}
  A \emph{multialgebraic theory} is a small category $\bb{M}$ with finite multiproducts, endowed with a distinguished small family of objects $(X_g)_{g\in G}$ such that every object of $\bb{M}$ belongs to a finite multiproduct of objects of this family.

  An \emph{$\bb{M}$-multialgebra} is a functor $F\colon\bb{M}\to\Set$ that is multicontinuous for finite multiproducts \cite{2}, i.e. for every finite sequence $X_1,\ldots,X_n$ of objects of $\bb{M}$ that has a multiproduct $(\gamma_{ij}\colon Y_j\to X_i)_{(i,j)\in[1,n]\times J}$, the map
  \[
    \langle(F\gamma_{ij})\rangle\colon
    \coprod_{j\in J} FY_j \to
    \prod_{i=1}^n FX_i
  \]
  is bijective.

  If $F$ and $H$ are $\bb{M}$-multialgebras, then an \emph{$\bb{M}$-homomorphism} from $F$ to $H$ is a natural transformation from $F$ to $H$.

  The category of $\bb{M}$-multialgebras and $\bb{M}$-homomorphisms is denoted by $\MulAlg(\bb{M})$.
\end{rmenv}

\begin{rmenv}{1.2 Examples}
\label{1.2}

  \begin{rmenv}{1.2.0}
  \label{1.2.0}
    Algebraic theories and algebras in the sense of F.W.~Lawvere \cite{6}, and those of $I$-terms in the sense of J.~Benabou \cite{1}.
  \end{rmenv}

  \begin{rmenv}{1.2.1 The multialgebraic theory of integral domains}
  \label{1.2.1}
    Let $\bb{D}_0$ be the category whose objects are pairs $(n,I)$ consisting of an integer $n\in\bb{N}$ and a prime ideal $I$ of $\bb{Z}[X_1,\ldots,X_n]$, and whose morphisms $(n,I)\to(m,J)$ are the injective homomorphisms of unital rings $\bb{Z}[X_1,\ldots,X_n]/I\to\bb{Z}[X_1,\ldots,X_m]/J$.
    These are of the form $\langle g_1,\ldots,g_n\rangle$, where $g_1,\ldots,g_n$ are polynomials in $\bb{Z}[X_1,\ldots,X_m]$ such that $f\in I$ if and only if $f(g_1,\ldots,g_n)\in J$ for all $f\in\bb{Z}[X_1,\ldots,X_n]$, and where $\langle g_1,\ldots,g_n\rangle$ denotes the quotient homomorphism of the homomorphism $g_1,\ldots,g_n\colon\bb{Z}[X_1,\ldots,X_n]\to\bb{Z}[X_1,\ldots,X_m]$.
    The composition of morphisms is given by composing the ring homomorphisms.
    The category $\bb{D}_0$ has finite multisums, since the family of objects $(0,(p))$, where $p$ runs over all prime numbers, is initial in $\bb{D}_0$, and the multisum of $(n,I)$ and $(m,J)$ exists, consisting of the objects $(n+m,K)$, where $K$ runs over the prime ideals of $\bb{Z}[X_1,\ldots,X_n,X_{n+1},\ldots,X_{n+m}]$ such that
    \[
      K\cap\bb{Z}[X_1,\ldots,X_n] = I
      \quad\text{and}\quad
      K\cap\bb{Z}[X_{n+1},\ldots,X_{n+m}] = J.
    \]
    We can see that every object of $\bb{D}_0$ belongs to a finite multisum of objects of the form $(1,I)$.
    The opposite category $\bb{D}_0^\op$ in which we distinguish the objects of the form $(1,I)$ is thus a multialgebraic theory, which we denote by $\bb{M}$.

\oldpage{195}
    Let $A$ be an integral domain.
    For each $(x_1,\ldots,x_n)\in A^n$, denote by
    \[
      I_{x_1,\ldots,x_n} =
      \big\{
        P(X_1,\ldots,X_n)\in\bb{Z}[X_1,\ldots,X_n]
        :
        P(x_1,\ldots,x_n)=0
      \big\}
    \]
    the prime ideal of polynomial relations with coefficients in $\bb{Z}$ between the $x_1,\ldots,x_n$.
    We define a functor $A^{(\,\,)}\colon\bb{M}\to\Set$ by
    \[
      A^{(n,I)}=\{(x_1,\ldots,x_n)\in A^n:I_{x_1,\ldots,x_n}=I\}
    \]
    and, for a morphism $\langle g_1,\ldots,g_n\rangle\colon(n,I)\to(m,J)$ of $\bb{D}_0$, by
    \[
      A^{\langle g_1,\ldots,g_n\rangle}(x_1,\ldots,x_m) =
      \big(
        g_1(x_1,\ldots,x_m),
        \ldots,
        g_n(x_1,\ldots,x_m)
      \big).
    \]
    This functor is an $\bb{M}$-multialgebra since we have
    \[
      \coprod_p A^{(0,(p))} \cong 1
      \quad\text{and}\quad
      \coprod_K A^{(n+m,K)} \cong A^{(n,I)}\times A^{(m,J)}
    \]
    (where the first coproduct is over all primes $p$, and the second coproduct is over all $K$ such that $K\cap\bb{Z}[X_1,\ldots,X_n] = I$ and $K\cap\bb{Z}[X_{n+1},\ldots,X_{n+m}] = J$).
    But we can prove that every $\bb{M}$-multialgebra is, up to isomorphism, of this form, and thus defines an integral domain.
    This correspondence is functorial, i.e. if $\Dom$ denotes the category of integral domains and injective homomorphisms, then we can define a functor $V\colon\Dom\to\MulAlg(\bb{M})$ by $VA=A^{(\,\,)}$ and $Vf(x_1,\ldots,x_n)=(f(x_1),\ldots,f(x_n))$.
    We can, with difficulty, directly prove that $V$ is an equivalence of categories, but this result is also an immediate consequence of \hyperref[3.2]{Theorem~3.2}.
  \end{rmenv}

  \begin{rmenv}{1.2.2 The multialgebraic theory of commutative local rings}
  \label{1.2.2}
    Let $\bb{L}_0$ be the category whose objects are pairs $(n,I)$ consisting of an integer $n\in\bb{N}$ and a prime ideal $I$ of $\bb{Z}[X_1,\ldots,X_n]$, and whose morphisms $(n,I)\to(m,J)$ are the homomorphisms of local rings $\bb{Z}[X_1,\ldots,X_n]_I\to\bb{Z}[X_1,\ldots,X_m]_J$ that are localisations of polynomial rings at prime ideals.
    These are of the form $[g_1,\ldots,g_n]$, where $g_1,\ldots,g_n$ are polynomials in $\bb{Z}[X_1,\ldots,X_n]$ such that $f\in I$ if and only if $f(g_1,\ldots,g_n)\in J$ for all $f\in\bb{Z}[X_1,\ldots,X_n]$, and where $[g_1,\ldots,g_n]$ denotes the extension to fractions of the homomorphism $g_1,\ldots,g_n\colon\bb{Z}[X_1,\ldots,X_n]\to\bb{Z}[X_1,\ldots,X_m]$.
    The composition of morphisms is given by composing the ring homomorphisms.
    The category $\bb{L}_0$ has finite multisums since the objects of the form $(0,(p))$ form an initial family of objects, and the multisum of $(n,I)$ and $(m,J)$ exists, consisting of objects $(n+m,K)$, where $K$ runs over the prime ideals of $\bb{Z}[X_1,\ldots,x_n,X_{n+1},\ldots,X_{n+m}]$ such that
    \[
      K\cap\bb{Z}[X_1,\ldots,X_n] = I
      \quad\text{and}\quad
      K\cap\bb{Z}[X_{n+1},\ldots,X_{n+m}] = J.
    \]
    We can see that every object of $\bb{L}_0$ belongs to a finite multisum of objects of the form $(1,I)$.
    The opposite category in which we distinguish the objects of the form $(1,I)$ is thus a multialgebraic theory, which we denote by $\bb{M}$.

    Let $A$ be a commutative local ring.
    For every $(x_1,\ldots,x_n)\in A^n$, denote by
    \[
      J_{x_1,\ldots,x_n} =
      \big\{
        P(X_1,\ldots,X_n)\in\bb{Z}[X_1,\ldots,X_n]
        :
        \mbox{$P(X_1,\ldots,x_n)$ is not invertible}
      \big\}
    \]
\oldpage{196}
    the prime ideal.
    We define a functor  $A^{[\,\,]}\colon\bb{M}\to\Set$ by
    \[
      A^{[n,I]}=\{(x_1,\ldots,x_n)\in A^n:J_{x_1,\ldots,x_n}=I\}
    \]
    and, for a morphism $[g_1,\ldots,g_n]\colon(n,I)\to(m,J)$ of $\bb{L}_0$, by
    \[
      A^{[g_1,\ldots,g_n]}(x_1,\ldots,x_m) =
      \big(
        g_1(x_1,\ldots,x_m),
        \ldots,
        g_n(x_1,\ldots,x_m)
      \big).
    \]
    This functor is an $\bb{M}$-multialgebra since we have
    \[
      \coprod_p A^{[0,(p)]} \cong 1
      \quad\text{and}\quad
      \coprod_K A^{[n+m,K]} \cong A^{[n,I]}\times A^{[m,J]}
    \]
    (where the first coproduct is over all primes $p$, and the second coproduct is over all $K$ such that $K\cap\bb{Z}[X_1,\ldots,X_n] = I$ and $K\cap\bb{Z}[X_{n+1},\ldots,X_{n+m}] = J$).
    But we can prove that every $\bb{M}$-multialgebra is, up to isomorphism, of this form, and thus defines a commutative local ring.
    This correspondence is functorial, i.e. if $\Locc$ denotes the category of commutative local rings and local homomorphisms, then we can define a functor $V\colon\Locc\to\MulAlg(\bb{M})$ by $VA=A^{[\,\,]}$ and $Vf(x_1,\ldots,x_n)=(f(x_1),\ldots,f(x_n))$.
    We can, with difficulty, directly prove that $V$ is an equivalence of categories, but this result is also an immediate consequence of \hyperref[3.2]{Theorem~3.2}.
  \end{rmenv}

  \begin{rmenv}{1.2.3 The multialgebraic theory of real pre-Hilbert spaces}
  \label{1.2.3}
    A finite sequence of vectors in a real pre-Hilbert space is said to be orthonormal if the vectors are all of norm~$1$ and pairwise orthogonal.
    A real matrix with $p$ rows and $n$ columns is said to be orthonormal if its columns form an orthonormal sequence in $\bb{R}^p$.
    Let $\Orth$ be the category of orthonormal matrices, whose objects are the natural numbers, and whose morphisms $n\to p$ are the orthonormal matrices with $p$ rows and $n$ columns, with composition being given by matrix multiplication.
    In particular, there is a unique morphism $0\to n$, namely the empty matrix.
    The category $\Orth$ is in fact equivalent to the category $\Euc$ of Euclidean spaces, which is a full subcategory of the category $\PHilb$ of real pre-Hilbert spaces.
    It thus has finite multisums, by \cite[{}1.1.3]{2}.
    The opposite category $\Orth^\op$ in which we distinguish the family of objects $(X_\rho)_{\rho\in\bb{R}_+}$ defined by $X_0=0$ and $X_\rho=1$ for $\rho>0$ is a multialgebraic theory, which we denote by $\bb{M}$.

    If $E$ is a real pre-Hilbert space, then we define a functor $E^{(\,\,)}\colon\bb{M}\to\Set$ by
    \[
      E^{(n)} =
      \big\{
        (x_1,\ldots,x_n)\in E^n
        :
        \mbox{$x_1,\ldots,x_n$ is orthonormal in $E$}
      \big\}
    \]
    and, for a morphism $A=(a_{ji})\colon n\to p$, by
    \[
      E^{(A)}(x_1,\ldots,x_p) =
      (a_{11}x_1+\ldots+a_{p1}x_p,\ldots,a_{1n}x_1+\ldots+a_{pn}x_p).
    \]
    We can show that this functor is an $\bb{M}$-multialgebra, and, conversely, that every $\bb{M}$-multialgebra is, up to isomorphism, of the above form, and thus defines a real pre-Hilbert space.
    We can thus establish an equivalence between the category $\PHilb$ of real pre-Hilbert spaces and orthogonal linear maps and the category $\MulAlg(\bb{M})$.
    This result is an immediate consequence of \hyperref[3.2]{Theorem~3.2}.
  \end{rmenv}

  \begin{rmenv}{1.2.4 The multialgebraic theory of totally ordered sets}
  \label{1.2.4}
    Let $\Delta_\st$ be the category whose objects are the finite ordinals, and whose morphisms are the
\oldpage{197}
    strictly increasing maps.
    The object $0$ is initial, and the category has finite multisums, with the multisum of $n$ and $p$ being given by the set of pairs of morphisms $(f\colon n\to q,g\colon p\to q)$ that are globally surjective.
    The opposite category $\Delta_\st^\op$ in which we distinguish the object $1$ is a multialgebraic theory, which we denote by $\bb{M}$.

    A totally ordered set $E$ determines an $\bb{M}$-multialgebra $E^{(\,\,)}\colon\bb{M}\to\Set$, defined by
    \[
      E^{(n)} =
      \big\{
        (x_1,\ldots,x_n)\in E^n
        :
        x_1<x_2<\ldots<x_n
      \big\}
    \]
    and, for $f\colon n\to p$, by
    \[
      E^{(f)}(x_1,\ldots,x_n) =
      \big(
        x_{f(1)},\ldots,x_{f(n)}
      \big).
    \]
    Conversely, every $\bb{M}$-multialgebra is, up to isomorphism, of the above form, and thus determines a totally ordered set.
    We thus establish a correspondence between the category $\TotOrd$ of totally ordered sets and strictly increasing maps and the category $\MulAlg(\bb{M})$.
  \end{rmenv}

\end{rmenv}

\begin{itenv}{1.3 Proposition}
\label{1.3}
  $\MulAlg(\bb{M})$ is a multireflexive full subcategory of $\Set^\bb{M}$ that is closed under connected limits, filtered colimits, and cokernels of equivalence relations.
\end{itenv}

\begin{proof}
  \begin{enumerate}[a)]
    \item Let $(F_k)_{k\in\bb{K}}$ be a connected diagram in $\MulAlg(\bb{M})$ whose limit in $\Set^\bb{M}$ is $F$.
      For every finite multiproduct $(\gamma_{ij}\colon Y_j\to X_i)_{(i,j)\in[1,n]\times J}$ in $\bb{M}$, we have \cite[{}3.5.4.a]{2}
      \[
        \begin{aligned}
          \coprod_{j\in J} FY_j
          &= \coprod_{j\in J} \varprojlim_{k\in\bb{K}} F_kY_j
          \cong \varprojlim_{k\in\bb{K}} \coprod_{j\in J} F_kY_j
          \cong \varprojlim_{k\in\bb{K}} \prod_{i=1}^n F_kX_i
        \\&\cong \prod_{i=1}^n \varprojlim_{k\in\bb{K}} F_kX_i
          = \prod_{i=1}^n FX_i.
        \end{aligned}
      \]
      The functor $F$ is thus an $\bb{M}$-multialgebra, and so $\MulAlg(\bb{M})$ is a full subcategory of $\Set^\bb{M}$ that is closed under connected limits.
      To show that $\MulAlg(\bb{M})$ is a multireflexive subcategory of $\Set^\bb{M}$ it suffices, by \cite[Theorem~3.6.1]{2}, to show that the solution-set condition is satisfied.
    \item Let $G$ be an $\bb{M}$-multialgebra, and $f\colon F\to G$ a morphism in $\Set^\bb{M}$.
      Since the category $\Set^\bb{M}$ is regular, $f$ factors as $f=gh$, where $h\colon F\to H$ is a quotient functor, and $g\colon H\to G$ is a sub-functor.
      For every object $X$ of $\bb{M}$, denote by $\bar{H}X$ the set of elements of $G(X)$ of the form $G\omega(y)$, where $\omega\colon Y\to X$ runs over the set of morphisms in $\bb{M}$ whose source is an object $Y$ belonging to a finite multiproduct $(\gamma_{ij}\colon Y_j\to X_i)_{(i,j)\in[1,n]\times J}$ of objects of the family $(X_g)_{g\in G}$, i.e. $Y=Y_{j_0}$, and where $y$ is an element of $GY$ such that $G\gamma_{ij_0}\in HX_i$ for all $i\in[1,n]$.
      It is immediate that this defines a sub-functor $\bar{H}$ of $G$ which contains the sub-functor $H$.
      We will show that $\bar{H}$ is an $\bb{M}$-multialgebra.
      Let $(\delta_{ik}\colon Z_k\to X_i)_{(i,k)\in[1,n]\times K}$ be a finite multiproduct in $\bb{M}$.
      The map $\langle(\bar{H}\delta_{ik})\rangle\colon\coprod_{k\in K}\bar{H}Z_k\to\prod_{i=1}^n\bar{H}X_i$ induced by the bijection $\langle(G\delta_{ik})\rangle\colon\coprod_{k\in K}GZ_k\to\prod_{i=1}^nGX_i$ is injective.
      To show that it is surjective, consider an element $(x_i)\in\prod_{i=1}^n\bar{H}X_i$, which we will show that the unique element $z\in GZ_{k_0}$ satisfying $G\delta_{ik_0}(z)=x_i$ for all $i\in[1,n]$ belongs to $\bar{H}Z_{k_0}$.
\oldpage{198}
      \[
        \begin{tikzcd}[column sep=huge]
          T_{l_0} \ar[rr,"\beta_{il_0}"] \ar[dr,swap,"\omega_0"]
          && Y_1 \ar[dr,"\omega_1"]
        \\&Z_{k_0} \ar[rr,"\delta_{ik_0}"] \ar[ddrr,bend right=20]
          && X_1
        \\T_l \ar[rr,swap,"\beta_{2l}"]
          && Y_2 \ar[dr,"\omega_2"]
        \\&Z_k \ar[rr,swap,"\delta_{2k}"] \ar[uurr,bend left=20]
          && X_2
        \end{tikzcd}
      \]
      For all $i\in[1,n]$, the element $x_i$ belongs to $\bar{H}X_i$, and so there exists a morphism $\omega_i\colon Y_i\to X_i$ and an element $y_i\in GY_i$ satisfying the conditions stated above in the definition of $\bar{H}$.
      Let $(\beta_{il}\colon T_l\to Y_i)_{(i,l)\in[1,n]\times L}$ and $t\in GT_{l_0}$ be such that they satisfy $G\beta_{il_0}(t)=y_i$ for all $i\in[1,n]$.
      The family of morphisms $\omega_i\beta_{il_0}\colon T_{l_0}\to X_i$ factors uniquely through a family of morphisms $(\delta_{ik_1}\colon Z_{k_1}\to X_i)_{i\in I}$ and a morphism $\omega_0\colon T_{l_0}\to Z_{k_1}$.
      The relations
      \[
        (G\delta_{ik_1})(G\omega_0(t))
        = (G\omega_i)(G\beta_{il_0}(t))
        = G\omega_i(y_i)
        = x_i
      \]
      imply that $k_1=k_0$ and that $G\omega_0(t)=z$.
      We can then easily show that $z\in\bar{H}Z_{k_0}$.
      This proves that $\bar{H}$ is an $\bb{M}$-multialgebra.
      We finally obtain a solution-set of morphisms from $F$ to $\MulAlg(\bb{M})$ by noting that, up to isomorphism, there exists a set of quotient functors $H$ of $F$, and thus a set of functors of the form $\bar{H}$.
    \item If $(F_k)_{k\in\bb{K}}$ is a filtered diagram of $\MulAlg(\bb{M})$ that has $F$ as its colimit in $\Set^\bb{M}$, then, for every finite multiproduct $(\gamma_{ij}\colon Y_j\to X_i)_{(i,j)\in[1,n]\times J}$ in $\bb{M}$, we have
      \[
        \begin{aligned}
          \coprod_{j\in J} FY_j
          &= \coprod_{j\in J} \varinjlim_{k\in\bb{K}} F_kY_j
          \cong \varinjlim_{k\in\bb{K}} \coprod_{j\in J} F_kY_j
          \cong \varinjlim_{k\in\bb{K}} \prod_{i=1}^n F_kX_i
        \\&\cong \prod_{i=1}^n \varinjlim_{k\in\bb{K}} F_kX_i
          = \prod_{i=1}^n FX_i
        \end{aligned}
      \]
      and so $F$ is an $\bb{M}$-multialgebra.
      Thus $\MulAlg(\bb{M})$ is closed under filtered colimits.
    \item Let $(m,n)\colon R\rightrightarrows F$ be an equivalence relation in $\MulAlg(\bb{M})$.
      Let $g\colon F\to G$ be its cokernel in $\Set^\bb{M}$.
      Then, for every object $X$ of $\bb{M}$, $RX$ is an equivalence relation on $FX$ whose quotient is $GX$.
      For every finite multiproduct $(\gamma_{ij}\colon Y_j\to X_i)_{(i,j)\in[1,n]\times J}$ in $\bb{M}$, we have
      \[
        \begin{aligned}
          \coprod_{j\in J} GY_j
          &\cong \coprod_{j\in J} (FY_j/RY_j)
          \cong \coprod_{j\in J} FY_j \Big/ \coprod_{j\in J} RY_j
          \cong \prod_{i=1}^n FX_i \Big/ \prod_{i=1}^n RX_i
        \\&\cong \prod_{i=1}^n (FX_i/RX_i)
          \cong \prod_{i=1}^n GX_i.
        \end{aligned}
      \]
      This proves that $G$ is an $\bb{M}$-multialgebra.
      Thus $\MulAlg(\bb{M})$ is closed in $\Set^\bb{M}$ under quotients of equivalence relations.
  \end{enumerate}
\end{proof}

\oldpage{199}
\begin{itenv}{1.4 Proposition}
\label{1.4}
  $\MulAlg(\bb{M})$ is a multicocomplete category, whose equivalence relations are effective, and whose pairs of coequalisable morphisms have cokernels, and it has a universal factorisation of morphisms into monomorphisms and regular epimorphisms.
\end{itenv}

\begin{proof}
  Let $(F_k)$ be a diagram in $\MulAlg(\bb{M})$.
  If $(\iota_k\colon F_k\to F)_{k\in\bb{K}}$ is its colimit in $\Set^\bb{M}$, and if $(f_j\colon F\to G_j)_{j\in J}$ is a universal family of morphisms from $F$ to $\MulAlg(\bb{M})$, then it is immediate that $(f_j\iota_k\colon F_k\to G_j)_{(j,k)\in J\times\bb{K}}$ is a multicolimit of $(F_k)_{k\in\bb{K}}$ in $\MulAlg(\bb{M})$.
  Let $(f,g)\colon F\rightrightarrows G$ be a pair of morphisms in $\MulAlg(\bb{M})$ that is coequalisable by a morphism $h\colon G\to H$ in $\MulAlg(\bb{M})$.
  Denote by $(k_i\colon G\to K_i)_{i\in I}$ a multicokernel of $(f,g)$ in $\MulAlg(\bb{M})$.
  The set $I$ is non-empty, and so the kernel pair $(m,n)\colon R\rightrightarrows G$ of the family $(k_i\colon G\to K_i)_{i\in I}$ exists.
  This is an equivalence relation.
  Let $k\colon G\to K$ be its cokernel.
  Every morphism $k_i$ is of the form $k_i=h_ki$.
  Consequently, $|I|=1$, and $k\cong k_i$ is a cokernel of $(f,g)$.
  The inclusion functor $\MulAlg(\bb{M})\to\Set^\bb{M}$ preserves kernel pairs and fibre products.
  It preserves and reflects regular epimorphisms since these are cokernels of their kernel pairs.
  Since the category $\Set^\bb{M}$ has universal regular factorisations, so too does the category $\MulAlg(\bb{M})$.
\end{proof}



\section{Multialgebraic forgetful functors}
\label{2}

The \emph{structure-forgetful functor} $U_\bb{M}\colon\MulAlg(\bb{M})\to\Set$ is defined by $U_\bb{M}F=\coprod_{g\in G}FX_g$ and, for $f\colon F\to H$ in $\MulAlg(\bb{M})$, by $U_\bb{M}f=\coprod_{g\in G}fX_g$.

\begin{itenv}{2.0 Proposition}
\label{2.0}
  The functor $U_\bb{M}\colon\MulAlg(\bb{M})\to\Set$ has a left multiadjoint;
  it is faithful and reflects isomorphisms;
  it preserves connected limits, filtered colimits, and regular epimorphisms.
\end{itenv}

\begin{proof}
  If we denote by $k_\bb{M}\colon\MulAlg(\bb{M})\to\Set$ the inclusion functor, by $\varphi\colon G\to M$ the functor defined by $\varphi g=X_g$, by $\Set^\varphi\colon\Set^\bb{M}\to\Set^G$ the functor associated to $\varphi$, and by $\Sigma\colon\Set^G\to\Set$ the disjoint sum of sets functor, then we have $U_\bb{M}=\Sigma\Set^\varphi k_\bb{M}$.
  The functor $k_\bb{M}$ has a left multiadjoint (\hyperref[1.3]{Proposition~1.3}), and the functor $\Set^\varphi$ has a left adjoint.
  The functor $\Sigma$ also has a left multiadjoint since, for any set $E$, and writing $\Part(E)$ to mean the set of partitions of $E$ indexed by $G$, the family of maps
  \[
    \left(1_E\colon E\to\bigcup_{g\in G}E_g\right)_{(E_g)_{g\in G}\in\Part(E)}
  \]
  is a universal family of morphisms from $E$ to $\Sigma$.
  We thus deduce that the composite functor $U_\bb{M}$ has a left multiadjoint.

  Let $f,h\colon F\rightrightarrows G$ be morphisms in $\MulAlg(\bb{M})$ such that $Uf=Uh$.
  Then we have $\coprod_{g\in G}f_{X_g}=\coprod_{g\in G}h_{X_g}$, and so $f_{X_g}=h_{X_g}$ for all $g\in G$.
  If $(X_i)_{i\in[1,n]}$ is a finite sequence of objects of $(X_g)_{g\in G}$ that has a multiproduct $(\gamma_{ij}\colon Y_j\to X_i)_{(i,j)\in[1,n]\times J}$, then we have
  \[
    \prod_{i=1}^n f_{X_i} = \prod_{i=1}^n h_{X_i}
  \]
  and consequently
  \[
    \coprod_{j\in J} f_{Y_j} = \coprod_{j\in J} h_{Y_j}
  \]
  and so $f_{Y_j}=h_{Y_j}$ for all $j\in J$.
  But, since every object $Y$ of $\bb{M}$ is of the above form $Y_j$, we have $f_Y=h_Y$.
  Thus $f=h$.

\oldpage{200}
  Let $f\colon F\to G$ be a morphism in $\MulAlg(\bb{M})$ such that $U_\bb{M}f$ is bijective.
  Then $\coprod_{g\in G}f_{X_g}$ is bijective, and so $f_{X_g}$ is bijective for all $g\in G$.
  If $(X_i)_{i\in[1,n]}$ is a finite sequence of objects of $(X_g)_{g\in G}$ that has a multiproduct $(\gamma_{ij}\colon Y_j\to X_i)_{(i,j)\in[1,n]\times J}$, then the map $\prod_{i=1}^nf_{X_i}$ is bijective, and so $\coprod_{j\in J}f_{Y_j}$ is bijective, and consequently $f_{Y_j}$ is bijective for all $j\in J$.
  Since every object $Y$ of $\bb{M}$ is of the above form $Y_j$, we deduce that $f_Y$ is bijective, and consequently that $f$ is an isomorphism.

  The functor $U_\bb{M}$ preserves connected limits since it has a left multiadjoint; it preserves filtered colimits and regular epimorphisms since the functors $\Sigma$, $\Set^\varphi$, and $k_\bb{M}$ all preserve them.
\end{proof}

\begin{rmenv}{2.1 Examples}
\label{2.1}
  It is immediate that the structure-forgetful functors $U_\bb{M}\colon\MulAlg(\bb{M})\to\Set$ for the multialgebraic theories $\bb{M}$ given in \hyperref[1.2]{1.2} are equivalent to the usual structure-forgetful functors.
\end{rmenv}



\section{The multialgebraic theory generated by a functor \texorpdfstring{$U\colon\bb{A}\to\Set$}{U:A->Set} that has a left multiadjoint}
\label{3}

Let $U\colon\bb{A}\to\Set$ be a functor that has a left multiadjoint.
For each set $E$, we choose a universal family of morphisms from $E$ to $U$, and consider only the morphisms that are diagonally universal to $U$ belonging to these families.
We denote by $\bb{L}_0$ the full subcategory of $\bb{A}$ whose objects are the targets of the diagonally universal morphisms to $U$ whose sources are the finite cardinals.

\begin{itenv}{3.0 Lemma}
\label{3.0}
  The category $\bb{L}_0$ is small, has finite multisums, and each of its objects belongs to a finite multisum of objects that are targets of diagonally universal morphisms whose source is the cardinal $1$.
\end{itenv}

\begin{proof}
  Let $(X_i)_{i\in[1,n]}$ be a finite sequence of objects of $\bb{L}_0$.
  For every $i\in[1,n]$, let $E_i$ be a finite cardinal, and $g_i\colon E_i\to UX_i$ a diagonally universal morphism from $E_i$ to $U$.
  Set $(E,\iota)=\coprod_{i=1}^nE_i$, and denote by $(h_j\colon E\to UY_j)_{j\in J}$ a universal family of morphisms form $E$ to $U$.
  Since the cardinal $E$ is finite, the objects $Y_j$ belong to $\bb{L}_0$.
  Set
  \[
    J' =
    \big\{
      j\in J
      :
      \mbox{$h_j\iota_i$ factors through $g_i$ in the form $h_j\iota_i=(U\gamma_{ji})g_i$ for all $i\in[1,n]$}
    \big\}.
  \]
  We will show that $(\gamma_{ji}\colon X_i\to Y_j)_{(i,j)\in[1,n]\times J'}$ is a multisum of $(X_i)_{i\in[1,n]}$.
  Let $(f_i\colon X_i\to Z)_{i\in[1,n]}$ be an inductive cone in $\bb{A}$ with base $(X_i)_{i\in[1,n]}$.
  Then there exists a unique morphism $g\colon E\to UZ$ such that $g\iota_i=(Uf_i)g_i$ for all $i\in I$.
  Denote by $h_j\colon E\to UY_j$ the diagonally universal morphism from $E$ to $U$, and by $f\colon Y_j\to Z$ the morphism in $\bb{A}$ that satisfies $(Uf)h_j=g$.
  The relation $(Uf)h_j\iota_i=g\iota_i=(Uf_i)g_i$ implies the existence of a morphism $\gamma_{ji}\colon X_i\to Y_j$ such that $(U\gamma_{ji})g_i=h_j\iota_i$ and $f\gamma_{ji}=f_i$ for all $i\in[1,n]$.
  Then $j\in J'$ and the inductive cone $(f_i\colon X_i\to Z)_{i\in[1,n]}$ factor uniquely through the inductive cone $(\gamma_{ij}\colon X_i\to Y_j)_{i\in[1,n]}$.
  This shows that $\bb{L}_0$ has finite multisums.

  Let $Y$ be an object of $\bb{L}_0$.
  Then there exists a finite cardinal $E$ and a diagonally universal morphism $h\colon E\to UY$.
  The cardinal $E$ is the finite sum of models of the cardinal $1$,
\oldpage{201}
  say $(E,\iota)=\coprod_{i=1}^nE_i$, with $E_i=1$.
  For all $i\in[1,n]$, the map $h\iota_i\colon E_i\to UY$ factors through a diagonally universal morphism $g_i\colon E_i\to UX_i$.
  We then find ourselves in the situation described above.
  We thus deduce that $Y$ belongs to a multisum of $(X_i)_{i\in[1,n]}$, with each of the $X_i$ being the target of a diagonally universal morphism whose source is the cardinal $1$.
\end{proof}

\begin{rmenv}{3.1 Notation}
\label{3.1}
  The multialgebraic theory $\bb{M}$ \emph{generated} by the functor $U\colon\bb{A}\to\Set$ is the opposite category of the full subcategory $\bb{L}_0$ of $\bb{A}$ whose objects are the targets of the diagonally universal morphisms to $U$ whose sources are the finite cardinals, endowed with the distinguished family of the objects that are targets of diagonally universal morphisms whose source is the cardinal $1$.
  Writing $J_0\colon\bb{L}_0\to\bb{A}$ for the inclusion functor, the \emph{comparison functor} $V\colon\bb{A}\to\MulAlg(\bb{M})$ is defined by
  \[
    V(\,\cdot\,) = \Hom_\bb{A}(J_0(-),\,\cdot\,).
  \]
  It satisfies $U_\bb{M}V\cong U$.
  If $V$ is an equivalence, then the functor $U$ is said to be a \emph{multialgebraic forgetful functor}.
\end{rmenv}

\begin{itenv}{3.2 Theorem}
\label{3.2}
  A functor $U\colon\bb{A}\to\Set$ is a multialgebraic forgetful functor if and only if
  \begin{enumerate}[1)]
    \item it has a left multiadjoint;
    \item it reflects isomorphisms;
    \item $\bb{A}$ has filtered colimits and kernel pairs, and its equivalence relations are effective; and
    \item it preserves filtered colimits and regular epimorphisms.
  \end{enumerate}
\end{itenv}

\begin{proof}
  Since the category $\bb{A}$ does not necessarily have products, we consider here the kernel pairs of a set of morphisms with the same source.
  The conditions are necessary by Propositions~\hyperref[1.3]{1.3}, \hyperref[1.4]{1.4}, and \hyperref[2.0]{2.0}.

  Now consider a functor $U$ satisfying the conditions.
  \begin{enumerate}[a)]
    \item We will show that $U$ reflects regular epimorphisms.
      Let $f\colon X\to Y$ be a morphism in $\bb{A}$ whose image $Uf$ is a surjective map.
      Denote by $(m,n)\colon R\rightrightarrows X$ the kernel pair of $f$, by $g\colon X\to Z$ the cokernel of $(m,n)$, and by $h\colon Z\to Y$ the unique morphism such that $hg=f$.
      \[
        \begin{tikzcd}
          R \rar[shift left,"m"] \rar[shift right,swap,"n"]
          & X \ar[rr,"f"] \ar[dr,swap,"g"]
          && Y
        \\&& Z \ar[ur,swap,"h"]
        \end{tikzcd}
      \]
      \[
        \begin{tikzcd}
          UR \rar[shift left,"Um"] \rar[shift right,swap,"Un"]
          & UX \ar[rr,"Uf"] \ar[dr,swap,"Ug"]
          && UY
        \\&& UZ \ar[ur,swap,"Uh"] \ar[ur,sloped,"\sim"]
        \end{tikzcd}
      \]
      Then $(m,n)$ is the kernel pair of $g$.
      Consequently, $(Um,Un)$ is the kernel pair of $Uf$, and also of $Ug$.
      Since $Uf$ and $Ug$ are surjective maps, they are both cokernels of $(Um,Un)$, and so $Uh$ is bijective.
      Then $h$ is an isomorphism, and $f\cong g$ is a regular epimorphism.
\oldpage{202}
    \item We will show that the functor $J_0\colon\bb{L}_0\to\bb{A}$ is dense.
      We will in fact show that $J_0$ is dense by filtered colimits and $J_0$-absolute cokernels \cite{4}.
      The objects of $\bb{L}_0$ are of finite presentation in $\bb{A}$, since, if $E$ is a finite cardinal, if $(g_i\colon E\to UA_i)_{i\in I}$ is a universal family of morphisms from $E$ to $U$, and if $A=\varinjlim_{k\in\bb{K}}A_k$ is a filtered colimit in $\bb{A}$, then we have
      \[
        \begin{aligned}
          \coprod_{i\in I} \Hom_\bb{A}(A_i,A)
          &\cong \Hom_\Set(E,UA)
          \cong \varinjlim_{k\in\bb{K}} \Hom_\Set(E,UA_k)
        \\&\cong \varinjlim_{k\in\bb{K}} \coprod_{i\in I} \Hom_\bb{A}(A_i,A_k)
          \cong \coprod_{i\in I} \varinjlim_{k\in\bb{K}} \Hom_\bb{A}(A_i,A_k)
        \end{aligned}
      \]
      and so, for all $i\in I$, we have $\Hom_\bb{A}(A_i,A)\cong\varinjlim_{k\in\bb{K}}\Hom_\bb{A}(A_i,A_k)$.
      We thus deduce that the filtered colimits of $\bb{A}$ are $J_0$-absolute \cite{4}.
      Let $A$ be an object of $\bb{A}$.
      Set $K=\{(A_0,x_0):\mbox{$A_0\in\bb{L}_0$ and $x_0\colon A_0\to A$}\}$, and denote by $P$ the set of finite subsets of $K$.
      Then $P$ is a filtered ordered set.
      For $K_0\in P$, consider a multisum of $(A_0)_{(A_0,x_0)\in K_0}$ in $\MulAlg(\bb{M})$, and denote by $(\iota_{A_0,x_0}\colon A_0\to S_{K_0})_{(A_0,x_0)\in K_0}$ the family of morphisms of this multisum, which factors as the family of morphisms $(x_0\colon A_0\to A)_{(A_0,x_0)\in K_0}$ and a morphism $x_{K_0}\colon S_{K_0}\to A$.
      \[
        \begin{tikzcd}[column sep=7em]
          A_0 \ar[ddr,swap,"\iota_{A_0,x_0}"] \ar[ddrrr,bend left=20,"x_0"]
        \\
        \\&S_{K_0} \ar[dr,swap,"\iota_{K_0}"] \ar[rr,"x_{K_0}"]
          && A
        \\&& S \ar[ru,"x"]
        \\A_1 \ar[uur,"\iota_{A_1,x_1}"] \ar[uurrr,bend right=20,swap,"x_1"]
        \end{tikzcd}
      \]

      The objects $S_{K_0}$ are in $\bb{L}_0$.
      For $K_0\subset K_1\in P$, we denote by $\iota_{K_1K_0}\colon S_{K_0}\to S_{K_1}$ the canonical morphism.
      Let $(\iota_{K_0}\colon S_{K_0}\to S)_{K_0\in P}$ be the filtered colimit of $(S_{K_0})_{K_0\in P}$, and $x\colon S\to A$ the morphism defined by $x\iota_{K_0}=x_{K_0}$ for all $K_0\in P$.
      For every object $A_0\in\bb{L}_0$, the map $\Hom_\bb{A}(A_0,x)$ is surjective.
      Since the map $Ux$ is equivalent to the sum $\coprod_{A_0}\Hom_\bb{A}(A_0,x)$, where $A_0$ runs over the targets of the diagonally universal morphisms whose source is $1$, it is also surjective.
      Since the functor $U$ reflects regular epimorphisms, we thus deduce that $x$ is a regular epimorphism.
      It is thus a $J_0$-absolute regular epimorphism.
      We denote by $\bb{L}$ the full subcategory of $\bb{A}$ whose objects are the filtered colimit of objects of $\bb{L}_0$.
      Every object of $\bb{A}$ is then the $J_0$-absolute regular quotient of an object of $\bb{L}$.
      We thus deduce that every object of $\bb{A}$ is the $J_0$-absolute cokernel of morphisms of $\bb{L}$ \cite[Lemma~5.6.1]{4}, and consequently $J_0$ is dense by filtered colimit and $J_0$-absolute cokernels \cite[Def.~2.0]{4}.
    \item The comparison functor $V\colon\bb{A}\to\MulAlg(\bb{M})$ is fully faithful since $J_0$ is dense;
      it preserves filtered colimits since the objects of $\bb{L}_0$ are of finite presentation in $\bb{A}$;
      and it preserves regular epimorphisms since $U$ preserves them, $U_\bb{M}$ reflects them, and since we have an isomorphism $U_\bb{M}V\cong U$.
      We will show that $V$ is an equivalence of categories.
      Denote by $\bb{L}_\bb{M}$ the full subcategory of $\MulAlg(\bb{M})$ whose objects are the filtered colimits of representable $\bb{M}$-multialgebras.
      Every object of $\bb{L}_\bb{M}$ is isomorphic to an object of the form $VA$, where $A$ is an object of $\bb{L}$.
      Let $F$ be an $\bb{M}$-multialgebra.
      By b) applied to $\bb{A}=\MulAlg(\bb{M})$, $F$ is the
\oldpage{203}
      regular quotient of an object of $\bb{L}_\bb{M}$.
      There thus exists an object $A_0\in\bb{L}$ and a regular epimorphism $q_0
      \colon VA_0\to F$.
      \[
        \begin{tikzcd}[sep=huge]
          B_0 \rar[shift left,"r"] \rar[shift right,swap,"s"]
          & A_0 \rar["p_0"]
          & A
        \\A_1 \uar["p_1"] \ar[ur,shift left,"f"] \ar[ur,shift right,swap,"g"]
        \\R_1 \uar[shift left,"h"] \uar[shift right,swap,"l"]
        \end{tikzcd}
      \]
      \[
        \begin{tikzcd}[sep=huge]
          F_0 \rar[shift left,"m"] \rar[shift right,swap,"n"]
          & VA_0 \rar[two heads,"q_0"]
          & F
        \\VA_1 \uar["q_1"] \ar[ur,shift left,"Vf"] \ar[ur,shift right,swap,"Vg"]
        \\VR_1 \uar[shift left,"Vh"] \uar[shift right,swap,"Vl"]
        \end{tikzcd}
      \]

      Let $(m,n)\colon F_0\rightrightarrows VA_0$ be the kernel pair of $q_0$.
      There exists, once again, an object $A_1$ of $\bb{L}$ and a regular epimorphism $q_1\colon VA_1\to F_0$.
      Let $f,g\colon A_1\rightrightarrows A_0$ be morphisms in $\bb{A}$ defined by $Vf=mq_1$ and $Vg=nq_1$, let $(h,l)\colon R_1\rightrightarrows A_1$ be the kernel pair of $(f,g)$, let $p_1\colon A_1\to B_0$ be the cokernel of $(h,l)$, and let $r,s\colon B_0\rightrightarrows A_0$ be the morphisms defined by $rp_1=f$ and $sp_1=g$.
      Then $(Vh,Vf)$ is the kernel pair of $(Vf,Vg)$.
      Since $(m,n)$ is a monomorphic pair, $(Vh,Vl)$ is the kernel pair of $q_1$.
      Since $V$ preserves kernel pairs and regular epimorphisms, the morphism $Vp_1$ is isomorphic to the morphism $q_1$, and so the pair $(Vr,Vs)$ is isomorphic to the pair $(m,n)$.
      Since the pair $(m,n)$ is an equivalence, so too is the pair $(r,s)$.
      It admits a cokernel $p_0\colon A_0\twoheadrightarrow A$.
      The two morphisms $Up_0$ and $q_0$ are thus isomorphic, and so the object $F$ is isomorphic to $VA$.
  \end{enumerate}
\end{proof}



\section{Multialgebraic categories}
\label{4}

A category is \emph{multialgebraic} if it is equivalent to a category $\MulAlg(\bb{M})$ of multialgebras for some multialgebraic theory $\bb{M}$.
By \hyperref[3]{\S3}, it is equivalent to ask that there exist a multialgebraic forgetful functor defined on the category.

\begin{itenv}{4.0 Theorem}
\label{4.0}
  A category is multialgebraic if and only if
  \begin{enumerate}[1)]
    \item it has filtered colimits and kernel pairs, and its equivalence relations are effective;
    \item it has finite multisums; and
    \item it has a proper generating set consisting of projective objects of finite presentation.
  \end{enumerate}
\end{itenv}

\begin{proof}
  Recall that an object $X$ is projective if the functor $\Hom(X,-)$ preserves regular epimorphisms, and is of finite presentation if the functor $\Hom(X,-)$ preserves filtered colimits \cite{5}.
  A category $\MulAlg(\bb{M})$ satisfies conditions 1), 2), and 3) by taking the generating set to be the set of representable $\bb{M}$-multialgebras $\Hom_\bb{M}(X,-)$, where $X\in\bb{M}$.
  Now let $\bb{A}$ be a category satisfying conditions 1), 2), and 3).
  Let $G$ be a proper generating set of $\bb{A}$ consisting of projective objects of finite presentation.
  Define the functor $U\colon\bb{A}\to\Set$ by
  \[
    U(-) = \coprod_{A_0\in G} \Hom_\bb{A}(A_0,-).
  \]
\oldpage{204}
  We will show that $U$ is a multialgebraic forgetful functor.
  The functor $U$ preserves filtered colimits since, for a filtered diagram $(A_i)_{i\in\bb{I}}$ of $\bb{A}$, we have
  \[
    \begin{aligned}
      U(\varinjlim_{i\in\bb{I}} A_i)
      &\cong \coprod_{A_0\in G} \Hom_\bb{A}(A_0,\varinjlim_{i\in\bb{I}} A_i)
      \cong \coprod_{A_0\in G} \varinjlim_{i\in\bb{I}} \Hom_\bb{A}(A_0,A_i)
    \\&\cong \varinjlim_{i\in\bb{I}} \coprod_{A_0\in G} \Hom_\bb{A}(A_0,A_i)
      = \varinjlim_{i\in\bb{I}} UA_i.
    \end{aligned}
  \]
  The functor $U$ preserves regular epimorphisms since, for a regular epimorphism $f$ in $\bb{A}$, the map $\Hom_\bb{A}(A_0,f)$ is surjective for all $A_0\in G$, and so the map $Uf=\coprod_{A_0\in G}\Hom_\bb{A}(A_0,f)$ is surjective too.
  The functor $U$ reflects isomorphism since, for a morphism $f$ in $\bb{A}$ such that $Uf$ is a bijection, for every $A_0\in G$, we have that $\Hom_\bb{A}(A_0,f)$ is a bijection, and so $f$ is an isomorphism.
  It remains to show that $U$ admits a left multiadjoint.
  Let $I$ be a set.
  For every family $(X_i)_{i\in I}$ of objects of $G$ indexed by $I$, choose a multisum $(\gamma_{ij}\colon X_i\to Y_j)_{(i,j)\in I\times J((X_i))}$ of $(X_i)_{i\in I}$ in $\bb{A}$, and for $j\in J((X_i))$ we define the map $g_j\colon I\to\coprod_{A_0\in G}\Hom_\bb{A}(A_0,Y_j)$ by $g_j(i)=\gamma_{ji}$.
  We will show that
  \[
    (g_j\colon I\to UY_j)_{j\in\coprod_{(X_i)\in GI}J((X_i))}
  \]
  is a universal family of morphisms from $I$ to $U$.
  Let $A$ be an object of $\bb{A}$, and let
  \[
    g\colon I\to UA = \coprod_{A_0\in G} \Hom_\bb{A}(A_0,A)
  \]
  be a map.
  For $i\in I$, let $X_i\in G$ be such that $g(i)\in\Hom_\bb{A}(X_i,A)$.
  We thus obtain an inductive cone $(g(i)\colon X_i\to A)_{i\in I}$ in $\bb{A}$ with base $(X_i)_{i\in I}$.
  There thus exists a unique pair $(j,f)$, where $j\in J((X_i))$ and $f\colon Y_j\to A$ satisfy $f\gamma_{ij}=g(i)$ for all $i\in I$.
  For $i\in I$, we have $(Uf)g_j(i)=Uf(\gamma_{ji})=f\gamma_{ji}=g(i)$, and so $(Uf)g_j=g$.
  Suppose further the existence of another factorisation $g=(Uf')g_{j'}$, where $(X'_i)_{i\in I}$ is a family of objects of $G$ indexed by $I$, $j'\in J((X_i))$, and $f\colon Y_{j'}\to A$.
  Then $g(i)\in\Hom_\bb{A}(X'_i,A)$, so $X'_i=X_i$ for all $i\in I$.
  Then $f'\gamma_{j'i}=(Uf')g_{j'}(i)=g(i)$, and so $j'=j$ and $f'=f$.
\end{proof}

\begin{rmenv}{4.1 Examples}
\label{4.1}
  Either \hyperref[4.0]{Theorem~4.0} or \hyperref[3.2]{Theorem~3.2} easily show that the following categories are multialgebraic, and that their structure-forgetful functor with values in $\Set$ is a multialgebraic forgetful functor.
  \begin{longtable}{p{0.5in}p{4.4in}}
    $\bb{K}$ & fields and homomorphisms
  \\$\bb{K}\mathrm{c}$ & commutative fields and homomorphisms
  \\$\bb{K}(p)$ & fields of characteristic~$p$ and homomorphisms
  \\$\bb{K}\mathrm{c}(0)$ & commutative fields of characteristic~$0$ and homomorphisms
  \\$\bb{L}\mathrm{oc}$ & local rings and local homomorphisms
  \\$\bb{L}\mathrm{occ}$ & commutative local rings and local homomorphisms
  \\$\bb{I}\mathrm{nt}$ & integral rings and injective homomorphisms
  \\$\bb{D}\mathrm{om}$ & integral domains and injective homomorphisms
  \\$\bb{R}\mathrm{ed}$ & reduced commutative rings and injective homomorphisms
  \\$\bb{P}\mathrm{rim}$ & primary commutative rings (every zero divisor is nilpotent) and injective homomorphisms
  \end{longtable}
\oldpage{205}
  \begin{longtable}{p{0.5in}p{4.4in}}
    $\bb{Q}\mathrm{Prim}$ & quasi-primary commutative rings ($xy=0$ implies that either $x$ or $y$ is nilpotent) and injective homomorphisms
  \\$\bb{K}\mathrm{dif}$ & differential fields and differential homomorphisms
  \\$\bb{L}\mathrm{ocdif}$ & differential local rings and differential local homomorphisms
  \\$\bb{D}\mathrm{omdif}$ & differential integral domains and injective differential homomorphisms
  \\etc.
  \end{longtable}
  \begin{longtable}{p{0.5in}p{4.4in}}
    $\bb{K}\mathrm{cO}$ & orderable commutative fields and homomorphisms
  \\$\bb{L}\mathrm{occO}$ & commutative local rings such that $1+x_1^2+\ldots+x_n^2$ is invertible for all $x_1,\ldots,x_n$ and local homomorphisms
  \\$\bb{O}\mathrm{rdtot}$ & totally ordered sets and strictly increasing maps
  \\$\bb{K}\mathrm{cord}$ & ordered fields and increasing homomorphisms
  \\$\bb{L}\mathrm{occOrdt}$ & totally ordered commutative local rings and strictly increasing local homomorphisms
  \\$\bb{D}\mathrm{omOrdt}$ & totally ordered integral domains and strictly increasing homomorphisms
  \\etc.
  \end{longtable}
  \begin{longtable}{p{0.5in}p{4.4in}}
    $\bb{G}\mathrm{rOrd}$ & ordered groups and proper increasing homomorphisms ($f(x)\geq0\implies x\geq0$)
  \\$\bb{A}\mathrm{bOrd}$ & ordered abelian groups and proper increasing homomorphisms
  \\$\bb{A}\mathrm{ncOrd}$ & ordered commutative rings and proper increasing homomorphisms
  \\etc.
  \end{longtable}
  \begin{longtable}{p{0.5in}p{4.4in}}
    $\bb{K}\mathrm{cv}$ & commutative fields with absolute values and homomorphisms that preserve the absolute value
  \\etc.
  \end{longtable}
  \begin{longtable}{p{0.5in}p{4.4in}}
    $\bb{N}\mathrm{orm}(\bb{R})$ & normed $\bb{R}$-vector spaces and linear maps that preserve the norm
  \\$\bb{Alg}\mathrm{Norm}(\bb{R})$ & normed $\bb{R}$-algebras and homomorphisms that preserve the norm
  \\$\bb{S}\mathrm{tell}(\bb{C})$ & $\bb{C}^*$-algebras and homomorphisms that preserve the norm
  \\etc.
  \end{longtable}
  \begin{longtable}{p{0.5in}p{4.4in}}
    $\bb{P}\mathrm{Hild}$ & pre-Hilbert spaces and orthogonal linear maps (linear maps that preserve the scalar product)
  \\$\bb{M}\mathrm{et}$ & metric spaces and isometries
  \\$\bb{T}\mathrm{rloc}$ & local lattices ($0\neq1$ and $[x\vee y=1\implies(\mbox{$x=1$ or $y=1$})]$) and local homomorphisms ($f(x)=1\implies x=1$)
  \\$\bb{T}\mathrm{rdloc}$ & distributive local lattices and local homomorphisms
  \\etc.
  \end{longtable}
\end{rmenv}



\section{Proper multialgebraic functors}
\label{5}

\begin{rmenv}{5.0 Definitions}
\label{5.0}
  If $\bb{M}$ and $\bb{N}$ are multialgebraic theories, then a \emph{proper morphism} of multialgebraic theories from $\bb{N}$ to $\bb{M}$ is a functor $m\colon\bb{N}\to\bb{M}$ that is bijective on objects and that preserves both the distinguished family of objects and all finite multiproducts.
  The functor $\MulAlg(m)\colon\MulAlg(\bb{M})\to\MulAlg(\bb{N})$ induced by the
\oldpage{206}
  functor $\Set^m\colon\Set^\bb{M}\to\Set^\bb{N}$ is said to be \emph{proper multialgebraic}.
  It satisfies
  \[
    U_\bb{N}\circ\MulAlg(m) = U_\bb{M}.
  \]
\end{rmenv}

\begin{itenv}{5.1 Theorem}
\label{5.1}
  Every proper multialgebraic functor $\MulAlg(m)\colon\MulAlg(\bb{M})\to\MulAlg(\bb{N})$ is faithful, reflects isomorphism, preserves filtered colimits and regular epimorphisms, and has a left adjoint.
\end{itenv}

\begin{proof}
  The first properties follow from \hyperref[1.3]{Proposition~1.3} and from the fact that $\MulAlg(m)$ is induced by $\Set^m$.
  Define the functors
  \[
    \begin{aligned}
      J_\bb{N}\colon \bb{N}^\op &\to \MulAlg(\bb{N})
    \\J_\bb{M}\colon \bb{M}^\op &\to \MulAlg(\bb{M})
    \end{aligned}
  \]
  by $J_\bb{N}(\,\cdot\,)=\Hom_\bb{N}(\,\cdot\,,-)$ and $J_\bb{M}(\,\cdot\,)=\Hom_\bb{M}(\,\cdot\,,-)$.
  Since the functor $J_\bb{N}$ is dense for filtered colimits and $J_\bb{N}$-absolute cokernels (part~(b) of the proof of \hyperref[3.2]{Theorem~3.2}), and since the category $\MulAlg(\bb{M})$ has filtered colimits and cokernels of pairs of coequalisable morphisms, the left Kan extension of $J_\bb{M}m^\op$ along $J_\bb{N}$ exists and determines a left adjoint functor of the functor $\MulAlg(m)$ \cite[Prop.~3.1]{4}.
\end{proof}

\begin{rmenv}{5.2 Examples}
\label{5.2}

  \begin{rmenv}{5.2.0}
  \label{5.2.0}
    Let $\bb{K}_0$ be the category whose objects are pairs $(n,I)$ consisting of a whole number $n$ and a prime ideal $I$ of $\bb{Z}[X_1,\ldots,X_n]$, and whose morphisms $(n,I)\to(m,J)$ are field homomorphisms $k(I)\to k(J)$, where $k(I)$ (resp. $k(J)$) denotes the field of fractions of the integral domain $\bb{Z}[X_1,\ldots,X_n]/I$ (resp. of $\bb{Z}[X_1,\ldots,X_m]/J$).
    This is a category with finite multisums, calculated as for $\bb{D}_0$ (\hyperref[1.2.1]{1.2.1}).
    The opposite category $\bb{K}_0^\op$ in which we distinguish the objects of the form $(1,I)$ is a multialgebraic theory.
    The category of multialgebras $\MulAlg(\bb{K}_0^\op)$ is equivalent to the category $\bb{K}\mathrm{c}$ of commutative fields.
    A proper morphism of multialgebraic theories $r\colon\bb{D}_0^\op\to\bb{K}_0^\op$ is defined by $r(n,I)=(n,I)$ and $r\langle g_1,\ldots,g_n\rangle=\mbox{the extension of $\langle g_1,\ldots,g_n\rangle$ to fractions}$.
    The proper multialgebraic functor
    \[
      \MulAlg(r)\colon \MulAlg(\bb{K}_0^\op) \to \MulAlg(\bb{D}_0^\op)
    \]
    is equivalent to the inclusion functor $\bb{K}\mathrm{c}\to\Dom$, whose left adjoint sends an integral domain to its field of fractions.
  \end{rmenv}

  \begin{rmenv}{5.2.1}
  \label{5.2.1}
    The proper morphism of multialgebraic theories $s\colon\bb{L}_0^\op\to\bb{K}_0^\op$ is the identity on objects, and sends $[g_1,\ldots,g_n]\colon\bb{Z}[X_1,\ldots,X_n]_I\to\bb{Z}[X_1,\ldots,X_m]_J$ to the quotient homomorphism $s[g_1,\ldots,g_n]\colon k(I)\to k(J)$.
    The proper multialgebraic functor $\MulAlg(s)\colon\MulAlg(\bb{K}_0^\op)\to\MulAlg(\bb{L}_0^\op)$ is equivalent to the inclusion functor $\bb{K}\mathrm{c}\to\Locc$, whose left adjoint sends a commutative local ring to its quotient by its maximal ideal.
  \end{rmenv}

  \begin{rmenv}{5.2.2}
    Let $\bb{P}_0$ be the category whose objects are pairs $(n,I)$ consisting of a whole number $n$ and a prime ideal $I$ of $\bb{Z}[X_1,\ldots,X_n]$, and whose morphisms $(n,I)\to(m,J)$ are the homomorphisms of unital rings $f\colon\bb{Z}[X_1,\ldots,X_n]\to\bb{Z}[X_1,\ldots,X_m]$ such that $f^{-1}(J)=I$.
    This is a category with finite multisums, calculated as for $\bb{D}_0$.
\oldpage{207}
    The opposite category $\bb{P}_0^\op$ in which we distinguish the objects of the form $(1,I)$ is a multialgebraic theory.
    The category of multialgebras $\MulAlg(\bb{P}_0^\op)$ is equivalent to the category $\bb{A}\mathrm{nc/Spec}$ whose objects are pairs $(A,P)$ consisting of a commutative unital ring $A$ and a prime ideal $P$ of $A$, and whose morphisms $(A,P)\to(B,Q)$ are the homomorphisms of unital rings $g\colon A\to B$ such that $g^{-1}(Q)=P$.
    We define a proper morphism of multialgebraic theories $t\colon\bb{P}_0^\op\to\bb{D}_0^\op$ by $t(n,I)=(n,I)$, and by $t(f)=\mbox{quotient of $f$}$.
    The proper multialgebraic functor $\MulAlg(t)\colon\MulAlg(\bb{D}_0^\op)\to\MulAlg(\bb{P}_0^\op)$ is equivalent to the functor $\Dom\to\bb{A}\mathrm{nc/Spec}$ that sends an integral domain $A$ to the pair $(A,\{0\})$, and whose left adjoint sends a pair $(A,P)$ to the integral domain $A/P$.
  \end{rmenv}

  \begin{rmenv}{5.2.3}
  \label{5.2.3}
    The proper morphism of multialgebraic theories $u\colon\bb{P}_0^\op\to\bb{L}_0^\op$ is defined by $u(n,I)=(n,I)$, and by $u(f)=\mbox{localisation of $f$}$.
    The functor
    \[
      \MulAlg(u)\colon \MulAlg(\bb{L}_0^\op) \to \MulAlg(\bb{P}_0^\op)
    \]
    is equivalent to the functor $\Locc\to\bb{A}\mathrm{nc/Spec}$ that sends a local ring $A$ to the pair $(A,M_A)$, where $M_A$ is the maximal ideal of $A$, and whose left adjoint sends a pair $(A,P)$ to the localised ring $A_P$.
  \end{rmenv}

  \begin{rmenv}{5.2.4}
  \label{5.2.4}
    The proper morphism of multialgebraic theories $rt=su\colon\bb{P}_0^\op\to\bb{K}_0^\op$ defines the functor $\MulAlg(rt)\colon\bb{K}\mathrm{c}\to\bb{A}\mathrm{nc/Spec}$ that sends a commutative field $K$ to the pair $(K,\{0\})$, and whose left adjoint sends a pair $(A,P)$ to the field $k(P)$.
  \end{rmenv}

\end{rmenv}



\section{Multimonadic categories of finite rank}
\label{6}

Multimonads and multimonadic categories are defined in \cite{3}.

\begin{rmenv}{6.0 Definitions}
\label{6.0}
  A multimonad $(S,\bb{T})=(S;(T,\eta,\mu))$ on $\Set$ is of \emph{finite rank} if the functor $S$ preserves cofiltered limits and the functor $T$ preserves filtered colimits.
  A category equivalent to $\Set_{/S}^\bb{T}$ and a functor equivalent to $U_S^\bb{T}\colon\Set_{/S}^\bb{T}\to\Set$ are said to be \emph{multimonadic of finite rank} over $\Set$.
\end{rmenv}

\begin{itenv}{6.1 Theorem}
\label{6.1}
  For a functor $U\colon\bb{A}\to\Set$, the following claims are equivalent:
  \begin{enumerate}[(i)]
    \item $U$ is multimonadic of finite rank;
    \item $U$ is multimonadic and $\bb{A}$ has filtered colimits preserved by $U$; and
    \item $U$ is a multialgebraic forgetful functor.
  \end{enumerate}
\end{itenv}

\begin{proof}
  The equivalence (ii)$\iff$(iii) follows immediately from \hyperref[3.2]{Theorem~3.2} and from \cite[Theorem~4.2]{3}.
  We will show the equivalence (i)$\iff$(ii) for $U_S^\bb{T}\colon\Set_{/S}^\bb{T}\to\Set$.
  First of all, it is immediate that the category $\Set_{/S}$ has filtered colimits preserved by $U_S$ if and only if the functor $S$ preserves cofiltered limits.
  If we assume (i), then $\Set_{/S}$ has filtered colimits preserved by $U_S$, and so $\Set_{/S}^\bb{T}$ has
\oldpage{208}
  filtered colimits preserved by $U$, and so $\Set_{/S}^\bb{T}$ has filtered colimits preserved by $U_S^\bb{T}$.
  Now assume (ii).
  Let $(X_i)_{i\in\bb{I}}$ be a filtered diagram in $\Set_{/S}$.
  Denote by $(\iota_i\colon U_SX_i\to E)_{i\in\bb{I}}$ the colimit of $(U_SX_i)_{i\in\bb{I}}$ in $\Set$, and by $(l_i\colon F^\bb{T}X_i\to(Y,y))_{i\in\bb{I}}$ the colimit of $(F^\bb{T}X_i)_{i\in\bb{I}}$ in $\Set_{/S}^\bb{T}$.
  Since the functor $U_S^\bb{T}$ preserves filtered colimits,
  \[
    (U_S^\bb{T}l_i\colon U_S^\bb{T}F^\bb{T}X_i \to U_SY)_{i\in\bb{I}}
  \]
  is a colimit of $(U_S^\bb{T}F^\bb{T}X_i)_{i\in\bb{I}}$.
  The maps
  \[
    (U_s\eta_{X_i}\colon U_SX_i \to U_S^\bb{T}F^\bb{T}X_i)_{i\in\bb{I}}
  \]
  determine, by colimits, a map $p\colon E\to U_SY$ such that $p\iota_i=(U_S^\bb{T}l_i)(U_S\eta_i)$ for all $i\in\bb{I}$.
  There thus exists a unique object $X$ of $\Set_{/S}$ and a unique morphism $\eta\colon X\to Y$ such that $U_S\eta=p$, and a unique diagram $(\gamma_i\colon X_i\to X)_{i\in I}$ such that $(X_i)_{i\in\bb{I}}$.
  Let $(f_i\colon X_i\to Z)_{i\in\bb{I}}$ be an inductive cone with base $(X_i)_{i\in\bb{I}}$.
  Then there exists a unique morphism $g\colon E\to U_SZ$ such that $g\iota_i=U_sf_i$ for all $i\in I$.
  Let $f\colon X'\to Z$ be the unique morphism such that $U_Sf=g$.
  Then
  \[
    \begin{aligned}
      U_S(\eta_Zf)\iota_i
      &= (U_S\eta_Z)(U_Sf)\iota_i
      = (U_S\eta_Z)(U_Sf_i)
      = U_S(\eta_Zf_i)
      = U_S((U^\bb{T}F^\bb{T}f_i)\eta_{x_i})
    \\&= (U_SU^\bb{T}(ll_i))(U_S\eta_{x_i})
      = (U_S^\bb{T}l)(U_S^\bb{T}l_i)(U_S\eta_{x_i})
      = (U_S^\bb{T})p\iota_i
    \\&= U_S((U^\bb{T}l)\eta)\iota_i.
    \end{aligned}
  \]
  We thus deduce the equality $(U^\bb{T}l)\eta=\eta_Zf$, and so $f$ is a morphism $X\to Z$ such that $f\gamma_i=f_i$ for all $i\in I$.
  Thus $\Set_{/S}$ has filtered colimits preserved by $U_S$.
  Since the functor $U_S$ reflects isomorphisms, it also reflects filtered colimits, and since $U$ preserves filtered colimits, $U^\bb{T}$ also preserves filtered colimits.
  We thus deduce that $S$ preserves cofiltered limits, and that $T$ preserves filtered colimits, i.e. that $(S,\bb{T})$ is of finite rank.
\end{proof}



\section{\texorpdfstring{$\alpha$}{alpha}-multialgebraic theories and categories}
\label{7}

We consider a regular infinite cardinal $\alpha$ (say $\alpha=\aleph_0$, $\alpha=\aleph_1$, \ldots).
A family is $\alpha$-small if its index set has cardinality less than $\alpha$.
A multiproduct of an $\alpha$-small family of objects is said to be $\alpha$-small.
A category has \emph{$\alpha$-small multiproducts} if every $\alpha$-small family of objects has a multiproduct.

\begin{rmenv}{7.0 Definitions}
\label{7.0}
  An \emph{$\alpha$-multialgebraic theory} is a small category $\bb{M}$ with $\alpha$-small multiproducts endowed with a small distinguished family $(X_g)_{g\in G}$ of objects such that every object of $\bb{M}$ belongs to an $\alpha$-small multiproduct of objects of this family.

  An \emph{$\bb{M}$-multialgebra} is then a functor $F\colon\bb{M}\to\Set$ that is multicontinuous for $\alpha$-small multiproducts \cite{2}.

  The category $\MulAlg(\bb{M})$ is defined as before.

  A \emph{proper morphism of $\alpha$-multialgebraic theories} is a functor that is bijective on objects and that preserves both the distinguished family of objects and all $\alpha$-small multiproducts.
\end{rmenv}

\oldpage{209}
\begin{rmenv}{7.1}
  All the above results remain true if we substitute:
  \begin{itemize}
    \item ``$\alpha$-small'' for ``finite'';
    \item ``$\alpha$-filtered'' for ``filtered'';
    \item ``$\alpha$-presentable'' for ``of finite presentation''; and
    \item ``rank-$\alpha$'' for ``finite rank''.
  \end{itemize}
\end{rmenv}

\begin{rmenv}{7.2 Examples of $\aleph_1$-multialgebraic categories}
  \begin{longtable}{p{0.6in}p{4.4in}}
    $\bb{M}\mathrm{etcompl}$ & complete metric spaces and isometries
  \\$\bb{M}\mathrm{etcomp}$ & compact metric spaces and isometries
  \\$\bb{B}\mathrm{an}(\bb{R})$ & real Banach spaces and linear maps that preserve the norm
  \\$\bb{A}\mathrm{lgBan}(\bb{R})$ & real Banach algebras and homomorphisms that preserve the norm
  \\$\bb{H}\mathrm{ilb}$ & Hilbert spaces and orthogonal linear maps
  \end{longtable}
\end{rmenv}





%% Bibliography %%

\nocite{*}

\begin{thebibliography}{8}

  \bibitem[0]{0}
  {Barr, M.}
  \newblock {\em Exact Categories and Categories of Sheaves.}
  \newblock Berlin--Heidelberg, LNM \textbf{236} (1971).

  \bibitem[1]{1}
  {Benabou, J.}
  \newblock {\em Structures alg\'{e}briques dans les cat\'{e}gories.}
  \newblock Thesis, Universit\'{e} de Paris (1966).

  \bibitem[2]{2}
  {Diers, Y.}
  \newblock Familles Universelles de Morphismes.
  \newblock {\em Ann. Soc. Sci. Bruxelles} \textbf{93} (1979).

  \bibitem[3]{3}
  {Diers, Y.}
  \newblock Multimonads and Multimonadic Categories.
  \newblock {\em J. Pure Appl. Algebra} \textbf{17} (1980), 153--170.

  \bibitem[4]{4}
  {Diers, Y.}
  \newblock Type de densit\'{e} d'un sous-cat\'{e}gorie pleine.
  \newblock {\em Ann. Soc. Bruxelles} \textbf{90} (1976), 25--47.

  \bibitem[5]{5}
  {Gabriel, P. and Ulmer, F.}
  \newblock {\em Lokal Pr\"{a}sentierbare Kategorien.}
  \newblock Berlin--Heidelberg, LNM \textbf{221} (1971).

  \bibitem[6]{6}
  {Lawvere, F.W.}
  \newblock Functorial Semantics of Algebraic Theories.
  \newblock {\em Proc. National Acad. Sci.} \textbf{50} (1963), 869--873.

  \bibitem[7]{7}
  {Schubert, H.}
  \newblock {\em Categories.}
  \newblock Berlin--Heidelberg (1972).

\end{thebibliography}

\end{document}
