\documentclass{article}

\usepackage{amssymb,amsmath}

\usepackage{hyperref}
\usepackage[nameinlink]{cleveref}
\usepackage{enumerate}
\usepackage{tikz-cd}
\usepackage{graphicx}

\usepackage{mathrsfs}
%% Fancy fonts --- feel free to remove! %%
\usepackage{Baskervaldx}
\usepackage{mathpazo}


\crefname{section}{Section}{Sections}
\crefname{equation}{}{}

%% Theorem environments %%

\usepackage{amsthm}

  \theoremstyle{plain}

  \newtheorem{innercustomtheorem}{Theorem}
  \crefname{innercustomtheorem}{Theorem}{Theorems}
  \newenvironment{theorem}[1]
    {\renewcommand\theinnercustomtheorem{#1}\innercustomtheorem}
    {\endinnercustomtheorem}

  \newtheorem{innercustomproposition}{Proposition}
  \crefname{innercustomproposition}{Proposition}{Propositions}
  \newenvironment{proposition}[1]
    {\renewcommand\theinnercustomproposition{#1}\innercustomproposition}
    {\endinnercustomproposition}

  \newtheorem{innercustomlemma}{Lemma}
  \crefname{innercustomlemma}{Lemma}{Lemmas}
  \newenvironment{lemma}[1]
    {\renewcommand\theinnercustomlemma{#1}\innercustomlemma}
    {\endinnercustomlemma}

  \newtheorem{innercustomcorollary}{Corollary}
  \crefname{innercustomcorollary}{Corollary}{Corollaries}
  \newenvironment{corollary}[1]
    {\renewcommand\theinnercustomcorollary{#1}\innercustomcorollary}
    {\endinnercustomcorollary}


  \theoremstyle{definition}

  \newtheorem*{remark}{Remark}

  \newtheorem{innercustomdefinition}{Definition}
  \crefname{innercustomdefinition}{Definition}{Definitions}
  \newenvironment{definition}[1]
    {\renewcommand\theinnercustomdefinition{#1}\innercustomdefinition}
    {\endinnercustomdefinition}


%% Shortcuts %%

\newcommand{\sh}{\mathscr}
\newcommand{\cat}{\mathcal}
\newcommand{\HH}{\mathrm{H}}

\renewcommand{\geq}{\geqslant}
\renewcommand{\leq}{\leqslant}

\DeclareMathOperator{\ann}{ann}
\DeclareMathOperator{\supp}{supp}

\newcommand{\todo}{\textbf{ !TODO! }}
\newcommand{\oldpage}[1]{\marginpar{\footnotesize$\Big\vert$ \textit{p.~#1}}}


%% Document %%

\usepackage{embedall}
\begin{document}

\renewcommand{\abstractname}{Translator's note.}

\title{Divisors in algebraic geometry}
\author{C.S. Seshadri}
\date{}
\maketitle

\begin{abstract}
  \renewcommand*{\thefootnote}{\fnsymbol{footnote}}
  \emph{This text is one of a series\footnote{\url{https://github.com/thosgood/translations}} of translations of various papers into English.}
  \emph{The translator takes full responsibility for any errors introduced in the passage from one language to another, and claims no rights to any of the mathematical content herein.}
  
  \emph{What follows is a translation (last updated \today) of the French paper:}

  \medskip\noindent
  \textsc{Seshadri, C. S.} Diviseurs en géométrie algébrique. \emph{Séminaire Claude Chevalley}, Volume~\textbf{4} (1958-1959), Talk no.~4, 9~p. {\footnotesize\url{http://www.numdam.org/item/SCC_1958-1959__4__A4_0/}}
\end{abstract}

\setcounter{footnote}{0}

\tableofcontents


%% Content %%

\bigskip\bigskip
\oldpage{4-01}
In the first part of this exposé, we will prove a theorem of Serre on complete varieties \cite{6}, following the methods of Grothendieck \cite{4}.
The second part is dedicated to generalities on divisors.
In the literature, we often call the divisors studied here ``locally principal'' divisors.

The algebraic spaces considered here are defined over an algebraically closed field $K$.
By ``variety'', we mean an irreducible algebraic space.
If $X$ is an algebraic space, we denote by $\sh{O}(X)$, $\sh{R}(X)$, etc. (or simply $\sh{O}$, $\sh{R}$, etc.) the sheaf of local rings, of regular functions, etc. on $X$ (to define $\sh{R}(X)$ we assume that $X$ is a variety).
By ``coherent sheaf'' on $X$, we mean a coherent sheaf of $\sh{O}$-modules on $X$.


\section{Preliminaries}
\label{section1}

\cite{4,5,6}
\medskip

If $M$ is a module over an integral ring $A$ (commutative and with $1$), then we say that an element $m\in M$ is a \emph{torsion element} if there exists some non-zero $a\in A$ such that $a\cdot m=0$.
We say that $M$ is a \emph{torsion module} (resp. \emph{torsion-free module}) if every element of $M$ is a torsion element (resp. if $M\neq0$ and no non-zero element of $M$ is a torsion element).
The torsion elements of $M$ form a torsion submodule of $M$ (denoted by $T(M)$);
if $M\neq0$, then $M/T(M)$ is a torsion-free module.
If $M$ is a torsion module of finite type over $A$, then the ideal $\ann M$ of $A$ (the ideal of $A$ given by the elements $a\in A$ such that $aM=0$) is non-zero.

Let $X$ be an algebraic space and $\sh{F}$ a sheaf of $\sh{O}$-modules on $X$.
We define $\supp\sh{F}$ to be the set of points $x\in X$ such that $\sh{F}_x\neq0$.
If $\sh{F}$ is coherent, then $\supp\sh{F}$ is a closed subset of $X$.
If $X$ is affine, then $\supp\sh{F}$ is the set defined by the ideal $\ann\HH^0(X,\sh{F})$ of the affine algebra $\HH^0(X,\sh{O})$, where $\HH^0(X,\sh{F})$ is considered as a module over $\HH^0(X,\sh{O})$.

A sheaf $\sh{F}$ of $\sh{O}$-modules on a \emph{variety} $X$ is said to be a \emph{torsion sheaf} (resp. \emph{torsion-free sheaf}) if, for every $x\in X$, the module $\sh{F}_x$ over the ring $\sh{O}_x$ is a torsion module (resp. torsion-free module).

\oldpage{4-02}
\begin{proposition}{1}
\label{proposition1}
  If $\sh{F}$ is a coherent sheaf on a variety $X$, then there exists a coherent subsheaf $T(\sh{F})$ of $\sh{F}$ (and only one) such that $(T(\sh{F}))_x = T(\sh{F}_x)$.
\end{proposition}

\begin{proof}
  \todo
\end{proof}


%% Bibliography %%

\nocite{*}
\bibliographystyle{acm}

\begin{thebibliography}{10}

  \bibitem{1}
  {\sc Cartier, P.}
  \newblock Diviseurs et d\'{e}rivations en g\'{e}om\'{e}trie alg\'{e}brique.
  \newblock {({\em Th\`{e}se Sc. math. Paris.} 1958)}
  \newblock (To appear in {\em Bull. Soc. math. France}).

  \bibitem{2}
  {\sc Chevalley, C.}
  \newblock Fondements de la G\'{e}om\'{e}trie alg\'{e}brique
  \newblock Paris, Secr\'{e}tariat math\'{e}matique, 1958, multigraphed.
  \newblock (Class taught at the Sorbonne in 1957--58).

  \bibitem{3}
  {\sc Godement, R.}
  \newblock Propri\'{e}t\'{e}s analytiques des localit\'{e}s
  \newblock In {\em S\'{e}minarie Cartan-Chevalley}, vol.~8, 1955--56.
  \newblock (Talk number 19).

  \bibitem{4}
  {\sc Grothendieck, A.}
  \newblock Sur les faisceaux alg\'{e}briques et les faisceaux analytiques
    coh\'{e}rents.
  \newblock In {\em S\'{e}minaire H. Cartan}, vol.~9.
  \newblock (Talk number 2).

  \bibitem{5}
  {\sc Serre, J.-P.}
  \newblock Faisceaux alg\'{e}briques coh\'{e}rents.
  \newblock {\em Ann. Math. (2)\/}, vol.~61 (1955), 197--279.

  \bibitem{6}
  {\sc Serre, J.-P.}
  \newblock Sur la cohomologie des vari\'{e}t\'{e}s alg\'{e}briques.
  \newblock {\em J. Math. pures et appl. (9)\/}, vol.~36 (1957), 1--16.

  \bibitem{7}
  {\sc Weil, A.}
  \newblock Fibre spaces in algebraic geometry
  \newblock (Notes taken by A. Wallace, 1952)
  \newblock Chicago, University of Chicago, 1955.

\end{thebibliography}

\end{document}
